% CHAPITRE 3
\chapter{Effets de la températures sur les variations journalière des flux de CO2}
\newpage

\section{Introduction}
Les flux de gaz et notamment les flux de CO2 sont fonctions de la température.
La température dépend quand à elle de l'énergie reçue par le soleil et donc varie de façon journalière, saisonnière et au delà !

Afin de palier à ces deux aspects un autre suivi a été mis en place : l'étude des flux de \COO à relativement haute fréquence \todo[inline]{combien ? qu'est ce qu'une haute fréquence ?} pendant 3 jours et sur 4 sites différents \todo[inline]{liste des sites ?}.

Nous avons donc avec ces deux suivis, une vision à la fois sur la variabilité spatiale, au sein d'un site ou inter-site, et une vision sur la variabilité temporelle quelle soit saisonnière, annuelle ou journalière.

Ce schéma n'est bien sur pas parfait, ainsi les sites étudiés restent des sites situés en France alors que la majorité des tourbières se situent à des latitudes plus élevées, dans les zones boréales et sub-boréale.

\todo[inline]{Proportion des tourbières qui ont été exploités ? qui sont encore à l'état naturel ? à mettre en regard avec la représentativité d'une tourbière comme La Guette. Est-elle représentative ? La majorité des tourbières sont perturbées... Sont-elles envahies par des végétaux vasculaires ?}

L'étude d'un système complexe de façon globale permet d'avoir une vision globale, cependant il est difficile de comprendre certains processus quand s'ils sont noyés dans un tel système. 
L'expérimentation, qu'elle soit sur le terrain ou en laboratoire permet de simplifier notre système afin de pouvoir déterminer l'impact de tel ou tel facteur plus particulièrement, afin de mieux comprendre tel ou tel processus.
Ainsi ont été mis en place différentes expérimentation bla bla bla.

\section{Présentation de l'expérimentation}

La respiration de l'écosytème (Re) est mesurée tous les quarts d'heure avec une méthode de chambre fermée.
La chambre, en plexiglas, est recouverte d'un isolant, un ventilateur placé à l'intérieur de la chambre permet d'homogénéiser l'air.
Ce dernier permet d'oculter la lumière du jour, et de conserver une température à l'intérieur de la chambre proche de la température extérieure.
Le \COO est mesuré à l'aide d'une sonde Vaisala (\plop précise).
Chaque mesure dure au maximum 5 minutes, délai permettant d'avoir une stabilisation du flux après la pose de la chambre et suffisant de points pour avoir une pente claire.

Les mesures sont faites en continu pendant 72h sur 4 embases. 
Chaque embase est donc mesuré une fois par heure et l'ordre des mesures a été déterminé de façon aléatoire.

En plus des mesures de \COO un piézomètre et une station météo a été installé à proximité des embases.
La station météo nous permet d'aquérir des données à haute fréquence (1 Hz, une mesure par seconde).
Les paramètres suivis sont, la radiation solaire, la température de l'air à 5 cm, la température du sol à différentes profondeurs (5, 10, 20, 30 cm) et l'humidité.

Des profils de températures réalisés (avec quelle sonde ?) ponctuellement dans les embases permettent de recaler chaque embase par rapport aux profils de la station.

Des mesures de NEE ont été testée, la première série sur la tourbière de LaGuette en utilisant le protocole de la variabilité spatiale (à préciser)
LE problème de ce protocole est l'augmentation de la température à l'intérieur de la chambre.
Cette augmentation peut engendrer dans les cas extrêmes une différence de température de plus de 10\degres C et entrainer l'arrêt de la photosynthèse dans la chambre.
(Probablement par fermeture des stomates des végétaux.)
Pour pallier à ce problème des "bloc de froid" ont été utilisé afin de minimiser la différence de température entre l'air à l'intérieur et à l'extérieur de la chambre.
Cette solution permet de diminuer la différence de température, mais il est difficile de contrôler précisément la température...
Un autre souci lors de l'expérimentation a été la perturbation de la végétation.
Répéter aussi régulièrement les mesures pertube la végétation sur 4 à 5 cm de part et d'autre de l'embase.

\section{synchronisation et profiles (article)}
