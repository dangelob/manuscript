% AVANT PROPOS

\chapter{Avant-propos}
%\addcontentsline{toc}{chapter}{Avant-propos}

Ce travail a été mené conjointement au LPC2E (Laboratoire de Physique et Chimie de l’Environnement et de l’Espace) dirigé par Michel TAGGER, et à l'ISTO (Institut des Sciences de la Terre d’Orléans) dirigé par Bruno SCAILLET, respectivement au sein des équipes « Atmosphère » et Biogéosystèmes Continentaux ». J'ai bénéficié d’un financement de la région Centre octroyé au LPC2E. 

La problématique de la thèse relève complètement des thématiques développées :
\begin{itemize}
\item au sein de l’OSUC (Observatoire des Sciences de l’Univers en région Centre) dirigé par Yves COQUET, et regroupant le LPC2E et l’ISTO. Ces travaux s’insèrent plus précisément dans la thématique fédérative « Atmosphère Terrestre et Interfaces »
\item au sein du Service National d’Observation Tourbières, labellisé par l’INSU SIC (Surfaces et Interfaces Continentales) en 2012, porté par l’OSUC et dont la coordination scientifique est assurée par Fatima LAGGOUN (ISTO).
\end{itemize}
