%%%%%%%%%%%%%%%%%%%%%%%%%%%%%%%%%%%%%%%%%%%%%%%%%%%%%%%%%%%%%%%%%%%%%%%%%%%%%%%%
% 4e de couverture
%%%%%%%%%%%%%%%%%%%%%%%%%%%%%%%%%%%%%%%%%%%%%%%%%%%%%%%%%%%%%%%%%%%%%%%%%%%%%%%%
% set-up -----------------------------------------------------------------------
\pagenumbering{gobble} % stop page numbering
\newgeometry{left=2cm,top=1.5cm,right=1.5cm,bottom=1.5cm} % set up margin
\usefont{T1}{phv}{m}{n} % Set up font style
{\parindent0pt % disables indentation for all the text between { and }

% Résumé ---------------------------------------------------------------------
\begin{center}
		\large{Benoît D'ANGELO}\\ 
		\textbf{Variabilité spatio-temporelle des émissions de GES dans une tourbière à Sphaignes : effets sur le bilan de carbone}
\end{center}

\begin{framed}
	\begin{minipage}{\dimexpr\textwidth-2\fboxrule-2\fboxsep}
	Les tourbières ne représentent que 2 à 3\% des terres émergées mais stockent entre 10 et 25\% du carbone accumulé dans les sols.
	Les conditions de saturation en eau importante de ces écosystèmes diminuent la décomposition des matières organiques et favorise leur préservation.
	Ces écosystèmes sont cependant soumis à des contraintes anthropiques et climatiques importantes qui posent la question de leur devenir ainsi que celui du stock de carbone qu'ils hébergent.
	Une meilleure compréhension de ces écosystèmes est nécessaire afin de déterminer quels sont les facteurs qui contrôlent leur émissions de gaz à effets de serre (GES) et surtout comment les contrôlent-ils.\par
	Ce travail a donc consisté à suivre les émissions de GES et les facteurs contrôlant d'une tourbière de Sologne (La tourbière de La Guette) afin d'établir son bilan de carbone.
	En parallèle des expérimentations sur l'hydrologie ont été menées afin d'en préciser les effets sur les flux, et un suivi ponctuel sur différents sites a été réalisé dans le but d'étudier la variabilité à l'échelle journalière.\par
	Les résultats de ces travaux montrent que la tourbière de La Guette fonctionne comme un puits de carbone et ce malgré un niveau de nappe élevé ce qui suggèrent un effet de l'histoire antérieure du site.
	Il montrent également l'importance de la variabilité spatiale des flux que l'on peut estimer à l'échelle d'un site.
	Les expérimentations confirment l'importance de l'hydrologie et mettent en avant l'importance à haut niveau de nappe d'eau de phénomènes liés au transport des gaz entre leurs zones de production et l'atmosphère.
	Enfin l'étude de la variabilité journalière montre que la sensibilité à la température de la respiration peut être différente le jour et la nuit et que la synchronisation entre les températures du sol et la respiration peuvent améliorer la représentation de cette dernière.
Mots cl\'es : Tourbière, Gaz à effet de serre, \coo, \chh, bilan de carbone
	\end{minipage}
\end{framed}

\vfill

% Abstract ---------------------------------------------------------------------
\begin{center}
	\large \textbf{Spatio-temporal variability of Greenhouse gases emissions in a Sphagnum peatland: effects on carbon balance}
\end{center}

\begin{framed}
	\begin{minipage}{\dimexpr\textwidth-2\fboxrule-2\fboxsep}
	R\'esum\'e : (1700 caract\`eres max.)\par
	Lorem ipsum dolor sit amet, consectetur adipiscing elit. Proin volutpat ipsum id purus ultrices lobortis. Maecenas ornare enim quis eros. Nunc eget mauris ut quam malesuada mattis. Vestibulum ante ipsum primis in faucibus orci luctus et ultrices posuere cubilia Curae; Integer vel tellus. Nam rutrum, purus non sodales rhoncus, quam magna imperdiet eros, sit amet euismod justo metus at orci. Suspendisse neque turpis, feugiat interdum, faucibus vel, aliquet quis, risus. Etiam est elit, eleifend a, consequat sit amet, scelerisque nec, odio. Quisque id odio quis libero iaculis tincidunt. Sed non mi. Morbi aliquam commodo nibh. Integer justo purus, pulvinar a, suscipit vel, iaculis a, justo. Morbi ut orci. Maecenas fringilla orci. Phasellus auctor, enim vitae tempus egestas, justo mi cursus sem, vel blandit leo turpis vitae quam. Etiam sit amet felis vitae eros ornare porttitor.\par
	Curabitur felis velit, aliquam at, aliquet in, iaculis vitae, velit. Nunc lobortis magna id ligula. Vestibulum ante ipsum primis in faucibus orci luctus et ultrices posuere cubilia Curae; Integer congue ultrices mi.
	Isdem diebus Apollinaris Domitiani gener, paulo ante agens palatii Caesaris curam, ad Mesopotamiam missus a socero per militares numeros immodice scrutabatur, an quaedam altiora meditantis iam Galli secreta susceperint scripta, qui conpertis Antiochiae gestis per minorem Armeniam lapsus Constantinopolim petit.\par
Mots cl\'es : mot 1, mot 2, ...
	\end{minipage}
\end{framed}

% logos and university name ----------------------------------------------------
\includegraphics[width=0.2\textwidth, valign=c]{./images/logos/pucvl}
\hfill
%{\LARGE\textbf{UNIVERSITÉ D'ORLÉANS}}
\begin{minipage}{.5\textwidth}
\begin{center}
LPC2E/CNRS\\
3A, Avenue de la Recherche Scientifique\\
45071 Orléans cedex 2\\
France \\
\end{center}
\end{minipage}
\hfill
\includegraphics[width=0.2\textwidth, valign=c]{./images/logos/univ}

}