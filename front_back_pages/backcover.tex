%%%%%%%%%%%%%%%%%%%%%%%%%%%%%%%%%%%%%%%%%%%%%%%%%%%%%%%%%%%%%%%%%%%%%%%%%%%%%%%%
% 4e de couverture
%%%%%%%%%%%%%%%%%%%%%%%%%%%%%%%%%%%%%%%%%%%%%%%%%%%%%%%%%%%%%%%%%%%%%%%%%%%%%%%%
% set-up -----------------------------------------------------------------------
\pagenumbering{gobble} % stop page numbering
\newgeometry{left=2cm,top=1.5cm,right=1.5cm,bottom=1.5cm} % set up margin
\usefont{T1}{phv}{m}{n} % Set up font style
{\parindent0pt % disables indentation for all the text between { and }

% Résumé ---------------------------------------------------------------------
\begin{center}
		\large{Benoît D'ANGELO}\\ 
		\textbf{Variabilité spatio-temporelle des émissions de GES dans une tourbière à Sphaignes : effets sur le bilan de carbone}
\end{center}

\begin{framed}
	\begin{minipage}{\dimexpr\textwidth-2\fboxrule-2\fboxsep}
	Les tourbières représentent 2 à 3\% des terres émergées et stockent entre 10 et 25\% du carbone des sols. 
%	La saturation en eau élevée de ces écosystèmes favorise la préservation des matières organiques. 
	Les tourbières sont soumises à des contraintes anthropiques et climatiques importantes qui posent la question de la pérennité de leur fonctionnement en puits de C et de leur stock. Une meilleure compréhension de ces écosystèmes est nécessaire pour déterminer les facteurs et les effets et interactions de ces facteurs sur les émissions de gaz à effet de serre (GES).
	Ce travail a consisté à suivre les émissions de GES et les facteurs contrôlant dans La tourbière de La Guette (Sologne) pour établir son bilan de C. En parallèle des expérimentations sur l'effet de l'hydrologie sur les flux ont été menées, enfin un suivi sur 4 sites a été réalisé pour étudier la variabilité à l'échelle journalière.
	Les résultats de ces travaux montrent que la tourbière de La Guette a fonctionné en source de C (-220 $\pm$ 33 gC m\textsuperscript{-2} an\textsuperscript{-1}) et ce malgré un niveau de nappe élevé. %, suggérant un effet de l'histoire antérieure du site. 
	Ils montrent également l'importance de la variabilité spatiale des flux estimés à l'échelle d'un site. Les expérimentations confirment l'importance de l'hydrologie et suggèrent à haut niveau de nappe d'eau des phénomènes liés au transport des gaz. Enfin l'étude de la variabilité journalière montre que la sensibilité de la respiration à la température peut être différente le jour et la nuit et que la synchronisation entre les températures du sol et la respiration peuvent améliorer la représentation de cette dernière.
%	Les tourbières ne représentent que 2 à 3\% des terres émergées mais stockent entre 10 et 25\% du carbone accumulé dans les sols.
%	Les conditions de saturation en eau importante de ces écosystèmes diminuent la décomposition des matières organiques et favorise leur préservation.
%	Ces écosystèmes sont cependant soumis à des contraintes anthropiques et climatiques importantes qui posent la question de leur devenir ainsi que celui du stock de carbone qu'ils hébergent.
%	Une meilleure compréhension de ces écosystèmes est nécessaire afin de déterminer quels sont les facteurs qui contrôlent leur émissions de gaz à effets de serre (GES) et surtout comment les contrôlent-ils.\par
%	Ce travail a donc consisté à suivre les émissions de GES et les facteurs contrôlant d'une tourbière de Sologne (La tourbière de La Guette) afin d'établir son bilan de carbone.
%	En parallèle des expérimentations sur l'hydrologie ont été menées afin d'en préciser les effets sur les flux, et un suivi ponctuel sur différents sites a été réalisé dans le but d'étudier la variabilité à l'échelle journalière.\par
%	Les résultats de ces travaux montrent que la tourbière de La Guette fonctionne comme un puits de carbone et ce malgré un niveau de nappe élevé ce qui suggèrent un effet de l'histoire antérieure du site.
%	Il montrent également l'importance de la variabilité spatiale des flux que l'on peut estimer à l'échelle d'un site.
%	Les expérimentations confirment l'importance de l'hydrologie et mettent en avant l'importance à haut niveau de nappe d'eau de phénomènes liés au transport des gaz entre leurs zones de production et l'atmosphère.
%	Enfin l'étude de la variabilité journalière montre que la sensibilité à la température de la respiration peut être différente le jour et la nuit et que la synchronisation entre les températures du sol et la respiration peuvent améliorer la représentation de cette dernière.

\textbf{Mots cl\'es :} Tourbière à Sphaignes, flux de \coo et \chh, bilan de carbone, facteurs biotiques et abiotiques, modélisation
	\end{minipage}
\end{framed}

\vfill

% Abstract ---------------------------------------------------------------------
\begin{center}
	\large \textbf{Spatio-temporal variability of Greenhouse gases emissions in a Sphagnum peatland: effects on carbon balance}
\end{center}

\begin{framed}
	\begin{minipage}{\dimexpr\textwidth-2\fboxrule-2\fboxsep}
	Peatlands cover only 2 to 3\% of the land area but store between 10 and 25\% of the soil carbon.
%	Waterlogged conditions in these ecosystems promotes organic matter preservation.
	The outcome of the anthropic and climatic pressure on these ecosystems is uncertain regarding their functions and storage.
	A better understanding of these ecosystems is needed to determine the factors and their interactions on greenhouse gas (GHG) emission.
	This work consist in monitoring GHG emissions and controlling factors in a \textit{Sphagnum} peatland to estimate its carbon balance.
	Experimentation on mesocosms were carried out to explore the effect of hydrology on the fluxes and a monitoring on 4 sites was made to study the daily variability.
	Results show that La Guette peatland was a carbon source (-220 $\pm$ 33 gC m\textsuperscript{-2} an\textsuperscript{-1}) in spite of the high water table level.
	The importance of the spatial variability measured in the site was also demonstrate.
	The hydroloy effect was confirmed by the mesocosms experiments and high water table level shows that gas transport might have an effect. 
	Finally the study of the daily variability show that the temperature sensitivity of the respiration might be different between day and night and that synchronising soil temperatures and respiration can improve the respiration representation.

\textbf{Keywords:} \textit{Sphagnum} peatland, \coo and \chh fluxes, carbon balance, biotic and abiotic factors
	\end{minipage} 
\end{framed}

% logos and university name ----------------------------------------------------
\includegraphics[width=0.15\textwidth, valign=c]{./images/logos/LPC2E}
\hfill
%{\LARGE\textbf{UNIVERSITÉ D'ORLÉANS}}
\begin{minipage}{.5\textwidth}
\begin{center}
LPC2E/CNRS\\
3A, Avenue de la Recherche Scientifique\\
45071 Orléans cedex 2\\
France \\
\end{center}
\end{minipage}
\hfill
\includegraphics[width=0.15\textwidth, valign=c]{./images/logos/region_centre}
%}