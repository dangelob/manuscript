% Formattage headers footers ---------------------------------------------------
% package fancyhdr
\fancyhf{}

% Définition des différents style de header
%\fancypagestyle{plain}{ % Redéfinition des entête et pied pour page chapitre
%\fancyhf{}
%\fancyfoot[C]{\thepage}
%\renewcommand{\headrulewidth}{0pt}
%\renewcommand{\footrulewidth}{0pt}
%}

\fancypagestyle{frontmatter}{% Définition des entête et pied de page pour frontmatter
  \renewcommand{\headrulewidth}{0pt}% No header rule
  \renewcommand{\footrulewidth}{0pt}% No footer rule
  \fancyhf{}% Clear header/footer
  \fancyfoot[C]{\thepage}%
}

\fancypagestyle{mainmatter}{% Définition des entête et pied de page pour mainmatter
  \fancyhf{}
  \fancyhead[LE]{\normalfont\nouppercase{\rightmark}}
  \fancyfoot[LE,RO]{\thepage} % partie gauche du pied de page
  \fancyhead[RO]{\normalfont\nouppercase{\leftmark}} % partie droite de l'en-tête
  \fancyfoot[RO]{\thepage}  % partie droite du pied de page
  \renewcommand{\headrulewidth}{0.4pt}
}

\fancypagestyle{backmatter}{% Définition des entête et pied de page pour frontmatter
  \renewcommand{\headrulewidth}{0pt}% No header rule
  \renewcommand{\footrulewidth}{0pt}% No footer rule
  \fancyhf{}% Clear header/footer
  \fancyfoot[C]{\thepage}%
}

%% style vide pas de footer (aucun numéro de page) pas de header
%\fancypagestyle{emptyPerso}{% Définition des entête et pied de page pour frontmatter
%  \renewcommand{\headrulewidth}{0pt}% No header rule
%  \renewcommand{\footrulewidth}{0pt}% No footer rule
%  \fancyhf{}% Clear header/footer
%}

% Formattage titles ------------------------------------------------------------
% package titlesec
%\titleformat{\chapter}{\Huge}{\thechapter}{1em}{}	% format les chapitres
%\titlespacing*{\chapter}{0pt}{3ex plus 1ex minus .2ex}{4.3ex plus .2ex}

\titleformat{\section}{\huge}{\thesection}{1em}{}	% format les sections
\titlespacing*{\section}{0pt}{4ex plus 1ex minus .2ex}{3.5ex plus .2ex}

% \titlespacing*{<command>}{<left>}{<before-sep>}{<after-sep>}
% <left> : left margin
% <before-sep> : vertical space before title
% <after-sep> : vertical space after title
% spacing: how to read {12pt plus 4pt minus 2pt}
%           12pt is what we would like the spacing to be
%           plus 4pt means that TeX can stretch it by at most 4pt
%           minus 2pt means that TeX can shrink it by at most 2pt
%       This is one example of the concept of, 'glue', in TeX
%

% Onglets ----------------------------------------------------------------------
\DeclareFixedFont{\ongletfont}{T1}{pag}{b}{n}{16pt}
%\newlength{\ongletwidth}
%\newlength{\ongletheight}
%\setlength{\ongletheight}{32pt}
%\setlength{\ongletwidth}{.96cm}
%
%\newcommand{\b@iteonglet}{%
%	\colorbox[gray]{.7}{% une boîte avec un fond gris contenant
%	% la boîte paragraphe de largeur et hauteur fixée :
%	\parbox[t][\ongletheight][s]{\ongletwidth}{%
%		\vfill%
%		\centering%
%		% on applique un effet miroir selon la parité de la page
%		\ifthenelse{\isodd{\value{page}}}{%
%		\ongletfont\thechapter}{%
%		\reflectbox{\ongletfont\thechapter}}%
%\vfill}}}