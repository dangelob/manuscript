%%%%%%%%%%%%%%%%%%%%%%%%%%%%%%%%%%%%%%%%%%%%%%%%%%%%%%%%%%%%%%%%%%%%%%%%%%%%%%%%
% Compilation :
% pdflatex
% bibtex
% makeindex -s perso_index.ist "phdthesis".idx
%%%%%%%%%%%%%%%%%%%%%%%%%%%%%%%%%%%%%%%%%%%%%%%%%%%%%%%%%%%%%%%%%%%%%%%%%%%%%%%%
% commandes pour passer le validateur CINES : 
% 	gs -DSAFER -dNOPAUSE -dBATCH -sDEVICE=pdfwrite -dEmbedAllFonts=true -sOutputFile=outputFile.pdf -f inputFile.pdf
% n'augmente pas la taille du fichier
% Alternative : 
% 	pdftk phdthesis.pdf output FichierCorrect.pdf
% (augmente fortement la taille du fichier)
%%%%%%%%%%%%%%%%%%%%%%%%%%%%%%%%%%%%%%%%%%%%%%%%%%%%%%%%%%%%%%%%%%%%%%%%%%%%%%%%
\documentclass[a4paper,12pt, twoside]{book}
% Packages base ----------------------------------------------------------------
\usepackage[T1]{fontenc}			%codage de caractères T1
\usepackage[utf8]{inputenc}			%caractères accentués
\usepackage[english,francais]{babel}		%adaptation de LaTeX au français
\usepackage{graphicx}				%inclure des graphiques
\usepackage[includeheadfoot, hmargin=25mm, vmargin=15mm, bindingoffset=0.5cm]{geometry}

% Packages needed for back and front page --------------------------------------
% Previous one +
\usepackage[export]{adjustbox}		%allow to align images by the top
\usepackage{framed}					%allow to do the box arround the title

% Other packages ---------------------------------------------------------------
\usepackage[load-configurations=abbreviations]{siunitx}		% Pour les unités
\sisetup{locale = FR,
detect-all, % permet de garder la même font que dans le texte
separate-uncertainty = true, % signe plus ou moins avant incertitudes
multi-part-units=single, % enlève les parenthèse autour nombre+incertitdes
list-final-separator= { et },   % Place "et" à la fin de la liste
list-units=single               % L'unité ne s'affiche qu'au dernier élément
}
%\usepackage{array}					% Outils supplémentaires pour les tableaux
%\usepackage{multicol}				% Plusieurs colonnes possibles
%\usepackage{indentfirst} 			% indenter le premier paragraphe après une nouvelle section
%\usepackage{latexsym}				% ajoute des symboles
\usepackage{fixltx2e}				% nécessaire pour la commande \textsubscript (générée par zotero - biblio)
\usepackage{fancyhdr}
\usepackage{textcomp} 				% améliore l'affichage des symbol (degré °)
\usepackage{caption}
\usepackage{subcaption}				% permettre sous-figures
\usepackage{amsmath}				% complement mathematique
\usepackage{amssymb}				% ajoute des symboles
\usepackage{makeidx}				% Pour l'index
\usepackage{xspace}					% Permet les espace après macro newcommand
\usepackage[authoryear]{natbib}
\bibliographystyle{apalike-fr}
\usepackage{setspace}				% Interligne double
\usepackage{emptypage}				% pre­vents page num­bers and head­ings from ap­pear­ing on empty pages
\usepackage{booktabs}				% Better tables
\usepackage{rotating}				% Rotation des tableaux
\usepackage{array}				% Rotation des tableaux
\usepackage{titlesec}				% Formattage des section
\usepackage{xcolor}					% Définition de couleur
\usepackage{float}					% Positionnement des flottants
\usepackage{wrapfig}				% Texte autour des figures

%Options: Sonny ?, Lenny, Glenn, Conny, Rejne, Bjarne, Bjornstrup ?
\usepackage[Glenn]{fncychap}


\usepackage{ifthen}

\usepackage[colorlinks = true, allcolors=dbleu]{hyperref}
\usepackage[french]{minitoc}		% Add chapter toc (load after hyperref)

% document formatting ----------------------------------------------------------
% Formattage headers footers ---------------------------------------------------
\fancyhf{}

% Définition des différents style de header
%\fancypagestyle{plain}{ % Redéfinition des entête et pied pour page chapitre
%\fancyhf{}
%\fancyfoot[C]{\thepage}
%\renewcommand{\headrulewidth}{0pt}
%\renewcommand{\footrulewidth}{0pt}
%}

\fancypagestyle{frontmatter}{% Définition des entête et pied de page pour frontmatter
  \renewcommand{\headrulewidth}{0pt}% No header rule
  \renewcommand{\footrulewidth}{0pt}% No footer rule
  \fancyhf{}% Clear header/footer
  \fancyfoot[C]{\thepage}%
}

\fancypagestyle{mainmatter}{% Définition des entête et pied de page pour mainmatter
  \fancyhf{}
  \fancyhead[LE]{\normalfont\nouppercase{\leftmark}}
  \fancyfoot[LE,RO]{\thepage} % partie gauche du pied de page
  \fancyhead[RO]{\normalfont\nouppercase{\rightmark}} % partie droite de l'en-tête
  \fancyfoot[RO]{\thepage}  % partie droite du pied de page
  \renewcommand{\headrulewidth}{0.4pt}
}

\fancypagestyle{backmatter}{% Définition des entête et pied de page pour frontmatter
  \renewcommand{\headrulewidth}{0pt}% No header rule
  \renewcommand{\footrulewidth}{0pt}% No footer rule
  \fancyhf{}% Clear header/footer
  \fancyfoot[C]{\thepage}%
}

%% style vide pas de footer (aucun numéro de page) pas de header
%\fancypagestyle{emptyPerso}{% Définition des entête et pied de page pour frontmatter
%  \renewcommand{\headrulewidth}{0pt}% No header rule
%  \renewcommand{\footrulewidth}{0pt}% No footer rule
%  \fancyhf{}% Clear header/footer
%}				% contient des macros formattage
% Ajout de commandes utiles
\newcommand\plop{\textbf{(Réf needed)\xspace}}
%\newcommand\coo{CO$_{2}$\xspace}
\newcommand\coo{CO\textsubscript{2}\xspace}
%\newcommand\chh{CH$_{4}$\xspace}
\newcommand\chh{CH\textsubscript{4}\xspace}
%\newcommand\qtn{Q$_{10}$\xspace}
\newcommand\qtn{Q\textsubscript{10}\xspace}

\newcommand\fchh{F\textsubscript{CH\textsubscript{4}}\xspace}

% Définition d'unités SIunitx
% µmolm2s1
\DeclareSIUnit{\uml}{\micro\mole\per\square\meter\per\second}
% µSm2s1
\DeclareSIUnit{\usml}{\micro\siemens\per\square\meter\per\second}
% gCm2s1
\DeclareSIUnit{\gcms}{\gram C m^{-2} s^{-1}}
% gCm2s1
\DeclareSIUnit{\gcma}{\gram C m^{-2} an^{-1}}
% gC m2
\DeclareSIUnit{\gcm}{\gram C m^{-2}}
% PgC an-1
\DeclareSIUnit{\pgca}{\peta\gram C an^{-1}}
% PgC 
\DeclareSIUnit{\pgc}{\peta\gram C}
% année
\DeclareSIUnit\year{an}

% Environnemment Définition
\definecolor{coulcitation}{RGB}{222,222,222}%

\newsavebox{\boiteremarque}
\newenvironment{pdef}{%
% clause begin
\begin{lrbox}{\boiteremarque}% début mise en boîte
\begin{minipage}{.85\textwidth}}{%
\setlength{\parindent}{10pt}%
% clause end
\end{minipage}
\end{lrbox}% fin mise en boîte
% production de la boîte encadrée
\begin{center}
%\fbox{\usebox{\boiteremarque}}
\setlength{\fboxsep}{10pt}%
\colorbox{coulcitation}{\usebox{\boiteremarque}}
\end{center}}

				% contient des macros "outils"

% temporary setup/package ------------------------------------------------------
%\geometry{showframe=true}
%\usepackage{showframe}
%\usepackage[textwidth=60pt,textsize=footnotesize]{todonotes}
\usepackage{lineno}					% Numéro de lignes

% Autres -----------------------------------------------------------------------
\makeindex							% "lancer" le log de l'index

\dominitoc[n]						% création des minitoc sans titre (opt n)
\mtcsetrules{minitoc}{off}			% retire les lignes horizontale des minitoc

%\setlength{\marginparwidth}{60pt}	% taille marge (todo ?)
%\setlength{\headheight}{20pt}		% taille header 28pt apparemment bien pour fancyhdr

% set up path to graphs
%\graphicspath{ {images/} }
\graphicspath{{./images/}}

% partie à compiler 
%\includeonly{content/chapitre1, content/introduction}

%%%%%%%%%%%%%%%%%%%%%%%%%%%%%%%%%%%%%%%%%%%%%%%%%%%%%%%%%%%%%%%%%%%%%%%%%%%%%%%%
%%%%%%%%%%%%%%%%%%%%%%%%%%%%%%%%%%%%%%%%%%%%%%%%%%%%%%%%%%%%%%%%%%%%%%%%%%%%%%%%
% Document start
%%%%%%%%%%%%%%%%%%%%%%%%%%%%%%%%%%%%%%%%%%%%%%%%%%%%%%%%%%%%%%%%%%%%%%%%%%%%%%%%
%%%%%%%%%%%%%%%%%%%%%%%%%%%%%%%%%%%%%%%%%%%%%%%%%%%%%%%%%%%%%%%%%%%%%%%%%%%%%%%%
\begin{document}

% setup ------------------------------------------------------------------------
% Changer la françisation de certaines commandes
\renewcommand{\tablename}{Tableau}
%\renewcommand{\figurename}{Nom de figure}
\renewcommand{\listfigurename}{Liste des figures}%
%\renewcommand{\listtablename}{Nouveau nom}%

% Proportion d'une page qui peut être un float
\renewcommand{\topfraction}{.80}

%%%%%%%%%%%%%%%%%%%%%%%%%%%%%%%%%%%%%%%%%%%%%%%%%%%%%%%%%%%%%%%%%%%%%%%%%%%%%%%%
% Frontmatter
%%%%%%%%%%%%%%%%%%%%%%%%%%%%%%%%%%%%%%%%%%%%%%%%%%%%%%%%%%%%%%%%%%%%%%%%%%%%%%%%
\frontmatter
%%%%%%%%%%%%%%%%%%%%%%%%%%%%%%%%%%%%%%%%%%%%%%%%%%%%%%%%%%%%%%%%%%%%%%%%%%%%%%%%
% 1e de couverture
%%%%%%%%%%%%%%%%%%%%%%%%%%%%%%%%%%%%%%%%%%%%%%%%%%%%%%%%%%%%%%%%%%%%%%%%%%%%%%%%
% set-up -----------------------------------------------------------------------
\pagenumbering{gobble} % stop page numbering
\newgeometry{left=2cm,top=1.5cm,right=1.5cm,bottom=1.5cm} % set up margin
\usefont{T1}{phv}{m}{n} % Set up font style
{\parindent0pt % disables indentation for all the text between { and }

% logos and university name ----------------------------------------------------
\includegraphics[width=0.2\textwidth, valign=c]{./images/logos/pucvl}
\hfill
{\LARGE\textbf{UNIVERSITÉ D'ORLÉANS}}
\hfill
\includegraphics[width=0.2\textwidth, valign=c]{./images/logos/univ}

% PhD Informations -------------------------------------------------------------
\begin{center}
	{\large\textit{\textbf{ÉCOLE DOCTORALE ÉNERGIE, MATÉRIAUX, SCIENCES DE LA TERRE ET DE L'UNIVERS}}}\\

	\vspace{1.1cm}

	{\large Laboratoire de Physique et Chimie de l'Environnement et de l'Espace} \\

	\vspace{1.1cm}

	{\Large\textbf{THÈSE}} présentée par :\\
	{\Large\textbf{Benoît D'ANGELO}}\\

	\vspace{1.1cm}

	\begin{tabular}{r l}
		soutenue le : & \textbf{15 décembre 2015} \\ [.8cm]
		pour obtenir le grade de : & \textbf{Docteur de l'universit\'e d'Orl\'eans}\\[.2cm]
		Discipline : &  \textbf{Sciences de la Terre et de l'Atmosphère}\\
	\end{tabular}
\end{center}

%\vspace{0.2cm}
\vfill

% Title box --------------------------------------------------------------------
\begin{framed}
	\begin{minipage}{\dimexpr\textwidth-2\fboxrule-2\fboxsep}
	\centering
		\vspace{0.2cm}
			{\Large\textbf{Variabilité spatio-temporelle des émissions de GES dans une tourbière à Sphaignes : \\effets sur le bilan de carbone}\\ \vspace{0.3cm}
			}
		\vspace{0.2cm}
	\end{minipage}
\end{framed}

} % end of the "no indentation" parametrization

\vfill

% Encadrement/Jury -------------------------------------------------------------
\begin{center}
\begin{tabular}{l l l l}
\multicolumn{2}{l}{\textbf{\textsc{Thèse} dirig\'ee par : }} & & \\[.5ex]
	& \textbf{Christophe \textsc{Guimbaud}} &  &  Professeur, Université d'Orléans\\
	& \textbf{Fatima \textsc{Laggoun}} & &  Directeur de Recherche, CNRS, Orléans\\[1ex]
\multicolumn{2}{l}{\textsc{\textbf{Rapporteurs :}}} & & \\[.5ex]
%\begin{tabular}{l p{1cm} p{10cm}}
	& \textbf{François \textsc{Gillet}} & &  Professeur, Université de Franche-Comté\\
	& \textbf{Denis \textsc{Loustau}} & &  Directeur de Recherche, INRA\\[1ex]
\hline\\ [-1ex]
\multicolumn{2}{l}{\textsc{\textbf{Jury : }}} & &\\[.5ex]
	&\textbf{Luca \textsc{Bragazza}} &  & Professeur, Université de Ferrare\\
	&\textbf{Christophe \textsc{Fléchard}} &  &  Chargé de recherche, INRA\\
	&\textbf{François \textsc{Gillet}} & &  Professeur, Université de Franche-Comté, Pr\'esident du jury\\
	&\textbf{Sébastien \textsc{Gogo}} &  &  Chercheur, Université d'Orléans\\
	&\textbf{Bertrand \textsc{Guenet}} &  &  Chargé de recherche, CNRS\\
	&\textbf{Denis \textsc{Loustau}} & &  Directeur de Recherche, INRA\\
%	&\textbf{Pr\'enom \textsc{Nom}} &  &  Titre, \'etablissement\\
%	&\textbf{Pr\'enom \textsc{Nom}} &  &  Titre, \'etablissement\\
%	&\textbf{Pr\'enom \textsc{Nom}} &  &  Titre, \'etablissement\\
%	&\textbf{Pr\'enom \textsc{Nom}} &  &  Titre, \'etablissement\\
\end{tabular}
\end{center}

\vfill

\restoregeometry % restore base margins
\usefont{T1}{cmr}{m}{n} % restore base font	% insertion page de garde
\cleardoublepage
\pagenumbering{roman} 					% On active la numérotation des pages

\pagestyle{frontmatter}					% frontmatter page style

\tableofcontents
\addcontentsline{toc}{chapter}{Table des matières}

\listoffigures
\addcontentsline{toc}{chapter}{Liste des figures}

\listoftables
\addcontentsline{toc}{chapter}{Liste des tableaux}

%% AVANT PROPOS

\chapter{Avant-propos et remerciements}
%\addcontentsline{toc}{chapter}{Avant-propos}

Ce travail a été mené conjointement au LPC2E (Laboratoire de Physique et Chimie de l’Environnement et de l’Espace) dirigé par Michel TAGGER, et à l'ISTO (Institut des Sciences de la Terre d’Orléans) dirigé par Bruno SCAILLET, respectivement au sein des équipes « Atmosphère » et Biogéosystèmes Continentaux ». J'ai bénéficié d’un financement de la région Centre octroyé au LPC2E. 

La problématique de la thèse relève complètement des thématiques développées :
\begin{itemize}
\item au sein de l’OSUC (Observatoire des Sciences de l’Univers en région Centre) dirigé par Yves COQUET, et regroupant le LPC2E et l’ISTO. Ces travaux s’insèrent plus précisément dans la thématique fédérative « Atmosphère Terrestre et Interfaces »
\item au sein du Service National d’Observation Tourbières, labellisé par l’INSU SIC (Surfaces et Interfaces Continentales) en 2012, porté par l’OSUC et dont la coordination scientifique est assurée par Fatima LAGGOUN (ISTO).
\end{itemize}

\vspace*{1cm}

Je tiens à remercier mes deux directeurs de thèse Christophe Guimbaud et Fatima Laggoun pour leur implication, leur confiance, pour m'avoir guidé pendant ces trois années de travail.

Un grand merci également à mon encadrant Sébastien Gogo, toujours présent pour répondre à mes interrogations, à mes doutes et qui m'a accompagné sur quasiment toutes les campagnes de terrain. Merci pour ces moments inoubliables.

Merci à l'ensemble des membres du jury, François Gillet et Denis Lousteau pour avoir lu et évalué avec attention mon travail.
Merci également à Luca Bragazza, Christophe Fléchard et Bertrand Guenet pour leurs remarques constructives et nos échanges.

Il me faut également remercier tous les collègues et les amis qui m'ont accompagné et aidé pendant ces trois années, chacun d'entre vous se reconnaîtra.
Merci à tous pour ces bons moments et ces rencontres.

Un énorme merci à ma famille, mon père, ma mère, mes deux soeurs, pour leur soutien indéfectible et inestimable. 

\textit{Last but not least}, Merci Claire pour m'avoir supporté moi et mes doutes pendant ces trois intenses années, pour ton soutien, ta patience et tout le reste !


% REMERCIEMENTS

\chapter{Remerciements}
%\addcontentsline{toc}{chapter}{Remerciements}
Fatima, Christophe, Sébastien
Franck, Fabien
Marielle
Emélie, 
Étienne, Zi, Tianyi, Sarah, Paul, Xiaole, Guillaume
Frédéric Stéphane Gille 
Catherine, Catherine, Marie-Noëlle, Olivier

%%%%%%%%%%%%%%%%%%%%%%%%%%%%%%%%%%%%%%%%%%%%%%%%%%%%%%%%%%%%%%%%%%%%%%%%%%%%%%%%
% Mainmatter
%%%%%%%%%%%%%%%%%%%%%%%%%%%%%%%%%%%%%%%%%%%%%%%%%%%%%%%%%%%%%%%%%%%%%%%%%%%%%%%%
\mainmatter
\doublespacing

\pagestyle{mainmatter}					% mainmatter page style

% Ces commandes permettent de modifier le texte renvoyé aux headers
\renewcommand{\chaptermark}[1]{\markboth{\chaptername\ \thechapter.\ #1}{}}
\renewcommand{\sectionmark}[1]{\markright{\thesection.\ #1}}

\linenumbers
% INTRODUCTION
% OBJECTIFS : Introduire le contexte générale, l'échelle globale des problématiques
% Cette introduction doit être parfaitement compréhensible par un béotien !

\chapter*{Introduction}
\markboth{Introduction}{}
\addcontentsline{toc}{chapter}{Introduction}
\newpage


%\subsection*{Contexte général}

%PLAN : 
%- Le changement global, une thématique actuelle
%	* Historique du changement climatique ?
%	* Consensus scientifique ?
%	* Qu'est-il attendu  en terme de changement (réchauffement ?)
%- Le cycle du carbone : des flux et des stocks
%	* importance relative des stocks et flux
%	* principaux facteur forcant à l'échelle globale ?
%- Et les zones humides et les tourbières dans tout ça ?
%	* Qu'est ce qu'une tourbière ?
% 	* Fonctionnement vis à vis du cycle de carbone 
% 	* Une sensibilité particulière (localisation, anthropisation)
% 	* Écosystème non pris en compte dans les modèles globaux
Vers 1610, Jan Baptist Van Helmont, chimiste, physiologiste et médecin, découvre le dioxide de carbone (\coo) qu'il nomme « gaz sylvestre » \citep{philippedesouabe-zyriane1988}.
À cette époque pré-industrielle (avant 1800), les concentrations en \coo sont généralement estimées à 280 ppm\footnote{Partie par million} \citep{Siegenthaler1987}.
En 1957, Charles David Keeling, scientifique américain, met au point et utilise pour la première fois un analyseur de gaz infra-rouge pour mesurer la concentration de \coo de l'atmosphère dans l'île d'Hawaii, à Mauna Loa.
La précision et la fréquence importante de ses mesures lui permirent de mettre en évidence pour la première fois les variations journalières et saisonnières des concentrations en \coo atmosphérique, mais d'évaluer également à plus long terme leur tendance haussière \citep{harris2010}.
Depuis l'époque pré-industrielle les concentration en \coo ont en effet légèrement augmenté et sont alors estimées à moins de \SI{320}{ppm} \citep{pales1965}.
Ce constat a probablement joué un rôle dans la prise de conscience, par la communauté scientifique, de l'importance et de l'intérêt de l'étude du changement climatique et plus largement des changements globaux.\index{changements globaux}
En 2013, le Groupe d'Experts Intergouvernemental sur l'évolution du Climat (GIEC) a publié son 5\ieme rapport sur le changement climatique qui souligne l'importance des émissions de Gaz à Effet de Serre (GES) dans le changement climatique \citep{stocker2013}.
Au printemps 2014, la barre symbolique des \SI{400}{ppm} a été dépassée dans tout l'hémisphère nord selon un communiqué de l'Organisation Météorologique Mondiale (\url{http://www.wmo.int/pages/mediacentre/press_releases/pr_991_fr.html}).

À l'échelle globale, l'humanité par la consommation des combustibles fossiles et par la production de ciment, émet dans l'atmosphère environ \SI{7.8}{\pgca}\footnote{\si{\pgc} : $10^{15}$ grammes de carbone} \citep{Ciais2014}.
Les flux « naturels » entre l'atmosphère et la biosphère sont d'un ordre de grandeur supérieur : \num{98} et \SI{123}{\pgca} pour la respiration (\coo et \chh principalement) et la photosynthèse au sens large \citep{Bond-Lamberty2010,Beer2010}. L'importance de ces flux renforce la nécessité de les comprendre et si possible de les prédire, car une modification de leur dynamique même faible pourrait avoir des conséquences importantes.
Les écosystèmes naturels, en plus d'en échanger de façon importante avec l'atmosphère, stockent du carbone de façon importante : entre \num{1500} et \SI{2000}{\pgc} pour les sols par rapport aux \num{750} à \SI{800}{\pgc} stockés dans l'atmosphère.

%Parmi les écosystèmes terrestres, les tourbières, dans leur fonctionnement naturel, stockent du carbone notamment grâce à leurs conditions de saturation en eau importantes.  

%Parmi les écosystèmes terrestres, les tourbières fonctionnent naturellement comme des puits de carbone : elles stockent du carbone grâce des conditions de saturation en eau importante.
%Elles ne représentent que \num{2} à \SI{3}{\percent} des terres émergées mais contiennent entre \num{270} et \SI{455}{\pgc}, faisant de ces écosystèmes des stocks importants \citep{gorham1991,turunen2002} :
%d'abord parce qu'ils sont relativement concentrés en terme de surface, mais également car situés majoritairement dans les hautes latitudes de l'hémisphère nord, là ou le réchauffement climatique attendu est le plus important.
Parmi les écosystèmes terrestres naturels, les \textbf{tourbières} sont les plus efficaces dans le stockage du carbone.
Ce fonctionnement naturel en \textbf{puits de carbone} est la conséquence de conditions de saturation en eau importante du milieu, empêchant la dégradation des matières organiques (majoritairement constituées de carbone)  qui se stockent sous forme de tourbe.
Ce stock est estimé entre \textbf{270 et  \SI{455}{\pgc}} ce qui représente \textbf{10 à \SI{25}{\percent} du carbone stocké dans les sols mondiaux} alors que ces écosystèmes ne représentent que \textbf{2 à \SI{3}{\percent} des terres émergées}.
%La concentration de ce stock accroît son importance, en effet les tourbières sont majoritairement situées dans les hautes latitudes de l'hémisphère nord, là où le changement climatique le plus important est attendu.
La concentration de ce stock dans les hautes latitudes de l'hémisphère nord, où sont localisées la majorités des tourbières, rend incertain son devenir. 
En effet ce sont dans ces zones que sont attendu les changements climatiques les plus importants \citep{Ciais2014}.
La pérennité de ces écosystèmes est également fragilisée par les nombreuses perturbations anthropiques qu'ils subissent ou qu'ils ont subit.
Longtemps été considérés comme néfastes et impropres, une grande partie d'entre eux ont été drainés pour être exploités : la tourbe a été utilisée comme combustible ou comme substrat horticole, les tourbières comme terres agricoles ou sylvicoles.

Autrefois étudiées pour les propriétés combustibles de la tourbe, les toubières sont aujourd'hui principalement étudiées afin de comprendre leur fonctionnement, l'effet des perturbations climatique et anthropique sur ce fonctionnement, notamment par rapport à leur fonction de puits de carbone.
%Autrefois étudiés pour les propriétés de combustible de la tourbe, les tourbières sont aujourd'hui principalement étudiées vis-à-vis des perturbations qu'elles subissent : perturbations humaines, hausse ou baisse du niveau de la nappe, apports azotés, réhabilitation, ou perturbations climatique, effet de la température, des précipitations.
%Parmi toutes ces questions, celle du devenir de ce stock de carbone reste incertaine.
La variabilité de ces écosystèmes rend la prédiction de leurs comportements délicate et aujourd'hui malgré leur importance ces écosystèmes ne sont pas pris en compte dans les modèles globaux.
Le dernier rapport du GIEC note ainsi que si les connaissances ont avancé, de nombreux processus ayant trait à la décomposition de la matière organique des sols sont toujours absents des modèles notamment en ce qui concerne le carbone des zones humides boréales et tropicales et des tourbières \citep{Ciais2014}.
Plus spécifiquement, si les facteurs de contrôle principaux des émissions de carbone dans ces écosystèmes sont connus : la température, le niveau de la nappe d'eau, la végétation, leurs variations et co-variations ne font pas consensus. 
Le rôle des variations du niveau de la nappe d'eau, particulièrement leur sens et leur intensité, restent à comprendre.
Tout comme l'effet des communautés végétales et de leur changements, comme par exemple l'envahissement d'une tourbière par une végétation vasculaire.
Pour mieux comprendre ces écosystèmes, à différentes échelles, l'investigation est donc nécessaire pour espérer pourvoir estimer leurs comportements face aux changements qu'ils subissent et vont subir.

%Le \coo est un gaz à effet de serre (GES) et son accumulation dans l'atmosphère... \textbf{force ? comparaison ? explication effet de serre ?}
%Car si à l'époque les concentration en \coo étaient inférieures à 320~ppm (partie par millions) elles ont dépassé, au printemps 2014, la barre symbolique des 400~ppm selon un communiqué de l'Organisation Météorologique Mondiale. Les concentrations pré-industrielles (avant 1800) sont quant à elles généralement estimées à 280 ppm \cite{Siegenthaler1987}.

%Aujourd'hui, que ce soit pour le comprendre, le caractériser ou bien le prédire, de nombreux \textbf{Combien ? cf fact sheet IPCC} scientifiques dans un grand nombre de disciplines, travaillent directement ou indirectement sur les changements globaux.
%Ils sont nombreux également à collaborer au sein du  Groupe d'experts Intergouvernemental sur l'Évolution du Climat (GIEC), qui rassemble, évalue et synthétise les connaissances internationales liées au sujet.
%Des expérimentateurs, des modélisateurs, des climatologues, des atmosphéristes \todo{atmosphéristes, vérifier que ce mot existe...}, des écologues, qu'ils travaillent sur des archives ou des données actuelles. \todo{à reformuler}

%%De manière générale, parmi les flux de C mesurés entre la biosphère et l'atmosphère, la respiration et la photosynthèse sont les plus  importants, 98 et 123 PgC/yr pour le flux de respiration globale (+ les feux) et la photosynthèse respectivement \cite{Bond-Lamberty2010,Beer2010}. Pour comparaison les flux liés à la production de ciment et aux ressources fossiles (charbon, pétrole et gaz) représentent 7.8 PgC/yr \cite{Ciais2014}.
%
%Étroitement lié aux changements globaux, le cycle du carbone est particulièrement étudié, quels sont les réservoirs, quels sont les flux et comment vont-ils évoluer ? 
%\textbf{schéma ?}
%

%Zones humides tourbières
%
%historique des tourbières, généralités sur l'histoire des tourbières vis à vis des hommes
%Sujets principaux qui ont menés à l’étude des tourbières jusqu'à nos jours (Exploitation, effet de serre)
%
%Pourquoi étudier les tourbières aujourd’hui ? 
%
%Les sols stockent entre \num{1500} et \SI{2000}{\giga\tonne C} et parmi eux, les tourbières, zones humides longtemps considérée néfastes et impropre, ont été drainées et exploitées.
%Pourtant, parmi les nombreux services écologiques qu'elles donnent \index{services ecologiques@services écologiques} (épuration du sol, régulation des flux hydriques, biodiversité), elles constituent un stock de carbone relativement important au regard de la surface qu'elles occupent. Ainsi il est généralement admis que les tourbières contiennent un quart à un tiers du carbone présent Chiffres \textbf{(surfaces...)} dans l'ensemble des terres émergées tandis qu'elle ne constituent que 3 \% des surfaces continentales \plop. Ce ratio relativement important, correspond à un stock d'environ 455 Gt \cite{gorham1991,turunen2002}.
%Il est à mettre perspective avec les autres stock du cycle du carbone. On observe que ce stock est du même ordre de grandeur que celui de la végétation 

%Les tourbières sont des écosystèmes qui, s'il ne s'étendent pas sur une surface très importante (leur surface est estimée à 3\% des surfaces émergées) contiennent, relativement à leur extension, une quantité de carbone importante (455 Gt selon \cite{Gorham1991}).
%En conséquence dans un contexte **d'augmentation des GES dans l'atm et de réchauffement**, l'évolution de ce stock, sa pérennité ou sa remobilisation est un sujet d'étude important. De plus cette importance n'est à ce jour pas prise en compte de façon spécifique dans les modèles climatiques globaux.
%
%En France les tourbières s'étendent sur environ 60 000 Ha (\plop).
%
%
%
%Transition modèles
%
%En octobre 2013 le GIEC a publié le rapport du groupe de travail I qui travaille sur les aspects scientifiques physiques du système et du changement climatique.
%S'il note que les connaissances ont avancé, il note également que de nombreux processus ayant trait à la décomposition du carbone sont toujours absents des modèles notamment en ce qui concerne le carbone des zones humides boréales et tropicales et des tourbières. \plop



%\subsection*{Objectif de la thèse et approche mise en oeuvre}

%Dans ce contexte l'objectif de ces travaux est de comprendre la dynamique des flux de carbone.
%Principalement le \coo qu'il soit gazeux, dissout ou particulaire et le \chh.
%Notamment caractériser la variabilité spatiale et temporelle et déterminer quels sont les facteurs de contrôles dominants.

\subsection*{Objectifs du travail}
%Les tourbières sont des écosystèmes particulièrement efficace pour stocker du carbone.
%Le devenir de leur stock de carbone et de leur fonction de puits reste incertaine vis-à-vis des changement globaux attendus.
%Plus particulièrement, la compréhension des processus contrôlant cette fonction puits, notamment pour les tourbières envahies par une végétation vasculaire, restent à éclaircir.
Dans ce contexte les objectifs de ce travail sont donc (i) de caractériser la variabilité spatio-temporelle des flux et d'établir le bilan de carbone de la tourbière de La Guette, (ii) de déterminer quels facteurs environnementaux contrôlent le fonctionnement comme puits ou source de carbone de cet écosystème notamment l'effet du niveau de la nappe sur les émissions lors de cycles de dessiccations réhumectations.
Pour ce faire une approche axée sur l'\textbf{observation} et l'\textbf{expérimentation} a été mise en oeuvre : 

%
%Dans ce contexte les objectifs de ce travail sont donc (i) de caractériser la variabilité spatio-temporelle des flux et d'établir le bilan de carbone de la tourbière de La Guette, (ii) de préciser l'effet du niveau de la nappe sur les émissions lors de cycles de dessiccations réhumectations.
%Pour ce faire une approche axée sur l'observation et l'expérimentation a été mise en oeuvre : 
\begin{itemize}
\item Dans un premier temps, a été mis en place un suivi sur deux années de la tourbière de La Guette permettant d'évaluer les flux de GES (\coo et \chh) et d'étudier leurs variations saisonnières et spatiales sur l'ensemble de l'écosystème. Ces estimations de flux ont ensuite pu être utilisées afin d'estimer le bilan de carbone de la tourbière.
\item Dans un second temps, à travers des expérimentations en mésocosmes et sur le terrain, l'effet du niveau de la nappe sur les flux de GES a été exploré, particulièrement lors de cycles de dessiccation-réhumectations.
\item Enfin un suivi des flux à haute fréquence sur plusieurs tourbières a été réalisé afin de déterminer les éventuelles différences de sensibilité des émissions de \coo entre le jour et la nuit et de tester à cette échelle une méthode d'estimation de la RE basée sur la synchronisation entre les signaux de flux et de température.
\end{itemize}



%Dans ce contexte, l'objectif de mes travaux de thèse est de mieux comprendre la dynamique du carbone au sein des tourbières.
%Tout d'abord en caractérisant la variabilité spatiale et temporelle des flux de carbone à travers l'établissement du bilan de carbone d'une tourbière de Sologne.
%De déterminer quels facteurs environnementaux contrôlent le fonctionnement comme puits ou source de carbone de cet écosystème.


%Enfin construire, dans un esprit de synthèse et d'ouverture et à l'aide des connaissances acquises, un modèle intégrateur permettant un lien avec les modèles globaux et notamment ORCHIDE, afin que ces écosystèmes puissent être pris en compte à cette échelle.

%Pour atteindre ces objectifs, nos travaux ont été articulés autour de deux axes principaux :
%dans un premier temps, l'\textbf{observation} pour une période de deux ans des flux de gaz (\coo et \chh) et de paramètres environnementaux servant à la caractérisation des variabilités spatiales et temporelles, ainsi qu'à l'étude des facteurs contrôlant.
%Certains facteurs contrôlant sont, dans un second temps, étudiés plus spécifiquement à travers un volet \textbf{expérimentation}.
%Ce dernier doit permettre une meilleure compréhension des processus clés avec notamment l'impact de l'hydrologie.
%Enfin un troisième volet axé sur la \textbf{modélisation}, avec le développement d'un modèle le plus mécaniste possible.

% Contenu des chapitres de la thèse .
Le document est structuré de la façon suivante :
\begin{itemize}
\item Le premier chapitre pose le contexte bibliographique dans lequel s'inscrit ce travail.
Il se découpe en trois parties ; la première définit les terminologies et les concepts principaux employés dans le manuscrit.
La seconde précise l'état des connaissances sur les tourbières vis à vis des flux de carbone.
Enfin la troisième partie replace ce travail au sein du contexte précédemment établi.
\item Le deuxième chapitre décrit les sites d'études et les méthodes et matériels employés dans ces travaux.
\item Le troisième chapitre présente la variabilité spatio-temporelle des flux et l'estimation du bilan de carbone de la tourbière de La Guette.
\item Le quatrième chapitre décrit l'effet de cycles de dessication/ré-humectation sur les flux de GES en mésocosmes.
\item Le cinquième chapitre se concentre sur des aspect méthodologique en ce qui concerne la respiration à l'échelle journalière, plus spécifiquement la prise en compte du temps de latence entre la vague de chaleur et les flux, et la différence entre les mesures faites le jour et la nuit.
\item Enfin la dernière partie du document présente la synthèse et l'interprétation des résultats obtenus, ainsi que les perspectives de ce travail.
%Le chapitre 1 est une synthèse bibliographique, un état de l'art des connaissances liées au sujet.
%Les chapitres 2 et 3 rassemblent les travaux du volet observation, ils concernent respectivement le suivi XX et le suivi YY 
%Les chapitres 4 et 5 développent la partie expérimentale à travers l'impact d'un assèchement et celui d'un rehaussement du niveau de l'eau.
%Le chapitre 6 concerne plus spécifiquement la modélisation, même si ce volet interviendra par ailleurs de façon transverse dans les autres chapitres.
%Enfin une conclusions et des perspectives seront exposées.
\end{itemize}

% % % % TRASH ? REUTILISATION APRES.
%
%Afin de répondre à notre problématique (La variation spatiale et temporelle des flux de \coo $\leftarrow$ Ceci est la thématique pas un problématique...), Trois approches complémentaires ont été mises en œuvres. 
%Tout d'abord la mise en place de suivis terrain, qui nous ont permit de mieux comprendre certains processus. 
%L'expérimentation ensuite, sur le terrain ou en laboratoire, afin d'étudier plus en détail l'impact d'un facteur forçant majeur qu'est l'hydrologie. 
%Et enfin la modélisation afin de synthétiser ces connaissances dans un cadre dont on espère pouvoir se resservir dans différents sites.

\setcounter{mtc}{5}
% % CHAPITRE 1 : SYNTHESE BIBLIO
%
%Réflexions
%
%NE PAS OUBLIER DE CALER UN MAXIMUM D'ORDRE DE GRANDEUR ! 
%	(émissions CO2, CH4, stocks flux, surface des tourbières, végétation ...)
%
%Bien différencier l'intro générale qui doit être lisible par un béotien, de l'intro au travail de la thèse qui doit être un état de l'art précis et documenté sur les travaux antérieurs (synthèse biblio)
%
%Ou caler la partie de biblio sur l'expérimentation ... (peut être dans la synthèse biblio, paragraphe "approche mise en oeuvre"
%INTRODUCTION GENERALE (à mettre dans le chapitre Intro)
%
%- Qu'est ce qu'une tourbière ? (Éventuellememt comment se forme-t-elle ?)
%	*Définition
%	*Formation/Évolution (stockage du C, battement de la nappe ?)
%	*Classification
%- Les tourbières et les hommes 
%	*Usages d'hier et d'aujourd'hui (Combustible, horticulture, matériau de construction)
%	*Les thématiques scientifiques (pourquoi les avoir étudier et les étudier en gros)
%	*Le contexte dans lequel va s'inscrire le travail qui suit
%
%SYNTHESE BIBLIOGRAPHIQUE
%
%- Quelles sont les grandes thématiques de recherche liées aux tourbières ?
%	*Exploitation
%	*Archives
%	*Émissions de GES
%- Plus précisément quelles sont les grands axes de recherche sur ces écosystèmes et liés aux émissions de GES.
%	*Processus de création de GES (CO2 et CH4) (Facteurs contrôlant généralement invoqués)
%	*Processus de migration des GES dans le profil
%	*Processus de stockage/capture
%- Approches mise en oeuvre
%	*Modélisation (empirique et mécaniste)
%	*Expérimentation (différentes techniques pour mesurer les émissions de GES, différentes techniques de chambre...)
%	*Variabilité spatio-temporelle (notion d'échelle)
%	
%	DOIS-JE TRAITER
%	La classification des tourbières ?
%	Hummock and Hollow ? Dire qu'on n'est pas dans ce niveau de détail ?
%	
%
%HISTORIQUE (études concernant les tourbières)
%
%1968-1969  Boelter : propriétés physique des tourbes
%1977 Boelter : hydrologie, caractéristique des sols organiques, chimie des écoulements
%1978 Ingram : Classification
%1981 Parkinson : Déjà l'amélioration d'une méthode pour la mesure des émissions de la respiration du sol
%1984 Clymo : Les limites à la croissance des tourbières
%1986 Chason : Conductivités hydraulique et propriétés physiques
%1989 Moore : Influence du niveau de la nappe sur les émissions de CO2 et de CH4
%//1990 Raich : Comparaison de 2 méthodes de chambre statique pour mesurer les flux de CO2
%1991 Gorham : Rôle des tourbières dans le cycle du carbon et réponse au changement climatique
%//1992 Raich : Flux de CO2 dans la respiration du sol et relation vis à vis du climat et de la végétation
%1992 Roulet : Flux de méthane (fen) et changement climatique
%1993 Bubier : Émission de méthane dans les zones humides
%1993 Bubier : Microtopographie et flux de méthane dans tourbières boréales.
%1993 Abbès : Sorption de l'ammonium ? (ammonia) par la tourbe et fractionnement de l'azote
%1994 Lloyd : Dépendance de la respiration du sol à la température
%1994 Bubier : Perspective écologique sur les émissions de méthane dans les zones humide de l'hémisphère nord
%//1994 Nay : Biais des méthodes de chambre pour la mesure des flux de CO2
%1995 Kirschbaum : Dépendance à la température de la décomposition de la MO (effet sur stock de C et changement Clim)
%1995 Bubier : Prédiction des émissions de méthane à partir de la distribution des bryophytes (tourbière hémisphère nord)
%1995 Bubier : Contrôles "écologique" sur les émissions de méthane dans les tourbières de l'hémisphère nord
%1995 Bubier : Relation entre la végétation avec les émissions de méthane et les gradients hydrochimique
%//1995 Bekku : Mesure de la respiration du sol avec une méthode de chambre fermée (IRGA)

% PLAN (2015-03-03)
%I. Définitions
%1 Tourbières/Tourbe (surface, type, localisation, biodiversité, services écologiques...)
%2 Classification
%3 Historique
%	a Utilisation
%	b Études scientifiques
%	
%Transition : Réaction aux changements globaux (comment fonctionnent-elles ?)
%
%II. Fonctionnement
%1 Stock
%2 Flux
%	a Entrants (Photosynthèse)
%	b Sortants (Méthanogénèse, Respiration)
%3 Facteurs Contrôlant
%	a Hydrologie (WTL,HR)
%	b Propriétés physiques (T, densités, conductivités thermiques...)
%	c Végétation (Bryophytes/Vasculaires)
%	d Météo

% GO TO intro générale ?
%Depuis quand sont-elles étudiées ?
%D'abord étudiées pour leurs propriétés physiques afin de connaitre leur qualité en tant que combustible.
%Elle sont maintenant majoritairement étudiée à travers le prisme des changements globaux.
%Ainsi les études concernent les flux de GES, ...

\chapter{Synth\`{e}se Bibliographique}

\minitoc

\newpage

\index{tourbières|(}
Dans ce chapitre, nous commenceront par donner une vue de ce que sont les tourbières : Que sont-elles ? Depuis quand sont-elles étudiées ? Pourquoi les a-t-on étudiés ?
Nous continuerons en entrant plus en détails sur leur fonctionnement vis à vis des flux de carbone.
Enfin nous verrons quels sont les facteurs contrôlant majeurs de ces flux.

\section{Les tourbières et le cycle du carbone}

\subsection{Zones humides et tourbières : définitions et terminologies}

\subsubsection{Définitions}

Les tourbières font partie d'un ensemble d'écosystèmes plus large que l'on appelle les zones humides\index{zone humide}.
Ces zones humides ne sont ni des écosystèmes terrestres au sens strict, ni des écosystèmes aquatiques.
Elles sont à la frontière entre les deux et sont caractérisées par un niveau de nappe élevé, proche de la surface du sol, voire au dessus.
L'omniprésence de l'eau joue fortement sur l'aération du milieu et contraint, de façon plus ou moins importante, l'accès à l'oxygène.
Elles ont été définie en 1971, lors de la convention dite de \textsc{Ramsar}\footnote{La convention de \textsc{ramsar} est un traité international visant à la conservation et l’utilisation rationnelle des zones humides.} de la façon suivante : 
\begin{pdef}
\textsc{Zones humides} :

«les zones humides sont des étendues de marais, de fagnes\footnotemark, de tourbières ou d'eaux naturelles ou artificielles, permanentes ou temporaires, où l'eau est stagnante ou courante, douce, saumâtre ou salée, y compris des étendues d'eau marine dont la profondeur à marée basse n'excède pas six mètres.»

\hfill {\scriptsize \citep{ramsar1987}}
\end{pdef}
\footnotetext{Marais tourbeux situé sur une hauteur}

Les zones humides regroupent donc des écosystèmes très variés parmi lesquels les marais, les mangroves, les plaines d'inondations et les tourbières.
Leurs particularités : niveau de nappe élevé et zone anaérobie importante, entraînent le développement d'une végétation spécifique, qui s'est adaptée aux milieux fortement humides ou inondés.

Les tourbières représentent 50 à \SI{70}{\percent} des zones humides \cite{joosten2002}.
Leur définition est variable selon les régions (\plop, exple).
%Cela ne facilite pas leur recensement et leur cartographie.
Deux définitions sont régulièrement utilisées :

\begin{pdef}
\textsc{Tourbière} :

Écosystèmes, avec ou sans végétation, possédant au moins \SI{30}{\cm} de tourbe naturellement accumulée.

\hfill {\scriptsize Définition traduite d'après \citet{joosten2002}}
\end{pdef}
%La première définie comme tourbières les écosystèmes possédant au moins \SI{30}{\cm} de tourbe (parfois 40).
Cette première définition correspond au \textit{peatland} anglo-saxon.
L'épaisseur de tourbe accolée à cette définition peut varier selon le pays, elle est par exemple établie à \SI{40}{\cm} au Canada \citep{nationalwetlandsworkinggroup1997}
\begin{pdef}
\textsc{Tourbière active} :

Écosystèmes dans lesquels un processus de tourbification est actif.

\hfill {\scriptsize Définition traduite d'après \citet{joosten2002}}
\end{pdef}
Cette seconde définition correspond au \textit{mire} anglo-saxon et peut être traduite en français par le terme de tourbière active.
Les concepts derrières ces deux définitions se chevauchent mais ne sont pas complètement similaires : une tourbière drainée peut avoir plus de \SI{30}{cm} de tourbe et ne plus former de tourbe, ne plus être active.
À l'inverse il peut exister des zones ou l'épaisseur de tourbe est inférieure à \SI{30}{cm} malgré un processus de tourbification actif.
Un même écosystème tourbeux pouvant d'ailleurs avoir à la fois des zones correspondant à la première définition et d'autres à la seconde.
%Elles sont généralement définies par rapport à la tourbe, qu'il convient donc de définir au préalable.
%La tourbe est un sol organique (histosol) formé suite à l'accumulation de litières végétales partiellement décomposées dans un milieux saturé en eau.
Les tourbières sont donc, selon la définition utilisée, des écosystèmes contenant ou des écosystèmes formant de la tourbe.
Mais qu'est ce que la tourbe ?

\begin{pdef}
\textsc{Tourbe} :

«Accumulation sédentaire de matériel composé d'au moins \SI{30}{\percent} (matière sèche), de matières organiques mortes.»

\hfill {\scriptsize Définition traduite d'après \citet{joosten2002}}
\end{pdef}

Le seuil de \SI{30}{\percent}, souvent utilisé pour rapprocher sa définition de celle d'un sol organique (histosol) au sens large, dans lesquels sont classés la majorité des sols tourbeux.
D'autres définitions existent, faisant la distinction entre sols organiques et tourbes avec un seuil à \SI{75}{\percent} \citep{andrejko1983} ou \SI{80}{\percent} \citep{landva1983}.
Il est également nécessaire de préciser que, au delà de la classification utilisée, ce que les écologues considèrent comme de la tourbe contient généralement \SI{80}{\percent} de matières organiques au minimum \citep{rydin2013b}.
Ce processus de formation est appelé la tourbification\index{tourbification} ou turfigénèse et les matières organiques accumulées proviennent majoritairement de la végétation.
On défini les matières organiques de la façon suivante : 
\begin{pdef}
\textsc{Matières organiques} :

Matières constituées d'un assemblage de composés ayant une ou plusieurs liaison C--H.
%Matières composées des éléments C, H, O, N, S et présentant des liaisons C, H. 
Elles sont composées de nombreux éléments dont des carbohydrates (sucres, cellulose \dots), des composés azotés (protéines, acides aminés \dots) et phénoliques (lignine \dots), des lipides (cires, résines, \dots) et d'autres\footnotemark.

%\hfill {\scriptsize Définition traduite d'après \citet{joosten2002}}
\end{pdef}
\footnotetext{Cette définition, utile pour définir simplement les matières organiques, est cependant limitée car elle inclue des composés traditionnellement considérés comme minéraux (le graphite) et exclue certains autres considérés comme organiques (acide oxalique) (Liste de diffusion ResMO).}

%La tourbe peut être considérée comme du charbon (?coal) dans ses stades les plus précoces. 
%En continuant à évoluer, à se compacter, se formerai  (combien de temps) d'abord de la lignite puis du charbon et enfin de l'anthracite (belanger 1988 dans Rydin et Jeglum 2006)
%Les propriétés physiques de la tourbe dépendent du type de végétation, mais également de sa profondeur dans le profil (pédogenèse, diagenèse) amha2010 dans rydin.

Ces variations de définitions ajoutées aux limites floues qui peuvent exister entre certain écosystèmes tourbeux et non-tourbeux rendent la cartographie de ces écosystèmes délicate.
Les estimations généralement citées évaluent la surface occupée par les tourbières à environ \SI{4000000}{\square\kilo\meter} \citep{lappalainen1996}.\index{tourbières!surface} 
Cette surface correspond à \num{2} à \SI{3}{\percent} de l'ensemble des terres émergées du globe.
Plus de \SI{85}{\percent} d'entre elles sont situés dans l'hémisphère nord, majoritairement dans les zones boréales et sub-boréales \citep{strack2008} (Figure~\ref{fig:peatlandGlobalDistribution}).
Ce sont sur ces écosystèmes que sera focalisé ce travail, laissant de côté les tourbières tropicales dont le fonctionnement est distinct et spécifique \plop.

\begin{figure}[t]
\centering
\includegraphics[width=\textwidth]{chap1/microtopo}
\caption{Micro-topographie dans les tourbières. Modifié d'après \citet{rydin2013a}}
\label{fig:microtopo}
\end{figure}

\begin{figure}
\centering
\includegraphics[width=\textwidth]{chap1/peatlandGlobalDistribution}
\caption{Global distribution of peatlands}
\label{fig:peatlandGlobalDistribution}\index{tourbières!distribution} 
\end{figure}

\subsubsection{La formation des tourbières}
\index{tourbières!formation} 
L'atterrissement\index{atterrissement} et la paludification\index{paludification} sont les deux processus principaux permettant la formation des tourbières.
Il s'agit pour le premier du comblement progressif d'une zone d'eau stagnante.
La paludification est la formation de tourbe directement sur un sol minéral, grâce à des conditions d'humidité importante.
Ces modes de formation ne sont pas exclusif, une tourbière pouvant se développer, selon les endroits considérés ou le temps, via des processus différents.

\subsubsection{Classifications}

Différentes classifications sont utilisées pour différencier ces écosystèmes.
La plus générale et la plus utilisée dans la littérature distingue les tourbières dite haute, ou de haut marais, correspondant au \textit{bog} anglais, et les tourbières basse, ou de bas marais, correspondant au \textit{fen} anglais.

Les tourbières de haut-marais ont généralement une épaisseur de tourbe supérieure à \SI{30}{\cm} et sont alimenté principalement alimentée par les précipitations : elles sont dites ombrotrophes.
Leur surface parfois bombée (tourbières élevées ou bombées) peut également être plate ou en pente.
Cette géométrie situe une partie au moins de l'écosystème au dessus du niveau de la nappe.
Elles ont une concentration en nutriments relativement faible : elles sont oligotrophes et sont fortement acide avec des eaux de surface dont le pH est autour de 4 voire moins.

Les tourbière de bas-marais ont une épaisseur généralement supérieure à \SI{30}{\cm} avec un niveau de nappe très proche de la surface du sol.
De forme concave ou en pente elles sont généralement alimentée en eaux par des sources ou par ruissellement et sont donc dites minerotrophes.
Le pH de leur eaux de surface varient de 4 à 8.
Les végétations dominantes de ces écosystèmes peuvent être des bryophytes, des graminées ou des arbustes bas.
%De nombreux critères existent pour classer les tourbières selon leur mode de formation, leur source d'eau, leur physico-chimie.
%La terminologie utilisée concernant ces écosystèmes n'a pas toujours été cohérente, de nombreux termes ont été utilisés parfois en contradiction les uns avec les autres \cite{joosten2002}.
%Il existe différents types de tourbières, notamment on distingue des tourbières tempérées/boréales des tourbières tropicales dont le fonctionnement diffère.
%Dans la suite de ce document seule les tourbières tempérées/boréales seront décrites et étudiées.


\subsection{Tourbières et fonctions environnementales}



\subsubsection{Biodiversité dans les tourbières}

%\subsection{Les tourbières, des écosystèmes particuliers}

%\subsubsection{Biodiversité}

%Ces écosystèmes sont le siège d'une biodiversité spécifique relativement importante et rendent un certain nombre de services écologiques.
%Parmi la végétation caractéristique de ces écosystèmes, les sphaignes, des bryophytes (des mousses) sont normalement présentes en abondance.
%Les sphaignes ont quelques particularités qu'il convient de mentionner.
%Ce sont des espèces ingénieures, capable de modifier le milieu dans lequel elle vivent afin de l'adapter à leur besoin.
%Plus spécifiquement elles sont capable d'acidifier leur milieu, de capter les nutriments provenant de l'eau de pluie et de les séquestrer afin de défavoriser d'autres végétaux.
Les tourbières sont le siège d'une biodiversité importante et spécifique.
Ainsi les Sphaignes, qui sont des bryophytes, (des mousses) sont caractéristiques des écosystèmes tourbeux.
Ce sont des espèces dites ingénieures, capable de modifier l'environnement dans lequel elles vivent afin de l'adapter à leurs besoins.
Les sphaignes sont ainsi capable d'abaisser le pH, de capter des nutriments et de les séquestrer et ce même quand elles n'en ont pas besoin afin d'empêcher d'autres espèces notamment vasculaire d'en profiter.
Plus précisément, le fait que les sphaignes captent les nutriments via leur capitulum leur permet de les intercepter avant qu'ils ne soient captés par d'éventuelles racines positionnées plus bas.
Les sphaignes, comme de nombreuse mousses ont des litières relativement récalcitrantes\footnote{il est d'usage de parler de litières récalcitrantes sans plus de précision. Il s'agit en fait de litières difficilement dégradables}.

\subsubsection{Qualité des eaux}


%\subsubsection{Puits de carbone}
\subsubsection{Puits de carbone}
\index{carbone!stock}
Par définition les tourbières stockent ou ont stocké du carbone.
C'est cette fonction de puits de carbone qui rend l'importance de ces écosystèmes non négligeable malgré la faible surface qu'ils représentent.
Les estimations du stock de carbone présent dans les tourbières tempérées/boréales sont comprises entre 270 et \SI{455}{\giga\tonne\,C} \citep{gorham1991,turunen2002}.
Les différences entre les estimations sont liées aux incertitudes de cartographie citées précédemment auxquelles s'ajoutent des incertitudes concernant l'épaisseur et la densité moyenne de la tourbe.
Le carbone stocké dans les tourbières représente 10 à \SI{25}{\percent} du carbone présent dans les sols et entre 30 et \SI{60}{\percent} du stock de carbone atmosphérique.

\begin{table}
\centering
\caption{Estimations des stocks de C pour différents environnements}
\label{table:CCycleStocks}
\begin{tabular}{llp{7cm}}\toprule
Compartiment & Stock (en Gt de C) & référence \\ \midrule
Tourbières & 270 -- 455 & \cite{gorham1991,turunen2002} \\ 
Végétation & 450 -- 650 & \cite{Robert2003}\\ 
Sols & 1500 -- 2000 & \cite{Robert2003,Post1982,Eswaran1993}\\ 
\coo atmosphérique & 750 -- 800 & \cite{Robert2003}\\ 
Permafrost & 1700 & \\ 
\bottomrule
\end{tabular}
\end{table}

Ce stock est un héritage datant des 10 derniers milliers d'années, l'holocène, période pendant laquelle se sont formés la majorité des tourbières \plop \citep{yu2010}.
Le fonctionnement naturel de ces écosystèmes permet le stockage du C.
C'est un des services écologiques que rendent les tourbières et que l'on appelle la fonction puits de carbone.
Cette fonction est liée an niveau élevé de la nappe d'eau, qui rend l'accès à l'oxygène est plus difficile diminuant d'autant l'activité aérobie, dont la respiration des micro-organismes et des plantes.
Cela ce traduit par une dégradation relativement faible des matières organiques.
Elle est également liée à la production de litière récalcitrante par les bryophytes.

En comparaison avec un sol forestier, l'accumulation de matières organiques n'est donc pas lié à une production primaire plus forte, mais bien à une dégradation des matières produites plus faible.

Ces perturbations peuvent induire des modifications de fonctionnement, notamment l'envahissement de ces écosystèmes par une végétation vasculaire, et changer cette fonction puits.

%\subsection{Le cycle global}
%
%Au cours des temps les tourbières ont donc accumulé du carbone... stock
%La vitesse de stockage a pu varier au cours du temps mais elle est estimé à XXXX, ainsi la majorité des tourbières actuelles ont un stock qui remonte à quelques milliers d'années.
%Les estimations précise du stock de C présent dans ces écosystèmes sont délicates, à la fois car la définition de ce qu'est une tourbière que varier selon les régions, mais également car leur étendue exacte n'est pas triviale à estimer, pas davantage que leur profondeur moyenne.
%Cependant il est usuellement admis que le stock de carbone se situe entre 270 et 500 Gt de C
%Les tourbières ont donc accumulées du carbone au cours des 10 derniers milliers d'années.
%Pour ce faire il a donc fallu que davantage de carbone soit capturé que de carbone libéré par l'écosystème.



\subsection{Les tourbières et les changements globaux}
On défini les changements globaux comme l'ensemble des modifications environnementales plus ou moins rapide, ayant lieu à l'échelle mondiale, quelle que soit leur origine. Les deux contraintes développées dans cette partie sont la pressions de l'homme : contrainte anthropique, et celle du climat : contrainte climatique.
\index{changements globaux}

\subsubsection{Contrainte anthropique}
\index{tourbières!utilisation} 

L'interaction entre les Hommes et les zones humides au sens large et les tourbières en particulier remonte probablement à l'aube de l'humanité.
De grandes découvertes archéologiques ont été faites dans ces écosystèmes témoins d'époques révolues.
Des chemins de rondins néolithique aux crannogs de l'époque romaine \citep{buckland1993}.
L’utilisation de la tourbe et des tourbières a du commencer relativement tôt, mais c'est à partir du 17\textsuperscript{e} siècle que le drainage de ces écosystèmes, pour les convertir en terres agricoles, s'est intensifié.
Au 19\textsuperscript{e} siècle, l'apparition de machines permettant une récolte industrialisée de la tourbe, a développé son utilisation comme combustible.
Enfin depuis le milieu du 20\textsuperscript{e} une part importante de ces écosystèmes ont été drainé pour développer la sylviculture.
Aujourd'hui l'exploitation principale de la tourbe est liée à son utilisation comme substrat horticole \citep{lappalainen1996,chapman2003}.
Suite à ces perturbations, la surface de tourbières altérée est estimée à \SI{500000}{\square\kilo\metre} environ, principalement du fait de leur reconversion pour l'agriculture et la sylviculture (Tableau~\ref{table:tourbeUsage}).
En France, suite à leur utilisation, principalement agricole, la surface des tourbières a été par deux entre 1945 et 1998, passant de \SI{1200}{\square\kilo\meter} à \SI{600}{\square\kilo\meter} \citep{lappalainen1996,manneville1999}.

Ces écosystèmes ont donc été et sont encore perturbés par différentes activités humaines.
\begin{table}[]
\centering
\caption{Surface de tourbe utilisée selon les usages considérés (tourbières non-tropicale). Modifié d'après \citet{joosten2002}.}
\label{table:tourbeUsage}
\begin{tabular}{lll}\toprule
Utilisation & Surface (\si{\square\kilo\meter})  & proportion (\%) \\ \midrule
Agriculture & \num{250000} & \num{50} \\ 
Sylviculture & \num{150000} & \num{30}\\ 
Extraction de tourbe & \num{50000} & \num{10}\\ 
Urbanisation & \num{20000} & \num{5}\\ 
Submersion & \num{15000} & \num{3}\\ 
Pertes indirectes (érosion, ...) & \num{5000} & \num{1}\\[1ex]
Total & \num{490000} & \num{100}\\
\bottomrule
\end{tabular}
\end{table}


\subsubsection{Contrainte climatique}

Comme nous l'avons dit, le stock de C accumulé par les tourbières s'est majoritairement constitué pendant l'Holocène.
À cette époque déjà ces écosystèmes étaient influencés par le climat, et leur développement n'a pas été linéaire sur ces douze derniers milliers d'années.
Il est reconnu que le développement des tourbières est très important au début de cette période entre il y a \num{12000} et \num{8000} ans \citep{smith2004,macdonald2006,yu2009}.
Cette période coïncide avec le maximum thermique holocène (HTM), période pendant laquelle le climat était plus chaud que aujourd’hui \citep{kaufman2004}.
Ce constat peu sembler paradoxal si l'on considère que dans la littérature concernant les tourbières et le réchauffement actuel, la crainte de voir ces écosystèmes se transformer en source de C est majoritaire.
Cependant ces même auteurs qui ont montré cette relation, entre le HTM et le développement important des tourbières, ne préjugent pas de l'effet du réchauffement actuel.
Notamment \citet{jones2010} expliquent que pendant cette période de maximum thermique, existe également une saisonnalité très importante, avec des été chauds et des hivers froid, qui a dû en minimisant la respiration hivernale de ces écosystèmes, jouer un rôle important dans leur développement.

Cette forte saisonnalité n'est pas attendue lors du réchauffement actuel.
L'effet estimé dans les hautes latitudes, semble plus important pendant l'hiver et l'automne, et tendrait donc à la minimiser \citep{christensen2007}.
Les effets directs attendus du réchauffement dans les hautes latitudes sont une augmentation des températures de 2 à \SI{8}{\degreeCelsius} dans les zones boréales, et de 2 à \SI{6}{\degreeCelsius} dans les zone tempérées, ainsi  qu'une augmentation probable des précipitations \citep{christensen2013,frolking2011}.
De façon plus indirecte est attendue la fonte du permafrost, l'augmentation de l'intensité et de la fréquence de feux et des changements dans les compositions des communautés végétales.

%L'impact anthropique direct n'est par la seule perturbation auxquelles sont soumises les tourbières.
%D'après les modèles de prédictions du GIEC, les tourbières, comme de nombreux autres écosystèmes, vont subir un changement climatique important dans les années à venir.
%Toujours d'après le GIEC, les changements les plus rapides que ce soit en terme de précipitations ou de température sont à attendre dans les zones boréales là ou se situent la majorité des tourbières.
%De ce constat découle un certain nombre de questions concernant ces écosystèmes.
%D'abord quel effet auront les changements climatiques et avec quelle variabilité régionale ?
%Cette question n'est pas évidente (paradoxe du sol plus froid ? augmentation photosynthèse)
%Quelle sera la sensibilité des tourbières ?
%Là encore leur diversité, leur répartition géographique rend difficile la réponse à cette question.
%Enfin découlant des précédentes, qu'elle est le devenir de la fonction puits de carbone.
\begin{figure}
\centering
\includegraphics[width=\textwidth]{chap1/holo_peat_ini}
\caption{Nombre de tourbières nouvellement formées pendant l'holocène. Modifié d'après \citep{macdonald2006}}
\label{fig:holo_peat_ini}
\end{figure}


\begin{figure}
\centering
\includegraphics[width=\textwidth]{chap1/ipcc2013_RCP45}
\caption{Projection des changements à l'horizon 2100, des moyennes et extrêmes annuels (sur terre) des températures de l'air et des précipitations : (a) température de surface moyenne par \si{\degreeCelsius} de changement global moyen, (b) 90\textsuperscript{e} percentile des températures journalières maximum par \si{\degreeCelsius} de changement de température moyenne maximale, (c) précipitations moyenne (en \si{\percent} par \si{\degreeCelsius} de changement de température moyenne) et (d) fraction de jours ayant des précipitations dépassant le 95\textsuperscript{e} percentile. Sources : (a) et (c) simulations CMIP5, scénario RCP4.5, (b) et (d) adaptation d'après \citet{orlowsky2012}(\textbf{IPCC2013}).}
\label{fig:ipcc2013_T_rain}
\end{figure}

%Toutes ces perturbations posent notamment la question de la pérennité de la fonction puit de carbone de ces écosystèmes.

Les tourbières, qui ont accumulées un stock de carbone important, sont donc soumises à des contraintes fortes.
Afin de mieux cerner le devenir de ce carbone, l'étude de ces écosystèmes, des flux de gaz qu'ils échangent avec l'atmosphère, est une nécessité.

\index{tourbières|)}

\section{Flux de gaz à effet de serre et facteurs contrôlants}

\subsection{GES et Tourbières}

Dans l'atmosphère le carbone est principalement présent dans l'atmosphère sous forme de dioxide de carbone (\coo) et de méthane (\chh).

La concentration en \coo dans l'atmosphère fluctuait avant l'ère industrielle entre 180 et \SI{290}{ppm}.
En 1750 au début de l'ère industrielle sa concentration était toujours de \SI{280}{ppm} avant d'augmenter pour atteindre \SI{391}{ppm} aujourd'hui (en 2011) \citep{Ciais2014}.
Différents processus permettent d'extraire du \coo de l'atmosphère, la photosynthèse, la dissolution du \coo dans l'océan et enfin l'altération de silicate et les réactions avec le carbonate de calcium.
Ces processus s'effectuent avec des échelles de temps différentes, en conséquence après une émission de \coo, il ne reste que \SI{40}{\percent} de cette émission après \SI{100}{ans}, mais il reste toujours plus de \SI{20}{\percent} après \SI{1000}{ans} et plus de \SI{10}{\percent} après \SI{10000}{ans} \citep{joos2013,Ciais2014} (Figure~\ref{fig:co2_decroissance}).

\begin{figure}
\centering
\includegraphics[width=\textwidth]{chap1/co2_decroissance}
\caption{Décroissance de la proportion de \coo de l'atmosphère suite à une émission idéalisée de \SI{100}{\peta\gram C}. les graphes (a) et (b) est une moyenne de modèles \citep{joos2013}, le graphe (c) est une moyenne d'autres modèles \citep{archer2009}. Modifié d'après \citep{Ciais2014}.}
\label{fig:co2_decroissance}
\end{figure}


La concentration en méthane de l'atmosphère est estimée à \SI{350}{ppb} il y a \SI{18000}{ans} environ lors de la dernière glaciation, à \SI{720}{ppb} en 1750, et à \SI{1800}{ppb} aujourd'hui (ou plutôt en 2011) \citep{Ciais2014}.
À l'inverse du \coo sa durée de vie dans l'atmosphère est limitée : moins de \SI{10}{ans} \citep{lelieveld1998,prather2012}.
Malgré sa faible durée dans l'atmosphère son potentiel de réchauffement global (PRG) est important 72 à 20 ans.
Les zones humides sont la première source naturelle de \chh atmosphérique pour avec un flux à l'échelle globale estimé entre \num{145} et \SI{285}{\tera\gram\per\year} \citep{lelieveld1998,wuebbles2002,Ciais2014} \textbf{(Tableau ?)}.
Les tourbières de l'hémisphère nord comptent pour \SI{46}{\tera\gram\per\year} \citep{gorham1991} \textbf{(pas de source plus récente ?)}.


À l'échelle globale, le stockage de C par les tourbières, prenant en compte à la fois le \coo et le \chh, est estimé à \SI{70}{\tera\gram\per\year} \citep{clymo1998}.
\subsection{Les flux entre l'atmosphère et les tourbières}

\subsubsection{De l'atmosphère à l'écosystème}
%\subsection{Assimilation du carbone atmosphérique}

%%%%% DELOCALISATION ?
%Le carbone est principalement présent dans l'atmosphère sous forme de dioxide de carbone (\coo) et de méthane (\chh).
%Comparé au CO2, le CH4 est un GES qui est bien moins présent dans l'atmosphère (CHIFFRES!).
%Cependant son "pouvoir de réchauffement" est bien plus important (effet radiatif CO2 x 100) (CHIFFRES !) (D'abord la vapeur d'eau, ensuite le CO2 et enfin le CH4)
%Il est usuellement convenu (???? ref) que dans une tourbière le méthane représente environ \SI{5}{\percent} du bilan de C.
%\textbf{Devenir du méthane atm}
%Le transfert du \coo atmosphérique vers la biosphère (de l'atmosphère à la tourbe) est principalement \plop liée à la photosynthèse.
%La photosynthèse est la réaction photochimique permettant l'assimilation du \coo par les végétaux chlorophylliens.
%\textbf{dans le but de ?}.
%
%\textbf{Détails ?}

\begin{figure}
\centering
\includegraphics[height=\textwidth, angle=90]{chap1/ges_flux}
\caption{schéma des flux de carbone entre une tourbière et l'atmosphère}
\label{fig:ges_flux}
\end{figure}


Avant de stocker et de conserver du carbone, le faut le capturer.
Ce transfert du carbone de l'atmosphère à la tourbe se fait sous la forme de \coo, assimilé par la photosynthèse, principalement des végétaux supérieurs, et éventuellement, bien que dans de moindre proportions, par des algues, des lichens ou des bactéries photosynthétiques \cite{girard2011}.
On peut écrire la réaction de photosynthèse de la façon suivante : 
$$\begin{aligned}
CO_{2} + H_{2}O + photons &\rightarrow CH_{2}O + O_{2}\\
\end{aligned} $$
Ce flux est généralement appelé \textbf{Production Primaire Brute} (PPB), \textit{Gross Primary Production}, (\textit{GPP}) en anglais (Figure~\ref{fig:ges_flux}).
Les tourbières sont des écosystèmes dont la production primaire est estimée à environ \SI{500}{\gcm} \citep{francez2000}. 
Si la photosynthèse est un processus majeur d'assimilation du \coo, il existe d'autres voies métaboliques permettant la capture du \coo de l'atmosphère.\index{photosynthèse}
Ainsi les micro-organismes chemolithotrophes (\textbf{expliciter}) sont capables d'assimiler le \coo en utilisant l'énergie issue de l'oxydation de composés inorganiques.

Les voies métaboliques permettant l'assimilation du \coo sont plutôt bien connues et le fait que les substrats de départ de varient pas (mal dit..) a permis une compréhension relativement fine du processus \citep{farquhar1980}.
Cependant une fois assimilé par la végétation le devenir du carbone est moins direct.
À plus long terme, et après son assimilation par la plante, le carbone est stocké principalement à travers la partie non décomposée des litières végétales.
Litières qui à force de compressions et de tassements va devenir de la tourbe.

Il n'y a pas de flux direct de \chh de l'atmosphère vers les écosystèmes terrestres, la majorité du méthane atmosphérique, \SI{90}{\percent}, réagit avec des radicaux hydroxyles, principalement dans la troposphère ou il sera un précurseur de l'ozone
%\subsection{Devenir du carbone assimilé}
%\subsubsection{libération du carbone ? Respiration}
\subsubsection{De l'écosystème à l'atmosphère}

Les sources de carbone émises par les tourbières vers l'atmosphère sont multiples.
D'abord différents gaz peuvent être émis, notamment le \coo et le \chh, éventuellement du N\textsubscript{2}O, et certains d'entre eux peuvent avoir plusieurs sources.

Le \coo est émis dans l'atmosphère à travers différents processus, la respiration aérobie (le plus gros contributeur), les respirations anaérobies ou fermentations (e.g. du glucose, de l'acétate), ou encore l'oxydation du méthane.
Les principales sources de \coo, sont représentées dans la figure~\ref{fig:ges_flux}.
La ou plutôt les respirations sont généralement séparées en deux.
D'un côté la respiration végétale, que ce soit celle de feuilles, des tiges, des racines et que l'on appelle la \textbf{respiration autotrophe}.
De l'autre rassemblé sous le vocable de \textbf{respiration hétérotrophe}, la respiration de la rhizosphère, liée à l'émission d'exsudats par les racines, la décomposition des litières et des matières organiques, la respiration de la faune et l'oxydation du \chh par les organismes méthanotrophes.
On appelle \textbf{Respiration de l'Écosystème} (RE) l'ensemble des respirations autotrophe et hétérotrophe, en incluant à la fois ses composantes aérienne et souterraine.\index{respiration!de l'écosystème}
On la distingue de la respiration du sol qui est définie comme l'ensemble des respirations de la colonne de sol, en excluant la partie aérienne.\index{respiration!du sol}
%
%Une autre source de \coo est l'oxydation du \chh lors de sa migration des zones anoxiques aux zones oxiques de la colonne de tourbe.
%Enfin dans les zones anaérobie, le \coo peut être produit par fermentation (respiration anaérobie).
La production de \coo est donc un signal intégré sur l'ensemble de la colonne de tourbe. 
C'est cette multitude de processus qui rend l'estimation de ce flux difficile, en effet chacune des respirations n'aura pas la même sensibilité vis à vis de facteurs contrôlant.
%La respiration de l'écosystème (RE) est définie comme l'ensemble des respirations de la colonne de tourbe, en incluant à la fois sa partie aérienne et sa partie souterraine. \index{respiration!de l'écosystème}
%La respiration du sol (SR) est elle définie comme l'ensemble des respirations de la colonne de tourbe, en excluant la partie aérienne.\index{respiration!du sol}
%La respiration du sol comprend donc principalement les respirations issues de la rhizosphère et des communautés de micro-organisme.

%Les tourbières sont des écosystèmes dont la production primaire est estimée à environ \SI{500}{\gcm} \cite{francez2000}. 

La strate muscinale pouvant jouer/participer/produire jusqu'à \SI{80}{\percent} de la production primaire \citep{francez2000}.
Cette production primaire n'est pas particulière élevée \plop et c'est en fait la faible décomposition des matières organiques qui permet aux tourbières de stocker du carbone.
L'accumulation moyenne estimée dans les tourbières boréales est de \SI{30}{\gcm}. Le taux d'accumulation varie en fonction des espèces végétales présentes (\plop), le niveau d'eau (\plop), ... (??)

Conséquence du niveau de nappe élevé des tourbières, le développement d'une zone anoxique importante dans la colonne de sol favorise la production de \chh.
En moyenne des flux de \chh mesurés dans les tourbières s'étendent de 0 à plus \SI{0.96}{\uml}, avec généralement des flux compris entre \num{0.0048} et \SI{0.077}{\uml} \citep{blodau2002}.
Le \chh est principalement produit à partir d'acétate (CH\textsubscript{3}COOH) ou de dihydrogène (H\textsubscript{2}), ces deux composés étant dérivés de la décomposition préalable de matières organiques \citep{lai2009}.

$$\begin{aligned}
CH_{3}COOH  &\rightarrow CH_{4} + CO_{2}\\
4H_{2} + CO_{2} &\rightarrow CH_{4} + 2H_{2}O\\
\end{aligned} $$
Le \chh produit est transporté dans l'atmosphère par diffusion, ébullition ou à travers certaines plantes \citep{joabsson1999,colmer2003}.
Pendant ce transport le \chh peut être oxydé par des organismes méthanotrophes.
Cette transformation produit tour à tour différents composés (méthanol, formaldéhyde, formate) aboutissant à la production de \coo \citep{whalen2005}.

$$
CH_{4} \rightarrow CH_{3}OH \rightarrow HCHO \rightarrow HCOOH \rightarrow CO_{2} \\
$$


Le méthane (Lai2009, seger1998, barlett1993 review)

%\subsubsection{storage ?}
%
%Le carbone assimilé par photosynthèse, utilisé par la plante puis évacué que se soit sous forme d'exudats racinaire ou de matériels morts, de litière, va en partie se dégrader.
%Continum de dégradation avec des matières organiques de plus en plus récalcitrantes avec la profondeur.
%
%La vitesse de stockage au cours du temps ?
%
%L'accumulation de matières organiques et donc de carbone dans les tourbières est donc fonction de la prépondérance relative de ces flux entre l'écosystème et l'atmosphère.

\subsection{Les facteurs majeurs contrôlant les flux}


Les facteurs qui contrôlent ces flux de carbone sont globalement connus : la température, le niveau de la nappe et la végétation.
%La température
L'augmentation de la vitesse de réaction de nombreuses réactions biochimiques avec la température est connue depuis longtemps.
Elle a été mise en évidence par un chimiste suédois en 1889 : Svante August Arrhenius sur la base de travaux réalisés par un autre chimiste, néerlandais, Jacobus Henricus Van't Hoff.
Depuis, de nombreuses mesures de terrain confirment cette relation \plop.
La photosynthèse et l'ensemble des respirations sont donc contrôlées, au moins en partie, par la température.
%L'hydrologie
L'hydrologie est un autre facteur contrôlant majeur.
Le niveau de la nappe, défini ici comme la distance entre la surface du sol de l'écosystème et le toit de l'aquifer/l'eau libre/la zone saturée, sépare la colonne de tourbe en une zone oxique, et une zone anoxique.
L'épaisseur relative de ces deux zones va influer sur la production du \coo, majoritairement produit dans la zone oxique, et du \chh produit dans la zone anoxique.
%DELOCALISATION ?
%La zone anoxique permet aux organismes anaérobies de se développer, notamment les Archaea\footnote{micro-organismes unicellulaires procaryotes} méthanogènes.
%L'activité de ces organisme est la plus importante juste sous la surface de l'eau, là ou ils trouvent, en plus de l'anoxie, des matières organiques de qualité (faiblement décomposées).
%La zone aérobie permet la respiration aérobie (\textbf{aérobie vs oxique}) des micro-organismes, des racines et de la faune.
%C'est donc dans cette zone qu'est produit la majorité du \coo.
%Lors de son transport de la zone anoxique vers la surface, le \chh passe par la zone oxique et y est en partie oxydé en \coo.
%(organismes méthanotrophes)
Le niveau de la nappe contraint également le teneur en eau du sol et la hauteur de la frange capillaire qui va influer sur la végétation \citep{laiho2006}.
%La végétation
La végétation est également un facteur important.
D'abord car elle exerce une influence directe sur les flux, avec d'un côté la photosynthèse et les respirations des plantes vivantes, ou la décomposition des plantes mortes.
La composition des communautés végétales va également influer sur le potentiel photosynthétique de l'écosystème, ce potentiel pouvant varier selon le végétal considéré \cite{moore2002}, et sur la vitesse de décomposition des litières qui peut également varier en fonction du végétal. 
De façon plus indirecte, la végétation peut également stimuler la respiration des micro-organismes présent dans la rhizosphère\footnote{zone du sol impacté par les racines} via la libération d'exsudats racinaires \cite{moore2002}.
Enfin certaines plantes vasculaires, adaptées aux conditions de saturations en eau, peuvent faciliter l'échange de gaz entre l'atmosphère et l'écosystème grâce à un espace intercellulaire agrandit, l'Aerenchyme.


%DELOCALISATION ?
%En effet certaines plantes présentes dans ces milieux humides ont développées un Aerenchyme, un espace intercellulaire agrandit permettant le transport d'oxygène des parties aériennes de la plantes aux parties submergées.
%Le transport peut également se faire dans l'autre sens et permettant par exemple le transport du \coo ou du \chh dans l'atmosphère.
%Ce passage au travers de la plante permet également au \chh d'éviter d'être oxydé avant d'atteindre l'atmosphère.
Cependant la sensibilité des flux à ces facteurs ne fait pas consensus et peut varier selon les conditions environnementales ou l'échelle de temps ou d'espace considérée.
Par la suite nous considérons les processus à l'échelle d'une colonne de sol ou d'un écosystème

\subsubsection{Facteurs contrôlant la production primaire brute}
\index{production primaire brute!contrôle}

\begin{figure}
\centering
\includegraphics[width=\textwidth]{chap1/prod_sphagnum}
\caption{Productivités moyennes des espèces de sphaignes en \si{\gram\per\square\metre\per\year}. Les barres d'erreurs représentent l'erreur standard. Le nombre d'observation est indiqué par les nombres à l'intérieur des barres. Les espèces en orange sont celles rencontrées sur le site d'étude. modifié d'après \citet{gunnarsson2005}}
\label{fig:prod_sphagnum}
\end{figure}

Le premier facteur contrôlant la PPB est bien sur la végétation et notamment la composition végétale des communautés présentes.
Les bryophytes n'ont pas la même productivité primaire que les graminées ou que les arbustes.
En plus de ces différences entre groupes de végétaux, il existe également des différences de productivité pour un même groupe selon le type de tourbière \citetext{\citealp{moore2002} dans \citealp{rydin2013b}} .
Alors que dans les tourbières de haut-marais, les sphaignes et les arbustes ont une productivité importante, les herbacées et graminées ont une productivité beaucoup plus faible.
À l'inverse ce sont les herbes et les graminées qui ont la plus forte productivité dans les tourbières de bas-marais pauvres.
Elles sont suivie par les sphaignes puis les arbustes.
Au sein même de ces groupes la productivité peut varier de façon importante, c'est ce que montrent \citet{gunnarsson2005} avec les sphaignes, dont la productivité, selon l'espèce et les conditions dans lesquelles elle vit, varie fortement (Figure~\ref{fig:prod_sphagnum}).

L'effet d'une variation du niveau de la nappe et de la température, jouant sur la végétation va également jouer sur la PPB.
Distinguer ces deux facteurs n'est pas anodin, la majorité des études réalisées sur le terrain montre les effets des deux facteurs combinés.
Ainsi \citet{cai2010} ont montrés que des conditions plus chaudes et sèches pouvaient augmenter la PPB.
L'effet du niveau de la nappe peut varier selon le contexte : Dans une étude des effets à long terme de variation du niveau de la nappe, \citet{ballantyne2014} montrent qu'une baisse du niveau de la nappe entraîne une augmentation de la PPB en facilitant l'accès des plantes vasculaire à l'oxygène et aux nutriments.
Paradoxalement, la hausse d'un niveau de nappe, initialement bas et entraînant un stress hydrique important, conduira également à une augmentation de la PPB \citep{strack2013}.
Ces effets sont variables selon les communautés végétales et le contexte dans lequel elles se trouvent.
Pour un gradient de niveau de nappe qui augmente dans une tourbière de haut-marais, \citet{weltzin2000} montent une diminution de la productivité des arbustes, tandis que celle des graminées n'est pas affectée.
À l'inverse, pour un gradient similaire dans une tourbière de bas-marais, la productivité des arbustes n'est pas affectés tandis que celle des graminées augmente.
Un opposition similaire est également relevé concernant les graminées soumises à un traitement infra-rouge afin de les réchauffer.
Ces dernières voient leur productivité diminuer dans la tourbière de haut-marais et augmenter dans la tourbière de bas-marais.
\citet{munir2015} isolent également l'effet de la température en utilisant des OTC (\textit{Open Top Chamber}).
Ces dispositifs, ressemblant à des serres ouvertes, permettent de réchauffer une zone de la tourbière.
Ils montrent que dans les zones sans manipulation du niveau de la nappe, le réchauffement des OTC, augmente la PPB.

\subsubsection{Facteurs contrôlant la respiration de l'écosystème}
\index{respiration!de l'écosystème!contrôle}

Un facteur majeur contrôlant la RE est la température.
Dans des conditions plus sèches et plus chaude \citet{cai2010} qui montrait une augmentation de la PPB, montre une augmentation plus importante encore de la RE.
\citet{updegraff2001} montrent, dans une expérimentation à base de mésocosme, que la respiration de l'écosystème est contrôlée presque exclusivement par la température du sol.
La modélisation de ce flux se fait donc généralement en utilisant la température que se soit celle de l'air \citep{bortoluzzi2006} ou celle du sol à différentes profondeurs \citep{gorres2014,zhu2015}.

Le niveau de nappe, conditionnant l'accès à l'oxygène, joue également un rôle important.
Un niveau qui diminue se traduit généralement pas une hausse de la RE que ce soit à long terme \citep{strack2006,ballantyne2014} ou à plus court terme \plop.

%Stratck2006 \\
%Augmentation de la respiration suite à un abaissement du niveau de l'eau (8ans plus tôt).
%
%Ballantyne2014 \\
%dans une expérimentation in-situ, montre une respiration de l'écosystème plus importante quand le niveau de la nappe est bas que lorsque le niveau de la nappe est haut.
%L'expérimentation se fait sur un site dont l'abaissement de la nappe est effectif depuis longtemps (80 ans plus tôt)
%Même résultat que strack, donc effet présent même sur le long terme.



\subsubsection{Facteurs contrôlant l'ENE}
\index{echange net de l'ecosystem@échange net de l'écosystème!contrôle}
On défini l'Échange Net de l'Écosystème (ENE) comme la différence entre la Photosynthèse Primaire Brute (PPB) et la Respiration de l'écosystème (RE).
Les facteurs contrôlants l'ENE sont donc les mêmes que ceux qui contrôlent ces 2 flux.
Cependant l'effet d'un même facteur de contrôle peut être différent vis à vis de PPB et de RE selon le contexte environnemental, que ce soit par rapport à la nature de l'effet ou son importance.
Ainsi une variation de l'ENE peut parfois est contrôlé majoritairement soit par la PPB soit par la RE soit par les deux.
Par exemple, une baisse du niveau de la nappe est souvent liée dans la littérature à une baisse de l'ENE.
Cependant certains attribuent cette baisse à une augmentation de la Respiration \citep{alm1999, ise2008} (aurela2013, oechel1993) quand d'autres l'attribuent à une diminution de la photosynthèse \citep{sonnentag2010,peichl2014}.
Enfin certain voient un effet à la fois de l'augmentation de la respiration et de la diminution de la photosynthèse \citep{strack2013}.

À noter un article particulièrement intéressant \citep{lund2012} dans lequel, dans un même site une baisse du niveau de la nappe 2 année différente entrainera une baisse de l'ENE dans les 2 cas, mais dans l'un des cas cette baisse est contrôlée par un augmentation de la respiration et dans l'autre cas cette baisse est contrôlée par une diminution de la photosynthèse.

Également un article de \citet{ballantyne2014} qui lui ne note pas d'effet d'une baisse du niveau de la nappe sur l'ENE car l'augmentation de la respiration est compensée par une augmentation de la photosynthèse.

\subsubsection{Facteurs contrôlant les flux de méthane}

Le niveau de la nappe et la température semblent être les facteurs prépondérant du contrôle des flux de méthane
%\subsubsection{L'hydrologie dans les tourbières et l'effet sur les flux}

%\subsubsection{La végétation dans les tourbières et l'effet sur les flux}


La prépondérance relative des ces différents flux, contrôlée par les conditions environnementale, va donc impacter le fonctionnement des tourbières. 
Soit elles stockent du carbone, en accumulant des matières organiques, et donc fonctionnent comme des puits ou soit elle relâchent du carbone et fonctionnent comme des sources.


L'étude individuelle de tel ou tel flux avec tel ou tel facteur contrôlant est nécessaire afin de comprendre ce qu'il se passe au niveau des processus.
Il est tout aussi nécessaire d'arriver à intégrer l'ensemble de la complexité naturelle.
C'est l'intérêt d'établir des bilans de carbone.

\subsection{Bilans de carbone}

Le calcul d'un bilan de carbone à l'échelle d'un écosystème permet de déterminer si l'équilibre (où le déséquilibre) des flux tend à stocker du carbone, le système fonctionnant alors comme un puits, ou à libérer du carbone, le système fonctionnant alors comme une source.
Il existe différentes façon de réaliser le bilan de carbone d'une tourbière que l'on peut séparer en deux approches principales.
La première approche consiste à utiliser l'archive tourbeuse pour estimer des vitesses d'accumulation de la tourbe.
Cette méthode permet d'étudier la fonction puits sur des temps long (derniers millénaires) et de lier d'éventuels changements dans les vitesses d'accumulation à des facteurs environnementaux.
La seconde approche se base d'avantage sur des mesures actuelles des différents flux afin d'étudier, sur des temps forcément plus court, l'évolution de la prépondérance puits/source d'un écosystème.
Les deux approches sont donc complémentaires.

\subsubsection{passé}
long-term apparent rate of carbon accumulation (LORCA) 
datations + dry bulk density + carbon content
(Tableau~\ref{table:lorca})

\begin{table}
\centering
\caption{Vitesse apparente d'accumulation du carbon à long terme en \si{\gcms}}
\label{table:lorca}
\begin{tabular}{llp{7cm}}\toprule
min -- max & moyenne & référence \\ \midrule
20 -- 140  & ? & Mitra2005 \\ %Xing
? & 18.6 &  Yu2009\\  %Xing
 & 17.2 & Gorham2012 \\  %Xing
 & 20 & Jones2010\\  %Xing
 & 16.2 & Borren2004\\  %Xing
 & 18.5 & Packalen2014\\ %Xing
 & 19.4 & Vitt2000\\ %Roulet2007
 & 19 & Turunen2004\\ %Roulet2007 (ombrotrophic bog)
5.74 -- 129.31 & 33.66 & Xing2015\\
\bottomrule
%CAR : 18.6 turunen2002 in Roulet2007
\end{tabular}
\end{table}
\textbf{tableau LORCA ajouter colonne contexte (exple: 7 tourbières ombrotrophes)}

\subsubsection{présent}
Dans cette approche on estime les flux actuels de carbone entrant et sortant de l'écosystème afin de déterminer un bilan.
Un certain nombre de flux de carbone sont présent au sein des écosystèmes terrestre (équation~\eqref{bdc})

\begin{equation}
BCNE=\frac{dC}{dt}=\overbrace{PPB - Re}^{ENE}  - F_{COD} - F_{COP} - F_{CH_{4}} \textcolor{gray}{- F_{CID} - F_{COV} - F_{CO}}
\label{bdc}
\end{equation}

\begin{itemize}
\item ENE : Échange Net de l'Écosystème
\item PPB : Production Primaire Brute
\item Re : Respiration de l'Écosystème
\vspace*{.2cm}
\item F$_{COP}$ : Flux de Carbone Organique Dissous
\item F$_{COP}$ : Flux de Carbone Organique Particulaire
\item F$_{CH_{4}}$ : Flux de Méthane
\vspace*{.2cm}
\item \textcolor{gray}{F$_{CID}$ : Flux de Carbone Inorganique Dissous}
\item \textcolor{gray}{F$_{COV}$ : Flux de Composés Organique Volatils}
\item \textcolor{gray}{F$_{CO}$ : Flux de Monoxyde de Carbone}
\end{itemize}

Les bilans les plus complets réalisées sur les tourbières comprennent la partie gazeuse, dissoute...

Dans les tourbières, les flux de \coo sont généralement les plus importants \plop, puis les flux de \chh et/ou de COD et enfin les flux de COP.

Pour estimer ces flux différentes techniques existent, notamment l'eddy covariance et les méthodes de chambre pour les flux de gaz.

D'autres méthodes, moins souvent utilisées, existent comme l'utilisation du ratio C:N (Kirk2015)


%\section{Objectifs du travail} %Synthèse bibliographique
% CHAPITRE 2
% SUIVI VARIABILITE SPATIALE

\chapter{Sites d'études et méthodologies employées}

\minitoc

\newpage

\section{Présentation de la tourbière de La Guette}

Le site d'étude, la tourbière de La Guette, est l'un des quatre sites du service national d'observation des tourbières (SNOT) qui vise à étudier la fonction puits de carbone des tourbières tempérées notamment vis-à-vis des changements globaux.

\begin{figure}[h]
\centering
\includegraphics[width=.75\textwidth]{chap2/SNO_siteLocalisation}
\caption{Site d'études SNO}
\label{fig:carte_europe}
\end{figure}
%\subsection{La Guette}

La tourbière de La Guette est située à Neuvy-sur-Barangeon, en Sologne, (N 47\textdegree19’44”, E 2\textdegree17’04”) dans le département du Cher (Figure~\ref{fig:carte_europe}).
Le site s'étend sur une surface d'une vingtaine d'hectare avec une géométrie relativement allongée \ref{fig:carte_LG}.
Cette surface la classe parmi les plus grandes de Sologne.
L'épaisseur moyenne de la tourbe est de \SI{80}{\centi\metre} avec des maximums locaux atteignant \SI{180}{\centi\metre}.
La tourbière de La Guette est probablement topogène \plop, formée par l'accumulation d'eau de pluie dans une cuvette imperméabilisée par une couche d'argile issue d'alluvions de la rivière du même nom (La Guette).
Les précipitations annuelles moyennes sur le site sont de \SI{880}{\milli\metre} et les températures moyenne annuelle de \SI{11}{\degreeCelsius}.
L'eau du site à une conductivité généralement inférieure à \SI{80}{\micro\siemens\per\square\metre} et un pH compris entre 4 et 5.
Ces caractéristiques classe la tourbière parmi les tourbières minérotrophes pauvres en nutriments (\textit{poor fen}).
Les datations effectuées sur le site permettent de dire que les premiers dépôts tourbeux remontent à environ 5 à 6000 ans.

Le site a subi un certain nombre de perturbations au cours de son existence.
D'abord la construction d'une route, avant 1945, qui coupe l'extrémité sud de la tourbière favorisant son drainage.
Le site est également brûlé par un incendie en 1976.
En 1979 des pins noirs (\textit{Pinus nigra}) sont plantés au nord du site
Enfin 2008 le récurage du fossé de drainage bordant la route semble entraîner une augmentation significative des pertes d'eau du système.

\begin{figure}
\includegraphics[width=\textwidth]{chap2/carteGLc}
\caption{Carte de la tourbière de La Guette}
\label{fig:carte_LG}
\end{figure}

Ces perturbations, ou au moins une partie d'entre elles, ont probablement favorisé l'envahissement du site par une végétation vasculaire, notamment arborée et composée de pins (\textit{Pinus sylvestris}) et de bouleaux (\textit{Betula verrucosa} et \textit{pubescens}).
\citet{viel2015} a pu calculé, grâce à l'étude de photos aérienne, la vitesse de fermeture du site, entre 1945 et 2010, estimée à \SI{2020}{\square\metre\per\year} avant l'incendie de 1976 et à \SI{3469}{\square\metre\per\year} après.
La tourbière est également envahie de façon importante par la molinie bleue (Molinia caerula) de la famille des \textit{Poaceae} (Figure~\ref{fig:mol}).
Leur présence favorisant la dégradation des matières organiques \citep{gogo2011}.

Sont également présentes sur le site un certain nombre d'espèces caractéristiques des tourbières comme les sphaignes, principalement \textit{Sphagnum cuspidatum} et \textit{Sphagnum rubellum}, qui forment des tapis.
Un tapis de sphaignes en cours de formation est visible sur la photo~\ref{fig:sphg_erio}.
Sur cette même photo sont également visible des Linaigrettes à feuilles étroites (Eriophorum augustifolium), une plante de la famille des \textit{Cyperaceae} caractéristique des marais et des landes tourbeuses \citep{rameau2008}.
Des bruyères sont également présentes de façon importante sur le site avec notamment \textit{Erica tetralix}, parfois appelée la Bruyère des marais, de la famille des \textit{Ericaceae} (Figure~\ref{fig:erica}).
De la même famille est présente sur le site, mais de façon moins omniprésente, la Callune (\textit{Calluna vulgaris}).
L'ensemble de ces espèces tendent à préférer les milieux riches en matières organiques et pauvres en nutriment (tela-botanica).

D'autres espèces sont présentes sur ce site notamment, \textit{Rhynchospora alba} de la famille des \textit{Cyperaceae}, \textit{Juncus bulbosus}(\textbf{image annexe ?}), de la famille de \textit{Juncaceae}, et des Droséras, une plante insectivore, de la famille des \textit{Droseraceae} (Annexe~\ref{sec:photos_veg}, Figure~\ref{fig:dro}) . 

\begin{figure}[htbp]
    \centering
    \begin{subfigure}[b]{.98\textwidth} % "0.45" donne ici la largeur de l'image
        \centering \includegraphics[trim=5cm 0cm 0cm 10cm, clip=true, width=\textwidth]{chap2/sphaigne_eriophorum_c.jpg}
        \caption{\textit{Sphagnum} -- \textit{Eriophorum augustifolium}}\label{fig:sphg_erio}
    \end{subfigure}
    
%    ~ % ce symbole ajoute un espacement horisontal entre les premières deux images
    \begin{subfigure}[b]{0.49\textwidth}
%        \centering \includegraphics[trim=0cm 0cm 0cm 0cm, clip=true, width=\textwidth]{chap2/molinia_caerulea_c.jpg}
        \centering \includegraphics[trim=0cm 4cm 0cm 0cm, clip=true, width=\textwidth]{chap2/erica_tetralix_c.jpg}
        \caption{\textit{Erica tetralix} -- \textit{Molinia caerulea}}\label{fig:erica}
    \end{subfigure}
    % la ligne blanche correspond au retour à la ligne après le deuxième image
    \begin{subfigure}[b]{0.49\textwidth}
        \centering \includegraphics[trim=0cm 2cm 0cm 2cm, clip=true, width=\textwidth]{chap2/molinia_caerulea_c.jpg}
%        \centering \includegraphics[trim=0cm 0cm 0cm 0cm, clip=true, width=\textwidth]{chap2/erica_tetralix_c.jpg}
        \caption{\textit{Molinia caerulea}}\label{fig:mol}
    \end{subfigure}
%    ~

%    \begin{subfigure}[b]{.8\textwidth}
%        \centering \includegraphics[trim=2.5cm 5cm 2.5cm 5cm, clip=true, width=\textwidth]{chap2/drosera_c.jpg}
%        \caption{drosera}\label{fig:dro}
%    \end{subfigure}
    \caption{Végétation présente sur le site de La Guette, et suivie lors des campagnes de mesure.}\label{fig:veg}
\end{figure}


\begin{figure}
\centering
\includegraphics[width=\textwidth]{chap2/pluvio}
\caption{Évolution du niveau de la pluviométrie, en \si{\mm}, des années 2011 à 2014}
\label{fig:pluvio}
\end{figure}

\begin{figure}
\centering
\includegraphics[width=\textwidth]{chap2/WTL}
\caption{Évolution du niveau de la nappe, en cm par rapport à la surface, des années 2011 à 2014}
\label{fig:WTL}
\end{figure}

Au cours des dernières années, les précipitations sont relativement différentes avec deux années plus sèche que la moyenne avant 2013 et deux années plus humide en 2013 et 2014 (Figure~\ref{fig:pluvio}).
On observe également cette dualité au niveau du niveau de la nappe.
Avant 2013 les étés sont marqués par des étiages important avec des baisses du niveau de nappe allant jusqu'à \SI{-60}{\cm} en 2012 (Figure~\ref{fig:WTL}).
Après 2013, les étiages sont beaucoup moins importants sur le site.



\begin{figure}
\centering
\includegraphics[width=\textwidth]{chap2/tair}
\caption{Évolution de la température de l'air (en \textdegree C) des années 2011 à 2014}
\label{fig:tair}
\end{figure}



Au sein de ses sites de nombreuses mesures ont été effectuée et notamment des mesures de flux de GES à la fois concernant le \coo et le \chh. La méthodologie étant transverse à de nombreuses expérimentations il convient de l'expliquer au préalable.

\section{Autres sites du service national d'observation}

Bien que moins étudiés, les autres sites du SNOT, Bernadouze, Frasne et Landemarais ont également fait l'objet d'un suivi ponctuel en 2013.
La tourbière de Bernadouze est situé à \SI{1400}{\metre} dans les Pyrénées, en Ariège (N 42\textdegree 48’09”, E 1\textdegree25’24”)).
Elle est relativement petite avec \SI{3.75}{\hectare} seulement.
La tourbière de Frasne est situé à \SI{840}{\metre} dans le Doubs et s'étend sur une surface de \SI{98}{\hectare}.
Enfin la tourbière de Landemarais est située en Ille-et-villaine, à \SI{154}{\metre} et sur \SI{23}{\hectare}.
les températures annuelles moyennes sur ces trois sites sont respectivement de 6, 7,5 et \SI{11}{\degreeCelsius}.
les précipitations annuelles étant de \num{1700}, \num{1400}, \SI{870}{\milli\meter}.


\section{Mesures de flux}
\label{sec:clsd_chbr_method}

\subsection{Présentation des méthodologies possibles}
De nombreuses techniques permettent de mesurer des flux de gaz, avec en premier lieu les méthodes de chambres.


Les chambres peuvent être ouvertes, c'est à dire que la mesure se fait lorsque le gaz à l'intérieur de la chambre à l'équilibre avec celui à l'extérieur, ou fermées, dans ce cas le gaz à l'intérieur de la chambre n'est pas à l'équilibre avec celui à l'extérieur.
Elles peuvent également être dynamique, lorsqu'un système de pompe, permettant notamment de transporter le gaz jusqu'à l'analyseur, est présent.
Ou statique si le système est sans flux artificiel.

Trois grandes techniques de chambre existent.
D'abord les chambres \textbf{dynamiques ouvertes} qui se basent sur un état d'équilibre et mesurent une différence de concentration d'un gaz dont une partie passe par la chambre et l'autre non. 
Cette méthode nécessite un système de pompe et donc le passage d'un flux.
Ensuite les chambres \textbf{dynamiques fermées} qui mesurent l'évolution de la concentration du gaz au sein de la chambre à l'aide d'un système de pompe permettant l'envoi du gaz dans un analyseur externe.
Enfin les chambres \textbf{statiques fermées} qui mesurent également l'évolution de la concentration du gaz au sein de la chambre sans qu'un système de pompe ne soit présent.
Dans ce cas soit l'analyseur est présent dans la chambre, soit des prélèvements sont fait à intervalles réguliers puis analysés par la suite en chromatographie gazeuse.

Il faut noter que les dénominations anglaises de ces méthodes doit faire l'objet d'une attention particulière.
\textit{Closed chamber} par exemple est parfois utilisé pour se référer à l'état ou non d'équilibre, comme défini dans ce document, mais parfois également pour désigner les méthodes de chambre sans système de flux ce qui peut prêter à confusion \cite{pumpanen2004}.
Souvent utilisées les dénominations \textit{open}/\textit{closed} et \textit{dynamic}/\textit{static} sont décrites dans \cite{luo2006161}, une autre convention peut être rencontrée : \textit{flow-through}/\textit{non-flow-through} et \textit{steady state}/\textit{non-steady state} \cite{livingston1995}

Ces différentes méthodes ont divers avantages et inconvénients.

Ces méthodes sont souvent utilisées car elles on un coût modeste, et sont très versatiles ce qui permet leur utilisation dans de nombreuses situations.
D'autres méthodes plus globales existent comme les méthodes d'Eddy Covariance.

Les méthodes d'Eddy Covariance se base sur...

Comparaison entre les méthodes de chambre et les méthodes d'Eddy Covariance.

\subsection{Les mesures de \coo}

Toutes les mesures de \coo présentées par la suite ont été faite avec les mêmes matériels et le même protocole.
Les chambres en \textbf{XXXX} ont été conçue (LPC2E) fabriquées (ISTO) au CNRS.
Ce sont des chambres transparentes, cylindrique, de \SI{30}{\centi\metre} de diamètre et \SI{30}{\centi\metre} de hauteur.
Les mesures de \coo à proprement parler ont été faite à l'aide d'une sonde Vaisala GMP 343\textregistered.
La sonde est directement inséré dans la chambre ainsi qu'une sonde Vaisala \textbf{XXXXX} mesurant d'humidité et la température dans la chambre.

Avant toute mesure, des embases sont installées sur le site.
Ce sont des cylindres de PVC d'une hauteur de \SI{15}{\centi\metre} inséré dans le sol sur 8 à \SI{10}{\centi\metre}.
La partie enterrée de ces cylindres ayant préalablement été percée d'une quarantaine de trou afin de minimiser les impacts de l'embase sur le développement racinaire et les écoulements d'eau.

La méthode mise en œuvre est celle de la chambre statique fermée.
Aucun système de pompe n'est donc utilisé, la chambre est posée sur l'embase, elle contient l'analyseur de \coo qui mesure la variation de la concentration en gaz au cours du temps.
Un ventilateur de faible puissance est également présent à l'intérieur de la chambre afin d'homogénéiser l'air présent dans la chambre.
1 à \SI{3}{\minute} sont nécessaires après la pose de la chambre afin d'éviter les effets pouvant y être liés.
Ensuite l’enregistrement est lancé, avec l'acquisition toutes les \SI{5}{\second} pendant \SI{5}{\minute} de la concentration en \coo, de la température et de l'humidité.
La mesure se déroule donc sur une période de temps relativement courte afin de minimiser le déséquilibre avec le milieu extérieur.
Dans ce but les mesures on parfois été encore raccourcie, 2 à \SI{3}{\minute} d'acquisition, si une pente claire se dégage rapidement, notamment lorsque les conditions météorologiques, chaudes et ensoleillées, laissaient supposer une différence importante vis à vis des conditions extérieures.

Généralement, deux acquisitions de \coo sont faites à la suite sur une embase.
La première, avec la chambre transparente nue, permettant l'enregistrement de l'ENE (Figure~\ref{fig:chb}-a).
La seconde avec la chambre recouverte d'une chaussette de tissu occultant, isolant la chambre de la lumière, permettant d'interrompre la photosynthèse et donc d'enregistrer les respirations (RE) (Figure~\ref{fig:chb}-b).

\begin{figure}
	\centering
	\begin{subfigure}[t]{0.5\textwidth}
		\centering
		\includegraphics[width=.8\textwidth, frame]{chap2/chb_ENE}
	\end{subfigure}%
	\begin{subfigure}[t]{0.5\textwidth}
		\centering
		\includegraphics[width=.8\textwidth, frame]{chap2/chb_ER}
	\end{subfigure}%

	\begin{subfigure}[t]{0.5\textwidth}
		\includegraphics[width=\textwidth]{chap2/chb_ENE_reg}
		\caption{Mesure de l'échange net de l'écosystème}
	\end{subfigure}%
	\begin{subfigure}[t]{0.5\textwidth}
		\includegraphics[width=\textwidth]{chap2/chb_ER_reg}
		\caption{Mesure de la respiration de l'écosystème}
	\end{subfigure}
%    \caption{Caption place holder}
\caption{Mesures de \coo}
\label{fig:chb}
\end{figure}


De nombreux écueils peuvent rendre une mesure inexploitable. D'abord le placement de la chambre, cela peut sembler trivial mais positionner la chambre au milieu d'herbacées et de bruyère n'est pas toujours évident. Plus anecdotiquement des sphaignes gelées, recouvrant les bords de l'embase rendent la pose de la chambre difficile voire impossible. Selon l'heure de la journée des gradients de concentrations peuvent être présent et augmenter localement les concentrations de \coo de façon importante allant jusqu'à saturer la sonde.

Au vu du volume de données acquises et souhaitant garder l'intérêt de mesure manuelle, à savoir le contrôle humain des flux et des conditions de mesure, il a été nécessaire de développer un outil de traitement facilitant le contrôle et le calcul des flux.
Ceci afin d'éviter de recourir à des seuils arbitraires (typiquement une valeur de R$^{2}$) pour le contrôle qualité des données, mais également de permettre une reproductibilité et un traçage des modifications effectuées sur les données brutes.
(donner des exemples)

\subsection{Les mesures de \chh}

\begin{figure}
\includegraphics[width=\textwidth]{chap2/SPIRIT_terrain}
\caption{SPIRIT}
\label{fig:SPIRIT}
\end{figure}

Les mesures de \chh ont été réalisée avec une chambre aux caractéristiques similaires à celles utilisées pour les mesures de \coo à l'exception de l'interface avec l'analyseur.
La méthode de la chambre dynamique fermée a été utilisée pour réaliser ces mesures, elle diffère donc légèrement de celle utilisée pour le \coo puisqu'elle nécessite la mise en oeuvre d'un système de pompe pour transporter le gaz jusqu'à l'analyseur.
Les mesures de concentration en \chh ont été réalisée à l'aide d'un instrument développé par le LPC2E, le SPIRIT (Figure~\ref{fig:SPIRIT}).
C'est un SPectrometre Infra Rouge In-situ Troposphérique (son premier objectif étant d'être emporté lors de campagne avion ou ballon ? pour mesurer le \chh de la troposphère.).
Il permet la mesure du \chh à haute fréquence.
Le fonctionnement détaillé de l'appareil est décrit dans \cite{guimbaud2011}.

\begin{equation}
F = \frac{dX}{dt} \times \frac{P}{R \times T} \times \frac{V}{S}
\end{equation}

%QUESTIONS :
%
%*Taille des embases ? Effets de bord ?
%*Perturbation du milieu ? (Mesure de végétation, pose de la chambre, mesure pièzo...)
%*Impact de la strate arborée ?
%*Validité des profils de température ?
%Méthode de Chambre fermée (Biais ?)
%
%Améliorations ? (Lister les amélioration à faire ou non)


\section{Facteurs contrôlants}
Afin de déterminer l'impact de facteurs contrôlants sur ces flux, mesurer les flux ne suffit pas il faut également mesurer les variables environnementales dont on pense qu'elles seront des facteurs contrôlants important.
La description des techniques et matériels communs aux différentes expérimentations utilisées est développée ci-dessous.
Par contre leur mise en œuvre ou caractéristiques spécifiques, comme la fréquence des mesures, sera décrite individuellement au niveau des parties détaillant chacune des expérimentations.

\subsection{acquisitions automatisées}

Les paramètres météorologiques ont été mesurés, en un point, au centre de la tourbière (Figure~\ref{fig:carte_LG})(\textbf{carte ?}) à l'aide d'une station d'acquisition Campbell installée sur le site en 2008.
Les variables ont été acquises à une fréquence horaire jusqu'au 20 février 2014 puis toutes les demi-heures par la suite. 
Les paramètres enregistrés sont la pression atmosphérique, l'humidité relative de l'air, la pluviométrie, l'irradiation solaire, la vitesse et la direction du vent. (\textbf{détail du matos ?}).
Cette même station à également permis l'acquisition de la température de l'air et de la tourbe à \num{-5}, \num{-10}, \num{-20} et \SIlist{-40}{\cm}.
Installées à la même époque, quatre sondes \textbf{OTT ?} de mesure du niveau de la nappe d'eau permettent le suivi du niveau de la nappe dans la tourbière.

%\subsection{Protocole d'estimation de la végétation}


%\begin{figure}
%\includegraphics[width=.5\textwidth]{chap2/mol_lon_bioM}
%\includegraphics[width=.5\textwidth]{chap2/mol_lon_bioM}
%\includegraphics[width=.5\textwidth]{chap2/mol_lon_surf}
%\includegraphics[width=.5\textwidth]{chap2/mol_lon_surf}
%\caption{Calibration de la biomasse herbacées pour \textit{molinia Caerulea} (a), pour \textit{eriophorum} (b) et de la surface de feuille pour \textit{molinia Caerulea} (c), pour \textit{eriophorum} (d) en fonction de la hauteur}
%\label{fig:cal_herb}
%\end{figure} %Sites d'études et méthodologies employées
% CHAPITRE 3
\chapter{Bilan de C de la tourbière de La Guette}

\minitoc

\newpage

\section{Introduction}

Parmi les écosystèmes tourbeux pour lesquels un bilan de carbone a été calculé, la majorité se situe dans les hautes latitudes \plop et/ou en montagne.
Le premier objectif de ce chapitre est d'établir le bilan de C de la tourbière de La Guette.
L'intérêt est double, d'une part car ce site est représentatif d'une grande partie des tourbières dans les perturbations qu'elle subie : son drainage et son envahissement par une végétation vasculaire (cf Chapitre 2).
D'autre part sa position en basse latitude la place dans des conditions environnementale qui, sans être identiques, peuvent se rapprocher de celles que subiront d'autres écosystèmes tourbeux suite au réchauffement climatique.
Le second objectif est de caractériser la variabilité spatiale de ces flux de GES à travers ce bilan de C.

\section{Procédure expérimentale et analytique}

\subsection{Méthodes de mesure}

\subsubsection{Mesures de flux de gaz}
\textbf{placer carte tourbière embase + quadrillage}

Les mesures des flux de \coo et de \chh ont été effectués en utilisant la méthode décrite dans la partie~\ref{sec:clsd_chbr_method}.
En juin 2011, 20 placettes ont été installées\footnote{je remercie ici Sébastien Gogo pour avoir installé ces placettes sur le terrain avant même mon arrivée.} selon un échantillonnage aléatoire stratifié:
La surface de la tourbière a été divisée selon une grille de 20 mailles et un point choisi aléatoirement dans chaque maille localise chaque placette.
Cette méthode permet de conserver un échantillonnage aléatoire tout en étant assuré d'avoir une représentativité homogène du site. 
Les placettes, délimitées par des piquets, occupaient une surface de \SI{4}{\square\metre} (2$\times$\SI{2}{\metre}), à l'intérieur de laquelle ont été installé de façon permanente un piézomètre et une embase permettant la mesure des flux de gaz.
Usuellement les placettes sont séparées en groupes micro-topographique. ce qui à l'avantage de permettre une distinction des capacités sources/puits relativement fine mais qui à généralement l'inconvénient du placement proche des embases les unes des autres.
Elles peuvent également être séparées en zone dans la tourbière, haut-marais par rapport à bas-marais, ou réhabilité par rapport à non-réhabilité.
Afin de gagner en représentativité spatiale, la taille du site le permettant, il a donc été décidé de positionner des placettes sur l'ensemble du site.
De plus, du fait de l'omniprésence de végétation vasculaire, et de la taille des chambres par rapport à la micro-topographie une telle approche était difficile à mettre en oeuvre.

Les flux de gaz mesurés sont le \coo et le \chh.
Des tests effectués sur la tourbière ayant montré des émissions de N$_{2}$O nulle, ce gaz n'a pas été étudié.
Les mesures de \coo ont été effectué de mars 2013 à février 2015, avec une fréquence quasiment mensuelle (20 campagnes, pour 24 mois de mesure). Quand aux mesures de \chh ont été effectuées avec une fréquence moindre (12 campagnes) principalement liée au difficulté de mise en oeuvre de l'instrument SPIRIT (lourd, difficilement transportable dans un milieu tourbeux).



Les facteurs contrôlant mesurés manuellement sont la pression atmosphérique, du PAR, des températures du sol à différentes profondeur, de la végétation.
Des prélèvements d'eau ont également été effectué chaque mois, une mesure du pH et de la conductivité dans cette eau a été réalisée sur le terrain après les mesures de flux puis les échantillons ont été congelés avant d'être analysé en terme de concentration de carbone dissous.
Ces mesures nécessitant d'accéder aux placettes régulièrement, des planches de bois ont été utilisées comme pontons mobiles, la dispersion des placettes sur l'ensemble du site rendant impossible une installation plus permanente.

Les mesures automatiquement acquise via une station météo campbell sont la température de l'air, température de la tourbe à \num{-5}, \num{-10}, \num{-20} et \SI{-40}{\centi\metre} profondeur, vitesse et direction du vent, humidité relative de l'air, irradiation solaire, pression atmosphérique.

\subsection{Modélisation du bilan de C}

\subsubsection{Démarche générale}

Afin de calculer le bilan de carbone du site il est nécessaire d'établir des modèles des flux afin de pourvoir interpoler les données acquises mensuellement sur l'ensemble des deux années de mesure.
Pour établir ces modèles empiriques les données acquises ont été moyennées par campagne de mesure.
Ceci permettant, dans un premier temps, de s'affranchir de la variabilité spatiale des flux pour se concentrer sur la variabilité temporelle.
Les relations entre flux et facteurs contrôlant ont ensuite été étudiées deux à deux.

La RE, et l'ENE sont mesurés directement sur le terrain.
Cependant afin d'établir le bilan de C tout en gardant une discrimination entre flux d'entré et de sortie la RE et la PPB (obtenue grâce à l'équation PPB = ENE - RE) ont été modélisé séparément.
%Les flux de \coo ont été modélisé en partant de l'équation ENE = PPB - RE, et le bilan a été établi en estimant de façon séparée la PPB et la RE.
%Cette séparation permettant de distinguer si une variation du bilan est liée à l'un ou l'autre des flux ou bien aux deux.
Les flux en phase gazeuse ont été modélisé en partant d'équations usuellement utilisées et dans lesquelles la température est le facteur contrôlant majeur.
Puis les résidus\footnote{Valeurs moyennes - Valeurs moyennes estimées} de ces modèles de base ont ensuite été étudiés en fonction des facteurs de contrôle restant.
Dans le cas ou une tendance est visible, le facteur est intégré.

Les modèles ont été comparés avec différents indicateurs, principalement Le R2, la NRMSE et l'AIC.
Le R$^{2}$ est utilisé comme indicateur de la proportion de la variabilité des données expliqué par le modèle, sa valeur est comprise entre 0 et 1.
La RMSE et sa normalisation par la moyenne NRMSE sont utilisés comme indicateur de l'écart entre les données mesurées et les données modélisées.
L'AIC (Akaile...) permet de déterminer si l'amélioration d'un modèle suite à l'ajout d'un paramètre est suffisamment intéressante pour que ce modèle plus complexe soit utilisé.

La température a été choisie comme base de départ à la construction des modèles de RE et PPBsat, à la fois car c'est le facteur de contrôle le plus souvent invoqué et à la fois car les corrélations avec les flux étaient les plus forte.
Concernant la respiration de l'écosystème, les températures utilisées dans la littérature sont variables.
La température qui semble le plus utilisée est la température du sol à \SI{-5}{\centi\metre}  \cite{ballantyne2014}\plop, même si d'autres, notamment la température de l'air et la température du sol à \SI{-10}{\centi\metre} le sont également régulièrement \cite{bortoluzzi2006,kim1992}.
Cette profondeur, \SI{-5}{\cm}, est régulièrement utilisée car c'est dans la tourbe, proche de la surface qu'est produit la majorité du \coo.
\textbf{production CO2 ? profils ?}
C'est également à des profondeurs relativement faibles que se situent la majorité des racines \plop qui peuvent contribuer à la respiration du sol \textbf{(de l'écosystème?)} pour 35 à \SI{60}{\percent} \cite{silvola1996,crow2005}.
La RE est estimée directement à partir des données acquises moyennées en partant de la température connue pour contrôler une grande partie de ce flux.
Différents modèles ont été testés parmi les plus souvent utilisés (linéaire, exponentiel, arrhénius).

Pas de consensus clair émerge de la littérature quand aux facteurs prépondérant dans le contrôle du \chh.
La température, peut être utilisée \cite{alm1999,bubier1995}, le niveau de la nappe \cite{bubier1993} ou la végétation \cite{bortoluzzi2006}.

Après cette phase de calibration, les facteurs de contrôle utilisés dans les modèles ont été évalué à l'aide de données indépendantes issues d'une autre expérimentation.
Cette dernière est également un suivi des même flux de gaz, sur le même site pendant l'année 2014.
Les méthodes de mesures des flux étant strictement identiques à celles utilisées pour établir le bilan de carbone.
En revanche le positionnement des placettes est beaucoup plus standard avec 4 placettes dans une station en amont et 4 en aval (plus de détails dans l'annexe \textbf{XXX}).
On ne parle pas ici de validation car les données utilisées bien qu'indépendante du jeu de données utilisé pour la calibration n'ont pas été acquise suivant un protocole identique, notamment au niveau de la répartition des embases sur le site.

Enfin les facteurs contrôlants ont été interpolés au pas de mesure de la station météo présente sur le site, c'est à dire à l'heure.
L'interpolation étant soit une simple interpolation linéaire entre les données mensuelles, soit une relation avec les facteurs acquis par la station météorologique.
À l'aide de ces interpolations et des équations les flux ont ensuite été recalculés, à l'échelle horaire, sur les 2 années de mesure puis les flux ont été sommés afin de calculer les bilans.


\section{Résultats}

\subsection{Évolution générale des flux et facteurs contrôlants sur la tourbière de La Guette}

\subsubsection{Les Facteurs contrôlant}

\begin{figure}
\centering
\includegraphics[width=\textwidth]{chap3/WTL_mean_evolution}
\caption{Évolution du niveau de la nappe moyen des 20 embases mesuré pendant la période de mesure (mars 2013 -- février 2015)}
\label{fig:WTL_mean_evolution}
\end{figure}

L'évolution du niveau de la nappe des 20 placettes, décrite dans la figure~\ref{fig:WTL_mean_evolution}, est marquée par un étiage d'une vingtaine de centimètres en moyenne en 2013 et l'absence d'un étiage net en 2014 avec un niveau de la nappe moyen ne descendant que rarement sous la barre des \SI{-10}{\cm}.
Ces observations sont cohérentes avec la figure~\ref{fig:WTL} représentant des données acquises à plus haute fréquence, et confirment la particularité de ces 2 années vis à vis des précédentes qui présentent des étiages bien plus fort.

\begin{figure}
\centering
\includegraphics[width=\textwidth]{chap3/T_mean_evolution}
\caption{Évolution des températures de l'air (Tair) et du sol à \SIlist{-5;-30;-50;-100}{\centi\metre} (T5, T30, T50 et T100 respectivement) moyenne mesurée lors des campagnes de terrain de mars 2013 à février 2015}
\label{fig:T_mean_evolution}
\end{figure}

La température de l'air mesurée manuellement montre une variabilité saisonnière cohérente avec celle mesurées par la station météo. 
%bien que les valeurs semblent systématiquement supérieures.
La variabilité saisonnière de la température est également visible dans le sol avec cependant un amortissement et une diminution de la variabilité avec la profondeur (figure~\ref{fig:T_mean_evolution})
%chiffres ?

\begin{figure}
\centering
\includegraphics[width=\textwidth]{chap3/cond_mean_evolution}
\caption{Évolution de la conductivité pendant la période de mesure (mars 2013 -- février 2015)}
\label{fig:cond_mean_evolution}
\end{figure}

La conductivité moyenne mesurée sur le site varie entre \SIlist{35;55}{\usml} (figure~\ref{fig:cond_mean_evolution}).


\begin{figure}
\centering
\includegraphics[width=\textwidth]{chap3/pH_mean_evolution}
\caption{Évolution du pH pendant la période de mesure (mars 2013 -- février 2015)}
\label{fig:pH_mean_evolution}
\end{figure}


En moyenne le pH mesuré sur la tourbière de La Guette est compris entre 4 et 5 (figure~\ref{fig:pH_mean_evolution}).
Ces valeurs sont cohérentes avec la classification \textit{poor-fen} du site .


%\begin{figure}
%\centering
%\includegraphics[width=\textwidth]{chap3/RH_mean_evolution}
%\caption{Évolution de la teneur en eau du sol pendant la période de mesure (mars 2013 -- février 2015)}
%\label{fig:RH_mean_evolution}
%\end{figure}

\begin{figure}
\centering
\includegraphics[width=\textwidth]{chap3/NPOC_mean_evolution}
\caption{Évolution de la concentration en carbone organique dissous dans l'eau du sol pendant la période de mesure (mars 2013 -- février 2015)}
\label{fig:NPOC_mean_evolution}
\end{figure}

La concentration en carbone organique dissous présente dans l'eau de la tourbière est compris en moyenne entre \num{10} et \SI{30}{\milli\gram\per\liter} (figure~\ref{fig:NPOC_mean_evolution}).

%\subsection{Évolution générale des flux de C sur la tourbière de La Guette}

\subsubsection{Les flux de carbone}

\begin{figure}
	\centering
	\begin{subfigure}[t]{\textwidth}
		\centering
		\includegraphics[width=\textwidth]{chap3/GPP_evolution_avg}
		\caption{Production primaire brute}
		\label{fig:GPP_evolution_avg}
	\end{subfigure}%
	
	\begin{subfigure}[t]{\textwidth}
		\centering
		\includegraphics[width=\textwidth]{chap3/ER_evolution_avg}
		\caption{Respiration de l'écosystème}
		\label{fig:ER_evolution_avg}
	\end{subfigure}
	
	\begin{subfigure}[t]{\textwidth}
		\centering
		\includegraphics[width=\textwidth]{chap3/NEE_evolution_avg}
		\caption{Échange net de l'écosystème}
		\label{fig:NEE_evolution_avg}
	\end{subfigure}
\caption{Évolution du niveau de PPB, RE et ENE pendant la période de mesure. Moyenne des 20 embases de mars 2013 à février 2015.}
\label{fig:flux_evolution_avg}
\end{figure}

%\subsubsection{PBB}
L'ensemble des mesures de \coo s'étendent de mars 2013 à février 2015.
Cependant de novembre 2013 à février 2014 les mesures ont été interrompue suite à des pannes/casses matérielles.
Malgré cela les périodes les plus critiques, notamment la saison de végétation, ont pu être mesurées pour les 2 années, permettant d'avoir une vision correcte/globale de chacune d'elle.
À noter également que pour l'ensemble des flux, la déviation standard augmente avec les valeurs mesurées.

En 2013, les valeurs de la PPB augmentent au printemps et une partie de l'été avec un maximum de \SI{999999(888)}{\uml} atteint fin juillet, avant de diminuer à partir d'août.
En 2014 le maximum de PPB, \SI{99999(888)}{\uml}, est atteint en juin, soit plus tôt que l'année précédente.
Puis pendant l'été et l'automne les valeurs décroissent jusqu'à être proche de 0.
En moyenne les valeurs de la PPB sont de \SI{7.12(519)}{\uml} en 2013 et de \SI{6.56(472)}{\uml} en 2014 (Figure~\ref{fig:GPP_evolution_avg}).

La RE en 2013 augmente pendant le printemps et une partie de l'été, elle atteint un maximum de \SI{99999(888)}{\uml} en juillet avant de diminuer.
En 2014 la RE atteint, comme la PPB, son maximum plus tôt, en juin à \SI{99999(888)}{\uml} avant de décroître jusqu'en hiver pour approcher des valeurs nulles.
La moyenne annuelle de RE en 2013 est de \SI{4.27(316)}{\uml}, ce qui est légèrement supérieure à celle de 2014 : \SI{3.63(256)}{\uml}(Figure~\ref{fig:ER_evolution_avg}).

Concernant l'ENE, en 2013 elle augmente jusqu'en juin avec un maximum à \SI{99999(888)}{\uml} avant de diminuer jusqu'à la fin de l'année.
Cependant, cette baisse est moins homogène que celle des deux flux précédents, avec notamment une augmentation de l'ENE entre juillet et août 2013.
Ceci étant, il faut également noter les valeurs importantes de la déviation standard particulièrement en juin et en août.
En 2014, l'ENE maximum est atteinte en juillet avec \SI{99999(888)}{\uml} avant qu'elle ne décroisse.
Cette baisse est cependant plus homogène qu'en 2013.
les moyennes de l'ENE en 2013 et 2014 sont très proche est sont respectivement de \SI{2.85(305)}{\uml} et \SI{2.93(277)}{\uml} (Figure~\ref{fig:NEE_evolution_avg}).

%\subsubsection{Le \chh}

Le \chh comme le \coo montre une variabilité saisonnière importante, cependant les flux mesurés sont un ordre de grandeur en dessous de ceux mesurés pour le \coo (Figure~\ref{fig:CH4_evolution_avg}).
À l'inverse de ce dernier, l'importance des flux de \chh mesurés en 2013 et 2014 est différente.
En 2013 les flux sont moins important qu'en 2014 avec des maximum de \SIlist{0.078;0.196}{\uml} respectivement.

\begin{figure}
\centering
\includegraphics[width=\textwidth]{chap3/CH4_evolution_avg}
\caption{Évolution des flux de méthane moyen (N ?) pendant la période de mesure (mars 2013 -- février 2015)}
\label{fig:CH4_evolution_avg}
\end{figure}

%\subsubsection{Le Carbone Organique Dissous (COD)}
\begin{figure}
\centering
\includegraphics[width=\textwidth]{chap3/Fl_FC}
\caption{Relations entre les flux de gaz et une sélection de facteurs contrôlant}
\label{fig:Fl_FC}
\end{figure}

\subsubsection{Les relation flux gazeux et facteurs contrôlants}

Comme précisé précédemment, le niveau de la nappe n'a que peu varié pendant les deux années de mesures.
De ce fait aucune relation claire ne se distingue entre les flux et le niveau de la nappe que ce soit pour le \coo (PPB et RE) ou le \chh (Figure~\ref{fig:Fl_FC}).
La PPB et la RE présentent cependant des relations avec la température de l'air, et l'indice de végétation, même si pour ce dernier les tendance sont moins évidentes, particulièrement pour la RE.
Le \chh quand à lui ne présente pas de relation avec la température de l'air, mais une tendance exponentielle est visible vis à vis de l'indice de végétation.
\textbf{(\chh et Température dans la tourbe ?)}

\subsection{Sélection des modèles}

\subsubsection{La Production Primaire Brute}

\begin{figure}
\centering
\includegraphics[width=\textwidth]{chap3/mdl_GPP_Tair}
\caption{PPBsat modèles Tair utilisant l'équation~\ref{eq:juneTair}}
\label{fig:mdl_GPP_Tair}
\end{figure}

L'estimation de la PPB se fait en deux étapes.
Dans un premier temps on estime le potentiel maximum de photosynthèse à un instant donné dans des conditions de lumière saturante (PPBsat).
Ce potentiel peut varier avec les conditions environnementales et a été déterminé en utilisant l'équation de \cite{june2004} qui relie la vitesse de transport des électrons photosynthétique à lumière saturante à la température :

\begin{equation}\label{eq:juneTair}
PPBsat = a * exp(\frac{Tair - b}{c})^2
\end{equation}

Avec a la vitesse de transport des électrons photosynthétique à lumière saturante, b la température optimale pour ce transport et c la différence de température à laquelle à laquelle PBBsat vaut e$^{-1}$ de sa valeur à la température optimale.
À partir de ce potentiel à lumière saturante, la PPB est estimée en prenant en compte la luminosité.
On utilise l'équation~\ref{eq:PPB_bubier} proposée par \cite{bubier1998} et régulièrement et majoritairement utilisée \cite{bortoluzzi2006,worrall2009}:

\begin{equation} \label{eq:PPB_bubier}
PPB = \frac{PPBsat * a * PAR}{PPBsat + a * PAR}
\end{equation}

L'utilisation de l'équation de June seule, avec la température de l'air comme variable explicative de la PPBsat, permet d'expliquer 66 \% des variations observées avec une erreur standard de l'estimation de \SI{32}{\percent} (Figure~\ref{fig:mdl_GPP_Tair}-a).
Les résidus de ce modèle se répartissent de façon relativement homogène et non biaisée (Figure~\ref{fig:mdl_GPP_Tair}-b).
Corrélés avec l'indice de végétation IV, ils présentent une tendance linéaire croissante (Figure~\ref{fig:mdl_GPP_Tair}-c).
On observe la même tendance avec le recouvrement de la strate herbacée avec une dispersion des points plus importante (Figure~\ref{fig:mdl_GPP_Tair}-d).
Par contre aucune tendance particulière n'est visible vis à vis du niveau de la nappe (Figure~\ref{fig:mdl_GPP_Tair}-e)
Le recouvrement des sphaignes (non présenté) ne montre également, aucune tendance avec les résidus de cette équation.
La PPB calculé à partir de l'équation~\ref{eq:juneTair} montre une erreur standard de \SI{31}{\percent}, du même ordre de grandeur que celle de PPBsat (Figure~\ref{fig:mdl_GPP_Tair}-f) et les résidus se répartissent de façon relativement homogène et non biaisée (Figure~\ref{fig:mdl_GPP_Tair}-g).
Cependant l'évaluation du modèle sur les données de tests montre une erreur standard de l'estimation plus forte qui atteint \SI{47}{\percent}(Figure~\ref{fig:mdl_GPP_Tair}-h).
Par ailleurs une forte incertitude est présent concernant l'estimation des paramètres qui ont tous une erreur standard importante, parfois plus importante que la valeur du paramètre, et une faible significativité (Tableau~\ref{table:mdl_par}).
Afin de prendre en compte la tendance linéaire entre les résidus et l'indice de végétation (IV) le modèle est adapté pour y intégré une fonction linéaire de la végétation :

\begin{equation}\label{eq:juneTairIV}
PPBsat = (a * IV + d) * exp(\frac{T - b}{c})^2
\end{equation}

\begin{figure}
\centering
\includegraphics[width=\textwidth]{chap3/mdl_GPP_TairIV}
\caption{PPBsat modèles Tair utilisant l'équation~\ref{eq:juneTairIV}}
\label{fig:mdl_GPP_TairIV}
\end{figure}

Cette nouvelle équation permet d'expliquer une part plus importante des variations de PPBsat (R$^{2}$ = 0,85) et augmente la proximité entre les données mesurées et les données modélisées : l'erreur standard diminue à \SI{21}{\percent}. (Figure~\ref{fig:mdl_GPP_TairIV}-a).
Les résidus de cette équation semblent répartis de façon moins homogène que précédemment.
On observe notamment un resserrement des points autour de zéro à l'exception d'un point de valeur supérieur à \num{4}.
Le biais reste malgré tout léger au regard de l'amélioration apportée.
Aucune tendance claire ne se dégage des résidus lorsqu'ils sont mis en relation avec des facteurs contrôlant tel que les recouvrements végétaux (sphaignes, herbacées), ou le niveau de la nappe (Figure~\ref{fig:mdl_GPP_TairIV}-c,d,e).
Comme précédemment, l'erreur standard de la GPP, de \SI{19}{\percent}, est du même ordre de grandeur que celle de PPBsat.
Pour PPBsat comme pour PPB l'erreur standard diminue donc avec l'ajout de l'indice de végétation lors de la calibration.
En revanche, l'évaluation sur les données de test de ce dernier modèle montre une erreur importante (\SI{58}{\percent}), supérieure à celle du modèle ne prenant pas en compte la végétation.
Cette évaluation montre également une tendance importante à sous-estimer les valeurs mesurées.
Néanmoins ce modèle intégrant la végétation permet de diminuer de façon importante l'erreur standard associée à l'estimation des paramètres de l'équation.
Dans la suite du texte le modèle permettant d'estimer la PPB à partir des équations~\ref{eq:juneTair} et \ref{eq:PPB_bubier} sera nommé PPB-1 et celui utilisant les équations~\ref{eq:juneTairIV} et \ref{eq:PPB_bubier} sera nommée PPB-2.


%\begin{figure}
%\centering
%\includegraphics[width=\textwidth]{chap3/GPP_mdl_mesmod}
%\caption{PPB modèles Tair}
%\label{fig:PPB_Tair_mdl}
%\end{figure}

%La PPB calculée à partir de l'équation~\ref{eq:juneTair} présente des résidus relativement homogène avec cependant d'avantage de points situés entre \num{-2} et \num{-4} qu'entre \num{2} et \num{4} (Figure~\ref{fig:mdl_GPP_Tair}--a,b).
%Une observation similaire peut être faire pour PPB calculé à partir de l'équation~\ref{eq:juneTairIV}, avec cependant des résidus resserrés entre \num{-2} et \num{2} et un point un peu plus extrême(Figure~\ref{fig:mdl_GPP_Tair}--c,d).

\subsubsection{La Respiration de l'Écosystème}

%%%%%%%%%%%%%%%%%%%%% RE


L'estimation de la RE s'effectue

\begin{equation} \label{eq:RE_T}
RE = a*exp(b*T)
\end{equation}

La température de l'air utilisée dans un modèle exponentiel permet d'expliquer une grande partie, 90 \%, des variations de la respiration de l'écosystème avec une erreur standard de \SI{18}{\percent} (Figure~\ref{fig:mdl_ER_Tair}--a).
Les résidus de cette équation semble répartis de façon non-biaisée, pas de tendance dans le nuage de point (Figure~\ref{fig:mdl_ER_Tair}--b).
L'évaluation de ce modèles montre une erreur standard de \SI{35}{\percent} avec une tendance à sous-estimer les valeurs mesurées.
Une légère tendance, moins claire que pour la PPBsat, est visible entre les résidus et l'indice de végétation ainsi qu'avec le recouvrement de la strate herbacée.
Très souvent utilisée, la température à \SI{-5}{\centi\metre} donne des résultats proche mais moins bons notamment avec une hétéroscédasticité des résidus (\textbf{Fig Annexe ?}).
%Dans les deux cas, le gain possible en ajoutant un paramètre semble limité.
On adapte l'équation~\ref{eq:RE_T} pour intégrer le signal de végétation :

\begin{equation} \label{eq:RE_TIV}
RE = (a*IV + c)*exp(b*T)
\end{equation}

\begin{equation} \label{eq:RE_TH}
RE = (a*H + c)*exp(b*T)
\end{equation}

Les calibrations de ces nouvelles équations sont présentées dans la figure~\ref{fig:mdl_ER_Tair}-a,b et \ref{fig:mdl_ER_Tair}-d,e respectivement.
Dans les deux cas, l'erreur diminue pour avoisiner \SI{13}{\percent}, avec des résidus qui se répartissent de façon non-biaisée.
L'évaluation de ces deux équations montre cependant des différences :
D'un côté l'équation~\ref{eq:RE_TIV} ne permet pas de diminuer l'erreur standard qui vaut \SI{34}{\percent}, et est donc très proche des \SI{35}{\percent} calculé pour l'évaluation du modèle n'intégrant pas la végétation.
De l'autre l'évaluation de l'équation~\ref{eq:RE_TH} montre une erreur standard plus faible s'établissant à \SI{23}{\percent}.
Les paramètres des différentes équations sont présentés dans le tableau~\ref{table:mdl_par}, les modèles RE-1, RE-2, et RE-3 correspondent respectivement aux équations~\ref{eq:RE_T}, \ref{eq:RE_TIV} et \ref{eq:RE_TH}.
À l'inverse de la PPB les paramètres des modèles de la RE ont, à l'exception du paramètre c du modèle RE-2, une significativité importante et une erreur standard faible.

\begin{figure}
\centering
\includegraphics[width=\textwidth]{chap3/mdl_ER_Tair}
\caption{RE modèles avec Tair}
\label{fig:mdl_ER_Tair}
\end{figure}

\begin{figure}
\centering
\includegraphics[width=\textwidth]{chap3/mdl_ER_Tair_IVH}
\caption{RE modèles avec Tair}
\label{fig:ER_mdl_TairIVH}
\end{figure}

%%%%%%%%%%%%%%%%%%%%% ENE
%\subsubsection{L'Échange Net de l'Écosystème}
%
%\begin{figure}
%\centering
%\includegraphics[width=\textwidth]{chap3/NEE_mdl_mesmod}
%\caption{ENE modèle T5 IV}
%\label{fig:ENE_mdl}
%\end{figure}
%
%L'ENE est ensuite modélisé en utilisant l'équation suivante :
%
%\begin{equation}
%ENE = PBB-RE
%\end{equation}
%
%Le résultat de cette équation (Figure~\ref{fig:ENE_mdl}), montre que ce modèle permet d'expliquer une grande partie des variations de l'ENE.
%Les résidus de cette équation sont répartis de manière a peu près homogène.

\subsubsection{Le flux de \chh}

\begin{figure}
\centering
\includegraphics[width=\textwidth]{chap3/mdl_CH4_IV}
\caption{CH4 modèle H}
\label{fig:CH4_mdl}
\end{figure}

Les relations entre les facteurs contrôlant mesurés et le \chh sont moins claires que celles concernant le \coo.
La corrélation la plus importante semble être liée à la végétation (Figure~\ref{fig:Fl_FC}). 
le méthane est également corrélé avec les températures, faiblement avec les température de surface puis plus fortement avec les températures du sol à plus forte profondeur.
Enfin il est anti-corrélé (R=-0.51) avec le niveau de la nappe.
Les relation \chh et végétation ont donc pu être modélisées avec 

\begin{equation} \label{eq:CH4_H}
F_{CH_{4}} = a*exp(b*IV)
\end{equation}

Avec les données acquises, l'indice de végétation est le meilleur prédicteur (Figure~\ref{fig:CH4_mdl}), il explique \SI{78}{\percent} de la variabilité du \chh avec une erreur standard de \SI{32}{\percent}.
Aucune tendance ne semble ce dégager entre les résidus de cette équations et les facteurs contrôlant mesurés.
L'évaluation de cette équation montre une tendance à sous-estimer les flux de \chh et une erreur standard qui double par rapport à la phase de calibration en atteignant \SI{68}{\percent}.
Les détails de l'estimation des paramètres de l'équation~\ref{eq:CH4_H} est visible dans le tableau~\ref{table:mdl_par} sous le nom FCH4.

%bortoluzzi veg
%Alm 1999 T -30 cm
%Bellisario relation inverse avec WTL (increased flux with lower WTL) T10
%Bubier 1993a WTL majeur
%Bubier 1995 Température humidité végétation

\subsubsection{Le COD}


\subsection{Le bilan de carbone de la tourbière de La Guette à l'échelle de l'écosystème}

\subsubsection{Bilan (param et valeur)}

\begin{table}
\centering
\caption{Valeur des paramètres des équations utilisées pour modéliser les flux et sensibilité relative (en \%) des flux en réponse à une variation de $\pm$\SI{10}{\percent} de chacun des paramètres des modèles.}
\label{table:mdl_par}
\begin{tabular}{llccccc}\toprule

& par & valeur & se & pval & \SI{-10}{\percent} & +\SI{+10}{\percent} \\ \midrule
\multicolumn{7}{l}{PPB-1 -- équations~\ref{eq:juneTair} et \ref{eq:PPB_bubier}}  \\ [+.5ex]
& a & 26.23 & 62.07 & 0.68 & -9.7 & +9.6 	\\
& b & 53.68 & 61.27 & 0.39 & +43.7 &-35.1 \\
& c & 27.21 & 28.56 & 0.35 &-22.5 & +21.9 \\
& i &  1.84 & 21.6  & 0.93 & -0.4 &  +0.4 \\[+1ex]
\multicolumn{7}{l}{PPB-2 -- équations~\ref{eq:juneTairIV} et \ref{eq:PPB_bubier}}  \\ [+.5ex]
& a & 39.44 & 18.89 & 0.05 & -11.8 & +11.5 \\
& b & 40.27 & 19.11 & 0.05 & +15.8 & -17.2 \\
& c & 25.23 & 14.35 & 0.1 & -8.1 & +6.7 \\
& d & -3.73 & 3.49 & 0.3 & +2.8 & -2.8 \\
& i &  0.26 & 0.25 & 0.31 & -1.3 & +1.1 \\[+1ex]
\multicolumn{7}{l}{RE-1 -- équation~\ref{eq:RE_T}}  \\ [+.5ex]
& a & 0.34 & 0.08 & 0 & -10 & +10 \\
& b & 0.10 & 0.01 & 0 & -22.6 & +29.9  \\[+1ex]
\multicolumn{7}{l}{RE-2 -- équation~\ref{eq:RE_TIV}}  \\ [+.5ex]
& a & 0.92 & 0.34 & 0.02 & -7.3 & +7.3 \\
& b & 0.09 & 0.01 & 0.00 & -19.5 & 24.7 \\
& c & 0.14 & 0.09 & 0.14 & +2.7 & -2.7  \\[+1ex]
\multicolumn{7}{l}{RE-3 -- équation~\ref{eq:RE_TH}}  \\ [+.5ex]
& a & 0 & 0 & 0.01 & -3.9 & +3.9 \\
& b & 0.08 & 0.01 & 0 & -18.8 & +23.6 \\
& c & 0.33 & 0.06 & 0 & -6.1 & +6.1 \\[+1ex]
\multicolumn{7}{l}{FCH4 -- équation~\ref{eq:CH4_H}}  \\ [+.5ex]
& a & 0 & 0 & 0.48 & -10 & +10 \\
& b & 13.01 & 2.82 & 0 & -43.9 & +79.2 \\[+1ex]
\bottomrule
\end{tabular}
\end{table}



\begin{table}
\centering
\caption{Bilan des flux en gCm2an1}
\label{table:flux}
\begin{tabular}{llllll}\toprule
ID & Flux & équation & 2013 & 2014 & moyen \\ \midrule
GPP-1 & GPP & \ref{eq:juneTair} et \ref{eq:PPB_bubier} & 1322 $\pm$ 410 & 1258 $\pm$ 390 & 1290 $\pm$ 400 \\
GPP-2 & & \ref{eq:juneTairIV} et \ref{eq:PPB_bubier} & 957 $\pm$ 182 & 1184 $\pm$ 225 & 1070 $\pm$ 203 \\[+1.5ex]
RE-1 & RE & \ref{eq:RE_T} & 1337 $\pm$ 241 & 1235 $\pm$ 222 & 1286 $\pm$ 231 \\
RE-2 & & \ref{eq:RE_TIV} & 1232 $\pm$ 160 & 1310 $\pm$ 170 & 1271 $\pm$ 165\\
RE-3 & & \ref{eq:RE_TH} & 1240 $\pm$ 161 & 1281 $\pm$ 167 & 1261 $\pm$ 164 \\[+1.5ex]
FCH4 & CH4 & \ref{eq:CH4_H} & 10 $\pm$ 3 & 24 $\pm$ 8 & 17 $\pm$ 5 \\
\bottomrule
\end{tabular}
\end{table}


\begin{table}
\centering
\caption{Bilan des flux en gCm2an1}
\label{table:bdc}
\begin{tabular}{llll}\toprule
combinaison de modèles & 2013 & 2014 & moyen \\ \midrule
PPB-1, RE-1, FCH4 &  \num{-25} & \num{-2} & \num{-14} \\
PPB-1, RE-3, FCH4 &  +\num{72} & \num{-48} & +\num{12} \\
PPB-2, RE-1, FCH4 &  \num{-390} & \num{-75} & \num{-233} \\
PPB-2, RE-3, FCH4 &  \num{-293} & \num{-122} & \num{-208} \\
\bottomrule
\end{tabular}
\end{table}

Les flux interpolés à l'heure puis sommés par années sont présentés dans le tableau~\ref{table:flux} pour les différents modèles.
Sur les deux années, selon le modèle utilisé, la PPB stocke du carbone pour 1070 à \SI{1290}{\gcma} pour PPB-2 et PPB-1 respectivement.
Dans le détail on observe une différence entre les deux modèles, celui utilisant uniquement la température de l'air (GPP-1) présente un stockage plus important en 2013 qu'en 2014.
À l'inverse le modèle prenant en compte la végétation (PPB-2) montre une relation inverse avec un stockage plus faible en 2013 par rapport à 2014.
Cette observation est également valable  pour la RE, la prise en compte de la végétation (modèles RE-2 et RE-3) conduit à une respiration plus forte en 2014 par rapport à RE-1.
Toujours concernant la RE, on observe une grande proximité dans les valeurs des flux interpolés sur les 2 années, quelque soit le modèle, avec un écart maximum de \SI{25}{\gcma}.
Les flux de \chh entachés d'une erreur importante, sont cependant beaucoup plus faible en valeur absolue que les précédents.
On y observe malgré tout un flux qui fait plus que doubler en 2014 par rapport à 2013.

\begin{figure}
\centering
\includegraphics[width=\textwidth]{chap3/BdC_GPP_interp}
\caption{Flux de \coo interpolé à partir de PPB-1 et PPB-2}
\label{fig:BdC_GPP_interp}
\end{figure}

\begin{figure}
\centering
\includegraphics[width=\textwidth]{chap3/BdC_ER_interp}
\caption{Flux de \coo interpolé à partir de RE-1, RE-2 et RE-3}
\label{fig:BdC_ER_interp}
\end{figure}



L'interpolation des flux de PPB montrent une variabilité saisonnière proche de celle mesurée sur le terrain (Figure~\ref{fig:BdC_GPP_interp}). 
Les valeurs les plus forte mesurées ne semblent pas atteinte par le modèle PPB-2 à l'inverse du modèle PPB-1.
Dans les deux cas les modèles semblent sur-estimer la valeurs de PPB mesurées fin 2014.
Pour la RE, l'interpolation suit également les variations saisonnière mesurée mensuellement (Figure~\ref{fig:BdC_ER_interp}).
Les gammes de valeurs mesurées sont très proche des gammes interpolées.
L'interpolation des flux de la RE est très proche quelque soit le modèle (Figure~\ref{fig:BdC_ER_interp}) utilisé.
On observe que l'intégration de la végétation dans les modèles RE-2 et RE-3 diminue la RE maximum modèlisée en 2013 par par rapport au modèle RE-1.

%Enfin pour l'ENE, les valeur interpolées suivent les variations saisonnière, malgré une interpolation qui semble sous-estimer les flux par rapport aux valeurs mesurées (Figure~\ref{fig:BdC_interp}-c).

%\begin{table}
%\centering
%\caption{Bilan en gCm2an1}
%\label{table:BdC}
%\begin{tabular}{lllllll}\toprule
%& année & PBB & RE & CH4 & COD & BCNE \\ \midrule
%%mdl 1 (Tair - Tair - IV) & 2013 & 1317.8 $\pm$ 266 & 1332.8 $\pm$ 238 & 10.1 $\pm$ 3 & & \\[+.5ex]
%%                         & 2014 & 1257.7 $\pm$  254& 1273.3 $\pm$ 228 & 24.6 $\pm$ 8 & & \\ [+1ex]
%%                         & total & 1287.7 $\pm$ 260& 1283.9 $\pm$ 229 & 17.4 $\pm$ 6& &\\[+2ex]
%mdl 2 (Tair IV - Tair - IV) & 2013 & 1035.2 $\pm$ 209  & 1332.8 $\pm$ 238 & 10.1 $\pm$ 3 & & \\[+.5ex]
%                            & 2014 & 1212.9 $\pm$ 245 & 1273.3 $\pm$ 228  & 24.6 $\pm$ 8 &  & \\ [+1ex]
%                            & total & 1124.1 $\pm$ 227 & 1283.9 $\pm$ 229  & 17.4 $\pm$ 6 & &\\[+2ex]
%%mdl 3 (Tair IV - T5 - IV) & 2013 & 1035.2 & 1358.8 & 10.1 & & \\[+.5ex]
%%                          & 2014 & 1212.9 & 1273.3 & 24.6 & & \\ [+1ex]
%%                          & total & 1124.1 & 1288.6 & 17.4 & &\\[+1ex]
%\bottomrule
%\end{tabular}
%\end{table}



\subsubsection{Évaluation du bilan}

%\subsubsection{sensibilité des paramètres}
\begin{table}
\centering
\caption{Sensibilité relative (en \%) du bilan de \coo (ENE) en réponse à une variation de $\pm$\SI{10}{\percent} de chacun des paramètres des modèles.}
\label{table:mdl_sensitiv_BdC}
\begin{tabular}{cccccccccc}\toprule
& \multicolumn{3}{l}{PPB} & \multicolumn{3}{l}{RE} & \multicolumn{3}{l}{\chh} \\ 
& & \SI{-10}{\percent} & +\SI{+10}{\percent} & & \SI{-10}{\percent} & +\SI{+10}{\percent} & & \SI{-10}{\percent} & +\SI{+10}{\percent} \\ \midrule
& \multicolumn{3}{l}{PPB-1} & \multicolumn{3}{l}{RE-1} & \multicolumn{3}{l}{FCH4} \\ [+.5ex]
& a & \num{-3263} & +\num{3243} & a & +\num{3371} & \num{-3371} & a & +\num{0.05} & \num{-0.05}\\
& b & +\num{14788} & \num{-11859} & b & +\num{7616} & \num{-10078} & b & +\num{0.2} & \num{-0.36}\\
& c & \num{-7597} & +\num{7398} & & & & & &\\
& i & +\num{119} & \num{-139} & & & & & &\\[+1ex]

& \multicolumn{3}{l}{PPB-2} & \multicolumn{3}{l}{RE-1} & \multicolumn{3}{l}{FCH4} \\ [+.5ex]
& a & +\num{59} & \num{-57} & a & \num{-60} & +\num{60} & a & \num{0} & \num{0}\\
& b & \num{-78} & +\num{85} & b & \num{-135} & +\num{178} & b & \num{0} & +\num{0.01}\\
& c & +\num{40} & \num{-33} & & & & & &\\
& d & \num{-14} & +\num{14} & & & & & &\\
& i & \num{6.22} & \num{-5.40} & & & & & &\\[+1ex]

& \multicolumn{3}{l}{PPB-1} & \multicolumn{3}{l}{RE-3} & \multicolumn{3}{l}{FCH4} \\ [+.5ex]
& a & \num{-426} & +\num{423} & a & +\num{168} & \num{-168} & a & +\num{0.01} & \num{-0.01}\\
& b & +\num{1931} & \num{-1548} & b & +\num{813} & \num{-1018} & b & +\num{0.03} & \num{-0.05}\\
& c & \num{-992} & +\num{966} & c & +\num{263} & \num{-263} & & &\\
& i & \num{-18} & +\num{15} & & & & & &\\[+1ex]

& \multicolumn{3}{l}{PPB-2} & \multicolumn{3}{l}{RE-3} & \multicolumn{3}{l}{FCH4} \\ [+.5ex]
& a & +\num{67} & \num{-65} & a & \num{-26} & +\num{26} & a & 0 & 0\\
& b & \num{-89} & +\num{97} & b & \num{-125} & +\num{157} & b & 0 & 0\\
& c & +\num{45} & \num{-38} & c & \num{-40} & +\num{40} & & &\\
& d & \num{-16} & +\num{16} & & & & & &\\
& i & +\num{7.1} & \num{-6.1} & & & & & &\\[+1ex]

\bottomrule
\end{tabular}
\end{table}

L'analyse de sensibilité, consistant à faire varier chaque paramètre des modèles de $\pm$\SI{10}{\percent}, les combinaisons de modèles PPB-1, PPB-2, RE-1 et RE-3 ont été testé (Tableau~\ref{table:mdl_sensitiv_BdC}).
Le modèle RE-2, très proche du modèle RE-3 n'a pas été testé.
\textbf{Attente du COD, les valeurs du tableau sont fausse pour le moment}

%, montre pour une équation exponentielle simple des valeurs attendues $\pm$\SI{10}{\percent} pour le paramètre a et \num{+29} à \SI{-22}{\percent} pour le b (Figure~\ref{table:mdl_sensitiv}).
%Pour la PPB issue des équations~\ref{eq:juneTairIV} et \ref{eq:PPB_bubier} le paramètre i à très peu d'effet sur le bilan, \num{0} à \SI{-1}{\percent}.
%Cependant l'effet sur le bilan augmente lorsque la végétation est prise en compte (équation~\ref{eq:juneTair} et \ref{eq:PPB_bubier}) : \num{-8} à \SI{-10}{\percent}.
%À l'inverse, la sensibilité de l'ensemble des autres paramètres (a, b, c) diminue lorsque l'indice de végétation est pris en compte.
%Le paramètre a est l'exception, passant de \num{-10} à \SI{-17}{\percent} pour une baisse de \SI{10}{\percent}.
%Considérant le modèle de PPB prenant en compte la végétation, la sensibilité maximum des différents paramètres du bilan est proche de \SI{30}{\percent}, et similaire pour la PPB et la RE.



%\begin{table}
%\centering
%\caption{Sensibilité relative (en \%) du bilan de \coo (ENE) en réponse à une variation de $\pm$\SI{10}{\percent} de chacun des paramètres des modèles.}
%\label{table:mdl_sensitiv_BdC}
%\begin{tabular}{llccccccccccc}\toprule
%& \multicolumn{4}{l}{PPB} & \multicolumn{4}{l}{RE} & \multicolumn{4}{l}{\chh} \\ 
%& par & valeur & \SI{-10}{\percent} & +\SI{+10}{\percent} & par & valeur & \SI{-10}{\percent} & +\SI{+10}{\percent} & par & valeur & \SI{-10}{\percent} & +\SI{+10}{\percent} \\ \midrule
%\multicolumn{13}{l}{PPB-2, ER-3, FCH4} \\ [+.5ex]
%& a & 0.34 & -766 & 766 & a & 26.23 & 741 & -736 & a & 17.82 & -12.28 & 12.28 \\
%& b & 0.10 & -1730 & 2289 & b & 53.68 & -3358 & 2693 & b & 0.03 & -15.08 & 17.68 \\
%& c &  & & & c & 27.21 & 1725 & -1680 & & & & \\
%& d &  & & & c & 27.21 & 1725 & -1680 & & & & \\
%& e &  & & & i & 1.84 & 31.56 & -26.92 & & & & \\[+1ex]
%%\multicolumn{13}{l}{mdl 2 équation~\ref{eq:RE_T5} et équation~\ref{eq:juneTairIV} Tair, Tair IVcov, H} \\ [+.5ex]
%%& a & 0.34 & -71.08 & 71.08 & a & 33.66 & 56.25 & -55.32 & a & 17.82 & -1.14 & 1.14 \\
%%& b & 0.10 & -160.59 & 212.51 & b & 42.45 & -119.84 & 123.69 & b & 0.03 & -1.40 & 1.64 \\
%%&  &  & & & c & 25.77 & 62.63 & -53.28 & & & & \\
%%&  &  & & & i & 0.33 & 6.94 & -6.02 & & & & \\[+1ex]
%%\multicolumn{13}{l}{mdl 3 équation~\ref{eq:RE_T5IV} et équation~\ref{eq:juneTairIV} Tair IVcov, Tair IVcov, H} \\ [+.5ex]
%%& a & 0.92 & -55.69 & 55.69 & a & 33.66 & 61.31 & -60.31 & a & 17.82 & -1.24 & 1.24 \\
%%& b & 0.09 & -149.40 & 189.01 & b & 42.45 & -130.64 & 134.84 & b & 0.03 & -1.53 & 1.79 \\
%%& c & 0.14 & -20.89 & 20.89 & c & 25.77 & 68.28 & -58.08 & & & & \\
%%&  &  & & & i & 0.33 & 7.57 & -6.56 & & & & \\
%\bottomrule
%\end{tabular}
%\end{table}


%\begin{table}
%\centering
%\caption{Sensibilité relative (en \%) des flux en réponse à une variation de $\pm$\SI{10}{\percent} de chacun des paramètres des modèles.}
%\label{table:mdl_sensitiv}
%\begin{tabular}{llccccccccccc}\toprule
%& \multicolumn{4}{l}{RE} & \multicolumn{4}{l}{GPP} & \multicolumn{4}{l}{\chh} \\ 
%& par & valeur & \SI{-10}{\percent} & +\SI{+10}{\percent} & par & valeur & \SI{-10}{\percent} & +\SI{+10}{\percent} & par & valeur & \SI{-10}{\percent} & +\SI{+10}{\percent} \\ \midrule
% & \multicolumn{4}{l}{Tair} & \multicolumn{4}{l}{Tair} & \multicolumn{4}{l}{H} \\ [+.5ex]
%& a & 0.34 & -10   & 10   & a & 26.23 & -9.7 &  9.6 & a & 17.82 &-10.0 & 10.0 \\
%& b & 0.10 & -22.6 & 29.9 & b & 53.68 & 43.7 &-35.1 & b &  0.03 &-12.3 & 14.4 \\
%&   &      &       &      & c & 27.21 &-22.5 & 21.9 &   &       &      &      \\
%&   &      &       &      & i &  1.84 & -0.4 &  0.4 &   &       &      &      \\[+1ex]
% & \multicolumn{4}{l}{Tair IVcov} & \multicolumn{4}{l}{Tair IVcov} & & & & \\ [+.5ex]
%& a & 0.92 & -7.3  & 7.3 & a & 33.66 & -9.0 &  8.9 & & & & \\
%& b & 0.09 & -19.5 & 24.7& b & 42.45 & 19.3 &-19.9 & & & & \\
%& c & 0.14 & -2.7  & 2.7 & c & 25.77 &-10.1 &  8.6 & & & & \\
%&   &      &       &     & i & 0.33  & -1.1 &  1.0 & & & & \\[+1ex]
%\bottomrule
%\end{tabular}
%\end{table}




%\subsubsection{pseudo-validation et erreur}

%\begin{figure}
%\centering
%\includegraphics[width=.5\textwidth]{chap3/ER_T5_val}
%\caption{Évaluation RE}
%\label{fig:RE_T5_val}
%\end{figure}
%\begin{figure}
%\centering
%\includegraphics[width=.5\textwidth]{chap3/GPP_TairIVcov_val}
%\caption{Évaluation GPP}
%\label{fig:GPP_TairIVcov_val}
%\end{figure}

\subsection{Variabilité spatiale du bilan}

\subsubsection{Représentativité locale}

\subsubsection{Modélisation par placette}

\subsubsection{Corrélation avec facteurs contrôlant}

\section{Discussion}

\subsection{Estimations des flux}

\subsubsection{PPB}

% Flux mesurés // biblio
%Les flux mesurés sont ...
%
%aurela 2007 inf \SI{5}{\uml}

% Flux modélisés
L'estimation des flux de PPB, est comprise entre 957 et \SI{1322}{\gcma} selon l'année et le modèle utilisé.
Ces valeurs sont très élevées comparées à des tourbières boréales comme celles étudiées par \cite{trudeau2014} ou encore \cite{peichl2014} dont les valeurs de PPB sont respectivement comprise entre 123 et \SI{131}{\gcma} et entre 203 et \SI{503}{\gcma}.
Une première hypothèse permettant d'expliquer une telle différence, est la différence entre les températures moyennes sur les sites.
Ces températures sont bien plus faible pour les sites des études pré-citées que sur la tourbière de La Guette.
Une seconde hypothèse, qui n'exclue par la première, serait la composition végétale de ces sites. 
La tourbière de La Guette envahie par une végétation vasculaire, notamment herbacée, est peut être plus proche d'une prairie tourbeuse.
En effet lorsque la compare à ce type d'écosystèmes les valeurs observées sont plutôt proche.
\cite{jacobs2007} estime des valeurs de PPB comprises entre 400 et \SI{2000}{\gcma} avec une moyenne de \SI{1300}{\gcma} dans des prairies tourbeuses hollandaise.
Sur des écosystèmes similiaires, au Danemark, \cite{gorres2014} trouve des valeurs de PPB plus importantes encore, entre 1555 et \SI{2590}{\gcma}.
Il apparait cohérent que la tourbière de La Guette, par sa position géographique, et donc le climat qu'elle subit, mais également par sa problématique d'envahissement important par une végétation vasculaire notamment herbacée, s'approche en terme de flux de site tourbeux utilisés comme prairie.
%jacobs 2007  sols tourbeux 1300 $\pm$ \SI{100}{\gcma} (grassland)
%
%jacobs 2007  sols tourbeux entre 400 et \SI{2000}{\gcma} 8.9- 9.5 °C 730-750 mm
%
%gorres 2014  entre 1555 et \SI{2590}{\gcma} 8.8-9.5°C -579-702 mm (peat grassland)
%
%trudeau 2014 123-\SI{131}{\gcma} -4.28°C 738 mm (3.5 ha oligotrophic patterned fen.)
%
%peichl2014 entre 203 et \SI{503}{\gcma} 1.2°C 523 mm (boreal fen)

% % Apport IV
L'apport de l'indice de végétation dans l'estimation de la PPB est important lors de la phase de calibration (observation faite également par \cite{bortoluzzi2006}).
Il permet de diminuer fortement l'incertitude sur les paramètres du modèles, d'augmenter leur significativité et d'améliorer la représentativité des données mesurées.
Cependant cette amélioration ne semble valoir que pour la série mesuré, en effet lors de l'évaluation sur un jeu de donnée indépendant, l'ajout de l'indice de végétation augmente l'erreur du modèle.
Dans le détail l'intégration de l'indice de végétation à un effet beaucoup plus important en 2013, avec un flux qui diminue de \SI{365}{\gcma}, qu'en 2014 ou la baisse n'est que de \SI{74}{\gcma}.

% % Incertitudes
Le modèle PPB-1 souffre d'une incertitude importante sur l'estimation de ses paramètres.
Cette incertitude est fortement diminué avec l'intégration (modèle PPB-2) de l'indice de végétation, cependant lorsque évalué sur des données indépendantes l'erreur standard qui lui est associé est plus forte que celle du modèle PPB-1.
Peu d'études sur des tourbières comparent différents modèles, \cite{worrall2009} en testant deux façons de calculer la PPB observe également une grande variabilité dans les résultats.

\subsubsection{RE}

% Flux mesurés // biblio
De la même façon que pour la PPB, les flux de la RE sont important si on les comparent à des tourbières boréales et s'approchent davantage des flux mesurés dans les prairies sur sols tourbeux.
La Re sur la tourbière de La Guette, comprise entre 1232 et \SI{1337}{\gcma} est plus importante que celle observée par \cite{peichl2014,trudeau2014} (pour reprendre les études citées précédemment) qui s'établissent respectivement entre 137 et \SI{443}{\gcma} et 206 et \SI{234}{\gcma}.
Elles sont en revanche plus faible que celle mesurées par \cite{jacobs2007},entre 500 et \SI{2000}{\gcma}, ou par \cite{gorres2014} : entre 2070 et \SI{3500}{\gcma}. 

%jacobs 2007 RE entre 0 et \SI{13}{\uml} 
%aurela 2007 RE inf \SI{3}{\uml}
%
%% Flux modélisés
%jacobs 2007  sols tourbeux 1520 $\pm$ \SI{30}{\gcma} (moyenne)
%
%jacobs 2007  sols tourbeux entre 500 et \SI{2000}{\gcma}
%
%peichl2014 entre 137 et \SI{443}{\gcma} 1.2 523 mm
%
%trudeau 2014 234-\SI{206}{\gcma} -4.28 ° 738 mm

% % Apport de la vég
À l'inverse de la PPB, l'intégration de la végétation pour modéliser la RE n'améliore que peu l'estimation de la RE lors de la phase de calibration : la différence entre les valeurs d'erreur standard est de \SI{5}{\percent}.
En revanche lors de la phase d'évaluation, l'utilisation du recouvrement des herbacées semble améliorer l'estimation de façon plus importante avec une différence de \SI{11}{\percent} entre les valeurs d'erreur standard.
Contrairement à la PPB la différence apportée par l'intégration de la végétation (RE-2 ou RE-3) est du même ordre de grandeur en 2013 et en 2014.
Sur les 2 années, l'effet de l'intégration de la végétation est limité avec une différence de \SI{25}{\gcma} au maximum (entre ER-1 et ER-3), soit moins de \SI{2}{\percent} du flux.


% % Incertitude
Les incertitudes sur le flux de la RE sont beaucoup moins importante que celle de la PPB.
Que ce soit au niveau de l'estimation des paramètres des modèles, à l'exception du paramètre c du modèles RE-2, leur p-value est inférieure à 0.05.
Ou au niveau de l'erreur calculée lors de la calibration, inférieure à \SI{15}{\percent}, mais également lors de leur évaluation sur des données indépendantes ou elle vaut moins de \SI{25}{\percent} (valeurs du modèles RE-3).

% % // biblio

\subsubsection{\chh}

% Flux mesurés // biblio

% Incertitudes

\subsubsection{COD}

\subsection{Estimations des bilans}

% Variations observées des bilans et importance relative des flux

La variation de bilan observée selon les équations utilisées sont du même ordre de grandeur que celle obtenues par \cite{worrall2009} quand il compare différentes méthode de calcul de bilan.


% Valeurs comparées à la biblio

En terme de bilan la tourbière étudiée par \cite{bortoluzzi2006} est un puits de carbone en 2004 et en 2005 entre 67 et \SI{183}{\gcma} selon les années et la dominance végétale (Eriophorum vs Sphagnum).
Hypothèse : cette différence peut s'expliquer par la différence de végétation, la tourbière de La Guette est dominée par la Molinie, et surtout la différence en terme de température moyenne annuelle (\SI{6.6}{\degreeCelsius} contre \SI{10.5}{\degreeCelsius} pour La Guette sur les 2 années de mesure).

% Incertitude/Sensibilité


\subsection{Représentativité du modèle à l'échelle de l'écosystème}

%\textbf{Valeur absolue des flux}
%
%Les flux de \coo mesurés sont du même ordre de grandeur que ceux mesurés par \cite{bortoluzzi2006} sur une tourbière jurassienne de montagne quoique systématiquement légèrement supérieurs.
%Les flux de \chh quand à ceux sont un ordre de grandeur en dessous de ceux mesuré à La Guette (\SI{0.03}{\uml} au maximum contre plus de \SI{0.2}{\uml}) (\textbf{comparé moyenne plutôt}).
%Les flux sont moins fort que \cite{gorres2014}, mais plus fort que \cite{lund2015}
%Les flux sont du même ordre de grandeur que \cite{gazovic2013} (\textbf{à détailler})
%
%Il faut noter que c'est le paramètre b de l'équation~\ref{eq:RE_TH} qui fait varier le bilan le plus sensiblement lorsqu'on lui applique une variation de $\pm$\SI{10}{\percent}.
%Cependant la significativité de ce paramètre est forte (\textbf{et donc ?}).
%
%
%\textbf{Différence entre 2013 et 2014}
%
%\textbf{apport d'un indice de végétation}





\subsection{Sensibilité et limitations du bilan}

Les incertitudes les plus fortes du bilan sont sur les flux de \chh avec une erreur standard de \SI{32}{\percent} lors de la calibration et de \SI{68}{\percent} lors de la validation.
Cette différence importante montre que l'estimation des flux de \chh à l'aide de l'indice de végétation à permis l'estimation de sa contribution au bilan de carbone de l'écosystème, mais que sont utilisation dans d'autre conditions est fortement limité.
L'importance faible du \chh dans le bilan de carbone de la tourbière rend ces erreurs moins critique que celles faite lors de l'estimation de la PPB.
Les incertitudes importantes sur la PPB, sont mise en évidence par les variations fortes des flux interpolés selon l'équation utilisée.
Elles sont la source des variations observées en terme de bilan.
L'ajout d'un indice de végétation diminue d'incertitude des paramètre du modèle, mais cet apport semble spécifique à l'étude, n'étant pas reflété par l'évaluation.
À l'inverse la RE semble relativement bien contrainte, sur les 2 années la différence entre les différentes équations utilisées ne dépassent pas \SI{25}{\gcma}.

\textbf{sensibilité du bilan au variation des paramètres}

\textbf{limitations}
Outre ces aspects ce bilan de carbone est aussi limité par sa représentativité. 
Ainsi la strate arborée fortement présente dans certaines zones n'est pas directement prise en compte.
De la même manière une partie restreinte de la tourbière mais néanmoins présente est constitué de touradons dont l'effet n'a pas été pris en compte.\plop (\textbf{biblio effet microtype}).


\begin{itemize}
\item pas de cartographie (pas grave si p7 maj bien représentée par mdl ecos)
\item extrapolation sur d'autres site difficile (cf validation)
\end{itemize}

\subsection{Représentativité locale du modèle}

Distribution des paramètres

Pourquoi certaines placette mieux que d'autres


\subsection{perspectives}

cartographie ?

 %Bilan de C de la tourbière de La Guette
\singlespacing
\chapter{Effets de l'hydrologie sur les flux de GES -- approche expérimentale}
\label{ch:4}

\minitoc

\newpage
\doublespacing

\section{Introduction}

L'hydrologie est reconnue comme un facteur contrôlant les flux de GES \citep{blodau2002}.
En effet de nombreuses études ont relié les émissions de \coo au niveau de la nappe d'eau (Tableau~\ref{table:bib_wtl}).
La majorité d'entre elles montrent qu'une tourbière dont le niveau de la nappe est abaissé, soit par un drainage, soit par une sécheresse, a un ENE plus faible.
Cependant, aucun consensus n'a encore été atteint concernant les origines de ces baisses de l'ENE.
\citet{strack2013} expliquent ainsi le fonctionnement en source de carbone d'une tourbière canadienne par des conditions plus chaudes et plus sèches que les moyennes observées à plus long terme sur le site.
Une observation similaire est faite par \citet{aurela2007} qui mesurent un ENE plus faible lors d'une année sèche, dans une tourbière à Carex au sud de la Finlande.
Ils attribuent également cette baisse de l'ENE aux conditions plus chaudes et plus sèches, qui permettent le développement d'une zone aérobie plus importante et favorise ainsi une RE plus élevée.
Lors d'un suivi de douze années sur une tourbière Suédoise, \citet{peichl2014} observent également une baisse de l'ENE lors d'une année ou le niveau de la nappe baisse de façon importante, en dessous de \SI{-30}{\centi\metre} de profondeur.

Ils expliquent cette baisse par une baisse de la PPB.
Cette observation va dans le même sens que celles de \citet{lund2012} sur un suivi de quatre années (2006--2009) dans une tourbière à sphaignes située au sud de la Suède. 
Dans cette étude, ils observent deux années de sécheresse, 2006 et 2008, pour lesquelles l'ENE est plus faible que la moyenne.
En 2006 ils observent également des valeurs de RE plus importantes que les autres années, ce qui explique l'ENE faible observée.
En revanche en 2008, ce n'est pas par la RE qu'ils expliquent les valeurs de l'ENE, mais par la PPB qui est plus faible cette année là.
Dans les deux cas la baisse du niveau de l'eau conduit à une baisse de l'ENE, cependant cette baisse a des origines différentes.
Les auteurs expliquent ces différences par le type de sécheresse : courte et intense pendant la saison de végétation de 2006 et d'intensité plus faible mais d'une durée plus longue en 2008.
À l'inverse des résultats précédemment cités, \citet{ballantyne2014} dans une étude sur les effets à long terme d'une baisse du niveau de la nappe, n'observent pas d'effets significatifs sur l'ENE tandis que les flux de RE et de PPB augmentent tous les deux.
Ces études montrent que si le niveau de la nappe est reconnu comme un facteur de contrôle des flux de \coo, il est difficile d'en dégager des liens de cause à effet répétables.

\begin{table}
\centering
\caption{Effet d'une baisse du niveau de la nappe d'eau (asséchement) dans les tourbières sur les flux de \coo. Les flèches rouges montantes décrivent une augmentation du flux et les flèches bleues une diminution.}
\label{table:bib_wtl}
\begin{tabular}{llll}\toprule
Référence & ENE & RE & PPB \\ \midrule
\citealp{strack2013}& \decarrow & \incarrow & \decarrow \\ 
\citealp{aurela2007} & \decarrow & \incarrow & NA \\ 
\citealp{peichl2014} & \decarrow & $\rightarrow$ & \decarrow \\ 
\citealp{lund2012} & \decarrow & \incarrow & $\rightarrow$ \\
\citealp{lund2012} & \decarrow & $\rightarrow$ & \decarrow \\
\citealp{ballantyne2014} & $\rightarrow$ & \incarrow & \incarrow \\ 

\bottomrule
\end{tabular}
\end{table}


Concernant le \chh, un niveau de nappe d'eau haut est généralement associé à des émissions importantes et un niveau de nappe bas à des émissions faibles.
Ceci est lié au fait que le niveau de la nappe d'eau contrôle l'épaisseur de la zone où le \chh est produit ainsi que celle ou il est oxydé \citep{pelletier2007}.
\citet{turetsky2008} montrent que l'effet des variations du niveau de nappe sur les flux de \chh n'est pas répétable.
Ils observent également que l'effet sur les flux de \chh est plus important lorsque le niveau de la nappe est augmenté que lorsqu'il est diminué ($\pm$ \SI{10}{\centi\metre}).
Pour expliquer cette observation, ils font l'hypothèse que, lorsque le niveau de la nappe d'eau est plus élevé, le transfert de chaleur dans le sol est plus rapide et permet de maintenir des températures plus élevées qui favorisent la production de \chh.
Cependant d'autres études, principalement dans des sites où le niveau de la nappe est proche de la surface du sol, montrent une absence de relation entre le niveau de la nappe et les émissions de \chh, voire une relation inverse, avec des flux plus faibles liés à des niveaux de nappe plus élevés \citep{kettunen1996,bellisario1999,treat2007}.
Pour expliquer ces observations, l'hypothèse avancée est que le \chh est piégé dans une porosité du sol fermée par la saturation importante en eau.
Là encore selon les conditions environnementales, la relation entre les flux de \chh et le niveau de la nappe n'est pas aisément généralisable.

Lors d'expérimentations consistant à manipuler le niveau de la nappe d'eau, la vitesse et/ou la manière de simuler une remontée du niveau de l'eau peut également influencer la réponse des flux de GES.
\citet{strack2009} ont ainsi observé en suivant les flux de \coo sur des mésocomes de tourbe, qu'une réhumectation graduelle alimentée par le bas de la colonne de sol conduisait à une baisse de la RE, alors qu'une hausse rapide par le haut de la colonne (simulant un événement pluvieux) conduisait à un pic de RE.
Une observation similaire d'augmentation importante de la RE après réhumectation a également été observée par  \citet{mcneil2003}.

Au cours des deux années de suivi des flux de \coo et de \chh dans la tourbière de La Guette (2013 et 2014), le niveau de la nappe d'eau est resté relativement élevé et a très faiblement varié en comparaison avec les années précédentes bien plus sèches (Figure~\ref{fig:WTL}).
En conséquence, l'effet des variations de nappe d'eau sur les flux de GES n'a pu être investigué.
L'effet de cycles de dessiccation/réhumectation sur les émissions de \coo et de \chh est cependant certain et on peut faire l'hypothèse qu'une baisse du niveau de la nappe entraînerait une augmentation des flux de RE, et possiblement une diminution des flux de \chh.
On peut également attendre un pic d'émission de la RE au moment de la réhumectation.

L'objectif de ce chapitre est donc de déterminer les effets de variations du niveau de la nappe sur les flux de GES à travers deux expérimentations simulant des phases de dessiccation et de réhumectation d'un sol tourbeux.

\section{Procédure expérimentale}

\begin{figure}
\centering
\includegraphics[width=\textwidth]{chap4/mesocosmes}\\
\includegraphics[width=\textwidth]{chap4/zi_exp_low}
\caption{Prélèvement des mésocosmes sur la tourbière de La Guette (en haut). Mésocosmes installés près du laboratoire : 6 témoins et 6 traités, avec des dispositifs pour intercepter la pluie (en bas).}
\label{fig:mesophoto}
\end{figure}

\begin{table}
\centering
\caption{Récapitulatif des différentes phases de dessiccation/réhumectation pour les deux expérimentations. La colonne code phase correspond à la première lettre de la phase (D pour dessiccation et R pour réhumectation) suivi d'un numéro représentant l'ordre du cycle. La phase EQ correspond au temps laissé aux mésocosmes pour s'équilibrer avec leur nouvel environnement.}
\label{table:recap_DR}
\begin{tabular}{llll}\toprule
 & Code phase & Dates & Campagnes \\ \midrule
\multicolumn{4}{l}{Expérimentation I (2013)} \\
 & EQ & 12 avril -- 31 mai & 1 \\ [+.5ex]
 & D1 & 1 juin -- 16 juillet & 2 à 15 \\
 & R1 & 17 -- 20 juillet & 16 à 19 \\
 & D2 & 21 -- 9 août & 20 à 24 \\ [+1ex]
\multicolumn{4}{l}{Expérimentation II (2014)} \\
 & EQ & 17 avril -- 29 juin & 1 à 3 \\ [+.5ex]
 & D1 & 30 juin -- 6 juillet & 4 à 5 \\
 & R1 & 7 -- 16 juillet & 6 à 10 \\ [+.5ex]
 & D2 & 17 -- 28 juillet & 11 à 14\\
 & R2 & 29 juillet -- 3 août & 15 à 17\\[+.5ex]
 & D3 & 4 -- 11 août & 18 à 19\\
 & R3 & 12 -- 14 août & 20 à 21\\
\bottomrule
\end{tabular}
\end{table}


L'étude des cycles de dessiccation/réhumectation est effectuée sur des mésocosmes cylindriques (\SI{30}{\centi\metre} de diamètre et de profondeur) dont la végétation est composée exclusivement de sphaignes.
Ces derniers ont été prélevés dans la tourbière de La Guette et installés en extérieur, dans des trous creusés dans le sol.
Au contraire d'échantillons en chambre climatique, cette méthode a l'inconvénient de ne pas permettre un contrôle total des variables expérimentales comme les apports d'eau ou la température.
Cependant, elle permet de maintenir les échantillons dans des conditions plus proches de celles présentes in-situ et notamment le rayonnement solaire, dont la luminosité est inatteignable en chambre climatique.
Deux expérimentations ont été réalisées, la première (expérimentation I) durant l'été 2013 avec un seul cycle long.
Cette expérimentation a été effectuée dans le cadre des stages de Master de Zi Yin de l'Université de Fudan en Chine, qui s'est occupée d'une grande partie de l'acquisition de données de \coo et des variables environnementales et de Paul Gaudry de l'Université d'Orléans qui s'est occupé de faire les mesures de \chh.
La seconde (expérimentation II) a été réalisée pendant l'été 2014 avec trois cycles, plus courts et a été effectuée dans le cadre des stages de Master de Tianyi Ji, de l'Université de Fudan en Chine qui s'est occupé de l'acquisition des données \coo, et de Sarah Williams qui a réalisé les mesures de \chh. 

Pour les deux expérimentations, les flux de \coo et de \chh ont donc été suivis ainsi que la température de l'air, du sol (à \SI{-5}{\centi\metre}), le niveau de nappe d'eau, et la teneur en eau du sol pendant les différentes phases de dessiccation et de réhumectation. 

\begin{center}
\begin{minipage}{.85\textwidth}
\setlength{\parindent}{-10pt}%

\onehalfspacing
\textbf{Remarque :} 
Pour l'expérimentation I les mesures ont été faites en insérant verticalement la sonde de teneur en eau d'une dizaine de centimètres dans le mésocosme.
La mesure est donc une intégration sur \SI{10}{\centi\metre}.
En revanche pour l'expérimentation II, la sonde a été insérée horizontalement sur un côté des mésocosmes à une profondeur fixe (\num{-5}, \num{-10} et \SI{-20}{\centi\metre}).
La mesure qui en résulte est donc plus spécifique à cette profondeur.
Pour les deux expérimentations les valeurs obtenues, ne sont pas à prendre de façon absolue, les sondes n'ayant pas été calibrées pour des sols tourbeux mais pour des sols minéraux.
\end{minipage}
\end{center}
Les placettes subissant les cycles de dessiccation seront nommées groupe « Dessiccation » et les placettes ne subissant pas les cycles, groupe « Contrôle ».
Pour le \coo et le \chh, l'analyse a été faite sur les flux moyennés sur une journée, les flux ayant été généralement mesurés deux fois par jour.

\subsection{Expérimentation I}
Six mésocosmes ont été prélevés le 12 avril 2013, dans la tourbière de La Guette.
Le prélèvement s'effectue à l'aide de cylindres de PVC, enfoncés délicatement dans la tourbe puis dégagés en creusant de chaque côté (Figure~\ref{fig:mesophoto}).
Enfin ils sont transportés au laboratoire où ils sont installés en extérieur et saturés en eau (eau prélevée dans la tourbière), afin que leurs conditions hydrologiques de départ soient les plus proches possibles.
Trois mésocosmes tirés au sort servent de contrôle, et trois vont subir un cycle de dessiccation/réhumectation.
Du 2 mai au 17 juillet 2013, les précipitations ont été interceptées dans trois mésocosmes à l'aide d'abris bâchés installés en cas de pluie et la nuit (Figure~\ref{fig:mesophoto}).
Au 17 juillet, de fortes précipitations sont simulées par l'ajout d'eau de pluie reconstituée\footnote{Cette eau est une eau créée artificiellement, à partir d'un mélange l'eau dé-ionisée, de sulfate de sodium, de nitrate d'ammonium, de chlorures de potassium, de calcium, de magnésium et de sodium pour reproduire la composition d'une eau de pluie.} dans les six mésocosmes (Tableau~\ref{table:recap_DR}).
La réhumectation s'est étalée sur quatre jours à raison d'un ajout de \SI{1.16}{\litre} d'eau par jour et par mésocosme reproduisant ainsi un événement pluvieux enregistré dans la tourbière de La Guette (\SI{81.8}{\milli\metre} sur cinq jours).

\subsection{Expérimentation II}
Le 17 avril 2014, six nouveaux mésocosmes ont été prélevés dans la tourbières de La Guette et installés près du laboratoire, en suivant le même protocole que pour l'expérimentation I.
Une station météo a été installée à côté des mésocosmes afin de mesurer avec un pas de 15 minutes la température de l'air, l'humidité relative, le rayonnement solaire, la vitesse et la direction du vent et les précipitations.
La pluviosité n'a pu être enregistrée à cause d'une panne du pluviomètre.
Cette station a permis également l'enregistrement des températures mesurées par les sondes T107 installées à \num{-5}, \num{-10}, et \SI{-20}{\centi\metre}.
Un abaissement manuel du niveau de la nappe a été mis en place pour cette expérimentation, dans le but de pouvoir suivre plusieurs cycles de dessiccation/réhumectation.
Pendant les phases d'assèchement les niveaux de nappes des placettes traitées étaient donc abaissés en moyenne de \SI{2}{\centi\metre} par jour, une intensité permettant de simuler plusieurs cycles.
La durée des différents cycles est présentée dans le tableau~\ref{table:recap_DR}.
Pendant les phases de réhumectation, de l'eau de pluie collectée à proximité des mésocosmes, est versée dans les mésocosmes jusqu'à ce que le niveau d'eau atteigne la limite haute de l'embase. 

\section{Résultats}

\subsection{Expérimentation I}

\subsubsection{Dynamique hydrologique}

Pendant la phase de dessiccation on observe une baisse du niveau de la nappe dans les placettes contrôles et dans les placettes traitements (Figure~\ref{fig:HMzi}--A, campagnes 2 à 15).
Cependant si les placettes du groupe « Dessiccation » ont un niveau de nappe qui diminue de façon régulière sur l'ensemble de cette phase, de \num{-3} à \SI{-25}{\centi\metre} ce n'est pas le cas des placettes du groupe « Contrôle ».
Ces dernières ont un niveau de la nappe d'eau qui reste à peu près constant ($\approx$ \SI{-3}{\centi\metre}) entre les campagnes 4 et 8, du fait d'épisodes pluvieux pendant cette période.
Puis le niveau de nappe diminue entre les campagnes 9 et 15, passant de \num{-7} à \SI{-22}{\centi\metre}.
Pendant la phase de réhumectation, les deux groupes ont un comportement similaire.
Leurs niveaux de nappe augmentent de \num{-22} à \SI{-1}{\centi\metre} pour le groupe « Contrôle » et de \num{-25} à \SI{-1}{\centi\metre} pour le groupe « Dessiccation ».
Dans la seconde phase d'assèchement le niveau de nappe baisse à nouveau pour les deux groupes, de façon régulière pour le groupe « Dessiccation » jusqu'à atteindre une profondeur de \SI{-30}{\centi\metre}, et de façon plus irrégulière à cause des pluies, pour le groupe « Contrôle ».

Cette dynamique d'assèchement est également visible à travers l'évolution de la teneur en eau du sol (Figure~\ref{fig:HMzi_T}--A).
Pour le groupe « Contrôle », la teneur en eau se maintient à \SI{100}{\percent} jusqu'à la campagne 5 puis elle diminue jusqu'à la campagne 15 où elle atteint \SI{43}{\percent}.
La teneur en eau du sol du groupe « Dessiccation » diminue dès la campagne 2 et atteint \SI{41}{\percent} à la fin de la phase de dessiccation (campagne 15).
À ce moment les deux groupes sont relativement proches.
Ils le restent lors de la phase de réhumectation pendant laquelle la teneur en eau du sol augmente.
Cette dernière augmente même pendant la 2\ieme phase de dessiccation, jusqu'à la campagne 22 pour le groupe « Contrôle » et 20 pour le groupe « Dessiccation », où elle atteint 100 et \SI{86}{\percent} respectivement.

La réponse hydrologique au cycle de dessiccation/réhumectation est différente selon qu'on l'observe à travers le niveau de la nappe ou la teneur en eau du sol (Figure~\ref{fig:wtlSWC_A}).
Pendant la dessiccation du groupe « Contrôle » le niveau de nappe reste, dans un premier temps, constant jusqu'à la campagne n°8 puis il diminue. 
Pendant la phase de réhumectation, ce même groupe suit un « chemin » identique, le niveau de nappe commence par augmenter avec une variation limitée de la teneur en eau du sol jusqu'à la campagne n°18, puis par la suite, la teneur en eau du sol augmente tandis que le niveau de nappe reste plus constant voire diminue.
Pour le groupe « Dessiccation », on observe une diminution conjointe du niveau de nappe et de la teneur en eau lors de la phase de dessiccation.
Cette relation n'est cependant pas strictement linéaire avec une teneur en eau qui varie peu pendant les trois premières campagnes, puis qui diminue jusqu'à la campagne n°8, avant de diminuer de manière moins importante jusqu'à la fin de la phase de dessiccation.
Le niveau de nappe du groupe « Dessiccation » diminue de façon régulière pendant cette phase.
À l'inverse du groupe « Contrôle » la phase de réhumectation du groupe « Dessiccation », ne suit pas le même chemin que lors de la dessiccation.
Pendant la réhumectation le chemin est très proche de celui observé pour le groupe « Contrôle » avec un niveau de nappe qui commence par augmenter, avant de se stabiliser et, pendant cette stabilisation, une augmentation de la teneur en eau du sol.
Au delà de la campagne n°20 le comportement des groupes diverge à nouveau.
Le groupe « Contrôle » semble reprendre le même chemin de dessiccation à l'exception d'un point.
Ce point, (campagne n°23) et liée à une baisse brusque du niveau de la nappe (\SI{-18}{\centi\metre}) et semble davantage sur le « chemin » du groupe « Dessiccation ».
Le groupe dessiccation quant à lui suit un chemin proche de sa première phase de dessiccation même si la teneur en eau du sol diminue moins rapidement par rapport au niveau de la nappe que précédemment.

\begin{figure}
\centering
\includegraphics[width=1.15\textwidth, center]{chap4/expA_flux}
\caption{Expérimentation I : Évolution de la moyenne journalière du niveau de nappe en cm (A), et des flux, \chh, RE, PPB, ENE en \si{\uml}, B, C, D, E de juin à août 2013, dans les placettes « Contrôle » et « Dessiccation ». Les phases de dessiccation (D) sont représentées en rouge et la phase de réhumectation (R), en bleu. Les numéros de 1 à 24 correspondent aux dates des campagnes.}
\label{fig:HMzi}
\end{figure}

\begin{figure}
\centering
\includegraphics[width=1.15\textwidth, center]{chap4/expA_SWC_T}
\caption{Expérimentation II : Évolution de la teneur en eau du sol à \SI{-5}{\centi\metre} (A), de la température de l'air (B), et de la température du sol à \SI{-5}{\centi\metre} (C) de juin à août 2013, dans les placettes « Contrôle » et « Dessiccation ». Les phases de dessiccation (D) sont représentées en rouge et la phase de réhumectation (R), en bleu. Les numéros de 1 à 24 correspondent aux dates des campagnes.}
\label{fig:HMzi_T}
\end{figure}


\begin{figure}
\centering
\includegraphics[width=1.15\textwidth, center]{chap4/expA_wtlSWC}
\caption{Relation entre les niveaux de nappe et la teneur en eau du sol lors de l'expérimentation I. Les numéros correspondent à l'ordre des campagnes de mesures et les lignes grises aux déviations standards.}
\label{fig:wtlSWC_A}
\end{figure}

\subsubsection{Les flux de \chh}

Les émissions de \chh varient dans l'ensemble de 0 et \SI{0.55}{\uml}.
Elles sont similaires entre les deux groupes (« Contrôle » et « Dessiccation ») jusqu'à la campagne n°8 à partir de laquelle elles divergent (Figure~\ref{fig:HMzi}--B).
À partir de cette campagne, les émissions du groupe « Contrôle » augmentent rapidement pour atteindre \SI{0.55(031)}{\uml} tandis que celles du groupe « Dessiccation » restent stables, voire diminue légèrement.
À la fin de la phase de dessiccation, mi-juillet, les deux groupes retrouvent des niveaux d'émission similaires compris entre \num{0.1} et \SI{0.2}{\uml}.
Ces niveaux restent constants pendant toute la phase de réhumectation, avant d'augmenter légèrement pendant la deuxième phase de dessiccation pour se situer entre \SI{0.25}{\uml} et \SI{0.2}{\uml}.

\subsubsection{La RE}

Pendant la phase de dessiccation, les flux de la RE tendent à augmenter quel que soit le groupe de placettes considéré (Figure~\ref{fig:HMzi}--C).
Ces valeurs inférieures à \SI{2.5}{\uml} début juin, atteignent environ \SI{7}{\uml} pour les deux groupes mi-juillet, avant la réhumectation.
La RE du groupe « Dessiccation » est supérieure à celle du groupe « Contrôle » pendant une grande partie du mois de juin.
Cependant la RE du groupe « Dessiccation » augmente régulièrement pendant l'ensemble de cette phase jusqu'à \SI{3.26(046)}{\uml}, tandis que les valeurs du groupe « Contrôle » restent, dans un premier temps, stables jusque fin juin (\SI{2.45(075)}{\uml}).
À partir de début juillet, les valeurs de RE du groupe « Contrôle » augmentent jusqu'à dépasser les valeurs du groupe « Dessiccation ».
La Re de ce groupe atteint un maximum à \SI{7.93(152)}{\uml} le 8 juillet avant de retrouver des valeurs proches de celles observées dans le groupe « Dessiccation ».
Cette augmentation brusque correspond temporellement à celle observée, pour le même groupe, dans les flux de \chh.
Lors de la phase de réhumectation, les flux de RE diminuent de façon très similaire pour les deux groupes pour atteindre \SI{2.75}{\uml} lors de la campagne n°19.
Ce minimum reste cependant plus élevé que les valeurs mesurées initialement (\SI{0.7}{\uml}).
Après la phase de réhumectation, les flux des deux groupes restent relativement proches et augmentent à mesure que le niveau de la nappe diminue à nouveau (Figure~\ref{fig:HMzi}--A).

\subsubsection{La PPB}

Pour les deux groupes, les flux de PPB restent stables pendant la phase de dessiccation (Figure~\ref{fig:HMzi}--D) :
entre 5 et \SI{6}{\uml} (\SI{5.29(076)}{\uml} de moyenne pour les deux groupes) jusqu'au 24 juin.
Ensuite, comme pour le \chh et la RE, les valeurs de la PPB du groupe « Contrôle » augmentent et s'écartent de celles mesurées dans le groupe « Dessiccation ».
À la fin de cette phase de dessiccation, les flux redeviennent identiques entre les traitements et atteignent un minimum proche de \SI{3}{\uml}.
Pendant la phase de réhumectation, la PPB augmente légèrement pour les deux groupes.
La PPB dans le groupe de contrôle a des valeurs supérieures à celles du groupe « Dessiccation ».
Pendant la deuxième phase de dessiccation, la PPB augmente pour les deux groupes, avec un maximum de \SI{5.83(161)}{\uml} pour le groupe « Dessiccation » et de \SI{10.17(330)}{\uml} pour le groupe « Contrôle ».

\subsubsection{L'ENE}

Pour l'ensemble de l'expérimentation, les flux d'ENE varient de la même façon et sont plus élevés dans le groupe « Contrôle » que ceux du groupe « Dessiccation » (Figure~\ref{fig:HMzi}--E).
Pendant la phase de dessiccation, l'ENE reste relativement constante jusque fin juin (campagne n°10) avec une valeur moyenne pour les deux groupes de \SI{1.18(058)}{\uml}.
Au delà du 30 juin (campagne n°10), l'ENE baisse pour les deux groupes pour atteindre un minimum proche de \SI{-4.5}{\uml} (campagne n°15).
Pendant la phase de réhumectation, l'ENE augmente rapidement pour atteindre \num{1.52(036)} et \SI{2.15(147)}{\uml} pour le groupe « Contrôle » et de groupe « Dessiccation » respectivement (campagne n°19).
Après la réhumectation, l'ENE du groupe « Contrôle » varie en suivant généralement les variations du niveau de nappe.
Pour le groupe « Dessiccation », l'ENE baisse par rapport au maximum atteint lors de la réhumectation puis se stabilise autour de 0.

\subsubsection{Météorologie}

Pendant la première phase de dessiccation (mois de juin), les températures de l'air restent plus ou moins stables autour d'une valeur de \SI{26}{\degreeCelsius} jusqu'à la campagne n°9, puis elles augmentent jusqu'à la fin de la phase de dessiccation où elles atteignent \SI{42}{\degreeCelsius} (Figure~\ref{fig:HMzi}--B).
Les températures de l'air diminuent pendant la réhumectation puis restent stables avec des valeurs proches de \SI{22}{\degreeCelsius}.
Les températures du sol à \SI{-5}{\centi\metre} de profondeur suivent les même tendances que la température de l'air, à l'exception d'une baisse moins prononcée suite à la réhumectation (Figure~\ref{fig:HMzi}--C).

\subsubsection{Synthèse des résultats de l'expérimentation I}

Les variations de la RE sont principalement liées aux variations du niveau de la nappe (Figure~\ref{fig:hm_wtl}--C).
Par conséquent, les variations de RE se répercutent sur l'ENE (Figure~\ref{fig:hm_wtl}--G).
L'effet des variations du niveau de la nappe sur la PPB est quasiment nul (Figure~\ref{fig:hm_wtl}--E), même si la PPB semble diminuer aux plus fortes profondeurs.
Pour le \chh il est difficile de distinguer des tendances générales entre les flux et les niveaux de nappe (Figure~\ref{fig:hm_wtl}--A).

\subsection{Expérimentation II}

Cette expérimentation est basée sur le suivi de trois phases de dessiccation chacune suivie d'une phase de réhumectation.

\subsubsection{Dynamique hydrologique}

\begin{figure}
\centering
\includegraphics[width=1.15\textwidth, center]{chap4/expB_wtlSWC}
\caption{Relation entre les niveaux de nappe et la teneur en eau du sol lors de l'expérimentation II. Les numéros correspondent à l'ordre des campagnes de mesures et les lignes grises aux déviations standards.}
\label{fig:wtlSWC_B}
\end{figure}

Contrairement à l'expérimentation I, le niveau de nappe du groupe « Contrôle » de l'expérimentation II reste relativement constant pendant l'ensemble de la période de mesures (Figure~\ref{fig:HMty}--A).
Le drainage artificiel du groupe « Dessiccation » conduit à une diminution du niveau de la nappe d'une quinzaine de centimètres en moyenne pour chaque cycle et un temps pluvieux permet au groupe « Contrôle » de garder un niveau de nappe constant et élevé, supérieur à \SI{-5}{\centi\metre} la plupart du temps.
Ce dernier n'a baissé dans les « Contrôle », avec la teneur en eau du sol, que lors des campagnes 2 et 3 où il atteint son point le plus bas à \SI{-8}{\centi\metre}.
Les niveaux de nappe minimum des différents cycles sont \num{-15}, \num{-19} et \SI{-13}{\centi\metre} respectivement pour D1, D2 et D3.

La teneur en eau du sol à \SI{-5}{\centi\metre} est constante, à \SI{100}{\percent} pour le groupe « Contrôle », à l'exception des campagnes n°2 et 3 ou elle baisse et atteint \SI{93}{\percent} (Figure~\ref{fig:HMty_T}--A).
Pour le groupe « Dessiccation », la teneur en eau du sol à \SI{-5}{\centi\metre} est proche de \SI{20}{\percent} pendant les phases de dessiccation et vaut \SI{100}{\percent} pendant les phases de réhumectation.
Les teneurs en eau mesurées à \num{-10} et \SI{-20}{\centi\metre} valent \SI{100}{\percent} pour l'ensemble de l'expérimentation.

Lors de cette expérimentation et compte tenu de la durée de chaque cycle, le nombre de points par cycle est moins important que pour l'expérimentation I.
Il est donc difficile de voir si le comportement et les relations teneur en eau de sol/niveau de nappe varient selon les phases d'un même cycle et entre les cycles (Figure~\ref{fig:wtlSWC_B}).

\begin{figure}
\centering
\includegraphics[width=1.15\textwidth, center]{chap4/expB_flux}
\caption{Expérimentation II : Moyenne journalière du niveau de nappe en cm (A), et des flux, \chh, RE, PPB, ENE en \si{\uml}, B, C, D, E. Les cadres et bandes colorées correspondent aux phases de dessiccation (D) en rouge et aux phases de réhumectation (R) en bleu. Les numéros présents sur le graphe A correspondent aux numéros des campagnes.}
\label{fig:HMty}
\end{figure}

\begin{figure}
\centering
\includegraphics[width=1.15\textwidth, center]{chap4/expB_SWC_T}
\caption{Expérimentation II : Évolution de la teneur en eau du sol à \SI{-5}{\centi\metre} (A), de la température de l'air (B), et des températures du sol à \num{-5}, \num{-10}, \SI{-20}{\centi\metre} (C, D, E). Les cadres et bandes colorées correspondent aux phases de dessiccation (D) en rouge et aux phases de réhumectation (R) en bleu. Les numéros présents sur le graphe A correspondent aux numéros des campagnes.}
\label{fig:HMty_T}
\end{figure}


\subsubsection{Les flux de \chh}

Les flux moyens de \chh varient entre \num{0.07} à \SI{0.34}{\uml} (Figure~\ref{fig:HMty}--B).
Dans l'ensemble, les flux du groupe « Contrôle », à l'exception de la première mesure, sont supérieurs aux flux du groupe « Dessiccation » : moyennes globales de \num{0.20(006)} et \SI{0.11(005)}{\uml}, respectivement).
Les émissions du groupe « Contrôle » tendent à augmenter sur la période de mesure.
Une tendance similaire, est également visible pour le groupe « Dessiccation ».
Concernant les cycles de dessiccation/réhumectation, il est difficile de dégager des comportements communs entre eux.
Le passage de la phase R1 à D2 semble provoquer une émission importante de \chh  (Figure~\ref{fig:HMty}--B).
Cette émission se maintient pour le groupe « Contrôle » et ne dure pas pour le groupe « Dessiccation ».
Pour le goupe « Dessiccation » il semble également y avoir un pic de \chh à la fin de la phase D3.
La relation entre le \chh et le niveau de nappe n'est pas plus visible en rassemblant l'ensemble des données (Figure~\ref{fig:hm_wtl}--B).

\subsubsection{La RE}

La RE varie pour les deux groupes entre \num{0.42} et \SI{5.12}{\uml} (Figure~\ref{fig:HMty}--C)).
Avant le démarrage des manipulations du niveau de la nappe, les valeurs des deux groupes sont très proches et augmentent tandis que le niveau de nappe diminue.
Pendant les phases de dessiccation, les valeurs du groupe « Dessiccation » sont systématiquement supérieures, de \num{1.5} à \SI{1.8}{\uml} en moyenne par phase, par rapport à celles du groupe « Contrôle ».
À l'inverse pendant les phases de réhumectation, les flux entre les deux groupes sont beaucoup plus proches avec une tendance de la RE du groupe « Contrôle » à être supérieure à celle du groupe « Dessiccation ».
La RE du groupe traité est systématiquement plus faible pendant les phases de réhumectation que pendant les phases de dessiccation.
En moyenne la RE vaut respectivement \num{2.28(100)} et \SI{3.86(080)}{\uml} pour les groupes « Contrôle » et « Dessiccation » pendant les phases de dessiccation et \num{1.70(062)} et \SI{1.51(098)}{\uml} pendant les phases de réhumectation.

\subsubsection{La PPB}

Sur l'ensemble de la période de mesure, la PPB est comprise entre \num{2.78} et \SI{7.96}{\uml} (Figure~\ref{fig:HMty}--D).
Avant le début des traitements, les flux des deux groupes sont similaires.
À partir de la première phase de dessiccation, la PPB du groupe « Contrôle » est supérieure à celle du groupe « Dessiccation ».
Pour les deux groupes, la PPB est plus importante lors des phases de dessiccation comparée aux phase de réhumectation, avec des moyennes respectives de \num{6.35(219)} contre \num{5.80(220)} pour le groupe « Contrôle » et de \num{5.95(146)} contre \SI{4.05(160)}{\uml} pour le groupe « Dessiccation ».

\subsubsection{L'ENE}

Les valeurs d'ENE mesurées sont comprises entre \num{0.11} et \SI{5.42}{\uml}, et augmentent au cours du temps.
Passé la période pré-traitement pendant laquelle les flux de l'ENE sont similaires pour les deux groupes, l'ENE du groupe « Contrôle » est systématiquement supérieure à celle du groupe « Dessiccation » (Figure~\ref{fig:HMty}--E).
L'évolution des deux groupes reste cependant relativement conjointe pendant la période de mesure avec, pour le groupe « Dessiccation », une diminution récurrente de l'ENE au début de chaque phase de dessiccation.

\subsubsection{Météorologie}

L'évolution des températures de la tourbe pendant l'expérimentation ne semble pas être liée aux traitements effectués (Figure~\ref{fig:HMty_T}--B à E).
La température de l'air varie entre 8 et \SI{33}{\degreeCelsius} et a tendance à diminuer entre les campagnes n°5 et 9, puis elle augmente (campagne n°10), avant de se stabiliser avec une tendance à la baisse pendant le reste de l'expérimentation.
À partir de la phase R1 et pour D2, R2 et D3 on observe des températures du sol plus importantes pour le groupe  « Contrôle » que pour le groupe « Dessiccation » particulièrement à \num{-5} et \SI{-10}{\centi\metre} de profondeur.

\begin{figure}
\centering
\begin{minipage}{.5\textwidth}
\centering
\includegraphics[width=\linewidth]{chap4/expA_fluxWTL}
\end{minipage}%
\begin{minipage}{.5\textwidth}
\centering
\includegraphics[width=\linewidth]{chap4/expB_fluxWTL}
\end{minipage}%
\caption{Relations entre les flux de GES, \chh (A et B), la RE (C et D), la PPB (E et F) et l'ENE (G et H), et le niveau de la nappe.}
\label{fig:hm_wtl}
\end{figure}

\begin{figure}[!tbp]
\centering
\hspace*{-2cm}
\begin{minipage}[]{.55\textwidth}
\includegraphics[width=\textwidth]{chap4/expA_res}
\end{minipage}
\hspace*{.1cm}
\begin{minipage}[]{.55\textwidth}
\includegraphics[width=\textwidth]{chap4/expB_res}
\end{minipage}
\hspace*{-2cm}
\caption{Relation entre les résidus d'équation du type $Flux=a*exp(b*Temp\acute{e}rature) $ reliant les flux de RE (A et B) et de \chh (C et D) au niveau de la nappe. La température de l'air est utilisée pour la RE des deux expérimentations (A et B), la température de la tourbe à \SI{-10}{\centi\metre} est utilisée pour l'expérimentation I et celle de la tourbe à \SI{-5}{\centi\metre} pour l'expérimentation II.}
\label{fig:HM_res}
\end{figure}

\subsection{Comparaison des deux expérimentations}

Pour le \chh, que ce soit pour l'expérimentation I ou II, aucune tendance ne semble se dégager vis à vis du niveau de la nappe (Figure~\ref{fig:hm_wtl}--A et B).
Une relation inverse est observée, pour les deux expérimentations, entre la RE et le niveau de la nappe (Figure~\ref{fig:hm_wtl}--C et D).
La PPB ne montre aucune tendance quelle que soit l'expérimentation.
Aux niveaux de nappe supérieurs à \SI{-20}{\centi\metre} de profondeur, correspondent des valeurs de PPB parmis les plus basses (Figure~\ref{fig:hm_wtl}--E).
Pour les deux expérimentations, une relation est visible entre le niveau de la nappe et l'ENE qui diminue lorsque le niveau de nappe augmente (Figure~\ref{fig:hm_wtl}--G et H, expérimentation I : R\textsuperscript{2}=\num{0.52} ; p-value < \num{0.001} et  expérimentation II : R\textsuperscript{2}=\num{0.26} ; p-value < \num{0.001}).
Afin de séparer les effets de la température et ceux du niveau de la nappe, les résidus des équations reliant les flux à la température ont été calculés pour le \chh et la RE, qui semble contrôler en grande majorité les flux de \coo (Figure~\ref{fig:HM_res}).
La relation entre les résidus de la RE et le niveau de la nappe est moins claire une fois l'effet de la température retiré (Figure~\ref{fig:HM_res}, comparée à la Figure~\ref{fig:hm_wtl}--C).
Malgré tout, on peut observer une tendance à la hausse des résidus entre 0 et \SI{-18}{\centi\metre} pour les deux groupes de l'expérimentation I, puis une cassure, et à nouveau une tendance à la hausse pour le groupe « Dessiccation ».
Une tendance à augmenter des résidus de la RE quand le niveau de nappe diminue est également visible pour le groupe « Dessiccation » de l'expérimentation II (Figure~\ref{fig:HM_res}--B).
Cette hausse semble cependant s'amortir rapidement au delà de \SI{-10}{\centi\metre}.
Pour le \chh, aucune tendance entre les résidus de l'équation et le niveau de la nappe n'est visible pour l'expérimentation II (Figure~\ref{fig:HM_res}--D).
Pour l'expérimentation I, il est difficile d'observer une tendance claire même s'il semble y avoir un maximum des résidus liés au \chh autour de \SI{-10}{\centi\metre}(Figure~\ref{fig:HM_res}--C).

\section{Discussion}

\subsection{Comparaison des flux de carbone à ceux mesurés sur le terrain}
%CH4
\subsubsection{\chh}

Les flux moyens de \chh mesurés dans les mésocosmes des deux expérimentations sont parmi les valeurs hautes mesurées dans la tourbière de La Guette : certaines mesures dépassant le maximum de \SI{0.2}{\uml} que nous avons mesuré \textit{in-situ} en 2014.
Ces valeurs sont également dans la tranche haute des valeurs mesurées dans d'autres tourbières.
\citet{blodau2002}, dans un article de synthèse sur le cycle du carbone de plusieurs tourbières de l'hémisphère nord, montre que les flux de \chh varient généralement entre \num{0.004} et \SI{0.14}{\uml}.
Les valeurs mesurées restent cependant cohérentes avec celles observées par \citet{lai2014} dans une tourbière canadienne (\num{0}--\SI{0.56}{\uml}) ou par \citet{gogo2011} dans la tourbière de La Guette avec des flux compris entre \num{0.03} et \SI{0.4}{\uml} et mesurés en 2009.

\subsubsection{\coo}

Pour le \coo, les flux sont généralement dans la gamme des valeurs mesurées dans la tourbière de La Guette.
Pour l'expérimentation I, l'ENE moyen est plus faible que celui mesuré sur le terrain l'année 2013 : \num{0.81} contre \SI{2.85}{\uml}.
En revanche, pour l'expérimentation II, l'ENE moyen est de \SI{2.71}{\uml} ce qui est proche de celui mesuré sur le terrain : \SI{2.93}{\uml}.
Comme pour la RE, les flux de PPB sont du même ordre de grandeur que ceux mesurés sur le terrain, mais dans la gamme basse : les maximas moyens mesurés dans les mésocosmes sont d'environ \num{7.5} pour des valeurs de \SI{13}{\uml} mesurées dans la tourbière.
Ces valeurs restent cohérentes avec la littérature (e.g. \citealp{bortoluzzi2006a}).

\subsection{Effet des variations du niveau de la nappe sur les flux de gaz}

\subsubsection{\chh}
Les flux de \chh sont plus élevés pendant les phases de dessiccation que lors des phases de réhumectation.
Cette observation va à l'encontre de l'hypothèse qui stipule qu'une baisse du niveau de la nappe fait baisser les flux de \chh, en augmentant la zone propice à son oxydation et en diminuant la zone propice à sa production \citep{aerts1997,pelletier2007,turetsky2008}.
\citet{kettunen1996}, dans une étude \textit{in-situ}, rapportent eux aussi une relation inverse entre les flux de \chh et le niveau de la nappe.
Ils expliquent cette observation par le fait qu'une baisse du niveau de la nappe peut permettre une libération du méthane accumulé dans une porosité précédemment scellée par la saturation en eau.
Des observations similaires sont rapportées par \citet{bellisario1999}, sur une tourbière où le niveau de la nappe d'eau varie entre \num{-1} et \SI{-13}{\centi\metre}, et par \citet{treat2007} où le niveau de nappe varie entre \num{-9} et \SI{-30}{\centi\metre}.
Ces derniers expliquent également l'augmentation des flux de \chh, suite à une baisse du niveau de la nappe, par une diminution de la pression de l'eau qui libère du \chh auparavant bloqué dans une porosité isolée de l'atmosphère.
Le point commun de ces travaux est un niveau de nappe relativement élevé, majoritairement supérieur à \SI{-30}{\centi\metre}.
Un niveau de nappe élevé semble influencer les émissions de \chh plus par son action sur le transport de ce gaz que sur le rapport production/oxydation du \chh.
Autrement dit, dans cette gamme de variation du niveau de la nappe d'eau (0, \SI{-20}{\centi\metre}), les variations de flux de \chh observées seraient davantage liées à des effets de pression de l'eau, ouvrant ou fermant une partie de la porosité du sol et permettant ou empêchant le transport de \chh.

Cette hypothèse permet d'expliquer, pour l'expérimentation II, le pic de \chh observé lors du passage de R1 à D2 (Groupe « Dessiccation », Figure~\ref{fig:HMzi}).
La baisse d'émission de \chh observée entre D3 et R3 s'expliquerait alors par un blocage du transport lié à la réhumectation.

Le fait que les groupes « Dessiccation », quelle que soit la phase et l'expérimentation, aient des flux de \chh plus faibles que les groupes « Contrôle » peut s'expliquer par le fait que les micro-organismes méthanogènes soient peu perturbés par les dessiccations dans les groupes « Contrôle » par rapport aux groupes « Dessiccation ».
Ceci est en cohérence avec les études montrant un effet positif de la présence d'eau sur les flux de \chh.
La production de \chh des groupes « Contrôle » est donc plus forte que celles des groupes « Dessiccation ».
De plus, après le premier abaissement du niveau de la nappe, une partie de la communauté des méthanogènes est probablement non active ou a migré dans le bas de la colonne de tourbe.
La production des groupes « Dessiccation » est donc localisée plus bas que celle des groupes « Contrôle ».

Ceci semble cohérent avec les observations faites pendant l'expérimentation I.
En effet malgré une dessiccation du groupe « Contrôle » pendant le mois de juin (baisse de la teneur en eau du sol), ce dernier est beaucoup plus réactif que le groupe « Dessiccation » lors de l'augmentation de température qui a lieu à partir de début juillet.
On peut faire l'hypothèse que les événements pluvieux subis par le groupe « Contrôle » lui ont permis de maintenir une communauté active de méthanogènes plus longtemps.
Avant la réhumectation les deux groupes ont des flux de \chh similaires, ils semblent donc avoir atteint un niveau d'assèchement proche.
L'état de leurs méthanogènes respectifs devrait également être similaire.
Pendant la réhumectation les méthanogènes se réactivent, mais les flux sont bloqués par la saturation en eau, le \chh est émis avec un retard lorsque le niveau d'eau diminue.

Il ressort de ces deux expérimentations qu'un niveau de nappe élevé favorise, sur le long terme, les émissions de \chh, mais d'autres effets peuvent interférer localement et notamment le bloquage ponctuel du transport de \chh par un niveau de nappe proche ou à la surface du sol.
Le \chh peut ensuite être émis lorsque le niveau de la nappe d'eau diminue.
Ces écarts temporels qui peuvent exister entre la production et l'émission du \chh rendent difficile d'établir une relation directe entre les flux de \chh et ses facteurs de contrôle, que ce soit la température ou le niveau de la nappe.

\subsubsection{\coo}

Dans les deux expérimentations, une baisse du niveau de la nappe conduit à une augmentation de la RE, ce qui est en accord avec la littérature, que ce soit des expérimentations en mésocosmes \citet{blodau2004,dinsmore2009} ou sur le terrain \citet{ballantyne2014}. 

Pour l'expérimentation I, cette augmentation de la RE conduit à une baisse de l'ENE pendant la première phase de dessiccation.
Pendant la phase de réhumectation les flux de RE diminuent.
Après la phase de dessiccation les flux de RE retrouvent la même intensité qu'avant la réhumectation.
Dans le même temps la PPB augmente et empêche l'ENE de décroître à nouveau.
La PPB du groupe « Contrôle » est supérieure à celle du groupe « Dessiccation ».
La dessiccation du groupe « Dessiccation » a davantage atteint la végétation que celle du groupe « Contrôle ».

Globalement, dans les deux expérimentations, l'ENE est plus faible dans les groupes « Dessiccation » que dans les groupes « Contrôle » ce qui est en accord avec la littérature (Tableau~\ref{table:bib_wtl})

\subsection{Effet des cycles hydrologiques multiples sur les flux de GES}

La multiplicité des cycles de l'expérimentation II semble montrer que la différence entre l'ENE observée dans les deux groupes, pendant les phases de réhumectation, tend à augmenter avec le temps.
Ce qui indiquerait une baisse de la résilience de l'écosystème après les événements de dessiccation.
Cependant davantage de points de mesures par cycle semble nécessaires pour établir un lien significatif entre fréquence des dessiccations et résilience du système.

\subsection{Conclusions}

Les deux expérimentations ont montré que malgré des dynamiques différentes, une baisse du niveau de la nappe d'eau conduisait à une augmentation de la RE et qu'un niveau de nappe haut favorisait les émissions de \chh.
Mais elles ont également mis en évidence que d'autres phénomènes interférent.
Notamment le blocage des flux par une saturation en eau élevée peut conduire à observer des émissions qui semblent contradictoires à celles que l'on attendrait (moins d'émissions de \chh quand le niveau de nappe est le plus élevé).
Au delà de la valeur absolue du niveau de la nappe, l'histoire et l'intensité de ses variations peut donc jouer un rôle important sur les flux de GES mesurés. %Effets de l'hydrologie sur les flux de CO2 et CH4
% CHAPITRE 5
\chapter{Effets de la végétation sur les flux}
\newpage

\section{Introduction}
\section{Mise en place d'un protocole}
\section{Impact des mesures de CO2 sur la végétation}
destruction... %Variation journalière de la respiration de l'écosystème
%% CHAPITRE 6
\chapter{Caractérisation de la variabilité spatio-temporelle des flux sur la tourbière de La Guette (Bilan de C)}
\newpage

\section{Introduction}
\section{Présentation du suivi}
Étudier la variabilité spatiale et temporelle des flux de carbone nécessite de mettre en place une observation régulière, un suivi adapté.
Qu'est ce qu'un suivi adapté ? Dans le cas des flux des gaz, on ne peut heureusement ou malheureusement pas suivre en permanence un site avec une fréquence importante (Qu'est ce qu'une fréquence importante).
Il faut donc choisir qu'elle(s) échelle(s) l'on souhaite étudier.
Lors de cette thèse nous avons choisi de mettre en place deux suivi différents et complémentaire.

Le premier est un suivi mensuel à l'échelle d'une tourbière. 
Plus précisément, les flux de \COO ont été mesurés, à minima, une fois par mois sur 20 points distribués sur l'étendue du site (La Tourbière de La Guette, Neuvy sur Barangeon, Cher, France).
À cause de contrainte technique et de leur importance relative \todo[inline]{Expliquer ici ou ailleurs que les flux de CH4 ne représente a priori que 5 \% du bilan de C sur une tourbière} les flux de \CHH ont été mesuré plus ponctuellement.
Ce suivi permet d'étudier la variabilité spatiale des flux au sein du site et cela sur une échelle temporelle de l'ordre de l'année qui permet d'avoir une vision sur les variations saisonnières.

Ce protocole d'observation seul ne nous permet cependant pas de rendre compte d'autres échelles de variabilité qui semble d'importance non négligeable. 
Tout d'abord la variabilité des flux journalière : 
Le système étant très lié à la photosynthèse les flux sont très dépendant de l’ensoleillement et donc des alternances jour/nuit
Ensuite qu'elle représentativité un site en particulier peut-il avoir par rapport à l'ensemble des sites existants ?

\subsection{Suivi des GES}
\subsection{Suivi des facteurs contrôlants}
Par ailleurs pour chaque embase le PAR a été mesuré, ainsi que la pression atmosphérique, la température de l'air et de la tourbe à différente profondeur (profils). Des mesures d'humidité ont également été effectué, 5 par embase afin de prendre en compte l'hétérogénéité présente.
\subsection{Suivi des flux liquides (DOC, POC)}
Mesures de DOC
Principe
Non Purgeable Organic Carbon
Pyrolyse oxydative
Ne mesure pas le CO2 dissous
L'acidification et le bullage permet d'expulser le CO2 dissous
Tout le carbone est ensuite pyrolysé le carbone est transformé en CO2 et mesuré
%% CHAPITRE 6
\chapter{Apport à la modélisation globale}
\newpage


\section{Introduction}
Enfin pour synthétiser l'ensemble de ces résultats, essayer d'en faire un tout cohérent et donc de comprendre notre système, nous l'avons modélisé en ayant pour objectif cet esprit de synthèse et de compréhension.
\section{Le modèle de Walter}
% Conclusions ------------------------------------------------------------------
% CONCLUSIONS
\chapter*{Synthèse et perspectives}
\markboth{Conclusions et perspectives}{}
\addcontentsline{toc}{chapter}{Conclusions et perspectives}
\newpage

L'étude des flux de carbone dans les écosystèmes tourbeux est complexe car assujetti à des facteurs de contrôle dont la prépondérance varie fortement selon l'échelle considérée et les conditions environnementales.

\section{Bilan du bilan (de C) ?}

Malgré tout les observations réalisées sur la tourbière de La Guette ont permis de mettre en évidence des flux de \coo particulièrement fort que ce soit pour la RE ou la PPB.
Cette force des flux de \coo est probablement liée à sa situation géographique locale et globale : une tourbière de plaine située à basse latitude et à ses problématiques de drainage et d'envahissement par une végétation vasculaire.
Ainsi la saisonnalité plus faible qu'en montagne permet aux flux de rester fort pendant une période de l'année plus importante.
Ces flux importants entraînent des variations forte en terme de bilan selon les méthodologies employées, il est cependant probable que la tourbière de La Guette fonctionne actuellement comme une source de carbone.

L'estimation du bilan à l'échelle saisonnière ne permet pas de reproduire les variations journalières, l'estimation du modèle pendant les 3 jours de mesures haute fréquence réalisés en 2013 est largement supérieure aux valeurs mesurées (Figure~\ref{fig:RE1_vs_JN})

La prise en compte de la végétation reste une difficulté importante, l'observation répétée nécessitant des mesures non destructives, souvent imprécises ou très coûteuses en temps.
Paradoxalement les zones de la tourbières fonctionnant en puits de carbone sont celle ou les herbacées sont dominantes.


\begin{figure}
\centering
\includegraphics[width=\textwidth]{conclusions/RE1_vs_JN}
\caption{Comparaison entre les valeurs estimées par le modèle RE-1 et les mesures faites à haute fréquence sur le site du 30 juillet au 2 août 2013}
\label{fig:RE1_vs_JN}
\end{figure}


\section{L'hydrologie}

L'effet de la restauration hydrologique de la tourbière de La Guette n'a pas pu être mis en évidence de part une pluviométrie forte et un niveau de nappe toujours important.
Les expérimentations

\subsection{Résilience de la tourbe par rapport aux 2 années sèches qui précèdent le BdC}
(lien chap 3 et 4)

%schéma conceptuel ? Modèles globaux (ORCHID, chloée)
%
%Flux fort
%
%sensibilité param forte
%
%Modèles multi annuel et prise en compte de la végétation
%
%Quid des variations journalières dans un bilan annuel ? (Figure~\ref{fig:RE1_vs_JN})


Les prendre en compte améliorerait-il les modèles

modèles globaux ?
\textbf{limitations des équations :}
Plus généralement, la majorité des tourbières sont sous la neige une partie de l'année, ce qui n'arrive que rarement sur la tourbière de La Guette et une partie possède également des zones d'eau libre, qui n'existent pas sur ce site.

modèles globaux et profondeur de tourbe


\section{Ouverture vers d'autre méthodes de mesures}
\begin{itemize}
\item chambre automatique (lien chap 5, et chap 3 ?)
\item tour eddy covariance (lien chap 5 et chap 3 ?)
\end{itemize}

\section{perspectives}

La suite du projet CARBIODIV permettra peut être de mettre en évidence l'effet de la restauration.

Un partenariat avec le LSCE commencé pendant ces travaux devra permettre de valoriser ces données à des échelles plus importante.
Des données on d'ors et déjà été envoyée à Chloé XX qui développe un code "tourbière" dans le modèle ORCHIDEE.

L'installation prochaine d'une tour eddy covariance sur le site permettra de comparer ce bilan à des mesures plus haute fréquence.


%%%%%%%%%%%%%%%%%%%%%%%%%%%%%%%%%%%%%%%%%%%%%%%%%%%%%%%%%%%%%%%%%%%%%%%%%%%%%%%%
% Backmatter
%%%%%%%%%%%%%%%%%%%%%%%%%%%%%%%%%%%%%%%%%%%%%%%%%%%%%%%%%%%%%%%%%%%%%%%%%%%%%%%%
\backmatter
\pagestyle{backmatter}					% backmatter page style

% Bibliographie ----------------------------------------------------------------
\singlespacing
\addcontentsline{toc}{chapter}{Références bibliographiques}
\bibliography{manuscript,manual}{}
\clearpage


% Index ------------------------------------------------------------------------
\phantomsection % Permet "d'accrocher" hyperref
\addcontentsline{toc}{chapter}{Index}
\printindex

% Annexes ----------------------------------------------------------------------
\appendix
%\setcounter{mtc}{15}
\chapter{Annexes}
%\addcontentsline{toc}{chapter}{Annexes}
\renewcommand{\thesection}{\Alph{section}}
% Annexe 1
% Photos

\section{Photos supplémentaires}
\label{sec:photos_veg}
%\minitoc

%\newpage

\begin{figure}[htbp]
    \centering
    \begin{subfigure}[b]{.8\textwidth}
        \centering \includegraphics[trim=2.5cm 5cm 2.5cm 5cm, clip=true, width=\textwidth]{chap2/drosera_c.jpg}
        \caption{drosera}\label{fig:dro}
    \end{subfigure}
    \caption{Végétation présente sur le site de La Guette, et suivie lors des campagnes de mesure.}\label{fig:veg}
\end{figure}

\section{protocole végétation}
\label{sec:protocole_veg}


Le suivi non-destructif d'une végétation n'est pas triviale et nécessite la mise en place de protocoles particuliers en fonction du type de végétation.
L'objectif est de pouvoir estimer une biomasse produite en impactant au minimum la végétation en place.
Pour l'ensemble des espèces végétales présentes dans les embases servant à la mesure des flux un recouvrement à été estimé, à l’œil.


\subsubsection{La strate arbustive}
Pour la strate arbustive des mesures de hauteur moyenne ont été effectuées, en mesurant depuis le niveau du sol, ou le toit des sphaignes, si elles étaient présentes, jusqu'au sommet de l'individu.
\begin{figure}
\includegraphics[width=.5\textwidth]{chap2/cal_tetra_eq}
\includegraphics[width=.5\textwidth]{chap2/cal_tetra_eq}
\caption{Calibration de la biomasse en fonction de la hauteur}
\label{fig:cal_arbu}
\end{figure}

\subsubsection{La strate herbacée}
Pour la strate herbacée, en 2013, 5 individus des deux espèces majoritaires (Eriophorum vaginatum ? augustifolium ?, Molinia Caerulea) ont été marqués afin de pourvoir les mesurer plusieurs fois au cours de la saison.
Cependant les difficultés à retrouver les individus marqués couplés à la mort d'un nombre important d'entre eux n'ont pas permis d'acquérir de résultats significatifs.
En conséquence en 2014 ces deux espèces ont fait l'objet de comptage exhaustif et de mesure de hauteur moyenne.


\begin{figure}
	\centering
	\begin{subfigure}[t]{0.5\textwidth}
		\centering
		\includegraphics[width=.8\textwidth, frame]{chap2/Cch_moli_A_1to4}
		\caption{image scannée}
	\end{subfigure}%
	\begin{subfigure}[t]{0.5\textwidth}
		\centering
		\includegraphics[width=.8\textwidth, frame]{chap2/Cch_moli_A_1to4_mod}
		\caption{image binarisée}
	\end{subfigure}
%    \caption{Caption place holder}
\caption{Scanne des feuilles}
\label{fig:scan_mol}
\end{figure}


\begin{figure}
	\centering
	\begin{subfigure}[t]{0.5\textwidth}
		\centering
		\includegraphics[width=\textwidth]{chap2/mol_lon_bioM}
		\caption{Molinia caerulea -- biomasse}
	\end{subfigure}%
	\begin{subfigure}[t]{0.5\textwidth}
		\centering
		\includegraphics[width=\textwidth]{chap2/mol_lon_bioM}
		\caption{Eriophorum -- biomasse}
	\end{subfigure}
	
	
	\begin{subfigure}[t]{0.5\textwidth}
		\centering
		\includegraphics[width=\textwidth]{chap2/mol_lon_surf}
		\caption{Molinia caerulea -- surface}
	\end{subfigure}%
	\begin{subfigure}[t]{0.5\textwidth}
		\centering
		\includegraphics[width=\textwidth]{chap2/mol_lon_surf}
		\caption{Eriphorum -- surface}
	\end{subfigure}
%    \caption{Caption place holder}
\caption{Calibration de la biomasse herbacées pour \textit{molinia Caerulea} (a), pour \textit{eriophorum} (b) et de la surface de feuille pour \textit{molinia Caerulea} (c), pour \textit{eriophorum} (d) en fonction de la hauteur}
\label{fig:cal_herb}
\end{figure}


\section{CARBIODIV}
\label{sec:carbiodiv}

\section{package m70r}
\label{sec:pckg_m70r}

% 4e de couverture -------------------------------------------------------------
%%%%%%%%%%%%%%%%%%%%%%%%%%%%%%%%%%%%%%%%%%%%%%%%%%%%%%%%%%%%%%%%%%%%%%%%%%%%%%%%
% 4e de couverture
%%%%%%%%%%%%%%%%%%%%%%%%%%%%%%%%%%%%%%%%%%%%%%%%%%%%%%%%%%%%%%%%%%%%%%%%%%%%%%%%
% set-up -----------------------------------------------------------------------
\pagenumbering{gobble} % stop page numbering
\newgeometry{left=2cm,top=1.5cm,right=1.5cm,bottom=1.5cm} % set up margin
\usefont{T1}{phv}{m}{n} % Set up font style
{\parindent0pt % disables indentation for all the text between { and }

% Résumé ---------------------------------------------------------------------
\begin{center}
		\large{[Pr\'enom NOM]}\\ 
		\textbf{[Titre de la th\`ese (en français)]}
\end{center}

\begin{framed}
	\begin{minipage}{\dimexpr\textwidth-2\fboxrule-2\fboxsep}
	R\'esum\'e : (1700 caract\`eres max.)\par
	Lorem ipsum dolor sit amet, consectetur adipiscing elit. Proin volutpat ipsum id purus ultrices lobortis. Maecenas ornare enim quis eros. Nunc eget mauris ut quam malesuada mattis. Vestibulum ante ipsum primis in faucibus orci luctus et ultrices posuere cubilia Curae; Integer vel tellus. Nam rutrum, purus non sodales rhoncus, quam magna imperdiet eros, sit amet euismod justo metus at orci. Suspendisse neque turpis, feugiat interdum, faucibus vel, aliquet quis, risus. Etiam est elit, eleifend a, consequat sit amet, scelerisque nec, odio. Quisque id odio quis libero iaculis tincidunt. Sed non mi. Morbi aliquam commodo nibh. Integer justo purus, pulvinar a, suscipit vel, iaculis a, justo. Morbi ut orci. Maecenas fringilla orci. Phasellus auctor, enim vitae tempus egestas, justo mi cursus sem, vel blandit leo turpis vitae quam. Etiam sit amet felis vitae eros ornare porttitor.\par
	Curabitur felis velit, aliquam at, aliquet in, iaculis vitae, velit. Nunc lobortis magna id ligula. Vestibulum ante ipsum primis in faucibus orci luctus et ultrices posuere cubilia Curae; Integer congue ultrices mi.
	Isdem diebus Apollinaris Domitiani gener, paulo ante agens palatii Caesaris curam, ad Mesopotamiam missus a socero per militares numeros immodice scrutabatur, an quaedam altiora meditantis iam Galli secreta susceperint scripta, qui conpertis Antiochiae gestis per minorem Armeniam lapsus Constantinopolim petit.\par
Mots cl\'es : mot 1, mot 2, ...
	\end{minipage}
\end{framed}

\vfill

% Abstract ---------------------------------------------------------------------
\begin{center}
	\large \textbf{[Titre de la th\`ese (en anglais)]}
\end{center}

\begin{framed}
	\begin{minipage}{\dimexpr\textwidth-2\fboxrule-2\fboxsep}
	R\'esum\'e : (1700 caract\`eres max.)\par
	Lorem ipsum dolor sit amet, consectetur adipiscing elit. Proin volutpat ipsum id purus ultrices lobortis. Maecenas ornare enim quis eros. Nunc eget mauris ut quam malesuada mattis. Vestibulum ante ipsum primis in faucibus orci luctus et ultrices posuere cubilia Curae; Integer vel tellus. Nam rutrum, purus non sodales rhoncus, quam magna imperdiet eros, sit amet euismod justo metus at orci. Suspendisse neque turpis, feugiat interdum, faucibus vel, aliquet quis, risus. Etiam est elit, eleifend a, consequat sit amet, scelerisque nec, odio. Quisque id odio quis libero iaculis tincidunt. Sed non mi. Morbi aliquam commodo nibh. Integer justo purus, pulvinar a, suscipit vel, iaculis a, justo. Morbi ut orci. Maecenas fringilla orci. Phasellus auctor, enim vitae tempus egestas, justo mi cursus sem, vel blandit leo turpis vitae quam. Etiam sit amet felis vitae eros ornare porttitor.\par
	Curabitur felis velit, aliquam at, aliquet in, iaculis vitae, velit. Nunc lobortis magna id ligula. Vestibulum ante ipsum primis in faucibus orci luctus et ultrices posuere cubilia Curae; Integer congue ultrices mi.
	Isdem diebus Apollinaris Domitiani gener, paulo ante agens palatii Caesaris curam, ad Mesopotamiam missus a socero per militares numeros immodice scrutabatur, an quaedam altiora meditantis iam Galli secreta susceperint scripta, qui conpertis Antiochiae gestis per minorem Armeniam lapsus Constantinopolim petit.\par
Mots cl\'es : mot 1, mot 2, ...
	\end{minipage}
\end{framed}

% logos and university name ----------------------------------------------------
\includegraphics[width=0.2\textwidth, valign=c]{./images/pucvl}
\hfill
%{\LARGE\textbf{UNIVERSITÉ D'ORLÉANS}}
\begin{minipage}{.5\textwidth}
\begin{center}
LPC2E/CNRS\\
3A, Avenue de la Recherche Scientifique\\
45071 Orléans cedex 2\\
France \\
\end{center}
\end{minipage}
\hfill
\includegraphics[width=0.2\textwidth, valign=c]{./images/univ}

}

%\listoftodos
\end{document}
%%%%%%%%%%%%%%%%%%%%%%%%%%%%%%%%%%%%%%%%%%%%%%%%%%%%%%%%%%%%%%%%%%%%%%%%%%%%%%%%
%%%%%%%%%%%%%%%%%%%%%%%%%%%%%%%%%%%%%%%%%%%%%%%%%%%%%%%%%%%%%%%%%%%%%%%%%%%%%%%%
% Document end
%%%%%%%%%%%%%%%%%%%%%%%%%%%%%%%%%%%%%%%%%%%%%%%%%%%%%%%%%%%%%%%%%%%%%%%%%%%%%%%%
%%%%%%%%%%%%%%%%%%%%%%%%%%%%%%%%%%%%%%%%%%%%%%%%%%%%%%%%%%%%%%%%%%%%%%%%%%%%%%%%