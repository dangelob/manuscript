% CHAPITRE 6
\chapter{Caractérisation de la variabilité spatio-temporelle des flux sur la tourbière de La Guette (Bilan de C)}
\newpage

\section{Introduction}
\section{Présentation du suivi}
Étudier la variabilité spatiale et temporelle des flux de carbone nécessite de mettre en place une observation régulière, un suivi adapté.
Qu'est ce qu'un suivi adapté ? Dans le cas des flux des gaz, on ne peut heureusement ou malheureusement pas suivre en permanence un site avec une fréquence importante (Qu'est ce qu'une fréquence importante).
Il faut donc choisir qu'elle(s) échelle(s) l'on souhaite étudier.
Lors de cette thèse nous avons choisi de mettre en place deux suivi différents et complémentaire.

Le premier est un suivi mensuel à l'échelle d'une tourbière. 
Plus précisément, les flux de \COO ont été mesurés, à minima, une fois par mois sur 20 points distribués sur l'étendue du site (La Tourbière de La Guette, Neuvy sur Barangeon, Cher, France).
À cause de contrainte technique et de leur importance relative \todo[inline]{Expliquer ici ou ailleurs que les flux de CH4 ne représente a priori que 5 \% du bilan de C sur une tourbière} les flux de \CHH ont été mesuré plus ponctuellement.
Ce suivi permet d'étudier la variabilité spatiale des flux au sein du site et cela sur une échelle temporelle de l'ordre de l'année qui permet d'avoir une vision sur les variations saisonnières.

Ce protocole d'observation seul ne nous permet cependant pas de rendre compte d'autres échelles de variabilité qui semble d'importance non négligeable. 
Tout d'abord la variabilité des flux journalière : 
Le système étant très lié à la photosynthèse les flux sont très dépendant de l’ensoleillement et donc des alternances jour/nuit
Ensuite qu'elle représentativité un site en particulier peut-il avoir par rapport à l'ensemble des sites existants ?

\subsection{Suivi des GES}
\subsection{Suivi des facteurs contrôlants}
Par ailleurs pour chaque embase le PAR a été mesuré, ainsi que la pression atmosphérique, la température de l'air et de la tourbe à différente profondeur (profils). Des mesures d'humidité ont également été effectué, 5 par embase afin de prendre en compte l'hétérogénéité présente.
\subsection{Suivi des flux liquides (DOC, POC)}
Mesures de DOC
Principe
Non Purgeable Organic Carbon
Pyrolyse oxydative
Ne mesure pas le CO2 dissous
L'acidification et le bullage permet d'expulser le CO2 dissous
Tout le carbone est ensuite pyrolysé le carbone est transformé en CO2 et mesuré