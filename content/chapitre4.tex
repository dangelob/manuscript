% CHAPITRE 4
\chapter{Effets de l'hydrologie sur les flux de CO2 et CH4}

\minitoc

\newpage

%\section{Manipulation du niveau de l'eau en mésocosmes}
\section{Introduction}

Au cours des deux années qu'ont durées des suivis des émissions de flux de \coo et de \chh sur la tourbière de La Guette, le niveau de la nappe n'a que très faiblement varié par rapport aux années précédentes bien plus sèches.
De cette faible ou absente variation a résulté du faible lien entre le niveau de la nappe et les flux de GES.
Malgré tout il est connu que ce facteur est un facteur contrôlant important des flux.

Ainsi de nombreuses études on reliées les émissions de \coo au niveau de la nappe avec, cependant, des résultats et des conclusions variables.
La majorité des études montrent qu'une tourbière dont le niveau de la nappe est abaissé, soit par un drainage, soit par une sécheresse, aura tendance à avoir un ENE plus faible :
Ainsi \citet{strack2013} expliquent des valeurs d'ENE plus faibles qu'escompté, par des mesures faites pendant une période relativement sèche.
Une observation similaire est faite par \citet{aurela2007} qui mesure un ENE plus faible lors d'une année sèche, sur une tourbière à Carex du sud de la Finlande.
Ils attribuent la variation de l'ENE à une augmentation de la RE et à une baisse de la PPB, dans des conditions plus chaudes et plus sèches.
\citet{peichl2014} observent également une baisse de l'ENE lors d'une année ou le niveau de la nappe baisse de façon importante, au de-là de \SI{-30}{\centi\metre}.
Ils expliquent cette baisse par une baisse de la PPB qui serait liée principalement à un effet de seuil présent à \SI{-30}{\centi\metre}.
À cette profondeur se situerait la limite de la sphère racinaire et la limite au delà de laquelle les forces capillaires ne sont plus assez forte pour maintenir un approvisionnement en eau suffisant pour les sphaignes.
Cette observation va dans le même sens que \citet{lund2012} qui attribuent également à une baisse de la PPB la baisse de l'ENE qu'ils observent en 2008 sur une tourbière située au sud de la Suède, les mesures de RE étant cette année là similaires à celles observées les autres années de mesure.
Mais dans cette même étude, pour la même tourbière, \citet{lund2012} expliquent une baisse de l'ENE, en 2006 cette fois, par une augmentation de la RE, tandis que cette année au contraire les flux de PPB sont ceux qui sont similaires à ceux des autres années.
Ce paradoxe apparent est interprété grâce au type de sécheresse : courte et intense pendant la saison de végétation de 2006 et d'intensité plus faible mais d'une durée plus longue en 2008.
Enfin et à l'inverse des résultats précédemment cités, \citet{ballantyne2014} dans une étude des effets à long terme d'une baisse du niveau de la nappe, n'observe que peu d'effet sur l'ENE tandis que les flux de RE et de PPB augmentent tous deux.
Toutes ces études montrent que si le niveau de la nappe est incontestablement reconnu comme un facteur de contrôle des flux de \coo, il est difficile d'en dégager, avec certitude, un comportement valable de façon générale.
Concernant le méthane, une baisse du niveau de la nappe est généralement liée à une baisse des émissions de \chh, et inversement \citep{strack2006,pelletier2007,turetsky2008}.
Cependant d'autres études, principalement dans des sites ou le niveau de la nappe est proche de la surface du sol, montrent une absence de relation entre le niveau de la nappe et les émissions de méthane, voire une relation inverse, avec des flux plus faibles liés à des niveaux de nappe plus élevés \citep{kettunen1996,bellisario1999,treat2007}.
Là encore selon les conditions environnementales, la relation entre les flux de \chh et le niveau de la nappe n'est pas généralisable.

L'objectif de ce chapitre est donc d'explorer plus en avant l'effet du niveau de la nappe sur les émissions de GES, effet peu ou pas visible \textit{in-situ}.
Plus précisément il s'agit de déterminer l'effet de cycles de dessication/ré-humectation sur les émissions de \coo et de \chh. 
%\citet{treat2007} montre également que selon l'échelle de temps le niveau de la nappe peut être corrélé soit négativement, soit positivement avec les flux de \chh.

\subsection{Procédure expérimentale}

L'étude des cycles de dessication/ré-humectation s'est effectué sur des mésocosmes, prélevés sur la tourbière de La Guette.
L'expérimentation a été testé durant l'été 2013 avec un seul cycle, on s'y référera par la suite comme l'expérimentation A.
Cette expérimentation a été renouvelée l'été 2014 afin d'augmenter à trois le nombre de cycles, on l'appelera l'expérimentation B (Tableau~\ref{table:recap_hm}).

\subsubsection{Expérimentation A}
Le 12 avril 2013, ont été prélevés, sur la tourbière de La Guette 6 mésocosmes.
Le prélèvement de mésocosmes s'effectue à l'aide de cylindres de PVC qui, dans un premier temps, posé sur le sol, permettent de faire un pré-découpage au couteau, puis dans un second temps sont insérés, délicatement, dans la tourbe. 
Les mésocosmes sont finalement dégagés en creusant de chaque côté (Figure~\ref{fig:mesophoto})

\begin{figure}
\centering
\includegraphics[width=\textwidth]{chap4/mesocosmes}
\caption{Prélèvement des mésocosmes}
\label{fig:mesophoto}
\end{figure}


\begin{table}
\centering
\caption{Récapitulatif des mesures pour les deux expérimentations}
\label{table:recap_hm}
\begin{tabular}{lll}\toprule
expérimentation & A & B \\ \midrule
année & 2013 & 2014 \\
réplicats & 6 & 6 \\
cycles & 1 & 3 \\
station météo & ? & oui\\

% & \multicolumn{2}{l}{expérimentation A} & \multicolumn{2}{l}{expérimentation B} \\ \midrule
%date &  \num{2013} & \num{1023} & \num{-308} \\
%2 &  \num{1045} & \num{1385} & \num{-340} \\
%3 &  \num{1323} & \num{1057} & \num{266} \\
%4 &  \num{1002} & \num{1262} & \num{-260} \\
\bottomrule
\end{tabular}
\end{table}


\begin{figure}
\centering
\includegraphics[width=.5\textwidth]{chap4/mesocarte}
\caption{Carte mésocosmes Zi}
\label{fig:mesocarte}
\end{figure}

\begin{figure}
\centering
\includegraphics[width=.5\textwidth]{chap4/mesoinstall}
\caption{Schéma d'un mésocosme}
\label{fig:mesoinstall}
\end{figure}

\subsection{Résultats}

\begin{figure}
\centering
\includegraphics[width=\textwidth]{chap4/HMzi_wtl}
\caption{Évolution du niveau de la nappe dans les mésocosmes Zi}
\label{fig:HMzi_wtl}
\end{figure}

\begin{figure}
\centering
\includegraphics[width=\textwidth]{chap4/HMzi_TES}
\caption{Évolution de la teneur en eau du sol dans les mésocosmes Zi}
\label{fig:HMzi_TES}
\end{figure}

\begin{figure}
\centering
\includegraphics[width=\textwidth]{chap4/HMzi_Tair}
\caption{Évolution de la température de l'air dans les mésocosmes Zi}
\label{fig:HMzi_Tair}
\end{figure}

\begin{figure}
\centering
\includegraphics[width=\textwidth]{chap4/HMzi_ch4}
\caption{Évolution du méthane dans les mésocosmes Zi}
\label{fig:HMzi_ch4}
\end{figure}

\begin{figure}
\centering
\includegraphics[width=\textwidth]{chap4/HMzi_ER}
\caption{Évolution de la RE dans les mésocosmes Zi}
\label{fig:HMzi_ER}
\end{figure}

\begin{figure}
\centering
\includegraphics[width=\textwidth]{chap4/HMzi_NEE}
\caption{Évolution de la NEE dans les mésocosmes Zi}
\label{fig:HMzi_NEE}
\end{figure}


\begin{figure}
\centering
\includegraphics[width=\textwidth]{chap4/HMty_wtl}
\caption{Évolution du niveau de la nappe dans les mésocosmes Tianyi}
\label{fig:HMty_wtl}
\end{figure}

\begin{figure}
\centering
\includegraphics[width=\textwidth]{chap4/HMty_NEE}
\caption{Évolution de la NEE dans les mésocosmes Tianyi}
\label{fig:HMty_NEE}
\end{figure}

\begin{figure}
\centering
\includegraphics[width=\textwidth]{chap4/HMty_ER}
\caption{Évolution de la RE dans les mésocosmes Tianyi}
\label{fig:HMty_ER}
\end{figure}

\subsubsection{Dessication}

\begin{itemize}
\item augmentation RE
\item diminution CH4
\end{itemize}

\subsubsection{Événement pluvieux}

\begin{itemize}
\item diminution RE
\item augmentation CH4 avec retard
\end{itemize}

\subsubsection{Effet cycles multiples}

\subsection{Discussion}

%\section{Manipulation du niveau de l'eau (teneur en eau) in-situ}
%\subsection{introduction}
%L'étude des effets de l'hydrologie sur les émissions de flux de GES a également pu être menée directement in-situ au sein du projet CARBIODIV (Restauration hydrologique de la tourbière de La Guette : effets sur l'évolution de la biodiversité et le stockage du carbone.) dont l'objectif est de restaurer le fonctionnement hydrologique de la tourbière de La Guette.
%\subsection{Procédure expérimentale}
%\subsubsection{Les travaux}
%\subsubsection{Les stations scientifiques}
%Deux stations ont été installées sur le site, dans deux sous-hydrosystèmes différents. Le premier en amont n'étant pas impacté par les travaux permet de contrôler les effets de site, et le second, en aval, enregistrera les effets de la restauration hydrologique.
%\subsection{Résultats}
%\subsection{Discussion}
