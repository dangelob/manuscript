% CHAPITRE 4
\chapter{Effets de l'hydrologie sur les flux de \coo et \chh}

\minitoc

\newpage

%\section{Manipulation du niveau de l'eau en mésocosmes}
\section{Introduction}

Au cours des deux années qu'ont duré les suivis des émissions de flux de \coo et de \chh sur la tourbière de La Guette, le niveau de la nappe n'a que très faiblement varié comparé aux années précédentes bien plus sèches.
En conséquence de cette faible variation peu de variations des flux ont pu y être relié.
Malgré tout il est connu que l'hydrologie est un facteur contrôlant important des flux.

Ainsi de nombreuses études on reliées les émissions de \coo au niveau de la nappe avec, cependant, des résultats et des conclusions variables.
La majorité des études montrent qu'une tourbière dont le niveau de la nappe est abaissé, soit par un drainage, soit par une sécheresse, aura tendance à avoir un ENE plus faible :
Ainsi \citet{strack2013} expliquent des valeurs d'ENE plus faibles qu'escompté, par des mesures faites pendant une période relativement sèche.
Une observation similaire est faite par \citet{aurela2007} qui mesure un ENE plus faible lors d'une année sèche, sur une tourbière à Carex du sud de la Finlande.
Ils attribuent la variation de l'ENE à une augmentation de la RE et à une baisse de la PPB, dans des conditions plus chaudes et plus sèches.
\citet{peichl2014} observent également une baisse de l'ENE lors d'une année ou le niveau de la nappe baisse de façon importante, au delà de \SI{-30}{\centi\metre}.
Ils expliquent cette baisse par une baisse de la PPB.
Cette observation va dans le même sens que \citet{lund2012} qui, lors de l'étude d'une tourbière située au sud de la suède, observent en 2008 une baisse de l'ENE.
Les mesures de RE faites cette année là étant similaires à celles effectuées les autres années, ils lient cette baisse à une diminution de la PPB.
En 2006, sur la même tourbière, \citet{lund2012} observent une autre baisse de l'ENE.
Mais cette fois, les mesures de PPB à leur tour similaires à celle des autres années n'expliquent pas cette baisse.
À l'inverse de 2008, cette baisse est expliqué par une augmentation de la RE
Ce paradoxe apparent est interprété grâce au type de sécheresse : courte et intense pendant la saison de végétation de 2006 et d'intensité plus faible mais d'une durée plus longue en 2008.
Enfin et à l'inverse des résultats précédemment cités, \citet{ballantyne2014} dans une étude des effets à long terme d'une baisse du niveau de la nappe, n'observe que peu d'effet sur l'ENE tandis que les flux de RE et de PPB augmentent tous deux.
Toutes ces études montrent que si le niveau de la nappe est incontestablement reconnu comme un facteur de contrôle des flux de \coo, il est difficile d'en dégager, avec certitude, un comportement valable de façon générale.

Concernant le méthane, une baisse du niveau de la nappe est généralement liée à une baisse des émissions de \chh, et inversement \citep{strack2006,pelletier2007,turetsky2008}.
Cependant d'autres études, principalement dans des sites ou le niveau de la nappe est proche de la surface du sol, montrent une absence de relation entre le niveau de la nappe et les émissions de méthane, voire une relation inverse, avec des flux plus faibles liés à des niveaux de nappe plus élevés \citep{kettunen1996,bellisario1999,treat2007}.
Là encore selon les conditions environnementales, la relation entre les flux de \chh et le niveau de la nappe n'est pas aisément généralisable.

L'objectif de ce chapitre est donc d'explorer plus en avant l'effet du niveau de la nappe sur les émissions de GES, effet peu ou pas visible \textit{in-situ}.
Plus précisément il s'agit de déterminer l'effet de cycles de dessication/ré-humectation sur les émissions de \coo et de \chh. 
%\citet{treat2007} montre également que selon l'échelle de temps le niveau de la nappe peut être corrélé soit négativement, soit positivement avec les flux de \chh.

\section{Procédure expérimentale}

L'étude des cycles de dessication/ré-humectation s'est effectué sur des mésocosmes, prélevés sur la tourbière de La Guette.
L'expérimentation a été testé durant l'été 2013 avec un seul cycle relativement long, on s'y référera par la suite comme l'expérimentation A.
Cette expérimentation a été renouvelée l'été 2014 afin d'augmenter à trois le nombre de cycles, ceux-ci étant plus court.
On appellera cette seconde expérimentation, l'expérimentation B (Tableau~\ref{table:recap_hm}).

\subsection{Expérimentation A}
Le 12 avril 2013, ont été prélevés, sur la tourbière de La Guette 6 mésocosmes.
Le prélèvement de mésocosmes s'effectue à l'aide de cylindres de PVC qui, dans un premier temps, posé sur le sol, permettent de faire un pré-découpage au couteau, puis dans un second temps sont insérés, délicatement, dans la tourbe. 
Les mésocosmes sont finalement dégagés en creusant de chaque côté (Figure~\ref{fig:mesophoto}).
Les mésocosmes sont transportés au laboratoire ou ils sont enterrés en extérieur et saturé en eau (eau prélevée dans la tourbière), afin que leur conditions hydrologique de départ soient les plus proche possible.
Trois mésocosmes sont choisi pour servir de contrôle, et trois vont subir un cycle de dessication/ré-humectation.
%À partir du 31 mars 2013 de l'eau a été pompé régulièrement dans les 3 mésocosmes traités pour simuler une sécheresse, jusqu'au 17 juillet.
À partir du 31 mars 2013 les précipitations ont été interceptées à l'aide d'abri bâchés installable rapidement en cas de pluie et la nuit.
Ces interceptions ont été faites jusqu'au 17 juillet dans les 3 mésocosmes traités pour simuler une sécheresse.
À cette date de l'eau est ajouté aux mésocosmes, que ce soit les contrôles ou les traitements, pour simuler de fortes précipitations.

\subsection{Expérimentation B}
Le 17 avril 2014, 6 nouveaux mésocosmes ont été prélevés sur la tourbières de La Guette et installé près du laboratoire, en suivant le même protocole que pour l'expérimentation A.
Une station météo a été installée à côté des mésocosmes afin de mesurer la température de l'air, l'humidité relative, l'irradiation solaire, la vitesse et la direction du vent et les précipitations toutes les 15 minutes.
Cette station permettait également l'enregistrement des températures mesurées par des sondes (T107) installées à \num{-5}, \num{-10}, et \SI{-20}{\centi\metre}.
Le premier cycle de dessication/réhumectation dura du 30 juin au 6 juillet pour la phase de dessication est du 7 au 16 juillet pour la phase de réhumectation.
Le deuxième cycle dura du 17 au 28 juillet et du 29 juillet au 3 aout, 
Enfin le dernier cycle fut mesuré du 4 au 11 aout pour la dessication et du 12 au 14 aout pour la réhumectation.



\begin{figure}
\centering
\includegraphics[width=\textwidth]{chap4/mesocosmes}
\caption{Prélèvement des mésocosmes}
\label{fig:mesophoto}
\end{figure}


\begin{table}
\centering
\caption{Récapitulatif des mesures pour les deux expérimentations}
\label{table:recap_hm}
\begin{tabular}{lll}\toprule
expérimentation & A & B \\ \midrule
année & 2013 & 2014 \\
réplicats & 6 & 6 \\
cycles & 1 & 3 \\
station météo & ? & oui\\
\bottomrule
\end{tabular}
\end{table}


\begin{figure}
\centering
\includegraphics[width=.5\textwidth]{chap4/mesocarte}
\caption{Carte mésocosmes Zi}
\label{fig:mesocarte}
\end{figure}

\section{Résultats}

\subsection{Expérimentation A}

Pendant la phase de dessication de l'expérimentation A, on observe une baisse du niveau de la nappe pour les placettes contrôles comme pour les placettes traitements (Figure~\ref{fig:HMzi}--A).
Malgré tout leur comportement est différent : les placettes contrôles ont un niveau de nappe relativement élevé jusqu'au 24 juin puis ce niveau baisse fortement alors que les placettes du groupe traité ont un niveau de nappe qui diminue de façon plus continue sur l'ensemble de la phase.
La remontée du niveau de la nappe s'effectue de façon similaire pour les deux groupes.
Enfin après la phase de ré-humectation, le niveau de la nappe baisse à nouveau, plus rapidement pour le groupe traitement que pour le groupe contrôle.


\begin{figure}
\centering
\includegraphics[width=1.15\textwidth, center]{chap4/expA_flux}
\caption{Évolution du niveau de la nappe dans les mésocosmes Zi}
\label{fig:HMzi}
\end{figure}


Les émissions de \chh, s'étendant de 0 et \SI{0.3}{\uml}, sont relativement similaires entre les deux groupes jusqu'au 24 juin 2013, date à partir de laquelle ils commencent à diverger (Figure~\ref{fig:HMzi}--B).
À cette date les émissions du groupe contrôle augmentent rapidement pour atteindre \SI{0.55(031)}{\uml} tandis que celles du groupe traité restent stable.
Finalement mi-juillet, à la fin de la phase de dessiccation les deux groupes retrouvent des niveaux d'émission similaires compris entre \num{0.1} et \SI{0.2}{\uml}.
Ces niveaux restent constant pendant toute la phase de réhumectation, avant d'augmenter légèrement par la suite ; ils ne dépassent pas \SI{0.25}{\uml} mais franchissent la barre des \SI{0.2}{\uml}.

Pendant la phase de dessication, les valeurs de la RE tendent à augmenter quel que soit le groupe de placettes considéré (Figure~\ref{fig:HMzi}--C).
Ces valeurs inférieures à \SI{2.5}{\uml} début juin, atteignent environ \SI{7}{\uml} pour les deux groupes mi-juillet, avant la réhumectation.
Dans le détail cependant les deux groupes ne se comportent pas de la même façon : la RE du groupe traité augmente régulièrement pendant l'ensemble de cette phase, tandis que les valeurs du groupe de contrôle restent, dans un premier temps, stable jusque fin juin.
La RE de ce groupe vaut alors \SI{2.45(075)}{\uml} contre \SI{3.26(046)}{\uml} pour le groupe traité.
Cet écart, pouvant varier dans le temps,  étant installé depuis le 9 juin
À partir de début juillet, les valeurs de RE du groupe de contrôle augmentent fortement dépassant les valeurs du groupe traité.
La Re de ce groupe atteint alors un maximum à \SI{7.93(152)}{\uml} le 8 juillet avant de retrouver des valeurs proche de celles observées dans le groupe traité.
Cette augmentation brusque correspond temporellement à celle observé, pour le même groupe, dans les flux de \chh
Lors de la phase de réhumectation, les flux de RE diminuent de façon très similaire pour les deux groupes ou ils atteignent un minimum proche de \SI{2.75}{\uml}.
Ce minimum reste cependant plus élevé que les valeurs mesurées initialement.
Après la phase de réhumectation, les flux des deux groupe restent relativement proches pendant le reste des mesures, où ils remontent à mesure que le niveau de la nappe diminue à nouveau.

Pour les deux groupes, les flux de PPB restent relativement stables pendant la phase de dessiccation (Figure~\ref{fig:HMzi}--D).
Jusqu'au 24 juin, les flux des deux groupes sont très proches et sont compris entre 5 et \SI{6}{\uml} (\SI{5.29(076)}{\uml} de moyenne pour les deux groupes).
Après cette date, et comme pour le \chh et la RE, les valeurs de la PPB du groupe de contrôle augmentent et s'écartent de celles mesurées dans le groupe traité, en attendant de les rejoindre avant la fin de cette phase de dessiccation.
Par ailleurs, à la fin de cette phase, les flux diminuent légèrement atteignant un minimum proche de \SI{3}{\uml}.
Pendant la phase de réhumectation, la PPB augmente très légèrement pour les deux groupes, le groupe de contrôle ayant des valeurs systématiquement supérieures à celles du groupe traité.
Après la réhumectation, la PPB augmente un peu pour le groupe traité, atteignant un maximum de \SI{5.83(161)}{\uml}, et plus fortement pour le groupe de contrôle ou le maximum atteint, de \SI{10.17(330)}{\uml}, est quasiment le double du précédent.


L'ENE mesuré sur les deux groupes est systématiquement supérieure pour le groupe contrôle, avec une évolution parallèle des flux pour les deux groupes (Figure~\ref{fig:HMzi}--E).
Pendant la phase de dessiccation l'ENE reste relativement constante jusque fin juin avec une valeur moyenne pour les deux groupes de \SI{1.18(058)}{\uml}.
L'écart entre le groupe contrôle et le groupe traitement tend à augmenter du 10 au 30 juin environ, avant que les valeurs du groupe de contrôle ne rejoignent celles du groupe traité.
Au delà du 24 juin, l'ENE baisse fortement pour les deux groupes pour atteindre un minimum proche de \SI{-4.5}{\uml}.
Pendant la phase de réhumectation l'ENE monte rapidement pour atteindre \num{1.52(036)} et \SI{2.15(147)}{\uml} pour le groupe de contrôle et de groupe traité respectivement.
Après la réhumectation, l'ENE du groupe contrôle varie en suivant généralement les variations du niveau de nappe.
Pour le groupe traité, l'ENE baisse par rapport au maximum atteint lors de la réhumectation puis se stabilise autour de 0.

Les variations du niveau de la nappe sont principalement liée à la RE (Figure~\ref{fig:test1}--A) et, à travers elle, à l'ENE (Figure~\ref{fig:test1}--C).
L'effet des variations du niveau de la nappe sur la PPB est quasiment nul (Figure~\ref{fig:test1}--B), même si la PPB semble diminuer aux plus fortes profondeurs.
Pour le \chh il est également difficile de distinguer des tendances générales entre les flux et les niveaux de nappe (Figure~\ref{fig:test1}--D).

\begin{figure}
\centering
\includegraphics[width=1.15\textwidth, center]{chap4/expB_flux}
\caption{Évolution du niveau de la nappe dans les mésocosmes Tianyi}
\label{fig:HMty}
\end{figure}

\subsection{Expérimentation B}

Contrairement à l'expérimentation A, et à cause d'une météo moins ensoleillée, le niveau de nappe du groupe de contrôle de l'expérimentation B reste relativement constant pendant l'ensemble de la période de mesure (Figure~\ref{fig:HMty}--A).
Le drainage artificiel du groupe traité permet d'abaisser le niveau de la nappe d'une quinzaine de centimètres en moyenne pour chaque cycle.

À l'exception du point mesuré lors de la première phase de dessiccation, les valeurs du groupe de contrôle sont systématiquement supérieures à celles du groupe traité (Figure~\ref{fig:HMty}--B).
Les émissions du groupe de contrôle tendent à augmenter sur la période de mesure.
Cette tendance est également visible pour le groupe traité.
Concernant les cycles de dessiccation/réhumectation, il est difficile de dégager des comportements communs, même s'il semble que l'assèchement conduit à une baisse des émissions.
Un pic d'émission de \chh est également à noter pour chaque cycle pendant la phase de dessiccation.

Avant le début des traitement d'assèchement, les flux de la RE sont proches pour les deux groupes tandis qu'après, leurs comportements diffèrent (Figure~\ref{fig:HMty}--C)).
Pendant les phases de dessiccation la RE du groupe traité est supérieure à celle du groupe de contrôle.
La relation semble s'inverser pendant les phases de réhumectation, du moins pour les cycles 1 et 3, car pendant la phase de réhumectation du cycle 2 les deux groupes ont des valeurs similaires.

Avant le début des traitements les flux des deux groupes sont similaires (Figure~\ref{fig:HMty}--D).
La première phase de dessiccation fait passer la PPB du groupe de contrôle au dessus du groupe traité.
Pour les deux groupes, la PPB est plus importante lors des phases de dessiccation comparée aux phase de réhumectation, avec des moyennes respectives de \num{6.35(219)} contre \num{5.80(220)} pour le groupe de contrôle et de \num{5.95(146)} contre \SI{4.05(160)}{\uml} pour le groupe traité.

Les flux de l'ENE sont similaires pour les deux groupes avant le début du traitement (Figure~\ref{fig:HMty}--E).
Le premier cycle de dessiccation/réhumectation divise les deux groupes : le groupe de contrôle ayant des valeurs d'ENE plus élevées que celles du groupe traité.
L'évolution des deux groupes reste cependant relativement conjointe pendant la période de mesure avec une tendance à la hausse.

\begin{figure}
\centering
\begin{minipage}{.5\textwidth}
  \centering
  \includegraphics[width=\linewidth]{chap4/expA_fluxWTL}
  \captionof{figure}{Expérimentation A}
  \label{fig:test1}
\end{minipage}%
\begin{minipage}{.5\textwidth}
  \centering
  \includegraphics[width=\linewidth]{chap4/expB_fluxWTL}
  \captionof{figure}{Expérimentation B}
  \label{fig:test2}
\end{minipage}
\end{figure}


\section{Discussion}

\subsection{Comparaison aux mesures \textit{in-situ}}
%CH4
À l'exception de la forte hausse, fin juin, début juillet, du groupe contrôle, les valeurs de méthane sont comprise entre 0 et \SI{0.3}{\uml}.
Ces valeurs sont un peu plus élevée, mais du même ordre de grandeur que celles mesurées sur le terrain.

Les émissions de \chh mesuré lors de l'expérimentation B sont du même ordre de grandeur, entre 0 et \SI{0.3}{\uml}, que celles mesurées \textit{in-situ}, sur la tourbière de La Guette.

% RE
Les valeurs de RE mesurées sont dans la gamme de celles mesurées sur le terrain, avec cependant des maximums moins importants.

Les valeurs RE mesurées sur les mésocomes sont plus faible que celle relevées directement sur la tourbière de La Guette avec un maximum moyen à 5 contre \SI{8}{\uml} mesuré sur le site.

% PPB

Comme pour la RE, les flux de PPB sont du même ordre de grandeur que ceux mesurés sur le terrain, mais légèrement plus faible : les maximas moyens mesurés dans les mésocosmes sont d'environ \num{7.5} pour des valeurs de \SI{13}{\uml} mesuré directement sur la tourbière.

\subsection{Effet des variations du niveau de la nappe sur les flux de gaz}

Les résultats de ces deux expérimentations montrent clairement une augmentation de la RE quand le niveau de la nappe diminue.
Ceci est en accord avec les résultats d'autres études que ce soit in-situ \citep{ballantyne2014} ou en mésocosmes \citep{blodau2004,dinsmore2009}.
Dans ces deux dernières publications, la baisse du niveau de la nappe diminue la PPB.
Pas de variations claires de la PPB n'est visible dans les données présentées, même si une légère tendance semble émergée aux plus fortes profondeur de nappe pour l'expérimentation A, ces observations restent à confirmer.
Cette absence d'effet du niveau de la nappe sur la PPB peut, peut être, être liée à la profondeur des mésocosmes (\SI{30}{\centi\metre}).
En effet dans \citet{blodau2004} et \citet{dinsmore2009}, les mésocosmes utilisés sont plus grands, 75 et \SI{41}{\centi\metre} respectivement, ont permis d'abaisser le niveau de l'eau au delà de \SI{-30}{\centi\metre}.
Cette limite a été rapportée plusieurs fois comme étant un seuil au delà duquel son observés des changements importants \citep{blodau2004,peichl2014}.
Ce seuil est expliqué comme étant la limite au delà de laquelle les forces de capillarités ne permettent plus d'alimenter en eau les sphaignes \citep{rydin2013a,ketcheson2014}.
Il résulte des constats précédents qu'une baisse du niveau de nappe, faisant augmenter la RE et ne changeant pas ou peu la PPB, conduit à une baisse de l'ENE.
Cette observation est cohérentes avec la littérature, que ce soit des expérimentations en mésocosmes \plop, ou in-situ \plop.
Malgré tout l'extrapolation de ses résultats à d'autres situations n'est pas évidente car fortement fonction du contexte.
D'autre études n'ont, par exemple, pas observé d'influence du niveau de la nappe sur la RE \plop.
Par ailleurs \citet{laiho2006} a montré l'importance du contexte et notamment celui de l'échelle de temps considéré qui peut impliquer des phénomènes différents et donc avoir des conséquences différentes.



\subsection{Dessication}

Lors des phases de dessiccations, les deux expérimentations montrent une hausse de la RE, que l'assèchement soit sur une période longue (expérimentation A) ou plus courte (expérimentation B).
À cause des conditions météorologiques présentes lors de l'expérimentation A, le groupe de contrôle subit également un assèchement et donc une hausse de sa RE.
À l'inverse les conditions météorologique moins ensoleillée de l'expérimentation B ont permis d'observer une différence nette entre d'assèchement du groupe traité et celui du groupe de contrôle.
Pour cette expérimentation, la différence entre la RE des deux groupes est statistiquement différente (p<0.05) pendant les phases de dessiccation mais non pendant les phases de réhumectation.
À l'exception du cycle 3 pour lequel la différence entre traitement et contrôle est également significative.
La RE est donc impactée de façon significative lors d'un assèchement et même si les différences ne sont significative que pour R3, il est intéressant de noter que pendant les phases de réhumectation la RE du groupe traité tend à être plus faible que celle du groupe de contrôle.
Ainsi les cycles de dessiccation/réhumectation semblent augmenter les extrêmes de la gamme de valeur de la RE.

La baisse du \chh observé dans l'expérimentation A, 


Discuter les hausses de CH4 et de RE à partir du 23 juin pour Zi.

\subsection{Réhumectation}

La réhumectation après une phase de dessication fait baisser la RE, que ce soit pour l'expérimentation A ou l'expérimentation B.

Il semble y avoir une émission de \chh suite aux phases de réhumectation, mais avec un certain temps de latence.
L'expérimentation A le montre  de façon relativement clair.
Par contre si des pics sont également visible dans l'expérimentation B, la faible durée des cycles ne permet pas de déterminer s'ils sont liés à la réhumectation ou à la phase de dessication qui suit.

\subsection{Effet cycles multiples}
