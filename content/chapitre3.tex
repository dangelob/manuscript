% CHAPITRE 3
\chapter{Bilan de C de la tourbière de La Guette}

\minitoc

\newpage

\section{Introduction}

Parmi les écosystèmes tourbeux pour lesquels un bilan de carbone a été calculé, la majorité se situe dans les hautes latitudes \plop et/ou en montagne.
Le premier objectif de ce chapitre est d'établir le bilan de C de la tourbière de La Guette.
L'intérêt est double, d'une part car ce site est représentatif d'une grande partie des tourbières dans les perturbations qu'elle subie : son drainage et son envahissement par une végétation vasculaire (cf Chapitre 2).
D'autre part sa position en basse latitude la place dans des conditions environnementale qui, sans être identiques, peuvent se rapprocher de celles que subiront d'autres écosystèmes tourbeux suite au réchauffement climatique.
Le second objectif est de caractériser la variabilité spatiale de ces flux de GES à travers ce bilan de C.

\section{Procédure expérimentale et analytique}

\subsection{Méthodes de mesure}

\subsubsection{Mesures de flux de gaz}
\textbf{placer carte tourbière embase + quadrillage}
La mesure des flux de \coo et de \chh ont été effectué en utilisant la méthode décrite dans la partie~\ref{sec:clsd_chbr_method}.
En juin 2011, 20 placettes ont été installées\footnote{je remercie ici Sébastien Gogo pour avoir installé ces placettes sur le terrain avant même mon arrivée.} selon un échantillonnage aléatoire stratifié:
La surface de la tourbière a été divisée selon une grille de 20 mailles et un point choisi aléatoirement dans chaque maille localise chaque placette.
Cette méthode permet de conserver un échantillonnage aléatoire tout en étant assuré d'avoir une représentativité homogène du site. 
Les placettes, délimitées par des piquets, occupaient une surface de \SI{4}{\square\metre} (2$\times$\SI{2}{\metre}), à l'intérieur de laquelle ont été installé de façon permanente un piézomètre et une embase permettant la mesure des flux de gaz.
Usuellement les placettes sont séparées en groupes micro-topographique. ce qui à l'avantage de permettre une distinction des capacités sources/puits relativement fine mais qui à généralement l'inconvénient du placement proche des embases les unes des autres.
Elles peuvent également être séparées en zone dans la tourbière, haut-marais par rapport à bas-marais, ou réhabilité par rapport à non-réhabilité.
Afin de gagner en représentativité spatiale, la taille du site le permettant, il a donc été décidé de positionner des placettes sur l'ensemble du site.
De plus, du fait de l'omniprésence de végétation vasculaire, et de la taille des chambres par rapport à la micro-topographie une telle approche était difficile à mettre en oeuvre.

Les mesures de \coo ont été effectué de mars 2013 à février 2015, avec une fréquence quasiment mensuelle (20 campagnes, pour 24 mois de mesure).

Les mesures de \chh ont été effectuées avec une fréquence moindre principalement liée au difficulté de mise en oeuvre de l'instrument SPIRIT (lourd, difficilement transportable dans un milieu tourbeux).

\subsubsection{Les facteurs contrôlants}

Les mesures manuelle effectuées sont la mesure de la pression atmosphérique, du PAR, des températures du sol à différentes profondeur, de la végétation.
Des prélèvements d'eau ont également été effectué chaque mois, une mesure du pH et de la conductivité dans cette eau a été réalisée sur le terrain après les mesures de flux puis les échantillons ont été congelés avant d'être analysé en terme de concentration de carbone dissous.
Ces mesures nécessitant d'accéder aux placettes régulièrement, des planches de bois ont été utilisées comme pontons mobiles, la dispersion des placettes sur le site rendant impossible une installation plus permanente.

Les mesures automatiquement acquise via une station météo campbell sont la température de l'air, température de la tourbe à X, X et X profondeur, vitesse et direction du vent, humidité relative de l'air, irradiation solaire, pression atmosphérique.

\subsection{Modélisation du bilan de C}

\subsubsection{Démarche générale}

Afin de calculer le bilan de carbone du site il est nécessaire d'établir des modèles empiriques des flux afin de pourvoir interpoler les données acquises mensuellement sur l'ensemble des deux années de mesure.
Pour établir ces modèles empiriques les données acquises ont été moyennées par campagne de mesure.
Ceci permettant, dans un premier temps, de s'affranchir de la variabilité spatiale des flux pour se concentrer sur la variabilité temporelle.
Les relations entre flux et facteurs contrôlant ont ensuite été étudiées deux à deux.

Les flux de \coo ont été modélisé en partant de l'équation ENE = PPB - RE, et le bilan a été établi en estimant de façon séparée la PPB et la RE.
Cette séparation permettant de distinguer si une variation du bilan est liée à l'un ou l'autre des flux ou bien aux deux.
Les flux en phase gazeuse ont été modélisé en partant d'équation usuellement utilisées et dans lesquelles la température est le facteur contrôlant majeur.
Puis les résidus\footnote{Valeurs moyennes - Valeurs moyennes estimées} de ces modèles de base ont ensuite été étudiés en fonction des facteurs de contrôle restant.
Dans le cas ou une tendance est visible, le facteur est intégré.
Les modèles ont été comparés avec différents indicateurs, principalement Le R2, la NRMSE et l'AIC.
Le R$^{2}$ est utilisé comme indicateur de la proportion de la variabilité des données expliqué par le modèle, sa valeur est comprise entre 0 et 1.
La RMSE et sa normalisation par la moyenne NRMSE sont utilisés comme indicateur de l'écart entre les données mesurées et les données modélisées.
L'AIC (Akaile...) permet de déterminer si l'amélioration d'un modèle suite à l'ajout d'un paramètre est suffisamment intéressante pour que ce modèle plus complexe soit utilisé.

La température a été choisie comme base de départ à la construction des modèles de RE et PPBsat, à la fois car c'est le facteur de contrôle le plus souvent invoqué et à la fois car les corrélations avec les flux étaient les plus forte.
Concernant la respiration de l'écosystème, les températures utilisées dans la littérature sont variables.
La température qui semble le plus utilisée est la température du sol à \SI{-5}{\centi\metre}  \cite{ballantyne2014}\plop, même si d'autres, notamment la température de l'air et la température du sol à \SI{-10}{\centi\metre} le sont également régulièrement \cite{bortoluzzi2006,kim1992}.
Cette profondeur, \SI{-5}{\cm}, est régulièrement utilisée car c'est dans la tourbe, proche de la surface qu'est produit la majorité du \coo.
\textbf{production CO2 ? profils ?}
C'est également à des profondeurs relativement faibles que se situent la majorité des racines \plop qui peuvent contribuer à la respiration du sol \textbf{(de l'écosystème?)} pour 35 à \SI{60}{\percent} \cite{silvola1996,crow2005}.

Après cette phase de calibration, les facteurs de contrôle utilisés dans les modèles ont été interpolés au pas de mesure de la station météo présente sur le site, c'est à dire à l'heure.
L'interpolation étant soit une simple interpolation linéaire entre les données mensuelles, soit une relation avec les facteurs acquis par la station météorologique.
À l'aide de ces interpolations et des équations les flux ont ensuite été recalculés sur les 2 années de mesure.

Enfin ces modèles ont été évalués sur des données issues d'une autre expérimentation.On ne parle pas ici de validation car les données utilisées bien qu'indépendante du jeu de données utilisé pour la calibration n'ont pas été acquise suivant un protocole identique, notamment au niveau de la répartition des embases sur le site.


\subsubsection{La Production Primaire Brute}

\begin{figure}
\centering
\includegraphics[width=\textwidth]{chap3/GPPsat_Tair_mesmod}
\caption{PPBsat modèles Tair utilisant l'équation~\ref{eq:juneTair}}
\label{fig:PPBsat_Tair_mdl}
\end{figure}

L'estimation de la PPB se fait en deux étapes.
Dans un premier temps on estime le potentiel maximum de photosynthèse à un instant donné dans des conditions de lumière saturante (PPBsat).
Ce potentiel peut varier avec les conditions environnementales et a été déterminé en utilisant l'équation de \cite{june2004} qui relie la vitesse de transport des électrons photosynthétique à lumière saturante à la température :

\begin{equation}\label{eq:juneTair}
PPBsat = a * exp(\frac{Tair - b}{c})^2
\end{equation}

Avec a la vitesse de transport des électrons photosynthétique à lumière saturante, b la température optimale pour ce transport et c la différence de température à laquelle à laquelle PBBsat vaut e$^{-1}$ de sa valeur à la température optimale.

L'utilisation de l'équation de June seule, avec la température de l'air comme variable explicative de la PPBsat, permet d'expliquer 66 \% des variations observées (Figure~\ref{fig:PPB_Tair_mdl}-a).
Les résidus de ce modèle se répartissent de façon relativement homogène (Figure~\ref{fig:PPB_Tair_mdl}-b).
Corrélés avec l'indice de végétation IV, ils présentent une tendance linéaire croissante (Figure~\ref{fig:PPB_Tair_mdl}-c), mais ne présentent pas de tendance particulière avec le niveau de la nappe (Figure~\ref{fig:PPB_Tair_mdl}-d)
Afin de prendre en compte cette tendance linéaire le modèle est adapté :

\begin{equation}\label{eq:juneTairIV}
PPBsat = (a * IV) * exp(\frac{T - b}{c})^2
\end{equation}

\begin{figure}
\centering
\includegraphics[width=\textwidth]{chap3/GPPsat_TairIV_mesmod}
\caption{PPBsat modèles Tair utilisant l'équation~\ref{eq:juneTairIV}}
\label{fig:PPBsat_TaIV_mdl}
\end{figure}

Cette nouvelle équation permet d'expliquer une part plus importante des variations de PPBsat (R$^{2}$ = 0,85) et augmente la proximité entre les données mesurées et les données modélisées (La RMSE diminue) (Figure~\ref{fig:PPBsat_TaIV_mdl}).
Les résidus de cette équation semblent répartis de façon moins homogène que précédemment, avec d'avantage de résidus présent entre \num{-1} et \num{-2} qu'entre \num{1} et 2, ainsi qu'un point de valeur supérieur à \num{4}.
Le biais reste malgré tout léger au regard de l'amélioration apportée.

À partir de ce potentiel à lumière saturante, la PPB est estimée en prenant en compte la luminosité.
On utilise l'équation~\ref{eq:PPB_bubier} proposée par \cite{bubier1998} et régulièrement et souvent utilisée \cite{bortoluzzi2006,worrall2009}:

\begin{equation} \label{eq:PPB_bubier}
PPB = \frac{PPBsat * a * PAR}{PPBsat + a * PAR}
\end{equation}

\begin{figure}
\centering
\includegraphics[width=\textwidth]{chap3/GPP_mdl_mesmod}
\caption{PPB modèles Tair}
\label{fig:PPB_Tair_mdl}
\end{figure}

La PPB calculée à partir de l'équation~\ref{eq:juneTair} présente des résidus relativement homogène avec cependant d'avantage de points situés entre \num{-2} et \num{-4} qu'entre \num{2} et \num{4} (Figure~\ref{fig:PPB_Tair_mdl}--a,b).
Une observation similaire peut être faire pour PPB calculé à partir de l'équation~\ref{eq:juneTairIV}, avec cependant des résidus resserrés entre \num{-2} et \num{2} et un point un peu plus extrême(Figure~\ref{fig:PPB_Tair_mdl}--c,d).

\subsubsection{La Respiration de l'Écosystème}

%%%%%%%%%%%%%%%%%%%%% RE

La RE est estimée directement à partir des données acquises moyennées en partant de la température connue pour contrôler une grande partie de ce flux.
Différents modèles ont été testés parmi les plus souvent utilisés (linéaire, exponentiel, arrhénius).


Les variations de la RE moyenne au cours du temps suivent les variations saisonnières de la température.

\begin{equation} \label{eq:RE_T}
RE = a*exp(b*T)
\end{equation}

La température de l'air utilisée dans un modèle exponentiel permet d'expliquer une grande partie, 90 \%, des variations de la respiration de l'écosystème (Figure~\ref{fig:ER_mdl}--a).
Les résidus de cette équation semble répartis de façon non-biaisée, pas de tendance dans le nuage de point (Figure~\ref{fig:ER_mdl}--b).
Une légère tendance, moins claire que pour la PPBsat, est visible entre les résidus et l'indice de végétation.
Très souvent utilisée, la température à \SI{-5}{\centi\metre} donne des résultats proche mais moins bons notamment avec une hétéroscédasticité des résidus.
La tendance des résidus avec l'indice de végétation est ici moins marqué encore qu'avec la température de l'air.
Dans les deux cas, le gain possible en ajoutant un paramètre semble limité et peu pertinent.

\begin{figure}
\centering
\includegraphics[width=\textwidth]{chap3/ER_mdl}
\caption{RE modèles avec T5}
\label{fig:ER_mdl}
\end{figure}

%%%%%%%%%%%%%%%%%%%%% ENE
\subsubsection{L'Échange Net de l'Écosystème}

\begin{figure}
\centering
\includegraphics[width=\textwidth]{chap3/NEE_mdl_mesmod}
\caption{ENE modèle T5 IV}
\label{fig:ENE_mdl}
\end{figure}

L'ENE est ensuite modélisé en utilisant l'équation suivante :

\begin{equation}
ENE = PBB-RE
\end{equation}

Le résultat de cette équation (Figure~\ref{fig:ENE_mdl}), montre que ce modèle permet d'expliquer une grande partie des variations de l'ENE.
Les résidus de cette équation sont répartis de manière a peu près homogène.

\subsubsection{Le flux de \chh}

\begin{figure}
\centering
\includegraphics[width=\textwidth]{chap3/CH4_H_mdl_mesmod}
\caption{CH4 modèle H}
\label{fig:CH4_mdl}
\end{figure}

Pas de consensus clair émerge de la littérature quand aux facteurs prépondérant dans le contrôle du \chh.
La température, peut être utilisée \cite{alm1999,bubier1995}, le niveau de la nappe \cite{bubier1993} ou la végétation \cite{bortoluzzi2006}.
Avec les données acquises, le recouvrement les herbacées H est le meilleur prédicteur (Figure~\ref{fig:CH4_mdl}).
le méthane est également corrélé avec les températures, faiblement avec les température de surface puis plus fortement avec les températures du sol à plus forte profondeur.
Enfin il est anti-corrélé (R=-0.51) avec le niveau de la nappe.

%bortoluzzi veg
%Alm 1999 T -30 cm
%Bellisario relation inverse avec WTL (increased flux with lower WTL) T10
%Bubier 1993a WTL majeur
%Bubier 1995 Température humidité végétation

\subsubsection{Le COD}

\section{Résultats}

\subsection{Évolution générale des flux et facteurs contrôlants sur la tourbière de La Guette}

\subsubsection{Les Facteurs contrôlant}

\begin{figure}
\centering
\includegraphics[width=\textwidth]{chap3/WTL_mean_evolution}
\caption{Évolution du niveau de la nappe moyen des 20 embases mesuré pendant la période de mesure (mars 2013 -- février 2015)}
\label{fig:WTL_mean_evolution}
\end{figure}

L'évolution du niveau de la nappe des 20 placettes, décrite dans la figure~\ref{fig:WTL_mean_evolution}, est marquée par un étiage d'une vingtaine de centimètres en moyenne en 2013 et l'absence d'un étiage net en 2014 avec un niveau de la nappe moyen ne descendant que rarement sous la barre des \SI{-10}{\cm}.
Ces observations sont cohérentes avec la figure~\ref{fig:WTL} représentant des données acquises à plus haute fréquence, et confirment la particularité de ces 2 années vis à vis des précédentes qui présentent des étiages bien plus fort.

\begin{figure}
\centering
\includegraphics[width=\textwidth]{chap3/T_mean_evolution}
\caption{Évolution des températures de l'air (Tair) et du sol à \SIlist{-5;-30;-50;-100}{\centi\metre} (T5, T30, T50 et T100 respectivement) moyenne mesurée lors des campagnes de terrain de mars 2013 à février 2015}
\label{fig:T_mean_evolution}
\end{figure}

La température de l'air mesurée manuellement montre une variabilité saisonnière cohérente avec celle mesurées par la station météo. 
%bien que les valeurs semblent systématiquement supérieures.
La variabilité saisonnière de la température est également visible dans le sol avec cependant un amortissement et une diminution de la variabilité avec la profondeur (figure~\ref{fig:T_mean_evolution})
%chiffres ?

\begin{figure}
\centering
\includegraphics[width=\textwidth]{chap3/cond_mean_evolution}
\caption{Évolution de la conductivité pendant la période de mesure (mars 2013 -- février 2015)}
\label{fig:cond_mean_evolution}
\end{figure}

La conductivité moyenne mesurée sur le site varie entre \SIlist{35;55}{\usml} (figure~\ref{fig:cond_mean_evolution}).


\begin{figure}
\centering
\includegraphics[width=\textwidth]{chap3/pH_mean_evolution}
\caption{Évolution du pH pendant la période de mesure (mars 2013 -- février 2015)}
\label{fig:pH_mean_evolution}
\end{figure}


En moyenne le pH mesuré sur la tourbière de La Guette est compris entre 4 et 5 (figure~\ref{fig:pH_mean_evolution}).
Ces valeurs sont cohérentes avec la classification \textit{poor-fen} du site .


\begin{figure}
\centering
\includegraphics[width=\textwidth]{chap3/RH_mean_evolution}
\caption{Évolution de la teneur en eau du sol pendant la période de mesure (mars 2013 -- février 2015)}
\label{fig:RH_mean_evolution}
\end{figure}

\begin{figure}
\centering
\includegraphics[width=\textwidth]{chap3/NPOC_mean_evolution}
\caption{Évolution de la teneur en eau du sol pendant la période de mesure (mars 2013 -- février 2015)}
\label{fig:NPOC_mean_evolution}
\end{figure}

%\subsection{Évolution générale des flux de C sur la tourbière de La Guette}

\subsubsection{Les flux de carbone}

\begin{figure}
	\centering
	\begin{subfigure}[t]{\textwidth}
		\centering
		\includegraphics[width=\textwidth]{chap3/GPP_evolution_avg}
		\caption{Production primaire brute}
		\label{fig:GPP_evolution_avg}
	\end{subfigure}%
	
	\begin{subfigure}[t]{\textwidth}
		\centering
		\includegraphics[width=\textwidth]{chap3/ER_evolution_avg}
		\caption{Respiration de l'écosystème}
		\label{fig:ER_evolution_avg}
	\end{subfigure}
	
	\begin{subfigure}[t]{\textwidth}
		\centering
		\includegraphics[width=\textwidth]{chap3/NEE_evolution_avg}
		\caption{Échange net de l'écosystème}
		\label{fig:NEE_evolution_avg}
	\end{subfigure}
\caption{Évolution du niveau de PPB, RE et ENE pendant la période de mesure. Moyenne des 20 embases de mars 2013 à février 2015.}
\label{fig:flux_evolution_avg}
\end{figure}

%\subsubsection{PBB}
L'ensemble des mesures de \coo s'étendent de mars 2013 à février 2015.
Cependant de novembre 2013 à février 2014 les mesures ont été interrompue suite à des pannes/casses matérielles.
Malgré cela les périodes les plus critiques, notamment la saison de végétation, ont pu être mesurées pour les 2 années, permettant d'avoir une vision correcte/globale de chacune d'elle.
À noter également que pour l'ensemble des flux, la déviation standard augmente avec les valeurs mesurées.

En 2013, les valeurs de la PPB augmentent au printemps et une partie de l'été avec un maximum de \SI{999999(888)}{\uml} atteint fin juillet, avant de diminuer à partir d'août.
En 2014 le maximum de PPB, \SI{99999(888)}{\uml}, est atteint en juin, soit plus tôt que l'année précédente.
Puis pendant l'été et l'automne les valeurs décroissent jusqu'à être proche de 0.
En moyenne les valeurs de la PPB sont de \SI{7.12(519)}{\uml} en 2013 et de \SI{6.56(472)}{\uml} en 2014 (Figure~\ref{fig:GPP_evolution_avg}).

La RE en 2013 augmente pendant le printemps et une partie de l'été, elle atteint un maximum de \SI{99999(888)}{\uml} en juillet avant de diminuer.
En 2014 la RE atteint, comme la PPB, son maximum plus tôt, en juin à \SI{99999(888)}{\uml} avant de décroître jusqu'en hiver pour approcher des valeurs nulles.
La moyenne annuelle de RE en 2013 est de \SI{4.27(316)}{\uml}, ce qui est légèrement supérieure à celle de 2014 : \SI{3.63(256)}{\uml}(Figure~\ref{fig:ER_evolution_avg}).

Concernant l'ENE, en 2013 elle augmente jusqu'en juin avec un maximum à \SI{99999(888)}{\uml} avant de diminuer jusqu'à la fin de l'année.
Cependant, cette baisse est moins homogène que celle des deux flux précédents, avec notamment une augmentation de l'ENE entre juillet et août 2013.
Ceci étant, il faut également noter les valeurs importantes de la déviation standard particulièrement en juin et en août.
En 2014, l'ENE maximum est atteinte en juillet avec \SI{99999(888)}{\uml} avant qu'elle ne décroisse.
Cette baisse est cependant plus homogène qu'en 2013.
les moyennes de l'ENE en 2013 et 2014 sont très proche est sont respectivement de \SI{2.85(305)}{\uml} et \SI{2.93(277)}{\uml} (Figure~\ref{fig:NEE_evolution_avg}).

%\subsubsection{Le \chh}

Le \chh comme le \coo montre une variabilité saisonnière importante, cependant les flux mesurés sont un ordre de grandeur en dessous de ceux mesurés pour le \coo.
À l'inverse de ce dernier, l'importance des flux de \chh mesurés en 2013 et 2014 est différente.
En 2013 les flux sont moins important qu'en 2014 avec des maximum de \SIlist{0.078;0.196}{\uml} respectivement.

\begin{figure}
\centering
\includegraphics[width=\textwidth]{chap3/CH4_evolution_avg}
\caption{Évolution des flux de méthane moyen (N ?) pendant la période de mesure (mars 2013 -- février 2015)}
\label{fig:CH4_evolution_avg}
\end{figure}

%\subsubsection{Le Carbone Organique Dissous (COD)}

\subsubsection{Les relation flux gazeux et facteurs contrôlants}

\begin{figure}
\centering
\includegraphics[width=\textwidth]{chap3/Fl_FC}
\caption{Relations entre les flux de gaz et une sélection de facteurs contrôlant}
\label{fig:Fl_FC}
\end{figure}

Comme précisé précédemment, le niveau de la nappe n'a que peu varié pendant les deux années de mesures.
De ce fait aucune relation claire ne se distingue entre les flux et le niveau de la nappe que ce soit pour le \coo (PPB et RE) ou le \chh (Figure~\ref{fig:Fl_FC}).
La PPB et la RE présentent cependant des relations avec la température de l'air, et l'indice de végétation, même si pour ce dernier les tendance sont moins évidentes, particulièrement pour la RE.
Le \chh quand à lui ne présente pas de relation avec la température de l'air, mais une tendance exponentielle est visible vis à vis de l'indice de végétation.
\textbf{(\chh et Température dans la tourbe ?)}

\subsection{Le bilan de carbone de la tourbière de La Guette à l'échelle de l'écosystème}

\subsubsection{Bilan (param et valeur)}

\begin{table}
\centering
\caption{Valeur des paramètres des équations utilisées pour modéliser les flux et sensibilité relative (en \%) des flux en réponse à une variation de $\pm$\SI{10}{\percent} de chacun des paramètres des modèles.}
\label{table:mdl_par}
\begin{tabular}{llccccc}\toprule

& par & valeur & se & pval & \SI{-10}{\percent} & +\SI{+10}{\percent} \\ \midrule
%\multicolumn{7}{l}{PPB -- Tair}  \\ [+.5ex]
%& a & 26.23 & 62.07 & & -9.7 & 9.6 	\\
%& b & 53.68 & 61.27 & & 43.7 &-35.1 \\
%& c & 27.21 & 28.56 & &-22.5 & 21.9 \\
%& i &  1.84 &       & & -0.4 &  0.4 \\[+1ex]
\multicolumn{7}{l}{PPB -- Tair IV}  \\ [+.5ex]
& a & 33.66 & 19.89 & & -9.0 & 8.9 \\
& b & 42.45 & 22.28 & & 19.3 & -19.9 \\
& c & 25.77 & 15.28 & & -10.1 & 8.6 \\
& i &  0.33 &       & & -1.1 & 1.0 \\[+1ex]
\multicolumn{7}{l}{ER -- Tair}  \\ [+.5ex]
& a & 0.34 & 0.08 &  & -10 & +10 \\
& b & 0.10 & 0.01  & & -22.6 & +29.9  \\[+1ex]
%\multicolumn{7}{l}{ER -- T5}  \\ [+.5ex]
%& a & 0.51 & 0.12 & & -10 & +10 \\
%& b & 0.12 & 0.01 & & -19.4 & 24.7 \\[+1ex]
\multicolumn{7}{l}{CH4 -- IV}  \\ [+.5ex]
& a & 0.00 & 0.00 & & -10 & +10 \\
& b & 13.01 & 2.82 & & -43.9 & 79.2 \\[+1ex]
\bottomrule
\end{tabular}
\end{table}

Afin de calculer le bilan à l'aide des équations XXXX, la température de l'air mesurée au niveau de la station météo à été utilisée.
Pour l'indice de végétation, le bilan a été calculé avec une interpolation linéaire entre les points.

\begin{figure}
\centering
\begin{subfigure}{.5\textwidth}
  \centering
  \includegraphics[width=\linewidth]{chap3/ER_T5_val}
  \caption{évalutation RE}
  \label{fig:sub1}
\end{subfigure}%
\begin{subfigure}{.5\textwidth}
  \centering
  \includegraphics[width=\linewidth]{chap3/GPP_TairIVcov_val}
  \caption{évaluation PPB}
  \label{fig:sub2}
\end{subfigure}
\caption{Évaluation modèles}
\label{fig:test}
\end{figure}


\begin{table}
\centering
\caption{Bilan en gCm2an1}
\label{table:BdC}
\begin{tabular}{lllllll}\toprule
& année & PBB & RE & CH4 & COD & BCNE \\ \midrule
mdl 1 (Tair - Tair - IV) & 2013 & 1317.8 & 1332.8 & 10.1 & & \\[+.5ex]
                         & 2014 & 1257.7 & 1273.3 & 24.6 & & \\ [+1ex]
                         & total & 1287.7 & 1283.9 & 17.4 & &\\[+2ex]
%mdl 2 (Tair IV - Tair - IV) & 2013 & 1035.2 & 1332.8 & 10.1 & & \\[+.5ex]
%                            & 2014 & 1212.9 & 1273.3 & 24.6 &  & \\ [+1ex]
%                            & total & 1124.1 & 1283.9 & 17.4 & &\\[+2ex]
%mdl 3 (Tair IV - T5 - IV) & 2013 & 1035.2 & 1358.8 & 10.1 & & \\[+.5ex]
%                          & 2014 & 1212.9 & 1273.3 & 24.6 & & \\ [+1ex]
%                          & total & 1124.1 & 1288.6 & 17.4 & &\\[+1ex]
\bottomrule
\end{tabular}
\end{table}




\begin{figure}
\centering
\includegraphics[width=\textwidth]{chap3/BdC_interp}
\caption{Flux de \coo interpolé à partir des équations~\ref{eq:juneTairIV} et \ref{eq:PPB_bubier} pour la GPP et de l'équation~\ref{eq:RE_T} pour RE.}
\label{fig:BdC_interp}
\end{figure}






\subsubsection{Évaluation du bilan}

%\subsubsection{sensibilité des paramètres}

L'analyse de sensibilité, consistant à faire varier chaque paramètre des modèles de $\pm$\SI{10}{\percent}, montre pour une équation exponentielle simple des valeurs attendues $\pm$\SI{10}{\percent} pour le paramètre a et \num{+29} à \SI{-22}{\percent} pour le b (Figure~\ref{table:mdl_sensitiv}).
Pour la PPB issue des équations~\ref{eq:juneTairIV} et \ref{eq:PPB_bubier} le paramètre i à très peu d'effet sur le bilan, \num{0} à \SI{-1}{\percent}.
Cependant l'effet sur le bilan augmente lorsque la végétation est prise en compte (équation~\ref{eq:juneTair} et \ref{eq:PPB_bubier}) : \num{-8} à \SI{-10}{\percent}.
À l'inverse, la sensibilité de l'ensemble des autres paramètres (a, b, c) diminue lorsque l'indice de végétation est pris en compte.
Le paramètre a est l'exception, passant de \num{-10} à \SI{-17}{\percent} pour une baisse de \SI{10}{\percent}.
Considérant le modèle de PPB prenant en compte la végétation, la sensibilité maximum des différents paramètres du bilan est proche de \SI{30}{\percent}, et similaire pour la PPB et la RE.

\begin{table}
\centering
\caption{Sensibilité relative (en \%) du bilan de \coo (ENE) en réponse à une variation de $\pm$\SI{10}{\percent} de chacun des paramètres des modèles.}
\label{table:mdl_sensitiv_BdC}
\begin{tabular}{llccccccccccc}\toprule
& \multicolumn{4}{l}{RE} & \multicolumn{4}{l}{GPP} & \multicolumn{4}{l}{\chh} \\ 
& par & valeur & \SI{-10}{\percent} & +\SI{+10}{\percent} & par & valeur & \SI{-10}{\percent} & +\SI{+10}{\percent} & par & valeur & \SI{-10}{\percent} & +\SI{+10}{\percent} \\ \midrule
\multicolumn{13}{l}{mdl 1 équation~\ref{eq:RE_T5} et équation~\ref{eq:juneTair} Tair, Tair, H} \\ [+.5ex]
& a & 0.34 & -766 & 766 & a & 26.23 & 741 & -736 & a & 17.82 & -12.28 & 12.28 \\
& b & 0.10 & -1730 & 2289 & b & 53.68 & -3358 & 2693 & b & 0.03 & -15.08 & 17.68 \\
&  &  & & & c & 27.21 & 1725 & -1680 & & & & \\
&  &  & & & i & 1.84 & 31.56 & -26.92 & & & & \\[+1ex]
\multicolumn{13}{l}{mdl 2 équation~\ref{eq:RE_T5} et équation~\ref{eq:juneTairIV} Tair, Tair IVcov, H} \\ [+.5ex]
& a & 0.34 & -71.08 & 71.08 & a & 33.66 & 56.25 & -55.32 & a & 17.82 & -1.14 & 1.14 \\
& b & 0.10 & -160.59 & 212.51 & b & 42.45 & -119.84 & 123.69 & b & 0.03 & -1.40 & 1.64 \\
&  &  & & & c & 25.77 & 62.63 & -53.28 & & & & \\
&  &  & & & i & 0.33 & 6.94 & -6.02 & & & & \\[+1ex]
\multicolumn{13}{l}{mdl 3 équation~\ref{eq:RE_T5IV} et équation~\ref{eq:juneTairIV} Tair IVcov, Tair IVcov, H} \\ [+.5ex]
& a & 0.92 & -55.69 & 55.69 & a & 33.66 & 61.31 & -60.31 & a & 17.82 & -1.24 & 1.24 \\
& b & 0.09 & -149.40 & 189.01 & b & 42.45 & -130.64 & 134.84 & b & 0.03 & -1.53 & 1.79 \\
& c & 0.14 & -20.89 & 20.89 & c & 25.77 & 68.28 & -58.08 & & & & \\
&  &  & & & i & 0.33 & 7.57 & -6.56 & & & & \\
\bottomrule
\end{tabular}
\end{table}


%\begin{table}
%\centering
%\caption{Sensibilité relative (en \%) des flux en réponse à une variation de $\pm$\SI{10}{\percent} de chacun des paramètres des modèles.}
%\label{table:mdl_sensitiv}
%\begin{tabular}{llccccccccccc}\toprule
%& \multicolumn{4}{l}{RE} & \multicolumn{4}{l}{GPP} & \multicolumn{4}{l}{\chh} \\ 
%& par & valeur & \SI{-10}{\percent} & +\SI{+10}{\percent} & par & valeur & \SI{-10}{\percent} & +\SI{+10}{\percent} & par & valeur & \SI{-10}{\percent} & +\SI{+10}{\percent} \\ \midrule
% & \multicolumn{4}{l}{Tair} & \multicolumn{4}{l}{Tair} & \multicolumn{4}{l}{H} \\ [+.5ex]
%& a & 0.34 & -10   & 10   & a & 26.23 & -9.7 &  9.6 & a & 17.82 &-10.0 & 10.0 \\
%& b & 0.10 & -22.6 & 29.9 & b & 53.68 & 43.7 &-35.1 & b &  0.03 &-12.3 & 14.4 \\
%&   &      &       &      & c & 27.21 &-22.5 & 21.9 &   &       &      &      \\
%&   &      &       &      & i &  1.84 & -0.4 &  0.4 &   &       &      &      \\[+1ex]
% & \multicolumn{4}{l}{Tair IVcov} & \multicolumn{4}{l}{Tair IVcov} & & & & \\ [+.5ex]
%& a & 0.92 & -7.3  & 7.3 & a & 33.66 & -9.0 &  8.9 & & & & \\
%& b & 0.09 & -19.5 & 24.7& b & 42.45 & 19.3 &-19.9 & & & & \\
%& c & 0.14 & -2.7  & 2.7 & c & 25.77 &-10.1 &  8.6 & & & & \\
%&   &      &       &     & i & 0.33  & -1.1 &  1.0 & & & & \\[+1ex]
%\bottomrule
%\end{tabular}
%\end{table}




%\subsubsection{pseudo-validation et erreur}

%\begin{figure}
%\centering
%\includegraphics[width=.5\textwidth]{chap3/ER_T5_val}
%\caption{Évaluation RE}
%\label{fig:RE_T5_val}
%\end{figure}
%\begin{figure}
%\centering
%\includegraphics[width=.5\textwidth]{chap3/GPP_TairIVcov_val}
%\caption{Évaluation GPP}
%\label{fig:GPP_TairIVcov_val}
%\end{figure}

\subsection{Variabilité spatiale du bilan}

\subsubsection{Représentativité locale}

\subsubsection{Modélisation par placette}

\subsubsection{Corrélation avec facteurs contrôlant}

\section{Discussion}

\subsection{Représentativité du modèle à l'échelle de l'écosystème}

Valeur absolue des flux

Différence entre 2013 et 2014

apport d'un indice de végétation

\subsection{Sensibilité et limitations du bilan}

sensibilité du bilan au variation des paramètres

limitations

\begin{itemize}
\item pas d'arbres
\item pas de zones à gros touradons
\item pas de cartographie (mais pas grave si les placette sont maj bien représentée par mdl ecos)
\item extrapolation sur d'autres site difficile (cf validation)
\end{itemize}

\subsection{Représentativité locale du modèle}

Distribution des paramètres

Pourquoi certaines placette mieux que d'autres




