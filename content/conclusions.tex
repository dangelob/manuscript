% CONCLUSIONS
\chapter*{Synthèse et perspectives}
\markboth{Conclusions et perspectives}{}
\addcontentsline{toc}{chapter}{Conclusions et perspectives}
\newpage

L'étude des flux de carbone dans les écosystèmes tourbeux est complexe car assujetti à des facteurs de contrôle dont la prépondérance varie fortement selon l'échelle considérée et les conditions environnementales.

\section{Bilan du bilan (de C) ?}

Malgré tout les observations réalisées sur la tourbière de La Guette ont permis de mettre en évidence des flux de \coo particulièrement fort que ce soit pour la RE ou la PPB.
Cette force des flux de \coo est probablement liée à sa situation géographique locale et globale : une tourbière de plaine située à basse latitude et à ses problématiques de drainage et d'envahissement par une végétation vasculaire.
Ainsi la saisonnalité plus faible qu'en montagne permet aux flux de rester fort pendant une période de l'année plus importante.
Ces flux importants entraînent des variations forte en terme de bilan selon les méthodologies employées, il est cependant probable que la tourbière de La Guette fonctionne actuellement comme une source de carbone.

L'estimation du bilan à l'échelle saisonnière ne permet pas de reproduire les variations journalières, l'estimation du modèle pendant les 3 jours de mesures haute fréquence réalisés en 2013 est largement supérieure aux valeurs mesurées (Figure~\ref{fig:RE1_vs_JN})

La prise en compte de la végétation reste une difficulté importante, l'observation répétée nécessitant des mesures non destructives, souvent imprécises ou très coûteuses en temps.
Paradoxalement les zones de la tourbières fonctionnant en puits de carbone sont celle ou les herbacées sont dominantes.


\begin{figure}
\centering
\includegraphics[width=\textwidth]{conclusions/RE1_vs_JN}
\caption{Comparaison entre les valeurs estimées par le modèle RE-1 et les mesures faites à haute fréquence sur le site du 30 juillet au 2 août 2013}
\label{fig:RE1_vs_JN}
\end{figure}


\section{L'hydrologie}

L'effet de la restauration hydrologique de la tourbière de La Guette n'a pas pu être mis en évidence de part une pluviométrie forte et un niveau de nappe toujours important.
Les expérimentations

\subsection{Résilience de la tourbe par rapport aux 2 années sèches qui précèdent le BdC}
(lien chap 3 et 4)

%schéma conceptuel ? Modèles globaux (ORCHID, chloée)
%
%Flux fort
%
%sensibilité param forte
%
%Modèles multi annuel et prise en compte de la végétation
%
%Quid des variations journalières dans un bilan annuel ? (Figure~\ref{fig:RE1_vs_JN})


Les prendre en compte améliorerait-il les modèles

modèles globaux ?
\textbf{limitations des équations :}
Plus généralement, la majorité des tourbières sont sous la neige une partie de l'année, ce qui n'arrive que rarement sur la tourbière de La Guette et une partie possède également des zones d'eau libre, qui n'existent pas sur ce site.

modèles globaux et profondeur de tourbe


\section{Ouverture vers d'autre méthodes de mesures}
\begin{itemize}
\item chambre automatique (lien chap 5, et chap 3 ?)
\item tour eddy covariance (lien chap 5 et chap 3 ?)
\end{itemize}

\section{perspectives}

La suite du projet CARBIODIV permettra peut être de mettre en évidence l'effet de la restauration.

Un partenariat avec le LSCE commencé pendant ces travaux devra permettre de valoriser ces données à des échelles plus importante.
Des données on d'ors et déjà été envoyée à Chloé XX qui développe un code "tourbière" dans le modèle ORCHIDEE.

L'installation prochaine d'une tour eddy covariance sur le site permettra de comparer ce bilan à des mesures plus haute fréquence.
