% INTRODUCTION
% OBJECTIFS : Introduire le contexte générale, l'échelle globale des problématiques
% Cette introduction doit être parfaitement compréhensible par un béotien !

\chapter*{Introduction}
\markboth{Introduction}{}
\addcontentsline{toc}{chapter}{Introduction}
\newpage


%\subsection*{Contexte général}

%PLAN : 
%- Le changement global, une thématique actuelle
%	* Historique du changement climatique ?
%	* Consensus scientifique ?
%	* Qu'est-il attendu  en terme de changement (réchauffement ?)
%- Le cycle du carbone : des flux et des stocks
%	* importance relative des stocks et flux
%	* principaux facteur forcant à l'échelle globale ?
%- Et les zones humides et les tourbières dans tout ça ?
%	* Qu'est ce qu'une tourbière ?
% 	* Fonctionnement vis à vis du cycle de carbone 
% 	* Une sensibilité particulière (localisation, anthropisation)
% 	* Écosystème non pris en compte dans les modèles globaux
Vers 1610, Jan Baptist Van Helmont, chimiste, physiologiste et médecin, découvre le dioxide de carbone (\coo) qu'il nomme « gaz sylvestre » \citep{philippedesouabe-zyriane1988}.
À cette époque pré-industrielle (avant 1800), les concentrations en \coo sont généralement estimées à 280 ppm\footnote{Partie par million} \citep{Siegenthaler1987}.
En 1957, Charles David Keeling, scientifique américain, met au point et utilise pour la première fois un analyseur de gaz infra-rouge pour mesurer la concentration de \coo de l'atmosphère dans l'île d'Hawaii, à Mauna Loa.
La précision et la fréquence importante de ses mesures lui permirent de mettre en évidence pour la première fois les variations journalières et saisonnières des concentrations en \coo atmosphérique, mais d'évaluer également à plus long terme leur tendance haussière \citep{harris2010}.
Depuis l'époque pré-industrielle les concentration en \coo ont en effet légèrement augmenté et sont alors estimées à moins de \SI{320}{ppm} \citep{pales1965}.
Ce constat a probablement joué un rôle dans la prise de conscience, par la communauté scientifique, de l'importance et de l'intérêt de l'étude du changement climatique et plus largement des changements globaux.\index{changements globaux}
En 2013, le Groupe d'Experts Intergouvernemental sur l'évolution du Climat (GIEC) a publié son 5\ieme rapport sur le changement climatique qui souligne l'importance des émissions de Gaz à Effet de Serre (GES) dans le changement climatique \citep{stocker2013}.
Au printemps 2014, la barre symbolique des \SI{400}{ppm} a été dépassée dans tout l'hémisphère nord selon un communiqué de l'Organisation Météorologique Mondiale (\url{http://www.wmo.int/pages/mediacentre/press_releases/pr_991_fr.html}).

À l'échelle globale, l'humanité par la consommation des combustibles fossiles et par la production de ciment, émet dans l'atmosphère environ \SI{7.8}{\pgca} (\SI{7.8e15}{\gram C an^{-1}}) \citep{Ciais2014}.
Les flux « naturels » entre l'atmosphère et la biosphère sont d'un ordre de grandeur supérieur : \num{98} et \SI{123}{\pgca} pour la respiration (\coo et \chh principalement) et la photosynthèse au sens large \citep{Bond-Lamberty2010,Beer2010}. L'importance de ces flux renforce la nécessité de les comprendre et si possible de les prédire, car une modification de leur dynamique même faible pourrait avoir des conséquences importantes.
Les écosystèmes naturels, en plus d'en échanger de façon importante avec l'atmosphère, stockent du carbone de façon importante : entre \num{1500} et \SI{2000}{\pgc} pour les sols par rapport aux \num{750} à \SI{800}{\pgc} stockés dans l'atmosphère.

%Parmi les écosystèmes terrestres, les tourbières, dans leur fonctionnement naturel, stockent du carbone notamment grâce à leurs conditions de saturation en eau importantes.  
Parmi les écosystèmes terrestres, les tourbières fonctionnent naturellement comme des puits de carbone : elles stockent du carbone grâce des conditions de saturation en eau importante.
Elles ne représentent que \num{2} à \SI{3}{\percent} des terres émergées mais contiennent entre \num{270} et \SI{455}{\pgc}, faisant de ces écosystèmes des stocks importants \citep{gorham1991,turunen2002} :
d'abord parce qu'ils sont relativement concentrés en terme de surface, mais également car situés majoritairement dans les hautes latitudes de l'hémisphère nord, là ou le réchauffement climatique attendu est le plus important.
Ces écosystèmes ont pendant longtemps été considérés comme néfastes et impropres.
D'ailleurs une grande partie d'entre eux ont été drainés pour être exploités, que ce soit pour utiliser la tourbe comme combustible ou comme substrat horticole, ou que ce soit pour utiliser les tourbières comme terres agricoles ou sylvicoles.
Autrefois étudiés pour les propriétés de combustible de la tourbe, les tourbières sont aujourd'hui principalement étudiées vis-à-vis des perturbations qu'elles subissent : perturbations humaines, hausse ou baisse du niveau de la nappe, apports azotés, réhabilitation, ou perturbations climatique, effet de la température, des précipitations.
Parmi toutes ces questions, celle du devenir de ce stock de carbone reste incertaine.
La variabilité de ces écosystèmes rend la prédiction de leurs comportements délicate et aujourd'hui malgré leur importance ces écosystèmes ne sont pas pris en compte dans les modèles globaux.
Le dernier rapport du GIEC note ainsi que si les connaissances ont avancé, de nombreux processus ayant trait à la décomposition de la matière organique des sols sont toujours absents des modèles notamment en ce qui concerne le carbone des zones humides boréales et tropicales et des tourbières \citep{Ciais2014}.
Mieux comprendre ces écosystèmes, à différentes échelles, l'investigation est donc nécessaire pour espérer pourvoir estimer leurs comportements face aux changements qu'ils subissent et vont subir.

%Le \coo est un gaz à effet de serre (GES) et son accumulation dans l'atmosphère... \textbf{force ? comparaison ? explication effet de serre ?}
%Car si à l'époque les concentration en \coo étaient inférieures à 320~ppm (partie par millions) elles ont dépassé, au printemps 2014, la barre symbolique des 400~ppm selon un communiqué de l'Organisation Météorologique Mondiale. Les concentrations pré-industrielles (avant 1800) sont quant à elles généralement estimées à 280 ppm \cite{Siegenthaler1987}.

%Aujourd'hui, que ce soit pour le comprendre, le caractériser ou bien le prédire, de nombreux \textbf{Combien ? cf fact sheet IPCC} scientifiques dans un grand nombre de disciplines, travaillent directement ou indirectement sur les changements globaux.
%Ils sont nombreux également à collaborer au sein du  Groupe d'experts Intergouvernemental sur l'Évolution du Climat (GIEC), qui rassemble, évalue et synthétise les connaissances internationales liées au sujet.
%Des expérimentateurs, des modélisateurs, des climatologues, des atmosphéristes \todo{atmosphéristes, vérifier que ce mot existe...}, des écologues, qu'ils travaillent sur des archives ou des données actuelles. \todo{à reformuler}

%%De manière générale, parmi les flux de C mesurés entre la biosphère et l'atmosphère, la respiration et la photosynthèse sont les plus  importants, 98 et 123 PgC/yr pour le flux de respiration globale (+ les feux) et la photosynthèse respectivement \cite{Bond-Lamberty2010,Beer2010}. Pour comparaison les flux liés à la production de ciment et aux ressources fossiles (charbon, pétrole et gaz) représentent 7.8 PgC/yr \cite{Ciais2014}.
%
%Étroitement lié aux changements globaux, le cycle du carbone est particulièrement étudié, quels sont les réservoirs, quels sont les flux et comment vont-ils évoluer ? 
%\textbf{schéma ?}
%

%Zones humides tourbières
%
%historique des tourbières, généralités sur l'histoire des tourbières vis à vis des hommes
%Sujets principaux qui ont menés à l’étude des tourbières jusqu'à nos jours (Exploitation, effet de serre)
%
%Pourquoi étudier les tourbières aujourd’hui ? 
%
%Les sols stockent entre \num{1500} et \SI{2000}{\giga\tonne C} et parmi eux, les tourbières, zones humides longtemps considérée néfastes et impropre, ont été drainées et exploitées.
%Pourtant, parmi les nombreux services écologiques qu'elles donnent \index{services ecologiques@services écologiques} (épuration du sol, régulation des flux hydriques, biodiversité), elles constituent un stock de carbone relativement important au regard de la surface qu'elles occupent. Ainsi il est généralement admis que les tourbières contiennent un quart à un tiers du carbone présent Chiffres \textbf{(surfaces...)} dans l'ensemble des terres émergées tandis qu'elle ne constituent que 3 \% des surfaces continentales \plop. Ce ratio relativement important, correspond à un stock d'environ 455 Gt \cite{gorham1991,turunen2002}.
%Il est à mettre perspective avec les autres stock du cycle du carbone. On observe que ce stock est du même ordre de grandeur que celui de la végétation 

%Les tourbières sont des écosystèmes qui, s'il ne s'étendent pas sur une surface très importante (leur surface est estimée à 3\% des surfaces émergées) contiennent, relativement à leur extension, une quantité de carbone importante (455 Gt selon \cite{Gorham1991}).
%En conséquence dans un contexte **d'augmentation des GES dans l'atm et de réchauffement**, l'évolution de ce stock, sa pérennité ou sa remobilisation est un sujet d'étude important. De plus cette importance n'est à ce jour pas prise en compte de façon spécifique dans les modèles climatiques globaux.
%
%En France les tourbières s'étendent sur environ 60 000 Ha (\plop).
%
%
%
%Transition modèles
%
%En octobre 2013 le GIEC a publié le rapport du groupe de travail I qui travaille sur les aspects scientifiques physiques du système et du changement climatique.
%S'il note que les connaissances ont avancé, il note également que de nombreux processus ayant trait à la décomposition du carbone sont toujours absents des modèles notamment en ce qui concerne le carbone des zones humides boréales et tropicales et des tourbières. \plop



%\subsection*{Objectif de la thèse et approche mise en oeuvre}

%Dans ce contexte l'objectif de ces travaux est de comprendre la dynamique des flux de carbone.
%Principalement le \coo qu'il soit gazeux, dissout ou particulaire et le \chh.
%Notamment caractériser la variabilité spatiale et temporelle et déterminer quels sont les facteurs de contrôles dominants.

\subsection*{Objectifs du travail}
Les tourbières sont des écosystèmes particulièrement efficace pour stocker du carbone.
Le devenir de leur stock de carbone et de leur fonction de puits reste incertaine vis-à-vis des changement globaux attendus.
Plus particulièrement, la compréhension des processus contrôlant cette fonction puits, notamment pour les tourbières envahies par une végétation vasculaire, restent à éclaircir.
Dans ce contexte les objectifs de ce travail sont donc (i) de caractériser la variabilité spatio-temporelle des flux et d'établir le bilan de carbone de la tourbière de La Guette, (ii) de déterminer quels facteurs environnementaux contrôlent le fonctionnement comme puits ou source de carbone de cet écosystème notamment l'effet du niveau de la nappe sur les émissions lors de cycles de dessiccations réhumectations.
Pour ce faire une approche axée sur l'\textbf{observation} et l'\textbf{expérimentation} a été mise en oeuvre : 

%
%Dans ce contexte les objectifs de ce travail sont donc (i) de caractériser la variabilité spatio-temporelle des flux et d'établir le bilan de carbone de la tourbière de La Guette, (ii) de préciser l'effet du niveau de la nappe sur les émissions lors de cycles de dessiccations réhumectations.
%Pour ce faire une approche axée sur l'observation et l'expérimentation a été mise en oeuvre : 
\begin{itemize}
\item Dans un premier temps, a été mis en place un suivi sur deux années de la tourbière de La Guette permettant d'évaluer les flux de GES (\coo et \chh) et d'étudier leurs variations saisonnières et spatiales sur l'ensemble de l'écosystème. Ces estimations de flux ont ensuite pu être utilisées afin d'estimer le bilan de carbone de la tourbière.
\item Dans un second temps, à travers des expérimentations en mésocosmes et sur le terrain, l'effet du niveau de la nappe sur les flux de GES a été exploré, particulièrement lors de cycles de dessiccation-réhumectations.
\item Enfin un suivi des flux à haute fréquence sur plusieurs tourbières a été réalisé afin de déterminer les éventuelles différences de sensibilité des émissions de \coo entre le jour et la nuit et de tester à cette échelle une méthode d'estimation de la RE basée sur la synchronisation entre les signaux de flux et de température.
\end{itemize}



%Dans ce contexte, l'objectif de mes travaux de thèse est de mieux comprendre la dynamique du carbone au sein des tourbières.
%Tout d'abord en caractérisant la variabilité spatiale et temporelle des flux de carbone à travers l'établissement du bilan de carbone d'une tourbière de Sologne.
%De déterminer quels facteurs environnementaux contrôlent le fonctionnement comme puits ou source de carbone de cet écosystème.


%Enfin construire, dans un esprit de synthèse et d'ouverture et à l'aide des connaissances acquises, un modèle intégrateur permettant un lien avec les modèles globaux et notamment ORCHIDE, afin que ces écosystèmes puissent être pris en compte à cette échelle.

%Pour atteindre ces objectifs, nos travaux ont été articulés autour de deux axes principaux :
%dans un premier temps, l'\textbf{observation} pour une période de deux ans des flux de gaz (\coo et \chh) et de paramètres environnementaux servant à la caractérisation des variabilités spatiales et temporelles, ainsi qu'à l'étude des facteurs contrôlant.
%Certains facteurs contrôlant sont, dans un second temps, étudiés plus spécifiquement à travers un volet \textbf{expérimentation}.
%Ce dernier doit permettre une meilleure compréhension des processus clés avec notamment l'impact de l'hydrologie.
%Enfin un troisième volet axé sur la \textbf{modélisation}, avec le développement d'un modèle le plus mécaniste possible.

% Contenu des chapitres de la thèse .
Le document est structuré de la façon suivante :
\begin{itemize}
\item Le premier chapitre pose le contexte bibliographique dans lequel s'inscrit ce travail.
Il se découpe en trois parties ; la première définit les terminologies et les concepts principaux employés dans le manuscrit.
La seconde précise l'état des connaissances sur les tourbières vis à vis des flux de carbone.
Enfin la troisième partie replace ce travail au sein du contexte précédemment établi.
\item Le deuxième chapitre décrit les sites d'études et les méthodes et matériels employés dans ces travaux.
\item Le troisième chapitre présente la variabilité spatio-temporelle des flux et l'estimation du bilan de carbone de la tourbière de La Guette.
\item Le quatrième chapitre décrit l'effet de cycles de dessication/ré-humectation sur les flux de GES en mésocosmes.
\item Le cinquième chapitre se concentre sur des aspect méthodologique en ce qui concerne la respiration à l'échelle journalière, plus spécifiquement la prise en compte du temps de latence entre la vague de chaleur et les flux, et la différence entre les mesures faites le jour et la nuit.
\item Enfin la dernière partie du document présente la synthèse et l'interprétation des résultats obtenus, ainsi que les perspectives de ce travail.
%Le chapitre 1 est une synthèse bibliographique, un état de l'art des connaissances liées au sujet.
%Les chapitres 2 et 3 rassemblent les travaux du volet observation, ils concernent respectivement le suivi XX et le suivi YY 
%Les chapitres 4 et 5 développent la partie expérimentale à travers l'impact d'un assèchement et celui d'un rehaussement du niveau de l'eau.
%Le chapitre 6 concerne plus spécifiquement la modélisation, même si ce volet interviendra par ailleurs de façon transverse dans les autres chapitres.
%Enfin une conclusions et des perspectives seront exposées.
\end{itemize}

% % % % TRASH ? REUTILISATION APRES.
%
%Afin de répondre à notre problématique (La variation spatiale et temporelle des flux de \coo $\leftarrow$ Ceci est la thématique pas un problématique...), Trois approches complémentaires ont été mises en œuvres. 
%Tout d'abord la mise en place de suivis terrain, qui nous ont permit de mieux comprendre certains processus. 
%L'expérimentation ensuite, sur le terrain ou en laboratoire, afin d'étudier plus en détail l'impact d'un facteur forçant majeur qu'est l'hydrologie. 
%Et enfin la modélisation afin de synthétiser ces connaissances dans un cadre dont on espère pouvoir se resservir dans différents sites.
