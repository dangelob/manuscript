\chapter*{Introduction}
\markboth{Introduction}{}
\addcontentsline{toc}{chapter}{Introduction}
\newpage

Vers 1610, Jan Baptist Van Helmont, chimiste, physiologiste et médecin, découvre le dioxyde de carbone (\coo) qu'il nomme « gaz sylvestre\footnote{Ce nom vient du fait que ce gaz était identifié comme provenant, entre autres, de la combustion du charbon de bois} » \citep{philippedesouabe-zyriane1988}.
À cette époque pré-industrielle (avant 1800), les concentrations en \coo sont estimées à 280 ppm\footnote{Partie par million} \citep{Siegenthaler1987}.
En 1957, Charles David Keeling, scientifique américain, met au point et utilise pour la première fois un analyseur de gaz infra-rouge pour mesurer la concentration de \coo de l'atmosphère dans l'île d'Hawaii, à Mauna Loa.
La précision et la fréquence importante de ses mesures lui permirent de mettre en évidence pour la première fois les variations journalières et saisonnières des concentrations en \coo atmosphérique, et d'évaluer également à plus long terme leur tendance à la hausse \citep{harris2010}.
Depuis l'époque pré-industrielle les concentrations en \coo ont en effet légèrement augmenté et sont alors estimées à \SI{315}{ppm} environ. \citep{pales1965}.
Ce constat a probablement joué un rôle dans la prise de conscience, par la communauté scientifique, de l'importance et de l'intérêt de l'étude du changement climatique et plus largement des changements globaux.\index{changements globaux}
En 2013, le Groupe d'experts Intergouvernemental sur l'Évolution du Climat (GIEC) a publié son 5\ieme rapport sur le changement climatique qui souligne l'importance des émissions de Gaz à Effet de Serre (GES) dans le changement climatique \citep{stocker2013}.
Au printemps 2014, la barre symbolique des \SI{400}{ppm} a été dépassée dans tout l'hémisphère nord selon un communiqué de l'Organisation Météorologique Mondiale (\url{http://www.wmo.int/pages/mediacentre/press_releases/pr_991_fr.html}).

À l'échelle globale, l'humanité, par la consommation des combustibles fossiles et par la production de ciment, émet dans l'atmosphère environ \SI{7.8}{\pgca}\footnote{\si{\pgc} : $10^{15}$ grammes de carbone} \citep{Ciais2014}.
Les flux « naturels » entre l'atmosphère et la biosphère sont d'un ordre de grandeur supérieur : \num{98} et \SI{123}{\pgca} respectivement, pour la respiration (\coo et \chh principalement) et la photosynthèse au sens large \citep{Bond-Lamberty2010,Beer2010}.
L'importance de ces flux renforce la nécessité de les comprendre et si possible de les prédire, car une modification de leur dynamique même faible pourrait avoir des conséquences importantes.
Les flux de carbone entre les écosystèmes naturels et l'atmosphère sont importants et les sols stockent entre entre \num{1500} et \SI{2000}{\pgc} qu'il faut mettre en perspective avec les \num{750} à \SI{800}{\pgc} stockés dans l'atmosphère.

Parmi les écosystèmes terrestres naturels, les \textbf{tourbières} sont les plus efficaces dans le stockage du carbone.
Ce fonctionnement naturel en \textbf{puits de carbone} est la conséquence de conditions de saturation en eau importante du milieu, empêchant la dégradation des matières organiques (majoritairement constituées de carbone)  qui se stockent sous forme de tourbe.
Ce stock est estimé entre \textbf{270 et  \SI{455}{\pgc}}, ce qui représente \textbf{10 à \SI{25}{\percent} du carbone stocké dans les sols mondiaux} alors que ces écosystèmes ne représentent que \textbf{2 à \SI{3}{\percent} des terres émergées}.
La concentration de ce stock sous les hautes latitudes de l'hémisphère nord, où sont localisées la majorité des tourbières, rend incertain son devenir. 
En effet ce sont dans ces zones que sont attendus les changements climatiques les plus importants \citep{Ciais2014}.
La pérennité de ces écosystèmes est également fragilisée par les nombreuses perturbations anthropiques qu'ils subissent ou qu'ils ont subit.
Longtemps considérées comme néfastes et impropres à la culture, une grande partie des tourbières ont été drainées pour être exploitées : la tourbe a été utilisée comme combustible ou comme substrat horticole, les tourbières comme terres agricoles ou sylvicoles.

Autrefois étudiées pour les propriétés combustibles de la tourbe, les tourbières sont aujourd'hui principalement étudiées afin de comprendre leur fonctionnement et l'effet des perturbations climatiques et anthropiques sur ce fonctionnement, notamment par rapport à leur fonction de puits de carbone.
La variabilité de ces écosystèmes rend la prédiction de leurs comportements délicate et aujourd'hui, malgré leur importance, ces écosystèmes ne sont pas pris en compte dans les modèles globaux.
Le dernier rapport du GIEC note ainsi que si les connaissances ont avancé, de nombreux processus ayant trait à la décomposition de la matière organique des sols sont toujours absents des modèles notamment en ce qui concerne le carbone des zones humides boréales et tropicales et des tourbières \citep{Ciais2014}.
Plus spécifiquement, si les facteurs de contrôle principaux des émissions de carbone dans ces écosystèmes sont connus : la température, le niveau de la nappe d'eau, la végétation, leurs variations et leurs interactions ne font pas consensus. 
Le \textbf{rôle des variations du niveau de la nappe d'eau}, particulièrement l'effet du sens de ces variations et leur intensité sur les flux de GES, restent à comprendre.
Tout comme\textbf{l'effet des communautés végétales} et de leurs changements, comme par exemple l'envahissement d'une tourbière par une végétation vasculaire.
Pour mieux comprendre ces écosystèmes, à différentes échelles, l'investigation est donc nécessaire pour estimer leur comportement face aux changements qu'ils subissent et vont subir.

\subsection*{Objectifs du travail}

Dans ce contexte les objectifs de ce travail sont donc (i) de caractériser la variabilité spatio-temporelle des flux et des variables environnementales qui pourraient les conditionner (ii) de déterminer quels facteurs environnementaux contrôlent le fonctionnement comme puits ou source de carbone de cet écosystème, notamment l'effet du niveau de la nappe d'eau sur les émissions lors de cycles de dessiccation/réhumectation, et (iii) d'établir le bilan de carbone de la tourbière de La Guette.
Pour ce faire une approche axée sur l'\textbf{observation} et l'\textbf{expérimentation} a été mise en oeuvre : 

\begin{itemize}

\item Dans un premier temps, a été mis en place un suivi sur deux années (2013 et 2014), dans 20 placettes couvrant la superficie de la tourbière de La Guette. 
Les 20 campagnes de terrain ont consisté à mesurer dans chaque placette les flux de GES ainsi que les variables environnementales (température du sol à différentes profondeurs, le niveau de la nappe d'eau, la végétation, les propriétés physico-chimiques de l'eau \dots).
Ces mesures ont ensuite pu être utilisées pour estimer le bilan de carbone de la tourbière.
\item Dans un second temps, à travers des expérimentations en mésocosmes, l'effet du niveau de la nappe sur les flux de GES a été exploré, particulièrement lors de cycles de dessiccation/réhumectation.
\item Enfin un suivi des flux à haute fréquence dans les quatre tourbières du Service National d'Observation a été réalisé afin de déterminer les éventuelles différences de sensibilité des émissions de \coo entre le jour et la nuit et de tester à cette échelle une méthode d'estimation de la respiration basée sur la synchronisation entre les signaux de flux et de température.
\end{itemize}

Le document est structuré de la façon suivante :
\begin{itemize}
\item Le premier chapitre pose le contexte bibliographique dans lequel s'inscrit ce travail.
Il se découpe en deux parties ; la première définit les terminologies et les concepts principaux employés dans le manuscrit.
La seconde précise l'état des connaissances sur les tourbières vis à vis des flux de carbone.
\item Le deuxième chapitre décrit les sites d'étude et les méthodes et matériels employés dans ces travaux.
\item Le troisième chapitre présente la variabilité spatio-temporelle des flux et l'estimation du bilan de carbone de la tourbière de La Guette.
\item Le quatrième chapitre décrit l'effet de cycles de dessication/réhumectation sur les flux de GES et COD en mésocosmes.
\item Le cinquième chapitre se concentre sur des aspects méthodologiques en ce qui concerne la respiration à l'échelle journalière, plus spécifiquement la prise en compte du temps de latence entre la vague de chaleur et les flux, et la différence entre les mesures faites le jour et la nuit.
\item Enfin la dernière partie du document présente la synthèse et l'interprétation des résultats obtenus, ainsi que les perspectives de ce travail.
\end{itemize}