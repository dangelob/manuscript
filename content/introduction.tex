% INTRODUCTION
% OBJECTIFS : Introduire le contexte générale, l'échelle globale des problématiques
% Cette introduction doit être parfaitement compréhensible par un béotien !

\chapter*{Introduction}
\markboth{Introduction}{}
\addcontentsline{toc}{chapter}{Introduction}
\newpage
\begin{linenumbers}


\section*{Contexte général}

%PLAN : 
%- Le changement global, une thématique actuelle
%	* Historique du changement climatique ?
%	* Consensus scientifique ?
%	* Qu'est-il attendu  en terme de changement (réchauffement ?)
%- Le cycle du carbone : des flux et des stocks
%	* importance relative des stocks et flux
%	* principaux facteur forcant à l'échelle globale ?
%- Et les zones humides et les tourbières dans tout ça ?
%	* Qu'est ce qu'une tourbière ?
% 	* Fonctionnement vis à vis du cycle de carbone 
% 	* Une sensibilité particulière (localisation, anthropisation)
% 	* Écosystème non pris en compte dans les modèles globaux

En 1957, Charles David Keeling, scientifique américain, met au point et utilise pour la première fois, un analyseur de gaz infra-rouge pour mesurer la concentration de \COO de l'atmosphère sur l'île d'Hawaii, à Mauna Loa.
La précision et la fréquence importante de ses mesures lui permirent de voir pour la première fois les variations journalière et saisonnière des concentrations en \COO atmosphérique, mais également à plus long terme leur tendance haussière \cite{harris2010}.
Le \COO est un gaz à effet de serre (GES) et son accumulation dans l'atmosphère... \todo[inline]{force ? comparaison ? explication effet de serre ?}
Ce constat a probablement joué un rôle considérable dans la prise de conscience, par la communauté scientifique, de l'importance et de l'intérêt de l'étude du changement climatique et plus largement des changements globaux.
Car si à l'époque les concentration en \COO était inférieure à 320~ppm (partie par millions) elles ont dépassées, au printemps 2014, la barre symbolique des 400~ppm selon un communiqué de l'Organisation Météorologique Mondiale. Les concentrations pré-industrielles (avant 1800) sont quand à elles généralement estimée à 280 ppm \cite{Siegenthaler1987}.

Aujourd'hui, que ce soit pour le comprendre, le caractériser ou bien le prédire, de nombreux \todo{Combien ? cf fact sheet IPCC} scientifiques dans un grand nombre de disciplines, travaillent directement ou indirectement sur les changements globaux.
Ils sont nombreux également à collaborer au sein du  Groupe d'experts Intergouvernemental sur l'Évolution du Climat (GIEC), qui rassemble, évalue et synthétise les connaissances internationales liée au sujet.
%Des expérimentateurs, des modélisateurs, des climatologues, des atmosphéristes \todo{atmosphéristes, vérifier que ce mot existe...}, des écologues, qu'ils travaillent sur des archives ou des données actuelles. \todo{à reformuler}

De manière générale, parmi les flux de C mesurés entre la biosphère et l'atmosphère, la respiration et la photosynthèse sont les plus  important, 98 et 123 PgC/yr pour le flux de respiration globale (+ les feux) et la photosynthèse respectivement \cite{Bond-Lamberty2010,Beer2010}. Pour comparaison les flux liés à la production de ciment et aux ressources fossiles (charbon, pétrole et gaz) représentent 7.8 PgC/yr \cite{Ciais2014}.

Étroitement lié aux changements globaux, le cycle du carbone est particulièrement étudié, quels sont les réservoirs, quels sont les flux et comment vont-ils évoluer ? 
\todo[inline]{schéma ?}


Zones humides tourbières

historique des tourbières, généralités sur l'histoire des tourbières vis à vis des hommes
Sujets principaux qui ont menés à l’étude des tourbières jusqu'à nos jours (Exploitation, effet de serre)

Pourquoi étudier les tourbières aujourd’hui ? 





L'étude des tourbières se poursuit car, en plus de rendre de nombreux services écologiques \index{services ecologiques@services écologiques} (épuration du sol, régulation des flux hydriques, biodiversité), elles constituent un stock de carbone relativement important au regard de la surface qu'elle occupent. Ainsi il est généralement admis que les tourbières contiennent un quart à un tiers du carbone présent \todo{Chiffres (surfaces...)} dans l'ensemble des terres émergées tandis qu'elle ne constituent que 3 \% des surfaces continentales \plop. Ce ratio relativement important, correspond à un stock d'environ 455 Gt \cite{gorham1991,turunen2002} est à mettre perspective avec les autres stock du cycle du carbone. On observe que ce stock est du même ordre de grandeur que celui de la végétation 

%Les tourbières sont des écosystèmes qui, s'il ne s'étendent pas sur une surface très importante (leur surface est estimée à 3\% des surfaces émergées) contiennent, relativement à leur extension, une quantité de carbone importante (455 Gt selon \cite{Gorham1991}).
En conséquence dans un contexte **d'augmentation des GES dans l'atm et de réchauffement**, l'évolution de ce stock, sa pérennité ou sa remobilisation est un sujet d'étude important. De plus cette importance n'est à ce jour pas prise en compte de façon spécifique dans les modèles climatiques globaux.

En France les tourbières s'étendent sur environ 60 000 Ha (\plop).

\begin{table}[!h]
\centering
\begin{tabular}{lll}\toprule
Compartiment & Stock (en Gt de C) & référence \\ \midrule
Tourbières & 455 & \cite{gorham1991} \\ 
Végétation & 450 -- 650 & \cite{Robert2003}\\ 
Sols & 1500 -- 2000 & \cite{Robert2003,Post1982,Eswaran1993}\\ 
\COO atmosphérique & 750 -- 800 & \cite{Robert2003}\\ 
Permafrost & 1700 & \\ \bottomrule
\hline 
\end{tabular}
\caption{Estimations des stocks de C pour différents environnements}
\end{table}

\todo{Pas d'entrée "journal" pour Post1982}


Transition modèles

En octobre 2013 le Groupe d'experts Intergouvernemental sur l'Évolution	du Climat (GIEC) a publié le rapport du groupe de travail I qui travaille sur les aspects scientifique physique du système et du changement climatique.
S'il note que les connaissance ont avancées, il note également que de nombreux processus ayant trait à la décomposition du carbone sont toujours absent des modèles notamment en ce qui concerne le carbone des zones humides boréales et tropicales et des tourbières. \plop



\section*{Objectif de la thèse et approche mise en oeuvre}

%Dans ce contexte l'objectif de ces travaux est de comprendre la dynamique des flux de carbone.
%Principalement le \COO qu'il soit gazeux, dissout ou particulaire et le \CHH.
%Notamment caractériser la variabilité spatiale et temporelle et déterminer quels sont les facteurs de contrôles dominants.

L'objectif de ces travaux est donc de mieux comprendre la dynamique du carbone au sein des tourbières.
Tout d'abord en caractérisant la variabilité spatiale et temporelle des flux de carbone à travers l'établissement de bilan de carbone.
De déterminer quels facteurs environnementaux contrôlent le fonctionnement comme puits ou source de carbone de ces écosystèmes.
Enfin construire, dans un esprit de synthèse et d'ouverture et à l'aide des connaissances acquise, un modèle intégrateur permettant un lien avec les modèles globaux et notamment ORCHIDE, afin que ces écosystèmes puissent être pris en compte à cette échelle.

Pour atteindre ces objectifs, nos travaux ont été articulés autour de trois volets \todo{volet... t'as pas mieux ? Branche ? -\_-"} principaux :
Dans un premier temps, l'\textbf{observation} régulière des flux de gaz (\COO et \CHH) ainsi que d'un certain nombre de paramètres environnementaux servant à la caractérisation des variabilités spatiales et temporelles, ainsi qu'à l'étude des facteurs contrôlant.
Certains facteurs contrôlant qui sont, dans un second temps, étudiés plus spécifiquement à travers un volet \textbf{expérimentation}.
Ce dernier doit permettre une meilleure compréhension de processus clé avec notamment l'impact de l'hydrologie.
Enfin un troisième volet axé sur la \textbf{modélisation}, avec le développement d'un modèle le plus mécaniste possible.


% Contenu des chapitres de la thèse .
Cette thèse est structurée de la façon suivante :
Le chapitre 1 est une synthèse bibliographique, un état de l'art des connaissances liées au sujet.
Les chapitres 2 et 3 rassemblent les travaux du volet observation, ils concernent respectivement le suivi XX et le suivi YY 
Les chapitres 4 et 5 développent la partie expérimentale à travers l'impact d'un assèchement et celui d'un rehaussement du niveau de l'eau.
Le chapitre 6 concerne plus spécifiquement la modélisation, même si ce volet interviendra par ailleurs de façon transverse dans les autres chapitres.
Enfin une conclusions et des perspectives seront exposées.

\end{linenumbers}

% % % % TRASH ? REUTILISATION APRES.
%
%Afin de répondre à notre problématique (La variation spatiale et temporelle des flux de \COO $\leftarrow$ Ceci est la thématique pas un problématique...), Trois approches complémentaires ont été mises en œuvres. 
%Tout d'abord la mise en place de suivis terrain, qui nous ont permit de mieux comprendre certains processus. 
%L'expérimentation ensuite, sur le terrain ou en laboratoire, afin d'étudier plus en détail l'impact d'un facteur forçant majeur qu'est l'hydrologie. 
%Et enfin la modélisation afin de synthétiser ces connaissances dans un cadre dont on espère pouvoir se resservir dans différents sites.
