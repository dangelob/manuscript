\singlespacing
\chapter{Synth\`{e}se bibliographique}
\label{ch:ch1}

\minitoc

\newpage
\doublespacing
\index{tourbières|(}

La première partie de ce chapitre traite des tourbières de façon générale : Que sont ces écosystèmes ?
Quelle terminologie y est associée ? Comment se forment-ils ? Quelle est leur extension ? Et quelles sont les perturbations qu'ils subissent ?
La seconde partie décrit plus spécifiquement les tourbières à travers le prisme des flux de carbone, principalement gazeux : 
Quels sont les liens entre la structure et le fonctionnement des tourbières et les flux de carbone ? 
Quels sont les facteurs qui contrôlent ces flux ? 
Quel est l'état des connaissances quant à l'estimation des bilans de carbone dans ces écosystèmes ?

\section{Les tourbières et le cycle du carbone}

Les tourbières sont des écosystèmes particulièrement liés au cycle du carbone.
En effet le carbone y est stocké de façon considérable grâce à un fonctionnement naturel propice à cette accumulation.
Ce lien est d'ailleurs d'une importance telle qu'il fait partie intégrante de leur définition.

\subsection{Zones humides et tourbières : définitions et terminologies}

\subsubsection{Définitions}

Les tourbières font partie d'un ensemble d'écosystèmes plus large que l'on appelle les zones humides\index{zone humide} (\textit{wetlands}).
Ces zones humides ne sont ni des écosystèmes terrestres au sens strict, ni des écosystèmes aquatiques.
Elles sont à la frontière entre ces deux mondes et sont caractérisées par un niveau de nappe élevé, proche de la surface du sol, voire au dessus.
Cette omniprésence de l'eau joue sur l'aération du sol et module ainsi la disponibilité en oxygène.
Les zones humides ont été définies en 1971, lors de la convention de \textsc{Ramsar}\footnote{La convention de \textsc{ramsar} est un traité international visant à la conservation et l’utilisation rationnelle des zones humides.} de la façon suivante : 
\begin{pdef}
\textsc{Zone humide} :

«les zones humides sont des étendues de marais, de fagnes\footnotemark, de tourbières ou d'eaux naturelles ou artificielles, permanentes ou temporaires, où l'eau est stagnante ou courante, douce, saumâtre ou salée, y compris des étendues d'eau marine dont la profondeur à marée basse n'excède pas six mètres.»

\hfill {\scriptsize \citep{ramsar1987}}
\end{pdef}
\footnotetext{Marais tourbeux situé sur une hauteur topographique}

Les zones humides regroupent donc des écosystèmes très variés parmi lesquels les marais, les mangroves, les plaines d'inondations et les tourbières.
Ces dernières sont des écosystèmes majoritairement continentaux (par opposition aux écosystèmes côtiers comme les deltas) et ont comme particularité d'avoir, comme toutes les zones humides, un niveau de nappe d'eau élevé, conséquence d'un bilan hydrique positif, et donc une zone anaérobie importante.
Ceci induit le développement de communautés microbiennes et végétales spécifiques, adaptées aux milieux humides ou inondés.

Les tourbières représentent 50 à \SI{70}{\percent} des zones humides \cite{joosten2002}.
Leur définition est variable selon les régions.
Deux définitions sont régulièrement utilisées :

\begin{pdef}
\textsc{Tourbière} :

Écosystème, avec ou sans végétation, possédant au moins \SI{30}{\cm} de tourbe naturellement accumulée.

\hfill {\scriptsize Définition traduite d'après \citet{joosten2002}}
\end{pdef}
Cette première définition correspond au \textit{peatland} anglo-saxon.
L'épaisseur de tourbe accolée à cette définition peut varier selon le pays, elle est par exemple établie à \SI{40}{\cm} au Canada \citep{nationalwetlandsworkinggroup1997}.
Une autre définition existe :

\begin{pdef}
\textsc{Tourbière active} :

Écosystème dans lequel un processus de tourbification est actif.

\hfill {\scriptsize Définition traduite d'après \citet{joosten2002}}
\end{pdef}
Cette seconde définition correspond au \textit{mire} anglo-saxon et peut être traduite en français par le terme de tourbière active.
Les concepts derrière ces deux définitions se chevauchent mais ne sont pas complètement similaires : une tourbière drainée peut, par exemple, avoir plus de \SI{30}{cm} de tourbe et ne plus former de tourbe, ne plus être active.
À l'inverse il peut exister des zones où l'épaisseur de tourbe est inférieure à \SI{30}{cm} malgré un processus de tourbification actif.
Un même écosystème tourbeux peut d'ailleurs contenir à la fois des zones qui correspondent à la première définition et d'autres à la seconde.
Les tourbières sont donc, selon la définition utilisée, des écosystèmes contenant ou des écosystèmes formant de la tourbe.
Mais qu'est-ce que la tourbe ?

\begin{pdef}
\textsc{Tourbe} :

«Accumulation sédentaire\footnotemark de matériel composé d'au moins \SI{30}{\percent} (matière sèche) de matières organiques mortes.»

\hfill {\scriptsize Définition traduite d'après \citet{joosten2002}}
\end{pdef}
\footnotetext{\citet{joosten2002} distinguent sédimentaire de sédentaire dans le sens où dans le premier cas la matière migre (dans la colonne d'eau par exemple) entre la zone où elle est produite et la zone où elle est stockée, ce qui n'est pas le cas pour le second cas où ces zones sont confondues.}

Le seuil de \SI{30}{\percent} est souvent utilisé pour rapprocher sa définition de celle d'un sol organique (histosol) au sens large, dans lequel est classée la majorité des sols tourbeux (selon la classification).
D'autres définitions existent, faisant la distinction entre sols organiques et tourbes avec un seuil à \SI{75}{\percent} \citep{andrejko1983} ou \SI{80}{\percent} \citep{landva1983}.
Il est également nécessaire de préciser qu'au delà de la classification utilisée, ce que les écologues considèrent comme de la tourbe contient généralement \SI{80}{\percent} de matières organiques au minimum \citep{rydin2013b}.
Ce processus de formation est appelé la tourbification\index{tourbification} ou turfigénèse et les matières organiques accumulées proviennent majoritairement de la végétation.
On définit les matières organiques de la façon suivante : 
\begin{pdef}
\textsc{Matières organiques} :

Matières constituées d'un assemblage d'éléments ayant une ou plusieurs liaison C--H formant de nombreux composés organiques dont des carbohydrates (sucres, cellulose \dots), des composés azotés (protéines, acides aminés \dots) et phénoliques (lignine \dots), des lipides (cires, résines, \dots) et d'autres\footnotemark.

\end{pdef}
\footnotetext{Cette définition, utile pour définir simplement les matières organiques, est cependant limitée car elle inclut des composés traditionnellement considérés comme minéraux (le graphite) et en exclut d'autres considérés comme organiques (acide oxalique) (Liste de diffusion ResMO (Réseau Matières Organiques \url{http://www6.inra.fr/reseau_matieres_organiques})).}

\subsubsection{Distribution des tourbières à l'échelle mondiale}

L'hétérogénéité des définitions ajoutée aux limites floues qui peuvent exister entre certains écosystèmes tourbeux et non-tourbeux rendent la cartographie de ces écosystèmes délicate.
Les estimations généralement citées évaluent la surface occupée par l'ensemble des tourbières à environ \SI{4000000}{\square\kilo\meter} \citep{lappalainen1996}.\index{tourbières!surface} 
Cette surface correspond à \num{2} à \SI{3}{\percent} de l'ensemble des terres émergées du globe.
Plus de \SI{85}{\percent} d'entre elles sont situées dans l'hémisphère nord, majoritairement dans les zones boréales et sub-boréales (\citealp{strack2008} et figure~\ref{fig:peatlandGlobalDistribution}).
Ce travail sera focalisé sur ces écosystèmes caractérisés par la présence importante de sphaignes.
Les sphaignes sont des bryophytes\footnote{Les bryophytes sont des végétaux caractérisés par un système vasculaire absent. Ces plantes n'ont pas de racines mais des rhizoïdes. On les appelle communément des mousses.} de la famille des \textit{Sphagnaceae}.
Les tourbières des forêts tropicales ne seront donc pas considérées, ces dernières ayant un fonctionnement spécifique.
En effet malgré des températures importantes, elle maintiennent un bilan hydrique positif grâce à des précipitations très importantes \citep{chimner2005}.

\begin{figure}
\centering
\includegraphics[width=\textwidth]{chap1/peatlandGlobalDistribution}
\caption{Distribution mondiale des tourbières en pourcentage de surface recouverte.}
\label{fig:peatlandGlobalDistribution}\index{tourbières!distribution} 
\end{figure}

\subsubsection{La formation des tourbières}

\begin{figure}
\centering
\includegraphics[width=\textwidth]{chap1/peat_formation}
\caption{Processus de formation des tourbières, à gauche l'atterrissement et à droite la paludification. Modifié d'après \citet{manneville1999}}
\label{fig:peat_formation}
\end{figure}

\index{tourbières!formation} 
Pour former une tourbière il faut la réunion de deux conditions majeures : un bilan hydrique positif (permettant de maintenir un niveau de nappe élevé et une anaérobie importante du milieu), et une décomposition des litières végétales plus lente que sa production.
Ces deux conditions sont réunies dans les deux processus de formation des tourbières généralement distingués : l'atterrissement\index{atterrissement} et la paludification\index{paludification} (Figure~\ref{fig:peat_formation}).
Il s'agit pour le premier du comblement progressif d'une zone d'eau stagnante.
Ce comblement est généralement lié à l'action combinée d'apports exogènes et d'une végétation colonisant les eaux en formant des tremblants\footnote{Radeau végétal, composé de végétation vivante et de débris qui peuvent masquer la surface de l'eau}.
La paludification est la formation de tourbe directement sur un sol minéral, grâce à des conditions d'humidité importante dans des zones peu perméables et topographiquement favorables (dépressions).
Ces modes de formation ne sont pas mutuellement exclusifs : une tourbière peut être le siège de l'un ou l'autre des processus, ou des deux, selon la zone spatiale ou la période de temps considérée.

\subsubsection{Classifications et terminologies}

Selon les disciplines, différentes classifications sont utilisées pour différencier ces tourbières à sphaignes.
La plus générale et la plus utilisée dans la littérature distingue les tourbières dite hautes, ou de haut-marais \textit{bog}, et les tourbières basses, ou de bas-marais \textit{fen}.
Ces deux catégories doivent davantage être vues comme un continuum plutôt qu'une séparation franche.

Les tourbières de haut-marais sont alimentées principalement par les précipitations : elles sont dites ombrotrophes.
Leur surface parfois bombée (tourbières élevées ou bombées) peut également être plate ou en pente.
Cette géométrie situe une partie au moins de l'écosystème au dessus du niveau de la nappe.
Elles ont une concentration en nutriments relativement faible (oligotrophes) et renferment des eaux acides dont le pH est compris entre \num{3.5} et \num{4.2}.
Les végétations dominantes sont constituées de sphaignes, de linaigrettes, et de petits arbustes  \citep{francez2000a,rydin2013}.

Les tourbières de bas-marais sont alimentées en eau par des nappes souterraines ou des eaux de ruissellement : elles sont dites minérotrophes.
Elles ont généralement un niveau de nappe très proche de la surface du sol et sont généralement de forme concave ou en pente.
Elles sont riches en nutriments (notamment en azote et phosphore) et le pH de leurs eaux de surface varie de 4 à 8.
Les végétations dominantes de ces écosystèmes peuvent être des bryophytes, des graminées ou des arbustes bas \citep{rydin2013}.

Au sein de ces écosystèmes la topographie est fortement variable et fait l'objet d'une terminologie particulière : on parle de buttes (\textit{hummock}) pour désigner des sur-élévations topographiques, de gouilles (\textit{hollow}) pour les dépressions et de replats (\textit{lawn}) pour les zones entre les deux (Figure~\ref{fig:microtopo}).
Ces différences micro-topographiques entraînent des différences de composition végétale, ainsi certaines espèces de sphaignes se développent préférentiellement sur les buttes (\textit{Sphagnum fuscum}) et d'autres dans les gouilles (\textit{Sphagnum cuspidatum}).


\begin{figure}[t]
\centering
\includegraphics[width=\textwidth]{chap1/microtopo}
\caption{Micro-topographie dans les tourbières. Modifié d'après \citet{rydin2013}}
\label{fig:microtopo}
\end{figure}

\subsection{Tourbières et fonctions environnementales}

\subsubsection{Fonction puits de carbone}
\index{carbone!stock}
Par définition les tourbières stockent ou ont stocké du carbone.
Cette fonction puits de carbone rend ces écosystèmes importants vis-à-vis des changements globaux malgré la faible surface qu'ils représentent (pour rappel 2 à \SI{3}{\percent} des terres émergées).
En effet le carbone stocké dans les tourbières tempérées et boréales est estimé entre 270 et \SI{455}{\giga\tonne\,C} (Tableau~\ref{table:CCycleStocks}).
Cela représente 10 à \SI{25}{\percent} du carbone présent dans les sols et entre 30 et \SI{60}{\percent} du stock de carbone atmosphérique.
Ce stock est un héritage datant des 10 derniers milliers d'années, l'holocène, période pendant laquelle s'est formée la majorité des tourbières \citep{yu2010,macdonald2006} (Figure~\ref{fig:holo_peat_ini}).

\begin{figure}
\centering
\includegraphics[width=\textwidth]{chap1/holo_peat_ini}
\caption{Nombre d'initiations de tourbières dans l'hémisphère nord, pendant l'holocène. Modifié d'après \citep{macdonald2006}.}
\label{fig:holo_peat_ini}
\end{figure}

L'accumulation du carbone nécessite donc que davantage de carbone soit assimilé, par photosynthèse, qu'émis par l'écosystème.
La production végétale des tourbières n'est pas particulièrement élevée \citep{huc1980} et n'explique pas l'accumulation du carbone.
La décomposition des litières végétales est en revanche plus faible que dans d'autres écosystèmes \citep{rydin2013}.
Ceci est rendu possible par les niveaux de nappe élevés de ces écosystèmes, minimisant les processus de dégradation aérobie en limitant l'accès à l'oxygène.
Cet effet est de plus renforcé par la végétation spécifique de ces écosystèmes, les sphaignes, qui produisent des litières difficilement dégradables, dites récalcitrantes, par rapport à celles produites par les végétaux vasculaires \citep{hobbie1996,liu2000,bragazza2007}.

\begin{table}
\centering
\caption{Estimations des stocks de C pour différents environnements}
\label{table:CCycleStocks}
\hspace*{-.75cm}
\begin{tabular}{llp{10cm}}\toprule
Compartiment & Stock (\si{\pgc}) & Références \\ \midrule
Tourbières & 270 -- 455 & {\footnotesize \citet{gorham1991,turunen2002}} \\ 
Végétation & 450 -- 650 &{\footnotesize  \citet{Robert2003}}\\ 
Sols & 1500 -- 2000 & {\footnotesize \citet{Post1982,Robert2003,Eswaran1993}}\\ 
\coo atmosphérique & 750 -- 800 & {\footnotesize \citet{Robert2003}}\\ 
Permafrost & 1700 &  {\footnotesize \citet{tarnocai2009}}\\ 
\bottomrule
\end{tabular}
\hspace*{-1cm}
\end{table}

\subsubsection{Végétation et biodiversité des tourbières}

Les sphaignes sont la végétation caractéristique des tourbières.
Ce sont des espèces dites ingénieures, capables de modifier l'environnement dans lequel elles se développent dans le but d'obtenir un avantage compétitif sur les autres végétaux \citep{vanbreemen1995}.
Les sphaignes sont notamment capables de capter les nutriments apportés par les précipitations via leur capitulum\footnote{partie apicale de la plante} et donc avant que ceux-ci n'atteignent les racines des plantes vasculaires \citep{malmer1994,svensson1995}.
Elles ont également la capacité de stocker ces nutriments ce qui diminue encore ceux qui seront disponibles dans le milieu \citep{rydin2013d}.
En plus de favoriser un environnement pauvre en nutriments, les sphaignes promeuvent un environnement acide en abaissant le pH.
Ces contraintes (pauvreté en nutriments et acidité) défavorisent l'implantation d'espèces peu tolérantes.

Malgré tout, ces écosystèmes sont le siège d'une biodiversité importante d'espèces végétales adaptées à ces milieux.
Parmi les plus répandues des graminoïdes (\textit{Eriophorum} spp., \textit{Scirpus cespitosus}, \textit{Rynchospora alba}, \textit{Carex} spp.) des arbustes (\textit{Erica tetralix}, \textit{Calluna vulgaris}, \textit{Andromeda polifolia}, \textit{Vaccinium} spp.) et bien d'autres encore 
 : des Carex (\textit{lasiocarpa}, \textit{rostrata}) des herbacées (\textit{Molinia caerulea}) des Phragmites (\textit{Phragmites australis}) Joncs (\textit{Juncus bulbosus}) et d'autres \citep{rydin2013dofip}

\subsubsection{Autres fonctions environnementales}

Les tourbières jouent également un rôle important vis-à-vis du cycle de l'eau.
Elles permettent par exemple de tamponner les effets d'une sécheresse ou d'une inondation en fournissant un peu d'eau dans le premier cas et en retenant une partie des excédents dans le second, régulant ainsi les écoulements d'eau \citep{joosten2002,parish2008}.
Elles ont également un effet sur la qualité de l'eau notamment en filtrant les matières en suspensions, en dégradant certains micro-polluants organiques et en fixant des métaux et métalloïdes grâce à leur forte capacité d'échange cationique.

\subsection{Les tourbières et les changements globaux}
On définit les changements globaux comme l'ensemble des modifications environnementales plus ou moins rapides, ayant lieu à l'échelle mondiale, quelle que soit leur origine. Les deux contraintes développées dans cette partie sont la pression de l'homme : contrainte anthropique, et celle du climat : contrainte climatique.
\index{changements globaux}

\subsubsection{Les contraintes anthropiques}
\index{tourbières!utilisation} 

Les interactions entre les Hommes et les zones humides au sens large et les tourbières en particulier remontent probablement à l'aube de l'humanité.
Des chemins de rondins néolithiques aux crannogs de l'époque romaine \citep{buckland1993}, de grandes découvertes archéologiques ont été faites dans les écosystèmes tourbeux témoins d'époques révolues.
L’utilisation de la tourbe et des tourbières a dû commencer relativement tôt, mais c'est à partir du 17\textsuperscript{e} siècle que le drainage de ces écosystèmes, pour les convertir en terres agricoles, s'est intensifié.
Au 19\textsuperscript{e} siècle, l'apparition de machines permettant une récolte industrialisée de la tourbe a développé son utilisation comme combustible.
Enfin depuis le milieu du 20\textsuperscript{e} siècle une part importante de ces écosystèmes a été drainée pour développer la sylviculture.
Aujourd'hui l'exploitation principale de la tourbe est liée à son utilisation comme substrat horticole \citep{lappalainen1996,chapman2003}.
Ces utilisations les ont fortement perturbés car elles nécessitent généralement de drainer ces écosystèmes, notamment pour pouvoir y faire rouler des engins mécanisés.
Aujourd'hui la surface de tourbières altérées est estimée à \SI{490000}{\square\kilo\metre} environ, principalement du fait de leur reconversion pour l'agriculture et la sylviculture (\citealp{joosten2002} et tableau~\ref{table:tourbeUsage}).
En France, suite à leur utilisation, principalement agricole, la surface des tourbières a été divisée par deux entre 1945 et 1998, passant de \SI{1200}{\square\kilo\meter} à \SI{600}{\square\kilo\meter} \citep{lappalainen1996,manneville1999}.

Le fonctionnement de ces écosystèmes a donc été et est encore perturbé par différentes activités humaines.
Leur importance est cependant reconnue et elles sont l'objet de nombreuses actions de préservation et/ou de réhabilitation.
\begin{table}[]
\centering
\caption{Surface de tourbe utilisée selon les usages considérés (tourbières non-tropicales). Modifié d'après \citet{joosten2002}.}
\label{table:tourbeUsage}
\begin{tabular}{lll}\toprule
Utilisation & Surface (\si{\square\kilo\meter})  & proportion (\%) \\ \midrule
Agriculture & \num{250000} & \num{50} \\ 
Sylviculture & \num{150000} & \num{30}\\ 
Extraction de tourbe & \num{50000} & \num{10}\\ 
Urbanisation & \num{20000} & \num{5}\\ 
Submersion & \num{15000} & \num{3}\\ 
Pertes indirectes (érosion, ...) & \num{5000} & \num{1}\\[1ex]
Total & \num{490000} & \num{100}\\
\bottomrule
\end{tabular}
\end{table}


\subsubsection{Les contraintes climatiques}

Comme indiqué précédemment, le stock de carbone des tourbières s'est majoritairement constitué pendant l'Holocène.
À cette époque déjà ces écosystèmes étaient influencés par le climat, et leur développement n'a pas été linéaire sur les douze derniers milliers d'années.
Il est reconnu que le développement des tourbières est très important au début de cette période \citep{smith2004,macdonald2006,yu2009}.
Plus particulièrement, entre \num{12000} et \num{8000} ans on recense la plus grande proportion d'initiation de tourbières (Figure~\ref{fig:holo_peat_ini}).
Cette période coïncide avec le maximum thermique holocène (HTM), période pendant laquelle le climat était plus chaud qu'aujourd’hui \citep{kaufman2004}.
Ce constat peut sembler paradoxal : en effet, dans la littérature concernant les tourbières et le réchauffement climatique actuel, il est craint que ces écosystèmes ne deviennent des sources de carbone.
Cependant ces mêmes auteurs qui ont montré cette relation entre le HTM et le développement important des tourbières, ne préjugent pas de l'effet du réchauffement actuel.
Notamment \citet{jones2010} expliquent que pendant cette période de maximum thermique, il existe également une saisonnalité très importante, avec des été chauds et des hivers froids, qui a dû en minimisant la respiration hivernale de ces écosystèmes, jouer un rôle important dans leur développement.
Cette forte saisonnalité n'est pas attendue lors du réchauffement actuel.
L'effet estimé sous les hautes latitudes semble plus important pendant l'hiver et l'automne, et tendrait donc à minimiser cette saisonnalité \citep{christensen2007}.
Les effets directs attendus du réchauffement %sous les hautes latitudes 
à l'horizon 2100, sont une augmentation des températures de 2 à \SI{8}{\degreeCelsius} dans les zones boréales, et de 2 à \SI{6}{\degreeCelsius} dans les zones tempérées, ainsi  qu'une augmentation probable des précipitations (Figure~\ref{fig:ipcc2013_T_rain}).
De façon plus indirecte sont attendus la fonte du permafrost, l'augmentation de l'intensité et de la fréquence de feux et des changements dans le recouvrement des communautés végétales \citep{christensen2013,frolking2011}.

\begin{figure}
\centering
\includegraphics[width=\textwidth]{chap1/ipcc2013_RCP45}
\caption{Projection des changements à l'horizon 2100, des moyennes et extrêmes annuels des températures de l'air et des précipitations : (a) température de surface moyenne par \si{\degreeCelsius} de changement global moyen, (b) 90\textsuperscript{e} percentile des températures journalières maximum par \si{\degreeCelsius} de changement de température moyenne maximale, (c) précipitations moyennes (en \si{\percent} par \si{\degreeCelsius} de changement de température moyenne) et (d) fraction de jours ayant des précipitations dépassant le 95\textsuperscript{e} percentile. Sources : (a) et (c) simulations CMIP5, scénario RCP4.5, (b) et (d) adaptation d'après \citet{orlowsky2012} dans \cite{christensen2013}.}
\label{fig:ipcc2013_T_rain}
\end{figure}

Les tourbières, qui ont accumulé un stock de carbone important, sont donc soumises à des contraintes fortes qu'elles soient anthropiques ou climatiques.
Afin de mieux comprendre le devenir de ce carbone, l'étude de ces écosystèmes, des flux de carbone qu'ils échangent avec l'atmosphère, est une nécessité.

\index{tourbières|)}

\singlespacing
\section{Flux de gaz à effet de serre et variables explicatives}
\doublespacing

Cette partie décrit dans un premier temps les relations entre les GES (\coo et \chh) et les tourbières, puis les facteurs qui contrôlent ces flux dans ces écosystèmes et enfin les bilans de carbone qui ont pu y être estimés.

\subsection{Les flux de GES entre l'atmosphère et les tourbières}

\subsubsection{Le \coo et le \chh dans l'atomsphère}

Dans l'atmosphère le carbone est principalement présent sous forme de dioxyde de carbone (\coo) et de méthane (\chh).
La concentration en \coo dans l'atmosphère fluctuait avant l'ère industrielle entre 180 et \SI{290}{ppm}.
En 1750 au début de l'ère industrielle sa concentration était de \SI{280}{ppm} environ avant d'augmenter pour atteindre \SI{391}{ppm} aujourd'hui (moyenne annuelle en 2011) \citep{Ciais2014}.
Différents processus naturels permettent d'extraire du \coo de l'atmosphère : la photosynthèse, la dissolution du \coo dans l'océan\footnote{$ CO_{2} + CO_{3}^{2-} + H_{2}O \rightleftharpoons 2HCO_{3}^{-} $}, les réactions avec les carbonates de calcium\footnote{$ CO_{2} + CaCO_{3} + H_{2}O \rightarrow Ca^{2+} + 2HCO_{3}^{-} $} et enfin l'altération de silicate et les réactions avec le carbonate de calcium\footnote{$ CO_{2} + CaSiO_{3} \rightarrow CaCO_{3} + SiO_{2} $}.
L'importance de ces processus varie selon l'échelle de temps considérée.
Pour une émission de \coo idéalisé de \SI{100}{\pgc}, \SI{60}{\percent} de ce \coo sera extrait de l'atmosphère en un siècle par l'effet combiné de la photosynthèse et des océans, ce qui laisse \SI{40}{\percent} de l'émission initiale dans l'atmosphère.
À l'horizon \num{1000} ans \SI{20}{\percent} de l'émission initiale sera toujours dans l'atmosphère et à \num{10000} ans, \SI{10}{\percent} (\citealp{joos2013,Ciais2014} et figure~\ref{fig:co2_decroissance}).

\begin{figure}
\centering
\includegraphics[width=\textwidth]{chap1/co2_decroissance}
\caption{Décroissance de la proportion de \coo de l'atmosphère suite à une émission idéalisée de \SI{100}{\peta\gram C}. les graphes (a) et (b) sont une moyenne de modèles \citep{joos2013}, le graphe (c) est une moyenne d'autres modèles \citep{archer2009}. Modifié d'après \citep{Ciais2014}.}
\label{fig:co2_decroissance}
\end{figure}

La concentration en \chh dans l'atmosphère est estimée à \SI{350}{ppb}\footnote{Partie par milliard (\textit{part per billion} en anglais)} il y a \SI{18000}{ans} environ lors de la dernière glaciation et à \SI{720}{ppb} en 1750.
En 2011 elle est estimée à \SI{1800}{ppb} \citep{Ciais2014}.
À l'inverse du \coo sa durée de vie dans l'atmosphère est limitée : moins de \SI{10}{ans} \citep{lelieveld1998,prather2012}.
Cependant son potentiel de réchauffement global\footnote{indice permettant de comparer le pouvoir de réchauffement des différents GES en donnant une équivalence par rapport au \coo. Le PRG du \coo vaut donc 1 par définition.} (PRG) est important notamment à court terme, 72 à 20 ans.
À plus long terme son effet relativement au \coo diminue et atteint 25 à l'horizon 100 ans.
Les zones humides sont la première source naturelle de \chh atmosphérique avec un flux à l'échelle globale estimé entre \num{145} et \SI{285}{\tera\gram\per\year} \citep{lelieveld1998,wuebbles2002,Ciais2014}. %\textbf{(Tableau ?)}.
Les tourbières de l'hémisphère nord émettent environ \SI{46}{\tera\gram\per\year} \citep{gorham1991}. % \textbf{(pas de source plus récente ?)}.

À l'échelle globale et pour l'ensemble des flux, le stockage du C par les tourbières est estimé à \SI{70}{\tera\gram\per\year} \citep{clymo1998}.

\subsubsection{De l'atmosphère à l'écosystème}

\begin{figure}
\centering
\includegraphics[width=\textwidth]{chap1/ges_flux}
\caption{Schéma des flux de carbone entre une tourbière et l'atmosphère, avec RE la respiration de l'écosystème, PPB, la production primaire brute, FCH4 le flux de \chh. Les sources de la respiration hétérotrophe (RH) sont l'oxydation du \chh (a), la respiration des organismes (b et c)  et la respiration hétérotrophe liée à la rhizosphère (d). La respiration autotrophe (RA) comprend la respiration des parties souterraines (e) et aériennes (f) et l'assimilation de carbone se fait par photosynthèse (g). Enfin le \chh peut être transporté via l'aérenchyme des plantes (h), via ébullition (i) ou diffusion (j). Les paramètres mesurés lors de ce travail sont la RE, l'ENE (qui est la différence entre PPB et RE), et F\chh.}
\label{fig:ges_flux}
\end{figure}

Le transfert du carbone de l'atmosphère à la tourbe se fait par le processus de photosynthèse\index{photosynthèse}, où le \coo est assimilé dans la matière organique\footnote{Il existe d'autres voies métaboliques permettant la capture du \coo de l'atmosphère.
Par exemple les micro-organismes chemolithotrophes sont capables d'assimiler le \coo en utilisant l'énergie issue de l'oxydation de composés inorganiques, ce que l'on appelle la chimiosynthèse, mais leur importance est négligeable.}.
Principalement par les végétaux vasculaires et les mousses, et éventuellement, bien que dans de moindres proportions, par des algues, des lichens ou des bactéries photosynthétiques \citep{girard2011}.
On peut écrire la réaction de photosynthèse de la façon suivante : 

$$\begin{aligned}
6CO_{2} + 6H_{2}O  & \rightarrow C_{6}H_{12}O_{6} + 6O_{2}\\
\end{aligned} $$


On définit la \textbf{Production Primaire Brute} (PPB), \textit{Gross Primary Production}, (\textit{GPP}) comme :

\begin{pdef}
\textsc{Production Primaire Brute (PPB)} :

Quantité de carbone extraite de l'atmosphère (principalement sous forme de \coo via la photosynthèse).
Ce carbone est en partie respiré et en partie transformé en matières organiques.
\end{pdef}

Les tourbières sont des écosystèmes dont la production primaire est estimée à environ \SI{500}{\gcm} ;
la production de la strate muscinale pouvant atteindre \SI{80}{\percent} \citep{francez2000a}.
La production primaire des tourbières n'est pas élevées.
C'est la faible décomposition des matières organiques qui permet aux tourbières de stocker du carbone.
Du fait de la production élevée de \chh dans les tourbières, il n'y a pas de flux significatif direct de \chh de l'atmosphère vers cet écosystème.
\SI{90}{\percent} du \chh présent dans l'atmosphère est oxydé lors de réactions avec des radicaux hydroxyles ayant lieu majoritairement dans la troposphère.

\subsubsection{De l'écosystème à l'atmosphère}

Les sources de carbone émis par les tourbières vers l'atmosphère sont multiples.
D'abord différents gaz peuvent être émis, notamment le \coo et le \chh et des molécules de carbone organique volatiles.
Le processus majeur de production de \coo se fait par respiration qui, au niveau cellulaire, peut être écrit sous la forme :

$$\begin{aligned}\label{eq:respi}
C_{6}H_{12}O_{6} + 6O_{2} &\rightarrow 6CO_{2} + 6H_{2}O \\
\end{aligned} $$

Ce gaz est produit principalement par la respiration aérobie et minoritairement par les respirations anaérobies, par fermentations (e.g. du glucose, de l'acétate), ou encore par oxydation du \chh \citep{lai2009}.
Les principales sources d'émissions du \coo sont représentées dans la figure~\ref{fig:ges_flux}.
À l'échelle macroscopique la respiration est séparée en deux.
D'un côté la respiration végétale (des feuilles, des tiges, des racines) que l'on appelle la \textbf{respiration autotrophe
}.
De l'autre, rassemblé sous le terme de \textbf{respiration hétérotrophe}, la respiration du sol, liée à l'excrétion d'exsudats par les racines, la décomposition des litières et des matières organiques par les micro-organismes et l'oxydation du \chh par les organismes méthanotrophes.

L'ensemble de ces respirations est défini comme : 
\begin{pdef}
\textsc{Respiration de l'Écosystème (RE)} :

Quantité de carbone émis sous forme de \coo par l'écosystème tourbeux dans l'atmosphère. 
Elle englobe la respiration autotrophe et hétérotrophe en incluant ses composantes aériennes et souterraines.
Ce flux est exprimé en quantité de carbone par unité de surface et de temps.
\end{pdef}
\index{respiration!de l'écosystème}
On distingue la respiration de l'écosystème de celle du sol en définissant la respiration du sol (RS) comme l'ensemble des respirations de la colonne de sol, à l'exclusion de la partie aérienne \citep{luo20063}.\index{respiration!du sol}
Cependant, dans la littérature la respiration du sol peut parfois être assimilée à la respiration de l'écosystème (RE)\citep{raich1992}.
Les études discriminant RS et RE montrent ainsi que dans des sols tourbeux, RS compte pour plus de \SI{60}{\percent} de RE \citep{lohila2003}.
La production de \coo est donc un signal multi-sources intégré sur l'ensemble de la colonne de tourbe.
Le transport du \coo produit se fait par diffusion suivant le gradient de concentration, fort dans le sol et plus faible dans l'atmosphère et par convection (gradient de température).
C'est cette multitude de processus qui rend l'analyse des variations spatio-temporelles de ce flux difficile.
En effet chacun des processus de respiration n'a pas la même sensibilité vis à vis de facteurs environnementaux.

Conséquence du niveau de nappe élevé des tourbières, le développement d'une zone anoxique importante dans la colonne de sol favorise la production de \chh.
Il est produit par des \textit{Archaea} méthanogènes, des organismes anaérobies vivants sous le niveau de la nappe \citep{garcia2000}.
En moyenne les flux de \chh mesurés dans les tourbières s'étendent de 0 à plus de \SI{0.96}{\uml}, avec généralement des flux compris entre \num{0.0048} et \SI{0.077}{\uml} \citep{blodau2002}.
Le \chh est principalement produit à partir d'acétate (CH\textsubscript{3}COOH) ou de hydrogène (H\textsubscript{2}) + \coo, ces deux composés étant dérivés de la décomposition préalable de matières organiques \citep{lai2009}.

$$\begin{aligned}
CH_{3}COOH  &\rightarrow CH_{4} + CO_{2}\\
4H_{2} + CO_{2} &\rightarrow CH_{4} + 2H_{2}O\\
\end{aligned} $$
Le \chh produit est transporté dans l'atmosphère par diffusion, convection ou ébullition (essentiellement à travers certaines plantes) \citep{joabsson1999,colmer2003}.
Pendant son transport, le \chh peut être oxydé par des organismes méthanotrophes.
Cette transformation produit tour à tour différents composés (méthanol, formaldéhyde, formate) aboutissant à la production de \coo \citep{whalen2005}.

$$
CH_{4} \rightarrow CH_{3}OH \rightarrow HCHO \rightarrow HCOOH \rightarrow CO_{2} \\
$$

On définit le flux net de \chh comme : 
\begin{pdef}
\textsc{Flux net de \chh (\fchh)} :

Quantité de carbone émise sous forme de \chh par l'écosystème dans l'atmosphère, suite au bilan des processus le produisant et le dégradant.
Ce flux est exprimé en quantité de carbone par unité de surface et de temps.
\end{pdef}

Au final, on peut noter que si le flux de carbone de l'atmosphère à l'écosystème a pour source quasiment unique la réaction de photosynthèse des plantes, le flux de carbone de l'écosystème vers l'atmosphère est multi-sources avec un nombre important de réactions de respirations et de fermentations.
La variabilité du premier flux résulte majoritairement de la nature et la structure des communautés végétales et de leurs sensibilités aux conditions environnementales.
Celle du second flux est multiple et est liée à la diversité des réactions permettant la dégradation des matières organiques et des communautés végétales ou microbiennes impliquées, de leur sensibilité aux conditions environnementales.

\subsection{Les facteurs majeurs contrôlant les flux de GES}

Dans cette partie seront décrits les facteurs qui contrôlent les flux de carbone en commençant à une échelle relativement fine pour atteindre celle de l'écosystème qui nous intéresse plus particulièrement.

Les facteurs majeurs qui contrôlent les flux de carbone sont globalement connus.
Comme bon nombre de réactions biochimiques, les vitesses de réactions des processus décrits précédemment sont fonction de la \textbf{température}.
Cette relation  a été mise en évidence par un chimiste suédois en 1889, Svante August Arrhenius, sur la base de travaux réalisés par un autre chimiste, néerlandais, Jacobus Henricus Van't Hoff.
Le \textbf{niveau de la nappe d'eau}, interface entre une zone oxique et une zone anoxique, et la \textbf{teneur en eau du sol} vont également influencer sur ces flux.
De même que la végétation, que ce soit de façon directe, comme siège de la photosynthèse et de la respiration autotrophe, ou indirecte en fournissant des nutriments via les exsudations racinaires et les litières.

\subsubsection{Facteurs contrôlant la photosynthèse}
\index{production primaire brute!contrôle}

\begin{figure}
\centering
\includegraphics[width=.8\textwidth]{chap1/PAR_photo}
\caption{Réponse idéalisée des vitesses d'assimilation du \coo en fonction flux de photons photosynthétiques. Sur la première partie de la courbe (partie linéaire) la vitesse d'assimilation du \coo est contrôlée par la vitesse de régénération du RuBP (Ribulose-1,5-bisphosphate) tandis que sur la seconde partie c'est l'activité d'une enzyme, la Rubisco (ribulose-1,5-bisphosphate carboxylase/oxygénase) qui la contrôle. Modifié d'après \citet{long1993}}
\label{fig:PAR_photo}
\end{figure}

À l'échelle des espèces végétales, la quantité de carbone assimilable par la photosynthèse est fonction de la quantité de lumière reçue \citep{long1993}.
La quantité de carbone assimilée augmente d'abord de façon linéaire avec le rayonnement, avant d'être limitée par l'activité d'une enzyme, la Rubisco\footnote{ribulose-1,5-bisphosphate carboxylase/oxygénase}, nécessaire à la fixation du \coo (Figure~\ref{fig:PAR_photo}).
Les limitations de l'assimilation, que ce soit la pente initiale de la partie linéaire, ou l'assimilation maximale, varient de façon importante en fonction de l'espèce végétale considérée \citep{wullschleger1993}.
La régénération de la Rubisco, qui limite la photosynthèse, est contrainte par la capacité de transport des électrons.
La vitesse de ce transport est fonction de la température et est traditionnellement décrite par une équation d'arrhenius modifiée, relativement complexe, ou par une équation simplifiée \citep{farquhar1980,june2004}.
À cette échelle, le niveau de l'eau va également influencer le développement de la végétation en facilitant plus ou moins leur accès à l'eau.
\citet{wagner1984} montrent par exemple que deux espèces de sphaignes ont des tolérances différentes à la dessiccation : l'espèce vivant dans les gouilles est plus résistante que celle vivant sur les buttes.
Dans des conditions expérimentales différentes, lors de re-végétalisation de deux tourbières, \cite{robroek2009} montrent que différentes espèces de sphaignes se développent de façon optimale à différents niveaux de nappe selon leurs affinités.
Cette variabilité entre espèces d'une même famille est elle même mise en évidence par leur variabilité en terme de productivité primaire (Figure~\ref{fig:prod_sphagnum}).
La productivité primaire varie également entre différentes communautés végétales : les bryophytes n'ont pas la même productivité primaire que les graminées ou que les arbustes \citetext{\citealp{moore2002} dans \citealp{rydin2013b}}.

Le niveau de la nappe d'eau et les propriétés physiques du sol contraignent également la teneur en eau du sol et la hauteur de la frange capillaire.
Cette dernière atteint généralement la surface du sol tant que le niveau de la nappe d'eau ne descend pas en dessous de \num{30} à \SI{40}{\centi\metre} de profondeur \citep{laiho2006}.
La hauteur du niveau d'eau va influencer le développement des différentes communautés végétales.
Un niveau d'eau élevé peut diminuer l'accès de la végétation vasculaire à l'oxygène par leurs racines alors qu'il sera propice au développement de sphaignes.
À l'inverse un niveau d'eau bas peut faciliter le développement de certains végétaux vasculaires au détriment des bryophytes.
Cette compétition entre espèces végétales peut déterminer l'évolution à long terme des communautés et impacter la PPB.
\citet{gornall2011} montrent que les effets des bryophytes sur le développement des plantes vasculaires sont en partie positifs et en partie négatifs ; les effets négatifs étant de plus en plus prépondérants quand l'épaisseur de la strate muscinale augmente.
La composition des communautés végétales va donc avoir un effet sur le potentiel photosynthétique de l'écosystème.
Ce potentiel peut varier selon le végétal considéré et les conditions environnementales dans lesquelles il se trouve \citep{moore2002}.

\begin{figure}
\centering
\includegraphics[width=\textwidth]{chap1/prod_sphagnum}
\caption{Productivité moyenne des espèces de sphaignes en grammes de matières sèches par mètre carré et par année.
Les barres d'erreurs représentent l'erreur standard. Le nombre d'observations est indiqué par les nombres à l'intérieur des barres. Les espèces en orange sont celles rencontrées sur le site d'étude. Modifié d'après \citet{gunnarsson2005}}
\label{fig:prod_sphagnum}
\end{figure}

À l'échelle de l'écosystème dans son ensemble la température, la végétation et le niveau de l'eau co-varient, ce qui rend la discrimination de leurs effets respectifs difficile.
L'effet d'une variation de température peut, selon l'échelle de temps considérée, influencer le niveau de nappe et la végétation.
Dans l'optique de discriminer l'effet de chacun de ces facteurs, \citet{munir2015} isolent l'effet de la température en utilisant des OTC\footnote{OTC ou chambres à toit ouvert, ce sont des hexagones en polycarbonate permettant un rehaussement \textit{in-situ} de la température moyenne de l'air.} (\textit{Open Top Chamber}).
Ils montrent que le réchauffement par les OTC augmente la PPB.
Néanmoins la majorité des études réalisées sur le terrain montrent les effets de variation de la température et du niveau de la nappe simultanément.
\citet{cai2010} ont par exemple montré que les conditions plus chaudes et sèches d'une année augmentaient la PPB.
Cependant l'effet du niveau de la nappe d'eau peut varier selon le contexte : dans une étude sur les effets à long terme d'une variation du niveau de la nappe, \citet{ballantyne2014} montrent qu'une baisse du niveau de la nappe entraîne une augmentation de la PPB en facilitant l'accès des plantes vasculaires à l'oxygène et aux nutriments.
Paradoxalement, un rehaussement du niveau de la nappe d'eau suite à un stress hydrique prolongé conduit également à une augmentation de la PPB \citep{strack2013}.
Pour un gradient croissant de niveaux de nappe d'eau dans un haut-marais, \citet{weltzin2000} montrent une diminution de la productivité des arbustes, tandis que celle des graminées n'est pas affectée.
À l'inverse, pour un gradient similaire dans un bas-marais, la productivité des arbustes n'est pas affectée tandis que celle des graminées augmente.
Des résultats similaires sont également relevés pour des graminées soumises à un réchauffement simulé.
La productivité des graminées diminue dans le haut-marais et augmente dans le bas-marais \citep{weltzin2000}.
Les effets du niveau de la nappe d'eau peuvent donc être variables selon les communautés végétales et le contexte (l'écosystème, le niveau initial) dans lequel elles se trouvent.

\subsubsection{Facteurs contrôlant la RE}
\index{respiration!de l'écosystème!contrôle}

La respiration est limitée par la quantité de substrat (organique labile) et l'accès à l'oxygène.
La qualité du substrat (la facilité qu'il aura à être dégradé) détermine ainsi la vitesse de respiration : 
Moins les substrats sont dégradables plus leur utilisation est lente et plus ils s'accumulent.
Inversement, plus les substrats sont facilement dégradable plus leur utilisation est rapide (respiration potentiellement élevée) 
Cependant ces derniers, rapidement utilisés et épuisés peuvent constituer un facteur limitant de la respiration.
Les sucres, par exemple, peuvent devenir un facteur limitant \citep{gornall2011}.
Les tourbières, du fait de la quantité de matières organiques qu'elles contiennent, constituent un vaste réservoir de substrat organique.
Réservoir de plus en plus difficile à dégrader avec la profondeur car contenant de plus en plus de matière récalcitrante à la dégradation.

À l'échelle de l'écosystème de nombreuses études ont mis en évidence une corrélation positive entre la respiration et la température \citep{singh1977,raich1992,luo2006}.
Cependant la diversité cumulée des processus, des communautés et des conditions environnementales qui influencent la respiration, font qu'aucune équation ne fait réellement consensus.
Cependant la majorité de ces études décrivent une augmentation exponentielle de la respiration avec la température.
Ainsi dans les tourbières, des observations \textit{in-situ} ont montré que dans des conditions plus chaudes, mais également plus sèches (ces deux conditions sont difficilement séparables sur le terrain) la RE a tendance à augmenter  \citep{aurela2007,cai2010,ward2013}.
D'autres observations sur des mésocosmes\footnote{carotte de grande taille non remaniée} de tourbe ont également montré une relation positive entre les variations de RE et celle de la température \citep{updegraff2001,weedon2013}.

Le niveau de la nappe d'eau conditionne l'accès des micro-organismes à l'oxygène, et de ce fait joue un rôle important : un niveau d'eau qui diminue se traduit généralement par une hausse de la RE que ce soit à long terme \citep{strack2006,ballantyne2014} ou à plus court terme \citep{aerts1997}.

De façon plus indirecte, le type de végétation influence la vitesse de décomposition des litières \citep{hobbie1996,liu2000,gogo2015}.
La végétation peut également stimuler la respiration des micro-organismes présents dans la rhizosphère\footnote{Zone du sol impactée par les racines} via la libération d'exsudats racinaires \citep{moore2002}.

\subsubsection{Facteurs contrôlant l'ENE}
\index{echange net de l'ecosystem@échange net de l'écosystème!contrôle}

À l'échelle de l'écosystème le bilan des flux de \coo gazeux est appelé l'échange net de l'écosystème

\begin{pdef}
\textsc{l'Échange Net de l'Écosystème (ENE)} :

Bilan de la quantité de \coo émise ou captée par l'écosystème. C'est la différence entre la Production Primaire Brute et la Respiration de l'Écosystème (ENE=PPB$-$RE).
Ce flux est exprimé en quantité de carbone par unité de surface et de temps.
\end{pdef}
Ce terme correspond, au référentiel près, au \textit{Net Ecosystem Exchange} anglais, qui prend l'atmosphère comme référence\footnote{Attention certains auteurs utilisent une autre convention} (ENE=$-$NEE) \citep{chapin2006}.

Les facteurs contrôlant l'ENE sont donc les mêmes que ceux qui contrôlent la PPB et la RE.
Cependant l'effet d'un même facteur de contrôle peut être différent vis à vis de PPB et de RE selon le contexte environnemental, que ce soit par rapport à la nature de l'effet ou son importance.
Ainsi une variation de l'ENE peut être contrôlée majoritairement soit par la PPB, soit par la RE, soit par les deux.
Par exemple, une baisse du niveau de la nappe est souvent liée dans la littérature à une baisse de l'ENE \citep{aurela2007,peichl2014}.
D'autres études ont montré que cette baisse de l'ENE est due à une augmentation de la respiration \citep{alm1999, ise2008}.
D'autres l'attribuent à une diminution de la photosynthèse \citep{sonnentag2010,peichl2014}.
La baisse de l'ENE peut résulter d'une augmentation de la respiration et de diminution de la photosynthèse \citep{strack2013}.
\citet{lund2012} montrent également que dans un même site, une baisse du niveau de la nappe deux années différentes entraînera une baisse de l'ENE dans les deux cas, mais que dans l'un des cas cette baisse est contrôlée par une augmentation de la respiration et que dans l'autre elle est contrôlée par une diminution de la photosynthèse.
Enfin une étude de \citet{ballantyne2014} ne montre pas d'effet d'une baisse du niveau de la nappe sur l'ENE, car l'augmentation de la respiration est compensée par une augmentation de la photosynthèse.
La réponse des flux de \coo vis-à-vis d'une variation du niveau de la nappe d'eau n'est donc pas triviale.

\subsubsection{Le \chh}

La production du \chh, par des \textit{Archaea} méthanogènes principalement à partir de dihydrogène et d'acétate, est contrôlée par la \textbf{disponibilité} de ces \textbf{substrats} \citep{segers1998}.
L'ajout de substrats (acétate, glucose, éthanol) pour les méthanogènes tend à augmenter les émissions de \chh \citep{coles2002}.
Le \textbf{niveau de la nappe d'eau} associé à l'anoxie, est un autre facteur influençant les flux de \chh.
Généralement, plus le niveau d'eau est élevé, plus la zone potentielle de production du \chh est importante et plus les émissions sont fortes \citep{pelletier2007}.
Par contre, une augmentation du niveau de la nappe au dessus de la surface du sol peut conduire à une diminution des émissions de \chh \citep{bubier1995}.
\citet{pelletier2007} montrent également que les flux sont plus importants lorsque le \chh est mesuré dans des zones avec \textbf{végétation}, et plus particulièrement des carex et des linaigrettes \citep{gogo2011}.
Ce lien avec la végétation est la conséquence d'une adaptation de certaines espèces aux conditions de saturation en eau qui peuvent faciliter l'échange de gaz entre l'écosystème et l'atmosphère grâce à un espace intercellulaire agrandi, l'Aérenchyme \citep{rydin2013d}.
Enfin la \textbf{température} joue généralement un rôle important en augmentant la vitesse de production du \chh.
La sensibilité à la température de la production de \chh varie selon le processus considéré et la communauté de méthanogènes associés \citep{segers1998}.
La température peut également faciliter le transport du \chh par ébullition et/ou via la végétation \citep{lai2009}.

Pour résumer, à l'échelle de l'écosystème un même facteur peut influencer ces différents flux, mais de différentes façons.
Parmi ces facteurs, l'effet du niveau de la nappe d'eau sur les flux de \coo et de \chh reste difficile à prédire.
Ce facteur contrôle l'amplitude des zones oxiques et anoxiques de la colonne de sol et donc la proportion de \coo et de \chh produite.
Il influence également la végétation, que ce soit à court terme (stress hydrique), ou à long terme (changement de communautés végétales).
L'effet d'une hausse du niveau de la nappe d'eau peut varier selon le niveau d'eau initial mais également la végétation présente sur le site.
Pour un même niveau moyen, plus la variation du niveau est importante plus les flux seront fort.
Des effets de chasse ont également été observés après simulation d’événements pluvieux \citep{strack2009}.
La question du niveau de la nappe est donc primordiale et sera explorée dans le chapitre~\ref{ch:4}.


\subsection{Bilans de C à l'échelle de l'écosystème}

Le fonctionnement naturel d'une tourbière active tend à accumuler du C atmosphérique dans l'écosystème, sous la forme de tourbe.
Ce fonctionnement est la conséquence d'entrées de carbone supérieures aux sorties, on parle alors d'un bilan positif, l'écosystème fonctionne en puits de carbone.
Lorsque les sorties sont supérieures aux entrées, le bilan devient négatif et l'écosystème fonctionne comme une source de carbone

Par convention, dans ce document les flux (RE, PPB et \fchh) sont exprimés en valeur absolue afin de faciliter l'étude de leurs variations.
Les bilans sont établis en prenant l'écosystème comme référence, le carbone entrant dans l'écosystème (PPB) est représenté positivement et le carbone sortant (RE, \fchh) négativement.

L'étude de ce bilan dans les tourbières est généralement approchée de deux manières : (i) en évaluant la vitesse d'accumulation du carbone sur une période plus ou moins longue et/ou (ii) en établissant un bilan entre les flux entrants et sortants de l'écosystème actuel.

\subsubsection{Bilan de carbone passé}

L'approche permettant de calculer le bilan de carbone passé d'une tourbière consiste à estimer dans l'archive tourbeuse des vitesses d'accumulation de la tourbe en datant des colonnes de tourbe et en mesurant la quantité de carbone qu'elles contiennent.
Cette méthode, appelée LORCA\footnote{Acronyme anglais pour vitesse apparente d'accumulation du carbone à long terme (\textit{LOng-term apparent Rate of Carbon Accumulation})}, permet d'évaluer la fonction puits sur des temps longs (derniers millénaires) de la comparer à l'actuelle et de relier d'éventuels changements dans les vitesses d'accumulation à des facteurs environnementaux.
Cette approche conduit généralement à des vitesses d'accumulation comprises entre 10 et \SI{30}{\gcma} (Figure~\ref{fig:lorca}).
Ces valeurs, exprimées dans la même unité que les bilans de carbone contemporains, doivent être comparées avec précaution à ces derniers.
En effet elles comprennent, à l'inverse des bilans contemporains, des milliers d'années de décomposition du carbone en profondeur, et ont donc des vitesses d'accumulation sous-estimées relativement à ces bilans \citep{yu2009}.
Selon l'échelle temporelle considérée, peut-être serait-il plus judicieux de dire que les bilans contemporains sont sur-estimés.

\begin{figure}
\centering
\includegraphics[width=\textwidth]{chap1/lorca}
\caption{Vitesse apparente d'accumulation du carbone à long terme durant l'Holocène. Les chiffres entre parenthèses représentent le nombre de mesures. Modifié d'après \citet{yu2009}}
\label{fig:lorca}
\end{figure}

\subsubsection{Bilans de carbone contemporains}

La seconde approche pour estimer le bilan de carbone d'écosystèmes est d'en estimer les flux actuels de carbone entrants et sortants.
Rappelons que les flux principaux dans le bilan de carbone d'une tourbière sont la PPB, la RE et le flux de \chh.
Cependant d'autres flux existent, notamment le flux de carbone organique dissout (COD), de carbone organique particulaire (COP), de carbone inorganique dissout (CID), de Composés Organiques Volatiles (COV), et de monoxyde de carbone (CO) \citep{chapin2006}.
Ils sont considérés comme négligeables, à l'exception du COD \citep{worrall2009}.
On définit ainsi le Bilan de Carbone Net de l'Écosystème (BCNE) comme :

\begin{equation}
BCNE\simeq\overbrace{PPB - RE}^{ENE} - F_{CH_{4}} - F_{COD}
\label{bdc}
\end{equation}

Avec : 
\begin{itemize}
\item ENE : Échange Net de l'Écosystème
\item PPB : Production Primaire Brute
\item RE : Respiration de l'Écosystème
\item F$_{CH_{4}}$ : Flux de Méthane
\item F$_{COD}$ : Flux de Carbone Organique Dissout
\end{itemize}

\begin{figure}
\centering
\includegraphics[width=\textwidth]{chap1/bib_necb}
\caption{Bilan de C dans différentes tourbières (en \si{\gcma}), en fonction de la température moyenne annuelle dans la littérature. Les données et références utilisées pour réaliser ce graphe sont détaillées dans l'annexe~\ref{sec:bibliodata}. La ligne de tirets sépare les écosystèmes stockant du carbone (au dessus) de ceux libérant du carbone (en dessous).}
\label{fig:bib_necb}
\end{figure}


Dans les tourbières, les flux de \coo sont généralement les plus importants puis les flux de \chh et/ou de COD et enfin les flux de COP \citep{worrall2009,koehler2011}.
Majoritairement réalisés dans les tourbières de haut-marais, les bilans de carbone rencontrés dans la littérature sont généralement compris entre 100 et \SI{-750}{\gcma} (Figure~\ref{fig:bib_necb}).
Peu de bilans de carbone ont été faits dans les tourbières en dessous de \SI{50}{\degree} de latitude en Europe (le nord de la France approximativement).
Le comportement de ces tourbières les plus au sud reste peu connu par rapport à celles situées à des latitudes plus hautes ou dans des climats plus froids.