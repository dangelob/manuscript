% % CHAPITRE 1 : SYNTHESE BIBLIO
%
%Réflexions
%
%NE PAS OUBLIER DE CALER UN MAXIMUM D'ORDRE DE GRANDEUR ! 
%	(émissions CO2, CH4, stocks flux, surface des tourbières, végétation ...)
%
%Bien différencier l'intro générale qui doit être lisible par un béotien, de l'intro au travail de la thèse qui doit être un état de l'art précis et documenté sur les travaux antérieurs (synthèse biblio)
%
%Ou caler la partie de biblio sur l'expérimentation ... (peut être dans la synthèse biblio, paragraphe "approche mise en oeuvre"
%INTRODUCTION GENERALE (à mettre dans le chapitre Intro)
%
%- Qu'est ce qu'une tourbière ? (Éventuellememt comment se forme-t-elle ?)
%	*Définition
%	*Formation/Évolution (stockage du C, battement de la nappe ?)
%	*Classification
%- Les tourbières et les hommes 
%	*Usages d'hier et d'aujourd'hui (Combustible, horticulture, matériau de construction)
%	*Les thématiques scientifiques (pourquoi les avoir étudier et les étudier en gros)
%	*Le contexte dans lequel va s'inscrire le travail qui suit
%
%SYNTHESE BIBLIOGRAPHIQUE
%
%- Quelles sont les grandes thématiques de recherche liées aux tourbières ?
%	*Exploitation
%	*Archives
%	*Émissions de GES
%- Plus précisément quelles sont les grands axes de recherche sur ces écosystèmes et liés aux émissions de GES.
%	*Processus de création de GES (CO2 et CH4) (Facteurs contrôlant généralement invoqués)
%	*Processus de migration des GES dans le profil
%	*Processus de stockage/capture
%- Approches mise en oeuvre
%	*Modélisation (empirique et mécaniste)
%	*Expérimentation (différentes techniques pour mesurer les émissions de GES, différentes techniques de chambre...)
%	*Variabilité spatio-temporelle (notion d'échelle)
%	
%	DOIS-JE TRAITER
%	La classification des tourbières ?
%	Hummock and Hollow ? Dire qu'on n'est pas dans ce niveau de détail ?
%	
%
%HISTORIQUE (études concernant les tourbières)
%
%1968-1969  Boelter : propriétés physique des tourbes
%1977 Boelter : hydrologie, caractéristique des sols organiques, chimie des écoulements
%1978 Ingram : Classification
%1981 Parkinson : Déjà l'amélioration d'une méthode pour la mesure des émissions de la respiration du sol
%1984 Clymo : Les limites à la croissance des tourbières
%1986 Chason : Conductivités hydraulique et propriétés physiques
%1989 Moore : Influence du niveau de la nappe sur les émissions de CO2 et de CH4
%//1990 Raich : Comparaison de 2 méthodes de chambre statique pour mesurer les flux de CO2
%1991 Gorham : Rôle des tourbières dans le cycle du carbon et réponse au changement climatique
%//1992 Raich : Flux de CO2 dans la respiration du sol et relation vis à vis du climat et de la végétation
%1992 Roulet : Flux de méthane (fen) et changement climatique
%1993 Bubier : Émission de méthane dans les zones humides
%1993 Bubier : Microtopographie et flux de méthane dans tourbières boréales.
%1993 Abbès : Sorption de l'ammonium ? (ammonia) par la tourbe et fractionnement de l'azote
%1994 Lloyd : Dépendance de la respiration du sol à la température
%1994 Bubier : Perspective écologique sur les émissions de méthane dans les zones humide de l'hémisphère nord
%//1994 Nay : Biais des méthodes de chambre pour la mesure des flux de CO2
%1995 Kirschbaum : Dépendance à la température de la décomposition de la MO (effet sur stock de C et changement Clim)
%1995 Bubier : Prédiction des émissions de méthane à partir de la distribution des bryophytes (tourbière hémisphère nord)
%1995 Bubier : Contrôles "écologique" sur les émissions de méthane dans les tourbières de l'hémisphère nord
%1995 Bubier : Relation entre la végétation avec les émissions de méthane et les gradients hydrochimique
%//1995 Bekku : Mesure de la respiration du sol avec une méthode de chambre fermée (IRGA)

% PLAN (2015-03-03)
%I. Définitions
%1 Tourbières/Tourbe (surface, type, localisation, biodiversité, services écologiques...)
%2 Classification
%3 Historique
%	a Utilisation
%	b Études scientifiques
%	
%Transition : Réaction aux changements globaux (comment fonctionnent-elles ?)
%
%II. Fonctionnement
%1 Stock
%2 Flux
%	a Entrants (Photosynthèse)
%	b Sortants (Méthanogénèse, Respiration)
%3 Facteurs Contrôlant
%	a Hydrologie (WTL,HR)
%	b Propriétés physiques (T, densités, conductivités thermiques...)
%	c Végétation (Bryophytes/Vasculaires)
%	d Météo

% GO TO intro générale ?
%Depuis quand sont-elles étudiées ?
%D'abord étudiées pour leurs propriétés physiques afin de connaitre leur qualité en tant que combustible.
%Elle sont maintenant majoritairement étudiée à travers le prisme des changements globaux.
%Ainsi les études concernent les flux de GES, ...

\chapter{Synth\`{e}se Bibliographique}
\newpage


Dans ce chapitre, nous commenceront par donner une vue de ce que sont les tourbières : Que sont-elles ? Depuis quand sont-elles étudiées ? Pourquoi les a-t-on étudiés ?
Nous continuerons en entrant plus en détails sur leur fonctionnement vis à vis des flux de carbone.
Enfin nous verrons quels sont les facteurs contrôlant majeurs de ces flux.

\section{Les tourbières et le cycle du carbone}

\subsection{Définitions}
Les tourbières font partie d'un ensemble d'écosystèmes plus large que l'on appelle les zones humides.
Les zones humides ne sont ni des écosystèmes terrestres au sens strict, ni des écosystèmes aquatiques.
Elles sont à la frontière, un mix des deux.
Les zone humides sont caractérisées par un niveau de nappe élevé, proche de la surface du sol, voire au dessus.
L'omniprésence de l'eau entraîne une autre caractéristique : la faible aération de ces zones contraint plus ou moins l'accès à l'oxygène.
Il résulte des deux points précédents le développement dans ces milieux d'une végétation spécifique qui s'est adaptée aux milieux fortement humides ou inondés.
Les zones humides regroupent des écosystèmes très variés parmi lesquels les marais, les mangroves, les plaines d'inondations et les tourbières.

Les tourbières représentent 50 à \SI{70}{\percent} des zones humides \cite{francez2000,joosten2002}.
Les estimations de la surface occupée par les tourbières est d'environ \SI{4000000}{\square\kilo\meter} \cite{lappalainen1996}.
Cette surface correspond à \SI{3}{\percent} de l'ensemble des terres émergées du globe.
Plus de \SI{85}{\percent} d'entre elles sont situés dans l'hémisphère nord, majoritairement dans les zones boréales et sub-boréales \cite{society2008}.
Les limites floues de ces écosystèmes rend difficile l'estimation de leur surface.
Elles rendent également ardue leur classification.
Il existe différents types de tourbières, notamment on distingue des tourbières tempérées/boréales des tourbières tropicales dont le fonctionnement diffère.
Dans la suite de ce document seule les tourbières tempérées/boréales seront décrites et étudiées.
De nombreux critères existent pour classer les tourbières selon leur mode de formation, leur source d'eau, leur physico-chimie.
La terminologie utilisée concernant ces écosystèmes n'a pas toujours été cohérente, de nombreux termes ont été utilisés parfois en contradiction les uns avec les autres \cite{joosten2002}, il est donc nécessaire de définir les termes utilisés par la suite. 
Une définition régulièrement utilisée pour caractériser ce qu'est une tourbière est : "Tout écosystème possédant au moins 30 cm de tourbe".
Cette définition correspond au \textit{peatland} anglo-saxon.
Une autre définition existe  : "écosystème dans lequel un processus de tourbification est actif" qui correspond au \textit{mire} anglo-saxon qui peut être traduit en français par tourbière active.
Les deux concepts se chevauchent mais ne sont pas complètement similaire : une tourbière drainée peut avoir plus de \SI{30}{cm} de tourbe et n'être plus active.
À l'inverse il peut exister des zones ou l'épaisseur de tourbe est inférieure à \SI{30}{cm} malgré un processus de tourbification actif.
Dans les deux cas ces définitions en appellent d'autre : Qu'est ce que la tourbe et la tourbification ?
La tourbe est le résultat de l'accumulation et de la, faible, dégradation de litières végétales.
C'est ce que l'on appelle la tourbification.


\subsection{Caractéristiques spécifiques}

\subsubsection{Biodiversité}

%Ces écosystèmes sont le siège d'une biodiversité spécifique relativement importante et rendent un certain nombre de services écologiques.
%Parmi la végétation caractéristique de ces écosystèmes, les sphaignes, des bryophytes (des mousses) sont normalement présentes en abondance.
%Les sphaignes ont quelques particularités qu'il convient de mentionner.
%Ce sont des espèces ingénieures, capable de modifier le milieu dans lequel elle vivent afin de l'adapter à leur besoin.
%Plus spécifiquement elles sont capable d'acidifier leur milieu, de capter les nutriments provenant de l'eau de pluie et de les séquestrer afin de défavoriser d'autres végétaux.
Les tourbières sont le siège d'une biodiversité importante et spécifique.
Ainsi les Sphaignes, qui sont des bryophytes, (des mousses) sont caractéristiques des écosystèmes tourbeux.
Ce sont des espèces dites ingénieures, capable de modifier l'environnement dans lequel elles vivent afin de l'adapter à leurs besoins.
Les sphaignes sont ainsi capable d'abaisser le pH, de capter des nutriments et de les séquestrer et ce même quand elles n'en ont pas besoin afin d'empêcher d'autres espèces notamment vasculaire d'en profiter.
Plus précisément, le fait que les sphaignes captent les nutriments via leur capitulum leur permet de les intercepter avant qu'ils ne soient captés par d'éventuelles racines positionnées plus bas.
Les sphaignes, comme de nombreuse mousses ont des litières relativement récalcitrante\footnote{il est d'usage de parler de litières récalcitrantes sans plus de précision. Il s'agit en fait de litières difficilement dégradables}.

\subsubsection{Puits de carbone}
Les tourbières ne comptent que pour \SI{3}{\percent} des surfaces terrestres émergées, malgré tout leur importance est plus grande que ce que leur surface peut laissé supposer.
En effet la tourbe, accumulation de matières organiques, stocke d'importantes quantités de carbone.
Les estimations du stock de carbone présent dans les tourbières tempérées/boréales se situent entre 270 et \SI{455}{\giga\tonne\,C} \cite{gorham1991,turunen2002}.
La différence entre les estimations s'expliquent notamment par la difficulté à cartographier ces écosystèmes à l'échelle globale, comme expliqué précédemment, mais aussi car il est difficile d'estimer une épaisseur moyenne de tourbe.
Néanmoins les tourbières stockent entre 10 et \SI{25}{\percent} du carbone présent dans les sols et entre 30 et \SI{60}{\percent} du stock de carbone atmosphérique.
Ce stock est un héritage datant des 10 derniers milliers d'années, l'holocène, période pendant laquelle se sont formés la majorité des tourbières.
Le fonctionnement naturel de ces écosystèmes permet le stockage du C.
C'est un des services écologiques que rendent les tourbières et que l'on appelle la fonction puits de carbone.
Cette fonction est liée an niveau élevé de la nappe d'eau, qui rend l'accès à l'oxygène est plus difficile diminuant d'autant l'activité aérobie, dont la respiration des micro-organismes et des plantes.
Cela ce traduit par une dégradation relativement faible des matières organiques.
Elle est également liée à la production de litière récalcitrante par les bryophytes.

En comparaison avec un sol forestier, l'accumulation de matières organiques n'est donc pas lié à une production primaire plus forte, mais bien à une dégradation des matières produites plus faible.

Ces perturbations peuvent induire des modifications de fonctionnement, notamment l'envahissement de ces écosystèmes par une végétation vasculaire, et changer cette fonction puit

%\subsection{Le cycle global}
%
%Au cours des temps les tourbières ont donc accumulé du carbone... stock
%La vitesse de stockage a pu varier au cours du temps mais elle est estimé à XXXX, ainsi la majorité des tourbières actuelles ont un stock qui remonte à quelques milliers d'années.
%Les estimations précise du stock de C présent dans ces écosystèmes sont délicates, à la fois car la définition de ce qu'est une tourbière que varier selon les régions, mais également car leur étendue exacte n'est pas triviale à estimer, pas davantage que leur profondeur moyenne.
%Cependant il est usuellement admis que le stock de carbone se situe entre 270 et 500 Gt de C
%Les tourbières ont donc accumulées du carbone au cours des 10 derniers milliers d'années.
%Pour ce faire il a donc fallu que davantage de carbone soit capturé que de carbone libéré par l'écosystème.



\subsection{Les tourbières et les changements globaux}

\subsubsection{Homme}
Ces écosystèmes ont été et sont encore perturbés par différentes activités humaines, notamment l'agriculture, l'utilisation de la tourbe comme combustible, et comme substrat horticole.

\subsubsection{Climat}
L'impact anthropique direct n'est par la seule perturbation auxquelles sont soumises les tourbières.
D'après les modèles de prédictions du GIEC, les tourbières, comme de nombreux autres écosystèmes, vont subir un changement climatique important dans les années à venir.
Toujours d'après le GIEC, les changements les plus rapides que ce soit en terme de précipitations ou de température sont à attendre dans les zones boréales dans lesquelles se situe la majorité des tourbières.
De ce constat découle un certain nombre de questions concernant ces écosystèmes et notamment le devenir de leur fonction puits de carbone.


Toutes ces perturbations posent notamment la question de la pérennité de la fonction puit de carbone de ces écosystèmes.

\section{Flux de gaz à effet de serre et facteurs contrôlants}

\subsection{Les flux entre l'atmosphère et les tourbières}

\subsubsection{Les flux entrants}
%\subsection{Assimilation du carbone atmosphérique}

Le carbone est principalement présent dans l'atmosphère sous forme de dioxide de carbone (\COO) et de méthane (\CHH).
Comparé au CO2, le CH4 est un GES qui est bien moins présent dans l'atmosphère (CHIFFRES!).
Cependant son "pouvoir de réchauffement" est bien plus important (effet radiatif CO2 x 100) (CHIFFRES !) (D'abord la vapeur d'eau, ensuite le CO2 et enfin le CH4)
Il est usuellement convenu (???? ref) que dans une tourbière le méthane représente environ \SI{5}{\percent} du bilan de C.
\textbf{Devenir du méthane atm}
Le transfert du \COO atmosphérique vers la biosphère (de l'atmosphère à la tourbe) est principalement \plop liée à la photosynthèse.
La photosynthèse est la réaction photochimique permettant l'assimilation du \COO par les végétaux chlorophylliens.
\textbf{dans le but de ?}.

\textbf{Détails ?}

Si la photosynthèse est un processus majeur d'assimilation du \COO, il existe d'autres vois métaboliques permettant la capture du \COO de l'atmosphère.
Ainsi les micro-organismes chemolithotrophes (\textbf{expliciter}) sont capables d'assimiler le \COO en utilisant l'énergie issue de l'oxydation de composés inorganiques.

Les voies métaboliques permettant l'assimilation du \COO sont plutôt bien connues (farquhar) et le fait que les substrats de départ de varient pas (sur ?) a permis une compréhension relativement fine du processus.
Cependant une fois assimilé par la végétation le devenir du carbone est moins direct.

%\subsection{Devenir du carbone assimilé}
%\subsubsection{libération du carbone ? Respiration}
\subsubsection{Les flux sortants}
Dans les tourbières le CO2 est produit par des sources multiples.
Ces sources sont la respiration des de la flore qu'elle soit aérienne ou souterraine et la respiration microbienne.
Une autre source de CO2 est l'oxydation du CH4 lors de sa migration des zones anoxiques aux zones oxiques de la colonne de tourbe.
Enfin dans les zones anaérobie, le CO2 peut être produit par fermentation (respiration anaérobie).
La production de CO2 est donc un signal intégré sur l'ensemble de la colonne de tourbe. 
C'est cette multitude de processus qui rend l'estimation de ce flux difficile, en effet chacune des respirations n'aura pas la même sensibilité vis à vis de facteurs contrôlant.
La respiration de l'écosystème (RE) est définie comme l'ensemble des respirations de la colonne de tourbe, en incluant à la fois sa partie aérienne et sa partie souterraine.
La respiration du sol (SR) est elle définie comme l'ensemble des respirations de la colonne de tourbe, en excluant la partie aérienne.
La respiration du sol comprend donc principalement les respirations issues de la rhizosphère et des communautés de micro-organisme.

Les tourbières sont des écosystèmes dont la production primaire est estimée à environ \SI{500}{\gcm} \cite{francez2000}. 


La strate muscinale pouvant jouer/participer/produire jusqu'à \SI{80}{\percent} de la production primaire \cite{francez2000}.
Cette production primaire n'est pas particulière élevée \plop et c'est en fait la faible décomposition des matières organiques qui permet aux tourbières de stocker du carbone.
L'accumulation moyenne estimée dans les tourbières boréales est de \SI{30}{\gcm}. Le taux d'accumulation varie en fonction des espèces végétales présentes (\plop), le niveau d'eau (\plop), ... (??)



\subsubsection{storage ?}

Le carbone assimilé par photosynthèse, utilisé par la plante puis évacué que se soit sous forme d'exudats racinaire ou de matériels morts, de litière, va en partie se dégrader.
Continum de dégradation avec des matières organiques de plus en plus récalcitrantes avec la profondeur.

La vitesse de stockage au cours du temps ?

L'accumulation de matières organiques et donc de carbone dans les tourbières est donc fonction de la prépondérance relative de ces flux entre l'écosystème et l'atmosphère.

\subsection{Les facteurs majeurs contrôlant les flux}



Ces flux sont contrôlés par différents facteurs.
Parmi ceux qui sont le plus souvent cité figure la température, le niveau de la nappe et la végétation.
%Plus rarement, la disponibilité du substrat, la texture du sol le pH ou enc.
%L'intrication des différents facteurs rend difficile d'isoler

%\subsubsection{Facteurs majeurs}
%La température
L'augmentation de la vitesse de réaction de nombreuses réactions biochimiques avec la température est connue depuis longtemps.
Elle a été mise en évidence par un chimiste suédois en 1889 : Svante August Arrhenius sur la base de travaux réalisés par un autre chimiste, néerlandais, Jacobus Henricus Van't Hoff.
Depuis, de nombreuses mesures de terrain confirment cette relation \plop
La photosynthèse et l'ensemble des respirations sont donc contrôlées, au moins en partie, par la température.
%L'hydrologie
L'hydrologie est comme nous l'avons précisé un peu plus haut, un facteur d'une grande importance dans les tourbières.
Nous distinguerons ici le niveau de la nappe qui est la hauteur sous la surface du sol permettant d'accéder à la zone saturée ? à l'eau "libre" ?
Et la teneur en eau du sol qui est une estimation de la quantité d'eau présente dans le sol.

L'effet du niveau de la nappe \par
Le niveau de la nappe est important car il sépare la colonne de tourbe en une zone oxique, ou il y a présence d'oxygène, et une zone anoxique dans laquelle l'oxygène est absent.
Ces deux zones vont avoir des comportements différents.
La zone anoxique, sous le niveau de la nappe, est une zone dans laquelle la production de CO2 est très faible car sans oxygène seule les processus de respiration anaérobie peuvent avoir lieu.
Par contre dans c'est dans cette zone que sera produit le méthane.
La zone oxique, proche de la surface, va permettre à la fois aux racines et aux micro-organismes de respirer.
Cette zone est donc l'endroit ou est produit la majorité du CO2, l'endroit ou la matière organique est le plus dégradée.
Lors de la migration du méthane dans la colonne de tourbe ce dernier aura tendance à être oxydé en CO2 lors de son passage dans cette zone oxique.
Certaines plantes permettent cependant au méthane de passer à travers l'aerenchyme et d'éviter ainsi d'être oxydé.

L'effet de l'humidité relative \par

Résilience de la tourbe\\
Les propriétés physique de la tourbe jouent bien évidemment un rôle important sur cette capacité de rétention d'eau.
Cependant dans le cas d'épisode de sécheresse important, il a été constaté que ces capacités n'était pas immédiatement recouverte en totalité.
%La végétation
Les communautés végétales évoluent en parallèle de l'évolution de la tourbière (succession végétale).
Les tourbières sont le siège d'une végétation caractéristique : Les sphaignes.
Ces bryophytes sont la clef de voûte de ces écosystèmes d'abord parce que leur litière sont moins facilement dégradable que celle des espèces vasculaires.
Ensuite parce qu'elle favorisent dans leur environnement local, les conditions favorable à leur développement. 
On les appelle d'ailleurs des espèces ingénieures.
Ces végétaux sans racines ont également une grande capacité à retenir l'eau (ce sont de véritables éponges) retenant également les nutriments. 
Ceci favorisant un milieu pauvre en nutriment et donc défavorable aux autres espèces (vasculaires?).
Il existe un grand nombre d'espèce de sphaignes (CHIFFRES+REF).
Par la suite il ne sera pas fait de distinction entre les différentes espèces présentes sur les différents sites étudiés.
Cependant dans de nombreuses tourbières on constate un envahissement par des végétaux vasculaires.
Ces plantes, sont souvent des pins, des bouleaux et des molinie ?
Elles ont un effet sur la production de CO2 principalement en aérant le sol, permettant à l'oxygène de migrer plus loin dans le profil, permettant à l'activité aérobie (plus efficace) d'agir sur une plus grande profondeur.
Ces végétaux peuvent également pomper de l'eau en quantité (arbre?) ?


D'autres facteurs à évoquer ?

%La température est le premier facteur contrôlant les flux.
%Comme pour toute (la majorité ? y a-t-il des réactions chimiques non influencées par la température ?) les réactions chimique la température influe sur les vitesses de réactions. 
%Plus la température augmente plus la vitesse de réaction augmente.
%La température à donc un rôle important à jouer au niveau des flux.

\subsubsection{Facteurs contrôlant la respiration de l'écosystème}
Updegraf2001\\
Montre, dans une expérimentation à base de mésocosme, que la respiration de l'écosystème est contrôlée presque exclusivement par la température du sol.

Cai2010\\
Mesures in-situ, sécheresse court terme, plus chaud et plus sec (1an).
Sensibilité à la température (Q10) identique l'année humide et l'année sèche.
Dans les conditions plus chaude et plus sèche Cai observe une augmentation de la Respiration (plus forte que celle de la photosynthèse)

Stratck2006 \\
Augmentation de la respiration suite à un abaissement du niveau de l'eau (8ans plus tôt).

Ballantyne2014 \\
dans une expérimentation in-situ, montre une respiration de l'écosystème plus importante quand le niveau de la nappe est bas que lorsque le niveau de la nappe est haut.
L'expérimentation se fait sur un site dont l'abaissement de la nappe est effectif depuis longtemps (80 ans plus tôt)
Même résultat que strack, donc effet présent même sur le long terme.

\subsubsection{Facteurs contrôlant la production primaire brute}
Si la diversité des réactions est moindre pour la photosynthèse, sa réponse aux variables environmentales à l'échelle de l'écosystème n'en est pas moins difficile à prédire.
Comme pour la respiration, l'augmentation de la température augmente la vitesse de réaction (Cai2010).
\plop
L'effet d'une variation du niveau de la nappe est cependant moins évidente.
La baisse du niveau de la nappe peut à la fois induire une augmentation de la PBB, notamment quand elle favorise la végétation vasculaire (Ballantyne2014).
Mais elle peut également la diminuer, lorsqu'elle induit un stress hydrique important (Strack \& Zuback 2013, Peichl 2014, Alm1999, Griffis2000, Weltzin2000)

\subsubsection{Facteurs contrôlant l'ENE}
Une baisse du niveau de la nappe induit souvent une baisse de l'ENE.
Cependant certain attribue cette baisse à une augmentation de la Respiration (, Aurela2013, Ballantyne2014, Alm1999, Ise2008, Oechel1993) quand d'autres l'attribue à une diminution de la photosynthèe Sonnentag2010, Peicl2014 
Enfin certain voient un effet à la fois de l'augmentation de la respiration et de la diminution de la photosynthèse (StrackZuback2013)

À noter un article intéressant (Lund2012) dans lequel, dans un même site une baisse du niveau de la nappe 2 année différente entrainera une baisse de l'ENE dans les 2 cas, mais dans l'un des cas cette baisse est contrôlée par un augmentation de la respiration et dans l'autre cas cette baisse est contrôlée par une diminution de la photosynthèse.

Également un article de Ballantyne2014 qui lui ne note pas d'effet d'une baisse du niveau de la nappe sur l'ENE car l'augmentation de la respiration est compensée par une augmentation de la photosynthèse.

\subsubsection{Facteurs contrôlant les flux de méthane}

%\subsubsection{L'hydrologie dans les tourbières et l'effet sur les flux}

%\subsubsection{La végétation dans les tourbières et l'effet sur les flux}


La prépondérance relative des ces différents flux, contrôlée par les conditions environnementale, va donc impacter le fonctionnement des tourbières. 
Soit elles stockent du carbone, en accumulant des matières organiques, et donc fonctionnent comme des puits ou soit elle relâchent du carbone et fonctionnent comme des sources.


L'étude individuelle de tel ou tel flux avec tel ou tel facteur contrôlant est nécessaire afin de comprendre ce qu'il se passe au niveau des processus.
Il est tout aussi nécessaire d'arriver à intégrer l'ensemble de la complexité naturelle.
C'est l'intérêt d'établir des bilans de carbone.

\subsection{Bilans de carbone}

Les flux gazeux entrants et sortant des écosystèmes tourbeux ont été précisé précédemment.
Il s'agit bien sur des respirations (\COO et \CHH) et de la photosynthèse.
Cependant d'autres flux de C peuvent jouer sur le bilan de carbone : 
Les flux dissous, le carbone organique dissous et de carbone inorganique dissous.
Les flux de carbone particulaire, et plus anecdotiquement les flux liées au composés organo-volatils (COV), au monoxyde de carbone.

Les bilans les plus complets réalisées sur les tourbières comprennent la partie gazeuse, dissoute...

\subsubsection{passé}

\subsubsection{présent}

