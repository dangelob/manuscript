% % CHAPITRE 1 : SYNTHESE BIBLIO
%
%Réflexions
%
%NE PAS OUBLIER DE CALER UN MAXIMUM D'ORDRE DE GRANDEUR ! 
%	(émissions CO2, CH4, stocks flux, surface des tourbières, végétation ...)
%
%Bien différencier l'intro générale qui doit être lisible par un béotien, de l'intro au travail de la thèse qui doit être un état de l'art précis et documenté sur les travaux antérieurs (synthèse biblio)
%
%Ou caler la partie de biblio sur l'expérimentation ... (peut être dans la synthèse biblio, paragraphe "approche mise en oeuvre"
%INTRODUCTION GENERALE (à mettre dans le chapitre Intro)
%
%- Qu'est ce qu'une tourbière ? (Éventuellememt comment se forme-t-elle ?)
%	*Définition
%	*Formation/Évolution (stockage du C, battement de la nappe ?)
%	*Classification
%- Les tourbières et les hommes 
%	*Usages d'hier et d'aujourd'hui (Combustible, horticulture, matériau de construction)
%	*Les thématiques scientifiques (pourquoi les avoir étudier et les étudier en gros)
%	*Le contexte dans lequel va s'inscrire le travail qui suit
%
%SYNTHESE BIBLIOGRAPHIQUE
%
%- Quelles sont les grandes thématiques de recherche liées aux tourbières ?
%	*Exploitation
%	*Archives
%	*Émissions de GES
%- Plus précisément quelles sont les grands axes de recherche sur ces écosystèmes et liés aux émissions de GES.
%	*Processus de création de GES (CO2 et CH4) (Facteurs contrôlant généralement invoqués)
%	*Processus de migration des GES dans le profil
%	*Processus de stockage/capture
%- Approches mise en oeuvre
%	*Modélisation (empirique et mécaniste)
%	*Expérimentation (différentes techniques pour mesurer les émissions de GES, différentes techniques de chambre...)
%	*Variabilité spatio-temporelle (notion d'échelle)
%	
%	DOIS-JE TRAITER
%	La classification des tourbières ?
%	Hummock and Hollow ? Dire qu'on n'est pas dans ce niveau de détail ?
%	
%
%HISTORIQUE (études concernant les tourbières)
%
%1968-1969  Boelter : propriétés physique des tourbes
%1977 Boelter : hydrologie, caractéristique des sols organiques, chimie des écoulements
%1978 Ingram : Classification
%1981 Parkinson : Déjà l'amélioration d'une méthode pour la mesure des émissions de la respiration du sol
%1984 Clymo : Les limites à la croissance des tourbières
%1986 Chason : Conductivités hydraulique et propriétés physiques
%1989 Moore : Influence du niveau de la nappe sur les émissions de CO2 et de CH4
%//1990 Raich : Comparaison de 2 méthodes de chambre statique pour mesurer les flux de CO2
%1991 Gorham : Rôle des tourbières dans le cycle du carbon et réponse au changement climatique
%//1992 Raich : Flux de CO2 dans la respiration du sol et relation vis à vis du climat et de la végétation
%1992 Roulet : Flux de méthane (fen) et changement climatique
%1993 Bubier : Émission de méthane dans les zones humides
%1993 Bubier : Microtopographie et flux de méthane dans tourbières boréales.
%1993 Abbès : Sorption de l'ammonium ? (ammonia) par la tourbe et fractionnement de l'azote
%1994 Lloyd : Dépendance de la respiration du sol à la température
%1994 Bubier : Perspective écologique sur les émissions de méthane dans les zones humide de l'hémisphère nord
%//1994 Nay : Biais des méthodes de chambre pour la mesure des flux de CO2
%1995 Kirschbaum : Dépendance à la température de la décomposition de la MO (effet sur stock de C et changement Clim)
%1995 Bubier : Prédiction des émissions de méthane à partir de la distribution des bryophytes (tourbière hémisphère nord)
%1995 Bubier : Contrôles "écologique" sur les émissions de méthane dans les tourbières de l'hémisphère nord
%1995 Bubier : Relation entre la végétation avec les émissions de méthane et les gradients hydrochimique
%//1995 Bekku : Mesure de la respiration du sol avec une méthode de chambre fermée (IRGA)

% PLAN (2015-03-03)
%I. Définitions
%1 Tourbières/Tourbe (surface, type, localisation, biodiversité, services écologiques...)
%2 Classification
%3 Historique
%	a Utilisation
%	b Études scientifiques
%	
%Transition : Réaction aux changements globaux (comment fonctionnent-elles ?)
%
%II. Fonctionnement
%1 Stock
%2 Flux
%	a Entrants (Photosynthèse)
%	b Sortants (Méthanogénèse, Respiration)
%3 Facteurs Contrôlant
%	a Hydrologie (WTL,HR)
%	b Propriétés physiques (T, densités, conductivités thermiques...)
%	c Végétation (Bryophytes/Vasculaires)
%	d Météo

\chapter{Synth\`{e}se Bibliographique}
\newpage


Dans ce chapitre, nous commenceront par donner une vue de ce que sont les tourbières : Que sont-elles ? Depuis quand sont-elles étudiées ? Pourquoi les a-t-on étudiés ?
Nous continuerons en entrant plus en détails sur leur fonctionnement vis à vis des flux de carbone.
Enfin nous verrons quels sont les facteurs contrôlant majeurs de ces flux.
\section{Qu'est ce qu'une tourbière ?}
Les tourbières font partie d'un ensemble d'écosystèmes que l'on appelle les zones humides.
Les zones humides se définissent comme n'étant ni des écosystèmes terrestres au sens strict ni des écosystèmes aquatiques.
Elles sont un mélange des deux. (ajout sur propriétés ?)
Cette dualité rend l'estimation de leur surface délicate, néanmoins on estime à \textcolor{red}{XX} leur étendu \plop.
Les zones humides regroupent des écosystèmes très variés parmis lesquels les marais, les mangroves, les plaines d'inondations et les tourbières.

Les tourbières représentent 50 à 70 \% des zones humides (\cite{francez2000}, wise use of peatland), s'étendant sur plus de 4000000 de km$^{2}$ (Lappalainen).
La terminologie utilisée concernant ces écosystèmes n'ayant pas toujours été cohérente \cite{joosten2002}, il est nécessaire de définir les termes utilisés par la suite. 
Une définition régulièrement utilisée pour caractériser ce qu'est une tourbière est : "Tout écosystème possédant au moins 30 cm de tourbe".
Cette définition correspond au "peatland" anglo-saxon.
Une autre définition existe  : "écosystème dans lequel un processus de tourbification est actif" qui correspond au "mire" anglo-saxon qui peut être traduit en français par tourbière active.
Les deux concepts se chevauchent mais ne sont pas complètement similaire : une tourbière drainée peut avoir plus de 30 cm de tourbe et n'être plus active.
À l'inverse il peut exister des zones ou l'épaisseur de tourbe est inférieure à 30 cm malgré un processus de tourbification actif.
Dans les deux cas ces définitions en appellent d'autre : Qu'est ce que la tourbe et la tourbification ?
La tourbe est le résultat de l'accumulation et de la, faible, dégradation de litières végétales.
C'est ce que l'on appelle la tourbification.

Il existe différents types de tourbières, notamment on distingue des tourbières tempérées/boréales des tourbières tropicales dont le fonctionnement diffère.
Dans la suite de ce document seule les tourbières tempérées/boréales seront décrites et étudiées.


La structure de la tourbe est liée à une production primaire particulièrement importante, comparée à un sol forestier par exemple, mais plutôt à une dégradation plus faible des matières organiques.
Cette faible dégradation est due notamment au fait que ces écosystèmes, sont saturés en eau une grande partie de l'année (ce sont des zones humides). 
Par conséquent, l'accès à l'oxygène est plus difficile diminuant d'autant l'activité aérobie, dont la respiration des micro-organismes et des plantes.

%Que sont-elles ? 
%Les tourbières font partie d'un ensemble d'écosystèmes que l'on appelle les zones humides.
%Dans ces zones humides, en plus des tourbières sont classés des écosystèmes comme les lacs, les deltas, les mangroves.
%Elles sont définies comme les écosystèmes possédant au moins 30 cm de tourbe.
%Cette règle peut cependant changer selon les régions.
La classification des tourbières n'est pas simple, de nombreux critères existent selon leur mode de formation, leur source d'eau, leur physico-chimie.
Au niveau terminologie, de nombreux termes ont été utilisés parfois en contradiction les uns avec les autres.
Les tourbières sont donc des zones dans lequelles l'eau à une rôle majeur.
%Notamment le niveau de l'eau étant, normalement, élevé il empêche la décomposition des litières en diminuant fortement l'accès à l'oxygène et donc la respiration.
Les tourbières sont le siège d'une biodiversité spécifique relativement importante et rendent un certain nombre de services écologiques.
Parmi la végétation caractéristique de ces écosystèmes, les sphaignes, des bryophytes (des mousses) sont normalement présentes en abondance.Ces écosystèmes on été et sont encore perturbés par différentes activités humaines, notamment l'agriculture, l'utilisation de la tourbe comme combustible, et comme substrat horticole.
Ces perturbations peuvent induire des modifications de fonctionnement, notamment l'envahissement de ces écosystèmes par une végétation vasculaire.
Ces écosystèmes s'étendent sur 4 000 000 de km2.

Depuis quand sont-elles étudiées
D'abord étudiées pour leurs propriétés physiques afin de connaitre leur qualité en tant que combustible.
Elle sont maintenant majoritairement étudiée à travers le prisme des changements globaux.
Ainsi les études concernent les flux de GES, ...

\subsection{Le CO2 processus de productions/dégradations et de transport}
Le CO2 est un des principaux gaz à effet de serre si bien que les autres sont souvent classés en fonction de ce dernier.

** Historique précis des études concernant les GES (CH4)

\subsubsection{La production de CO2 : La respiration de l'écosystème (processus de production)}
Dans les tourbières le CO2 est produit par des sources multiples.
Ces sources sont la respiration des de la flore qu'elle soit aérienne ou souterraine et la respiration microbienne.
Une autre source de CO2 est l'oxydation du CH4 lors de sa migration des zones anoxiques aux zones oxiques de la colonne de tourbe.
Enfin dans les zones anaérobie, le CO2 peut être produit par fermentation (respiration anaérobie).
La production de CO2 est donc un signal intégré sur l'ensemble de la colonne de tourbe. 
C'est cette multitude de processus qui rend l'estimation de ce flux difficile, en effet chacune des respirations n'aura pas la même sensibilité vis à vis de facteurs contrôlant.
La respiration de l'écosystème (RE) est définie comme l'ensemble des respirations de la colonne de tourbe, en incluant à la fois sa partie aérienne et sa partie souterraine.
La respiration du sol (SR) est elle définie comme l'ensemble des respirations de la colonne de tourbe, en excluant la partie aérienne.
La respiration du sol comprend donc principalement les respirations issues de la rhizosphère et des communautés de micro-organisme.

Les tourbières sont des écosystèmes dont la production primaire est estimée à environ 500~g~C.m$^{-2}$ \cite{francez2000}. 
La strate muscinale pouvant jouer/participer/produire jusqu'à 80\% de la production primaire \cite{francez2000}.
Cette production primaire n'est pas particulière élevée \plop et c'est en fait la faible décomposition des matières organiques qui permet aux tourbières de stocker du carbone.
L'accumulation moyenne estimée dans les tourbières boréales est de 30~g~C.m$^{-2}$. Le taux d'accumulation varie en fonction des espèces végétales présentes (\plop), le niveau d'eau (\plop), ... (??)

\subsubsection{La capture du CO2 :(processus de stockage)}
C'est évidemment par photosynthèse que le CO2 est pompé de l'atmosphère pour être stocké dans tissus des végétaux avant d'être en partie non dégradé et donc stockés dans les litières puis dans la tourbe à proprement parler
La vitesse de stockage a pu varier au cours du temps mais elle est estimé à XXXX, ainsi la majorité des tourbières actuelles ont un stock qui remonte à quelques milliers d'années.
Les estimations précise du stock de C présent dans ces écosystèmes sont délicates, à la fois car la définition de ce qu'est une tourbière que varier selon les régions, mais également car leur étendue exacte n'est pas triviale à estimer, pas davantage que leur profondeur moyenne.
Cependant il est usuellement admis que le stock de carbone se situe entre 270 et 500 Gt de C

\subsection{Le CH4 processus de productions/dégradations et de transport}
Comparé au CO2, le CH4 est un GES qui est bien moins présent dans l'atmosphère (CHIFFRES!).
Cependant son "pouvoir de réchauffement" est bien plus important (effet radiatif CO2 x 100) (CHIFFRES !) (D'abord la vapeur d'eau, ensuite le CO2 et enfin le CH4)
Il est usuellement convenu (???? ref) que dans une tourbière le méthane représente environ 5\% du bilan de C.

\section{Les facteurs majeurs contrôlant les flux}
\subsection{La température et les flux}
La température est le premier facteur contrôlant les flux.
Comme pour toute (la majorité ? y a-t-il des réactions chimiques non influencées par la température ?) les réactions chimique la température influe sur les vitesses de réactions. 
Plus la température augmente plus la vitesse de réaction augmente.
La température à donc un rôle important à jouer au niveau des flux.

température et ER
Concernant la, ou plutôt les respirations de l'écosystème, l'influence de la température sera différente selon les sources considérés.
Ainsi à la fois les plantes et les communauté de micro-organisme ne réagiront probablement pas de la même façon, au mêmes moments et avec les même intensités.

température et NEE
Pour la NEE même s'il semble y avoir moins de sources possible puisque seule la végétation photosynthétique est concernée, l'influence de la température est également fonction de la végétation présente.
De plus ce signal est plus ou moins covariant avec la luminosité ce qui ne facilite pas son interprétation.

Synchronisation


\subsection{L'hydrologie dans les tourbières et l'effet sur les flux}
L'hydrologie est comme nous l'avons précisé un peu plus haut, un facteur d'une grande importance dans les tourbières.
Nous distinguerons ici le niveau de la nappe qui est la hauteur sous la surface du sol permettant d'accéder à la zone saturée ? à l'eau "lirbe" ?
Et l'humidité du sol qui est une estimation de la quantité d'eau présente dans la zone non-saturée.

\subsubsection{L'effet du niveau de la nappe}
Le niveau de la nappe est important car il sépare la colonne de tourbe en une zone oxique, ou il y a présence d'oxygène, et une zone anoxique dans laquelle l'oxygène est absent.
Ces deux zones vont avoir des comportements différents.
La zone anoxique, sous le niveau de la nappe, est une zone dans laquelle la production de CO2 est très faible car sans oxygène seule les processus de respiration anaérobie peuvent avoir lieu.
Par contre dans c'est dans cette zone que sera produit le méthane.
La zone oxique, proche de la surface, va permettre à la fois aux racines et aux micro-organismes de respirer.
Cette zone est donc l'endroit ou est produit la majorité du CO2, l'endroit ou la matière organique est le plus dégradée.
Lors de la migration du méthane dans la colonne de tourbe ce dernier aura tendance à être oxydé en CO2 lors de son passage dans cette zone oxique.
Certaines plantes permettent cependant au méthane de passer à travers l'aerenchyme et d'éviter ainsi d'être oxydé.

\subsubsection{L'effet de l'humidité relative}

\subsubsection{Résilience de la tourbe}
Les propriétés physique de la tourbe jouent bien évidemment un rôle important sur cette capacité de rétention d'eau.
Cependant dans le cas d'épisode de sécheresse important, il a été constaté que ces capacités n'était pas immédiatement recouverte en totalité.


\subsection{La végétation dans les tourbières et l'effet sur les flux}
Les communautés végétales évoluent en parallèle de l'évolution de la tourbière (succession végétale).
Les tourbières sont le siège d'une végétation caractéristique : Les sphaignes.
Ces bryophytes sont la clef de voûte de ces écosystèmes d'abord parce que leur litière sont moins facilement dégradable que celle des espèces vasculaires.
Ensuite parce qu'elle favorisent dans leur environnement local, les conditions favorable à leur développement. 
On les appelle d'ailleurs des espèces ingénieures.
Ces végétaux sans racines ont également une grande capacité à retenir l'eau (ce sont de véritables éponges) retenant également les nutriments. 
Ceci favorisant un milieu pauvre en nutriment et donc défavorable aux autres espèces (vasculaires?).
Il existe un grand nombre d'espèce de sphaignes (CHIFFRES+REF).
Par la suite il ne sera pas fait de distinction entre les différentes espèces présentes sur les différents sites étudiés.
Cependant dans de nombreuses tourbières on constate un envahissement par des végétaux vasculaires.
Ces plantes, sont souvent des pins, des bouleaux et des molinie ?
Elles ont un effet sur la production de CO2 principalement en aérant le sol, permettant à l'oxygène de migrer plus loin dans le profil, permettant à l'activité aérobie (plus efficace) d'agir sur une plus grande profondeur.
Ces végétaux peuvent également pomper de l'eau en quantité (arbre?) ?


% % % % % % % % % % % % % % %

\todo[inline]{Descriptif et comparaison des méthodes permettant de mesurer les flux de gaz}


